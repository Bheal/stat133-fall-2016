\documentclass[12pt]{beamer}\usepackage[]{graphicx}\usepackage[]{color}
%% maxwidth is the original width if it is less than linewidth
%% otherwise use linewidth (to make sure the graphics do not exceed the margin)
\makeatletter
\def\maxwidth{ %
  \ifdim\Gin@nat@width>\linewidth
    \linewidth
  \else
    \Gin@nat@width
  \fi
}
\makeatother

\definecolor{fgcolor}{rgb}{0.345, 0.345, 0.345}
\newcommand{\hlnum}[1]{\textcolor[rgb]{0.686,0.059,0.569}{#1}}%
\newcommand{\hlstr}[1]{\textcolor[rgb]{0.192,0.494,0.8}{#1}}%
\newcommand{\hlcom}[1]{\textcolor[rgb]{0.678,0.584,0.686}{\textit{#1}}}%
\newcommand{\hlopt}[1]{\textcolor[rgb]{0,0,0}{#1}}%
\newcommand{\hlstd}[1]{\textcolor[rgb]{0.345,0.345,0.345}{#1}}%
\newcommand{\hlkwa}[1]{\textcolor[rgb]{0.161,0.373,0.58}{\textbf{#1}}}%
\newcommand{\hlkwb}[1]{\textcolor[rgb]{0.69,0.353,0.396}{#1}}%
\newcommand{\hlkwc}[1]{\textcolor[rgb]{0.333,0.667,0.333}{#1}}%
\newcommand{\hlkwd}[1]{\textcolor[rgb]{0.737,0.353,0.396}{\textbf{#1}}}%
\let\hlipl\hlkwb

\usepackage{framed}
\makeatletter
\newenvironment{kframe}{%
 \def\at@end@of@kframe{}%
 \ifinner\ifhmode%
  \def\at@end@of@kframe{\end{minipage}}%
  \begin{minipage}{\columnwidth}%
 \fi\fi%
 \def\FrameCommand##1{\hskip\@totalleftmargin \hskip-\fboxsep
 \colorbox{shadecolor}{##1}\hskip-\fboxsep
     % There is no \\@totalrightmargin, so:
     \hskip-\linewidth \hskip-\@totalleftmargin \hskip\columnwidth}%
 \MakeFramed {\advance\hsize-\width
   \@totalleftmargin\z@ \linewidth\hsize
   \@setminipage}}%
 {\par\unskip\endMakeFramed%
 \at@end@of@kframe}
\makeatother

\definecolor{shadecolor}{rgb}{.97, .97, .97}
\definecolor{messagecolor}{rgb}{0, 0, 0}
\definecolor{warningcolor}{rgb}{1, 0, 1}
\definecolor{errorcolor}{rgb}{1, 0, 0}
\newenvironment{knitrout}{}{} % an empty environment to be redefined in TeX

\usepackage{alltt}
\usepackage{graphicx}
\usepackage{tikz}
\setbeameroption{hide notes}
\setbeamertemplate{note page}[plain]
\usepackage{listings}

% get rid of junk
\usetheme{default}
\usefonttheme[onlymath]{serif}
\beamertemplatenavigationsymbolsempty
\hypersetup{pdfpagemode=UseNone} % don't show bookmarks on initial view

% named colors
\definecolor{offwhite}{RGB}{255,250,240}
\definecolor{gray}{RGB}{155,155,155}

\ifx\notescolors\undefined % slides

  \definecolor{foreground}{RGB}{80,80,80}
  \definecolor{background}{RGB}{255,255,255}
  \definecolor{title}{RGB}{255,199,0}
  \definecolor{subtitle}{RGB}{89,132,212}
  \definecolor{hilit}{RGB}{248,117,79}
  \definecolor{vhilit}{RGB}{255,111,207}
  \definecolor{lolit}{RGB}{200,200,200}
  \definecolor{lit}{RGB}{255,199,0}
  \definecolor{mdlit}{RGB}{89,132,212}
  \definecolor{link}{RGB}{248,117,79}

\else % notes
  \definecolor{background}{RGB}{255,255,255}
  \definecolor{foreground}{RGB}{24,24,24}
  \definecolor{title}{RGB}{27,94,134}
  \definecolor{subtitle}{RGB}{22,175,124}
  \definecolor{hilit}{RGB}{122,0,128}
  \definecolor{vhilit}{RGB}{255,0,128}
  \definecolor{lolit}{RGB}{95,95,95}
\fi
\definecolor{nhilit}{RGB}{128,0,128}  % hilit color in notes
\definecolor{nvhilit}{RGB}{255,0,128} % vhilit for notes

\newcommand{\hilit}{\color{hilit}}
\newcommand{\vhilit}{\color{vhilit}}
\newcommand{\nhilit}{\color{nhilit}}
\newcommand{\nvhilit}{\color{nvhilit}}
\newcommand{\lit}{\color{lit}}
\newcommand{\mdlit}{\color{mdlit}}
\newcommand{\lolit}{\color{lolit}}

% use those colors
\setbeamercolor{titlelike}{fg=title}
\setbeamercolor{subtitle}{fg=subtitle}
\setbeamercolor{frametitle}{fg=gray}
\setbeamercolor{structure}{fg=subtitle}
\setbeamercolor{institute}{fg=lolit}
\setbeamercolor{normal text}{fg=foreground,bg=background}
%\setbeamercolor{item}{fg=foreground} % color of bullets
%\setbeamercolor{subitem}{fg=hilit}
%\setbeamercolor{itemize/enumerate subbody}{fg=lolit}
\setbeamertemplate{itemize subitem}{{\textendash}}
\setbeamerfont{itemize/enumerate subbody}{size=\footnotesize}
\setbeamerfont{itemize/enumerate subitem}{size=\footnotesize}

% center title of slides
\setbeamertemplate{blocks}[rounded]
\setbeamertemplate{frametitle}[default][center]
% margins
\setbeamersize{text margin left=25pt,text margin right=25pt}

% page number
\setbeamertemplate{footline}{%
    \raisebox{5pt}{\makebox[\paperwidth]{\hfill\makebox[20pt]{\lolit
          \scriptsize\insertframenumber}}}\hspace*{5pt}}

% add a bit of space at the top of the notes page
\addtobeamertemplate{note page}{\setlength{\parskip}{12pt}}

% default link color
\hypersetup{colorlinks, urlcolor={link}}

\ifx\notescolors\undefined % slides
  % set up listing environment
  \lstset{language=bash,
          basicstyle=\ttfamily\scriptsize,
          frame=single,
          commentstyle=,
          backgroundcolor=\color{darkgray},
          showspaces=false,
          showstringspaces=false
          }
\else % notes
  \lstset{language=bash,
          basicstyle=\ttfamily\scriptsize,
          frame=single,
          commentstyle=,
          backgroundcolor=\color{offwhite},
          showspaces=false,
          showstringspaces=false
          }
\fi

% a few macros
\newcommand{\code}[1]{\texttt{#1}}
\newcommand{\hicode}[1]{{\hilit \texttt{#1}}}
\newcommand{\bb}[1]{\begin{block}{#1}}
\newcommand{\eb}{\end{block}}
\newcommand{\bi}{\begin{itemize}}
%\newcommand{\bbi}{\vspace{24pt} \begin{itemize} \itemsep8pt}
\newcommand{\bbi}{\vspace{4pt} \begin{itemize} \itemsep8pt}
\newcommand{\ei}{\end{itemize}}
\newcommand{\bv}{\begin{verbatim}}
\newcommand{\ev}{\end{verbatim}}
\newcommand{\ig}{\includegraphics}
\newcommand{\subt}[1]{{\footnotesize \color{subtitle} {#1}}}
\newcommand{\ttsm}{\tt \small}
\newcommand{\ttfn}{\tt \footnotesize}
\newcommand{\figh}[2]{\centerline{\includegraphics[height=#2\textheight]{#1}}}
\newcommand{\figw}[2]{\centerline{\includegraphics[width=#2\textwidth]{#1}}}



%------------------------------------------------
% end of header
%------------------------------------------------

\title{XML - Part 1}
\subtitle{STAT 133}
\author{\href{http://www.gastonsanchez.com}{Gaston Sanchez}}
\institute{\href{https://github.com/ucb-stat133/stat133-fall-2016}{\tt \scriptsize \color{foreground} github.com/ucb-stat133/stat133-fall-2016}}
\date{}
\IfFileExists{upquote.sty}{\usepackage{upquote}}{}
\begin{document}


{
  \setbeamertemplate{footline}{} % no page number here
  \frame{
    \titlepage
  } 
}

%------------------------------------------------

\begin{frame}
\begin{center}
\Huge{\hilit{XML}}
\end{center}
\end{frame}

%------------------------------------------------

\begin{frame}
\begin{center}

\bb{XML \& HTML}
The goal of these slides is to give you a \textbf{crash introduction to XML and HTML} so you can get a good grasp of those formats for the following lectures
\eb

\end{center}
\end{frame}

%------------------------------------------------

\begin{frame}
\frametitle{Datasets}

\bb{You'll have some sort of (raw) data to work with}
\eb
\begin{center}
\ig[width=10cm]{images/tabular_nontabular.pdf}
\end{center}

\end{frame}

%------------------------------------------------

\begin{frame}
\frametitle{Motivation}

\bb{Two main limitations of field-delimited files}
\bbi
  \item In plain text formats there is no information to describe the location of the data values
  \item There is no recognizable label for each data value within the file
  \item Serious limitations to store data with hierarchical structure 
\ei
\eb

\end{frame}

%------------------------------------------------

\begin{frame}
\frametitle{Hierarchical data}
\begin{center}
\ig[width=10cm]{images/hierarchical_data.pdf}
\end{center}
\end{frame}

%------------------------------------------------

\begin{frame}[fragile]
\frametitle{Hierarchical data}

Field-delimited files have limitations with hierarchical data
\bigskip

\begin{verbatim}
                   John      33  male
                   Julia     32  female
    John   Julia   Jack       6  male
    John   Julia   Jill       4  female
    John   Julia   John jnr   2  male
                   David     45  male
                   Debbie    42  female
    David  Debbie  Donald    16  male
    David  Debbie  Dianne    12  female
\end{verbatim}

\end{frame}

%------------------------------------------------

\begin{frame}
\frametitle{XML format}

\bb{XML advantages}
\bbi
  \item XML is a storage format that is still based on plain text
  \item In XML formats every single value is distinctly labeled
  \item Moreover, every single value is self-described
  \item The information is organized in a much more sophisticated manner
\ei
\eb

\end{frame}

%------------------------------------------------

\begin{frame}[fragile]
\frametitle{Hierarchical data}

An example of hierarchical data in XML
\begin{knitrout}\footnotesize
\definecolor{shadecolor}{rgb}{0.969, 0.969, 0.969}\color{fgcolor}\begin{kframe}
\begin{alltt}
<family>
  <parent gender=\hlstr{"male"} name=\hlstr{"John"} age=\hlstr{"33"} />
  <parent gender=\hlstr{"female"} name=\hlstr{"Julia"} age=\hlstr{"32"} />
  <child gender=\hlstr{"male"} name=\hlstr{"Jack"} age=\hlstr{"6"} />
  <child gender=\hlstr{"female"} name=\hlstr{"Jill"} age=\hlstr{"4"} />
  <child gender=\hlstr{"male"} name=\hlstr{"John jnr"} age=\hlstr{"2"} />
</family>
<family>
  <parent gender=\hlstr{"male"} name=\hlstr{"David"} age=\hlstr{"45"} />
  <parent gender=\hlstr{"female"} name=\hlstr{"Debbie"} age=\hlstr{"42"} />
  <child gender=\hlstr{"male"} name=\hlstr{"Donald"} age=\hlstr{"16"} />
  <child gender=\hlstr{"female"} name=\hlstr{"Dianne"} age=\hlstr{"12"} />
</family>
\end{alltt}
\end{kframe}
\end{knitrout}

\end{frame}

%------------------------------------------------

\begin{frame}
\frametitle{XML and HTML}

\bb{Why should you care about XML and HTML?}
\bbi
  \item Large amounts of data and information are stored, shared and distributed using HTML and XML-dialects
  \item They are widely adopted and used in many applications
  \item Working with data from the Web means dealing with HTML
\ei
\eb

\end{frame}

%------------------------------------------------

\begin{frame}
 \begin{center}
  {\Huge \hilit XML}
  
  \bigskip
  {\Large \hilit eXtensible Markup Language}
 \end{center}
\end{frame}

%------------------------------------------------

\begin{frame}
\frametitle{Some Definitions}

\begin{quotation}
``XML is a markup language that defines a set of rules for encoding documents in a format that is both human-readable and machine-readable''
\end{quotation}

{\footnotesize 
\hspace{8mm} \url{http://en.wikipedia.org/wiki/XML}
}

\bigskip
\begin{quotation}
``XML is a data description language used for describing data''
\end{quotation}

{\footnotesize 
\hspace{8mm} {\hilit Paul Murrell} \\
\hspace{8mm} {\lolit Introduction to Data Technologies}
}

\end{frame}

%------------------------------------------------

\begin{frame}
\frametitle{Some Definitions}

\begin{quotation}
``XML is a very general structure with which we can define any number of new formats to represent arbitrary data''
\end{quotation}

\begin{quotation}
``XML is a standard for the semantic, hierarchical representation of data''
\end{quotation}

{\footnotesize 
\hspace{8mm} {\hilit Deb Nolan \& Duncan Temple Lang} \\
\hspace{8mm} {\lolit XML and Web Technologies for Data Sciences with R}
}

\end{frame}

%------------------------------------------------

\begin{frame}
\frametitle{About XML}

\bb{XML}
XML stands for \textbf{eXtensible Markup Language}
\eb

\bb{Broadly speaking ...}
XML provides a flexible framework to create formats for describing and representing data
\eb

\end{frame}

%------------------------------------------------

\begin{frame}
\frametitle{Markups}

\bb{Markup}
A \textbf{markup} is a sequence of characters or other symbols inserted at certain places in a document to indicate either: 
\bi
 \item how the content should be displayed when printed or in screen
 \item describe the document's structure
\ei
\eb

\end{frame}

%------------------------------------------------

\begin{frame}
\frametitle{Markups}

\bb{Markup Language}
A markup language is a system for \textbf{annotating} (i.e. \textit{marking}) a document in a way that the content is distinguished from its representation (eg LaTeX, PostScript, HTML, SVG)
\eb

\end{frame}

%------------------------------------------------

\begin{frame}[fragile]
\frametitle{LaTeX example}

\begin{columns}[t]
\begin{column}{0.1\textwidth}
\end{column}
\begin{column}{0.6\textwidth}
\begin{knitrout}\tiny
\definecolor{shadecolor}{rgb}{0.969, 0.969, 0.969}\color{fgcolor}\begin{kframe}
\begin{alltt}
\textbackslash{}documentclass\{article\}
\textbackslash{}usepackage\{graphicx\}

\textbackslash{}begin\{document\}

\textbackslash{}title\{Introduction to XML\}
\textbackslash{}author\{First Last\}
\textbackslash{}maketitle

\textbackslash{}section\{Introduction\}
Here is the text of your introduction.

\textbackslash{}begin\{equation\}
    \textbackslash{}label\{simple_equation\}
    \textbackslash{}alpha = \textbackslash{}sqrt\{ \textbackslash{}beta \}
\textbackslash{}end\{equation\}

\textbackslash{}subsection\{Subsection Heading Here\}
Write your subsection text here.

\textbackslash{}begin\{figure\}
    \textbackslash{}centering
    \textbackslash{}includegraphics[width=3.0in]\{myfigure\}
    \textbackslash{}caption\{Simulation Results\}
    \textbackslash{}label\{simulationfigure\}
\textbackslash{}end\{figure\}

\textbackslash{}end\{document\}
\end{alltt}
\end{kframe}
\end{knitrout}
\end{column}

\begin{column}{0.1\textwidth}
\end{column}

\end{columns}

\end{frame}

%------------------------------------------------

\begin{frame}[fragile]
\frametitle{Markups}

\bb{XML Markups}
In XML (as well as in HTML) the marks (aka \textit{tags}) are defined using angle brackets: {\Large {\hilit \code{<>}}}
\eb

\bigskip

\code{{\hilit <mark>}Text marked with special tag{\hilit </mark>} }

\end{frame}

%------------------------------------------------

\begin{frame}[fragile]
\frametitle{Extensible}

\bb{Extensible?}
The concept of \textit{extensibility} means that we can define our own marks, the order in which they occur, and how they should be processed. For example:
 \bi
  \item \code{<my\_mark>}
  \item \code{<awesome>}
  \item \code{<boring>}
  \item \code{<cool>}
 \ei
\eb

\end{frame}

%------------------------------------------------

\begin{frame}
\frametitle{About XML}

\bb{XML is NOT}
\bbi
 \item a programming language
 \item a network transfer protocol
 \item a database
\ei
\eb

\end{frame}

%------------------------------------------------

\begin{frame}
\frametitle{About XML}

\bb{XML is}
\bbi
 \item more than a markup language
 \item a generic language that provides structure and syntax for representing any type of information
 \item a meta-language: it allows us to create or define other languages
\ei
\eb

\end{frame}

%------------------------------------------------

\begin{frame}
\frametitle{XML Applications}

\bb{Some XML dialects}
\bbi
 \item \textbf{KML} (\textit{Keyhole Markup Language}) for describing geo-spatial information used in Google Earth, Google Maps, Google Sky
 \item \textbf{SVG} (\textit{Scalable Vector Graphics}) for visual graphical displays of two-dimensional graphics with support for interactivity and animation
 \item \textbf{PMML} (\textit{Predictive Model Markup Language}) for describing and exchanging models produced by data mining and machine learning algorithms
\ei
\eb

\end{frame}

%------------------------------------------------

\begin{frame}[fragile]
\frametitle{Keyhole Markup Language example}

\begin{knitrout}\footnotesize
\definecolor{shadecolor}{rgb}{0.969, 0.969, 0.969}\color{fgcolor}\begin{kframe}
\begin{alltt}
<?xml version=\hlstr{"1.0"} encoding=\hlstr{"UTF-8"}?>
<kml xmlns=\hlstr{"http://www.opengis.net/kml/2.2"}>
<Document>
<Placemark>
  <name>New York City</name>
  <description>New York City</description>
  <Point>
    <coordinates>-74.006393,40.714172,0</coordinates>
  </Point>
</Placemark>
</Document>
</kml>
\end{alltt}
\end{kframe}
\end{knitrout}

\end{frame}

%------------------------------------------------

\begin{frame}[fragile]
\frametitle{Scalable Vector Graphics example}

\begin{knitrout}\scriptsize
\definecolor{shadecolor}{rgb}{0.969, 0.969, 0.969}\color{fgcolor}\begin{kframe}
\begin{alltt}
<svg width=\hlstr{"100"} height=\hlstr{"100"}>
  <circle cx=\hlstr{"50"} cy=\hlstr{"50"} r=\hlstr{"40"} stroke=\hlstr{"green"} stroke-width=\hlstr{"4"} />
</svg>
  
  
<svg width=\hlstr{"400"} height=\hlstr{"110"}>
  <rect width=\hlstr{"300"} height=\hlstr{"100"} style=\hlstr{"fill:\hlkwd{rgb}(0,0,255)"} />
</svg>
\end{alltt}
\end{kframe}
\end{knitrout}

\end{frame}

%------------------------------------------------

\begin{frame}
\begin{center}
\Huge{\hilit{Minimalist Example}}
\end{center}
\end{frame}

%------------------------------------------------

\begin{frame}
\frametitle{}
\begin{center}
\ig[width=10cm]{images/goodwillhunting.jpg}
\end{center}
\end{frame}

%------------------------------------------------

\begin{frame}[fragile]
\frametitle{XML Example}

\begin{block}{Ultra Simple XML}
{\Large \begin{verbatim}
<movie>
  Good Will Hunting
</movie>
\end{verbatim}
}
\end{block}

\end{frame}

%------------------------------------------------

\begin{frame}[fragile]
\frametitle{XML Example}

\begin{block}{Ultra Simple XML}
\begin{verbatim}
<movie>
  Good Will Hunting
</movie>
\end{verbatim}
\end{block}

\bigskip

\bi
 \item one single element {\textit{movie}}
 \item start-tag: {\hilit \code{<movie>}}
 \item end-tag: {\hilit \code{</movie>}}
 \item content: \code{Good Will Hunting}
\ei

\end{frame}

%------------------------------------------------

\begin{frame}[fragile]
\frametitle{XML Example}

\bb{Ultra Simple XML}
\begin{verbatim}
<movie mins="126" lang="en">
  Good Will Hunting
</movie>
\end{verbatim}
\eb

\bigskip

\bi
 \item xml elements can have \textbf{attributes}
 \item attributes:  \code{\hilit mins} (minutes) and \code{\hilit lang} (language)
 \item attributes are \textit{attached} to the element's start tag
 \item attribute values \textbf{must be quoted!}
\ei

\end{frame}

%------------------------------------------------

\begin{frame}[fragile]
\frametitle{XML Example}

\bb{Minimalist XML}
\begin{verbatim}
<movie mins="126" lang="en">
  <title>Good Will Hunting</title>
  <director>Gus Van Sant</director>
  <year>1998</year>
  <genre>drama</genre>
</movie>
\end{verbatim}
\eb

\bigskip

\bi
 \item an xml element may contain other elements
 \item \textit{movie} contains several elements: \textit{title, director, year, genre}
\ei

\end{frame}

%------------------------------------------------

\begin{frame}[fragile]
\frametitle{XML Example}

\bb{Simple XML}
\begin{verbatim}
<movie mins="126" lang="en">
  <title>Good Will Hunting</title>
  <director>
    <first_name>Gus</first_name>
    <last_name>Van Sant</last_name>
  </director>
  <year>1998</year>
  <genre>drama</genre>
</movie>
\end{verbatim}
\eb

\bigskip

\bi
 \item Now \textit{director} has two child elements: \textit{first\_name} and \textit{last\_name}
\ei

\end{frame}

%------------------------------------------------

\begin{frame}[fragile]
\frametitle{XML Hierarchy Structure}

\begin{block}{Conceptual XML}
\begin{verbatim}
<Root>
  <child_1>...</child_1>
  <child_2>...</child_2>
    <subchild>...</subchild>
  <child_3>...</child_3>
</Root>
\end{verbatim}
\end{block}

\bigskip

\bi
 \item An XML document can be represented with a \textbf{tree structure}
 \item An XML document must have \textbf{one single Root} element
 \item The \code{Root} may contain \code{child} elements
 \item A \code{child} element may contain \code{subchild} elements
\ei

\end{frame}

%------------------------------------------------

\begin{frame}
\frametitle{}
\begin{center}
\ig[width=10cm]{images/xml_movie_tree1.pdf}
\end{center}
\end{frame}

%------------------------------------------------

\begin{frame}
\frametitle{}
\begin{center}
\ig[width=10cm]{images/xml_movie_tree2.pdf}
\end{center}
\end{frame}

%------------------------------------------------

\begin{frame}
\frametitle{Well-Formedness}

\bb{Well-formed XML}
We say that an XML document is \textbf{well-formed} when it obeys the basic syntax rules of XML. Some of those rules are:
\bi
 \item one root element containing the rest of elements
 \item properly nested elements
 \item self-closing tags
 \item attributes appear in start-tags of elements
 \item attribute values must be quoted
 \item element names and attribute names are case sensitive
\ei
\eb

\end{frame}

%------------------------------------------------

\begin{frame}[fragile]
\frametitle{Well-Formedness}

\begin{verbatim}
<movie mins="126" lang="en">
  <title>Good Will Hunting</title>
  <director>
    <first_name>Gus</first_name>
    <last_name>Van Sant</last_name>
  </director>
  <year>1998</year>
  <genre>drama</genre>
</movie>
\end{verbatim}

\end{frame}

%------------------------------------------------

\begin{frame}
\frametitle{Well-Formedness}

\bb{Importance of Well-formed XML}
Not well-formed XML documents produce potentially fatal errors or warnings when parsed.

\bigskip
Documents may be well-formed but not valid. Well-formed just guarantees that the document meets the basic XML structure, not that the content is valid.
\eb

\end{frame}

%------------------------------------------------

\begin{frame}
\begin{center}
\Huge{\hilit{Additional XML Elements}}
\end{center}
\end{frame}

%------------------------------------------------

\begin{frame}[fragile]
\frametitle{Some Additional Elements}

{ \small
\begin{verbatim}
<?xml version="1.0"? encoding="UTF-8" ?>
<![CDATA[ a > 5 & b < 10 ]]>
<?GS print(format = TRUE)>
<!DOCTYPE Movie>
<!-- This is a commet -->
<movie mins="126" lang="en">
  <title>Good Will Hunting</title>
  <director>
    <first_name>Gus</first_name>
    <last_name>Van Sant</last_name>
  </director>
  <year>1998</year>
  <genre>drama</genre>
</movie>
\end{verbatim}
}

\end{frame}

%------------------------------------------------

\begin{frame}
\frametitle{Additional Optional XML Elements}

{\small 
\begin{tabular}{l l}
  \hline
  Markup & Description \\
  \hline
  \code{<?xml >} & XML Declaration \\
  & \textit{identifies content as an XML document} \\
  \code{<?PI >} & Processing Instruction \\
  & \textit{processing instructions passed to application} \code{PI} \\
  \code{<!DOCTYPE >} & Document-type Declaration \\
  & \textit{defines the structure of an XML document} \\
  \code{<![CDATA[ ]]>} & CDATA Character Data \\
  & \textit{anything inside a CDATA is ignored by the parser} \\
  \code{<!--  -->} & Comment \\
  & \textit{for writing comments} \\
  \hline
\end{tabular}
}

\end{frame}

%------------------------------------------------

\begin{frame}
\frametitle{DTD}

\bb{Document-Type Declaration}
The Document-type Declaration identifies the \textbf{type} of the document. The \textit{type} indicates the structure of a \textbf{valid} document: 

\bi
 \item what elements are allowed to be present
 \item how elements can be combined
 \item how elements must be ordered
\ei
\eb

Basically, the DTD specifies what the format allows to do.
\end{frame}

%------------------------------------------------

\begin{frame}
\begin{center}
\Huge{\hilit{Wrapping Up}}
\end{center}
\end{frame}

%------------------------------------------------

\begin{frame}
\frametitle{About XML}

\bb{About XML}
\bi
 \item designed to store and transfer data
 \item designed to be self-descriptive
 \item tags are not predefined and can be extended
\ei
\eb

\end{frame}

%------------------------------------------------

\begin{frame}
\frametitle{Characteristics of XML}

\bb{XML is}
\bi
 \item a generic language that provides structure and syntax for many markup dialects
 \item is a syntax or format for defining markup languages
 \item a standard for the semantic, hierarchical representation of data
 \item provides a general approach for representing all types of information  dialects
\ei
\eb

\end{frame}

%------------------------------------------------

\begin{frame}[fragile]
\frametitle{XML document example}

\bb{Simple XML}
\begin{verbatim}
<?xml version="1.0"?>
<!DOCTYPE movies>
<movie mins="126" lang="en">
  <!-- this is a comment -->
  <title>Good Will Hunting</title>
  <director>
    <first_name>Gus</first_name>
    <last_name>Van Sant</last_name>
  </director>
  <year>1998</year>
  <genre>drama</genre>
</movie>
\end{verbatim}
\eb

\end{frame}

%------------------------------------------------

\begin{frame}
\frametitle{XML Tree Structure}

\bb{Each Node can have:}
\bi
 \item a Name
 \item any number of attributes
 \item optional content
 \item other nested elements
\ei
\eb

\bb{Traversing the tree}
There's a \textbf{unique} path from the root node to any given node 
\eb

\end{frame}

%------------------------------------------------

\begin{frame}
\begin{center}
\Huge{\hilit{HTML}}
\end{center}
\end{frame}

%------------------------------------------------

\begin{frame}
\frametitle{HTML}

\bb{About HTML}
\bi
 \item HyperText Markup Language
 \item standard markup language used to create web pages
 \item HTML describes the structure of a website semantically along with cues for presentation
 \item Web browsers can read HTML files and render them into visible or audible web pages
\ei
\eb

\end{frame}

%------------------------------------------------

\begin{frame}[fragile]
\frametitle{Hello World example}

\begin{knitrout}\footnotesize
\definecolor{shadecolor}{rgb}{0.969, 0.969, 0.969}\color{fgcolor}\begin{kframe}
\begin{alltt}
<!DOCTYPE html>
<html>
  <head>
    <title>This is a title</title>
  </head>
  <body>
    <p>Hello world!</p>
  </body>
</html>
\end{alltt}
\end{kframe}
\end{knitrout}

\end{frame}

%------------------------------------------------

\begin{frame}
\frametitle{HTML}

\bi
 \item Open a new text file
 \item Add osme HTML content (e.g. hello world example)
 \item Save your file with extension \code{.html}
 \item Click on your html file
 \item Should be displayed in your browser
\ei

\end{frame}

%------------------------------------------------

\begin{frame}[fragile]
\frametitle{Header Element}

Header of the HTML document: is declared with the tag {\hilit \code{<head>...</head>}}
\begin{knitrout}\footnotesize
\definecolor{shadecolor}{rgb}{0.969, 0.969, 0.969}\color{fgcolor}\begin{kframe}
\begin{alltt}
<head>
  <title>The Title</title>
</head>
\end{alltt}
\end{kframe}
\end{knitrout}

\end{frame}

%------------------------------------------------

\begin{frame}[fragile]
\frametitle{Headings}

HTML headings are defined with the {\hilit \code{<h1>, <h2>, ... <h6>}} tags:
\begin{knitrout}\footnotesize
\definecolor{shadecolor}{rgb}{0.969, 0.969, 0.969}\color{fgcolor}\begin{kframe}
\begin{alltt}
<h1>Heading level 1</h1>
<h2>Heading level 2</h2>
<h3>Heading level 3</h3>
<h4>Heading level 4</h4>
<h5>Heading level 5</h5>
<h6>Heading level 6</h6>
\end{alltt}
\end{kframe}
\end{knitrout}

\end{frame}

%------------------------------------------------

\begin{frame}[fragile]
\frametitle{Paragraphs}

Paragraphs are defined with the {\hilit \code{<p>}} tag:
\begin{knitrout}\footnotesize
\definecolor{shadecolor}{rgb}{0.969, 0.969, 0.969}\color{fgcolor}\begin{kframe}
\begin{alltt}
<p>This is the first paragraph</p>
  
<p>
  This is the second paragraph.
  The quick brown fox jumps over the lazy dog.
</p>
\end{alltt}
\end{kframe}
\end{knitrout}

\end{frame}

%------------------------------------------------

\begin{frame}[fragile]
\frametitle{Links and comments}

Links require the anchor tag {\hilit \code{<a>}} and the attribute {\hilit \code{href=}}
\begin{knitrout}\footnotesize
\definecolor{shadecolor}{rgb}{0.969, 0.969, 0.969}\color{fgcolor}\begin{kframe}
\begin{alltt}
<a href=\hlstr{"https://www.wikipedia.org/"}>A link to Wikipedia!</a>
\end{alltt}
\end{kframe}
\end{knitrout}

\bigskip
Comments:
\begin{knitrout}\footnotesize
\definecolor{shadecolor}{rgb}{0.969, 0.969, 0.969}\color{fgcolor}\begin{kframe}
\begin{alltt}
<!-- This is a comment -->
  
<!-- 
  This is also a comment 
-->
\end{alltt}
\end{kframe}
\end{knitrout}

\end{frame}

%------------------------------------------------

\begin{frame}[fragile]
\frametitle{Images}

Images are included with the {\hilit \code{<img>}} tag and the attribtue {\hilit \code{src=}}:
\begin{knitrout}\footnotesize
\definecolor{shadecolor}{rgb}{0.969, 0.969, 0.969}\color{fgcolor}\begin{kframe}
\begin{alltt}
<img src=\hlstr{"image.gif"}>
\end{alltt}
\end{kframe}
\end{knitrout}

\bigskip
Image with a link:
\begin{knitrout}\footnotesize
\definecolor{shadecolor}{rgb}{0.969, 0.969, 0.969}\color{fgcolor}\begin{kframe}
\begin{alltt}
<a href=\hlstr{"http://example.org"}>
  <img src=\hlstr{"image.gif"} alt=\hlstr{"descriptive text"}>
</a>
\end{alltt}
\end{kframe}
\end{knitrout}

\end{frame}

%------------------------------------------------

\begin{frame}[fragile]
\frametitle{HTML Example}

\begin{knitrout}\footnotesize
\definecolor{shadecolor}{rgb}{0.969, 0.969, 0.969}\color{fgcolor}\begin{kframe}
\begin{alltt}
<!DOCTYPE html>
<html>
  <head>
    <title>This is a title</title>
  </head>
  <!-- this is a commetn -->
  <body>
    <h1>Heading level 1</h1>
    <h2>Heading level 2</h2>
    <h3>Heading level 3</h3>
    <h4>Heading level 4</h4>
    <h5>Heading level 5</h5>
    <h6>Heading level 6</h6>  

    <p>Hello world!</p>
    <img src=\hlstr{"http://www.r-statistics.com/wp-content/uploads/2013/05/R_logo-e1369060981211.png"}>
  
    <a href=\hlstr{"https://www.r-project.org/"}>This is a link</a>
  </body>
</html>
\end{alltt}
\end{kframe}
\end{knitrout}

\end{frame}

%------------------------------------------------

\begin{frame}
\frametitle{Some References}

\bi
 \item XML Files website {\scriptsize (\url{http://www.xmlfiles.com})} \\
 {\lolit by Jan Egil Refsnes}
 \item XML in a  Nutshell \\
 {\lolit by Elliotte Rusty Harold; W. Scott Means}
 \item XML Tutorial {\scriptsize (\url{http://www.w3schools.com/xml/default.asp})} \\
 {\lolit by w3schools}
 \item Introduction to Data Technologies \\
 {\lolit by Paul Murrell}
 \item XML and Web Technologies for Data Sciences with R \\
 {\lolit by Deb Nolan and Duncan Temple Lang}
\ei

\end{frame}

%------------------------------------------------


\end{document}
