\documentclass[12pt]{beamer}\usepackage[]{graphicx}\usepackage[]{color}
%% maxwidth is the original width if it is less than linewidth
%% otherwise use linewidth (to make sure the graphics do not exceed the margin)
\makeatletter
\def\maxwidth{ %
  \ifdim\Gin@nat@width>\linewidth
    \linewidth
  \else
    \Gin@nat@width
  \fi
}
\makeatother

\definecolor{fgcolor}{rgb}{0.345, 0.345, 0.345}
\newcommand{\hlnum}[1]{\textcolor[rgb]{0.686,0.059,0.569}{#1}}%
\newcommand{\hlstr}[1]{\textcolor[rgb]{0.192,0.494,0.8}{#1}}%
\newcommand{\hlcom}[1]{\textcolor[rgb]{0.678,0.584,0.686}{\textit{#1}}}%
\newcommand{\hlopt}[1]{\textcolor[rgb]{0,0,0}{#1}}%
\newcommand{\hlstd}[1]{\textcolor[rgb]{0.345,0.345,0.345}{#1}}%
\newcommand{\hlkwa}[1]{\textcolor[rgb]{0.161,0.373,0.58}{\textbf{#1}}}%
\newcommand{\hlkwb}[1]{\textcolor[rgb]{0.69,0.353,0.396}{#1}}%
\newcommand{\hlkwc}[1]{\textcolor[rgb]{0.333,0.667,0.333}{#1}}%
\newcommand{\hlkwd}[1]{\textcolor[rgb]{0.737,0.353,0.396}{\textbf{#1}}}%
\let\hlipl\hlkwb

\usepackage{framed}
\makeatletter
\newenvironment{kframe}{%
 \def\at@end@of@kframe{}%
 \ifinner\ifhmode%
  \def\at@end@of@kframe{\end{minipage}}%
  \begin{minipage}{\columnwidth}%
 \fi\fi%
 \def\FrameCommand##1{\hskip\@totalleftmargin \hskip-\fboxsep
 \colorbox{shadecolor}{##1}\hskip-\fboxsep
     % There is no \\@totalrightmargin, so:
     \hskip-\linewidth \hskip-\@totalleftmargin \hskip\columnwidth}%
 \MakeFramed {\advance\hsize-\width
   \@totalleftmargin\z@ \linewidth\hsize
   \@setminipage}}%
 {\par\unskip\endMakeFramed%
 \at@end@of@kframe}
\makeatother

\definecolor{shadecolor}{rgb}{.97, .97, .97}
\definecolor{messagecolor}{rgb}{0, 0, 0}
\definecolor{warningcolor}{rgb}{1, 0, 1}
\definecolor{errorcolor}{rgb}{1, 0, 0}
\newenvironment{knitrout}{}{} % an empty environment to be redefined in TeX

\usepackage{alltt}
\usepackage{graphicx}
\usepackage{tikz}
\setbeameroption{hide notes}
\setbeamertemplate{note page}[plain]
\usepackage{listings}

% get rid of junk
\usetheme{default}
\usefonttheme[onlymath]{serif}
\beamertemplatenavigationsymbolsempty
\hypersetup{pdfpagemode=UseNone} % don't show bookmarks on initial view

% named colors
\definecolor{offwhite}{RGB}{255,250,240}
\definecolor{gray}{RGB}{155,155,155}

\ifx\notescolors\undefined % slides

  \definecolor{foreground}{RGB}{80,80,80}
  \definecolor{background}{RGB}{255,255,255}
  \definecolor{title}{RGB}{255,199,0}
  \definecolor{subtitle}{RGB}{89,132,212}
  \definecolor{hilit}{RGB}{248,117,79}
  \definecolor{vhilit}{RGB}{255,111,207}
  \definecolor{lolit}{RGB}{200,200,200}
  \definecolor{lit}{RGB}{255,199,0}
  \definecolor{mdlit}{RGB}{89,132,212}
  \definecolor{link}{RGB}{248,117,79}

\else % notes
  \definecolor{background}{RGB}{255,255,255}
  \definecolor{foreground}{RGB}{24,24,24}
  \definecolor{title}{RGB}{27,94,134}
  \definecolor{subtitle}{RGB}{22,175,124}
  \definecolor{hilit}{RGB}{122,0,128}
  \definecolor{vhilit}{RGB}{255,0,128}
  \definecolor{lolit}{RGB}{95,95,95}
\fi
\definecolor{nhilit}{RGB}{128,0,128}  % hilit color in notes
\definecolor{nvhilit}{RGB}{255,0,128} % vhilit for notes

\newcommand{\hilit}{\color{hilit}}
\newcommand{\vhilit}{\color{vhilit}}
\newcommand{\nhilit}{\color{nhilit}}
\newcommand{\nvhilit}{\color{nvhilit}}
\newcommand{\lit}{\color{lit}}
\newcommand{\mdlit}{\color{mdlit}}
\newcommand{\lolit}{\color{lolit}}

% use those colors
\setbeamercolor{titlelike}{fg=title}
\setbeamercolor{subtitle}{fg=subtitle}
\setbeamercolor{frametitle}{fg=gray}
\setbeamercolor{structure}{fg=subtitle}
\setbeamercolor{institute}{fg=lolit}
\setbeamercolor{normal text}{fg=foreground,bg=background}
%\setbeamercolor{item}{fg=foreground} % color of bullets
%\setbeamercolor{subitem}{fg=hilit}
%\setbeamercolor{itemize/enumerate subbody}{fg=lolit}
\setbeamertemplate{itemize subitem}{{\textendash}}
\setbeamerfont{itemize/enumerate subbody}{size=\footnotesize}
\setbeamerfont{itemize/enumerate subitem}{size=\footnotesize}

% center title of slides
\setbeamertemplate{blocks}[rounded]
\setbeamertemplate{frametitle}[default][center]
% margins
\setbeamersize{text margin left=25pt,text margin right=25pt}

% page number
\setbeamertemplate{footline}{%
    \raisebox{5pt}{\makebox[\paperwidth]{\hfill\makebox[20pt]{\lolit
          \scriptsize\insertframenumber}}}\hspace*{5pt}}

% add a bit of space at the top of the notes page
\addtobeamertemplate{note page}{\setlength{\parskip}{12pt}}

% default link color
\hypersetup{colorlinks, urlcolor={link}}

\ifx\notescolors\undefined % slides
  % set up listing environment
  \lstset{language=bash,
          basicstyle=\ttfamily\scriptsize,
          frame=single,
          commentstyle=,
          backgroundcolor=\color{darkgray},
          showspaces=false,
          showstringspaces=false
          }
\else % notes
  \lstset{language=bash,
          basicstyle=\ttfamily\scriptsize,
          frame=single,
          commentstyle=,
          backgroundcolor=\color{offwhite},
          showspaces=false,
          showstringspaces=false
          }
\fi

% a few macros
\newcommand{\code}[1]{\texttt{#1}}
\newcommand{\hicode}[1]{{\hilit \texttt{#1}}}
\newcommand{\bb}[1]{\begin{block}{#1}}
\newcommand{\eb}{\end{block}}
\newcommand{\bi}{\begin{itemize}}
%\newcommand{\bbi}{\vspace{24pt} \begin{itemize} \itemsep8pt}
\newcommand{\bbi}{\vspace{4pt} \begin{itemize} \itemsep8pt}
\newcommand{\ei}{\end{itemize}}
\newcommand{\bv}{\begin{verbatim}}
\newcommand{\ev}{\end{verbatim}}
\newcommand{\ig}{\includegraphics}
\newcommand{\subt}[1]{{\footnotesize \color{subtitle} {#1}}}
\newcommand{\ttsm}{\tt \small}
\newcommand{\ttfn}{\tt \footnotesize}
\newcommand{\figh}[2]{\centerline{\includegraphics[height=#2\textheight]{#1}}}
\newcommand{\figw}[2]{\centerline{\includegraphics[width=#2\textwidth]{#1}}}



%------------------------------------------------
% end of header
%------------------------------------------------

\title{Parsing XML}
\subtitle{STAT 133}
\author{\href{http://www.gastonsanchez.com}{Gaston Sanchez}}
\institute{\href{https://github.com/ucb-stat133/stat133-fall-2016}{\tt \scriptsize \color{foreground} github.com/ucb-stat133/stat133-fall-2016}}
\date{}
\IfFileExists{upquote.sty}{\usepackage{upquote}}{}
\begin{document}


{
  \setbeamertemplate{footline}{} % no page number here
  \frame{
    \titlepage
  } 
}

%------------------------------------------------

\begin{frame}
\begin{center}
\Huge{\hilit{Parsing XML \\ and HTML Content}}
\end{center}
\end{frame}

%------------------------------------------------

\begin{frame}
\frametitle{Motivation}

\bb{In a nutshell}
We'll cover a variety of situations you most likely will find yourself dealing with:
\bi
 \item R package XML
 \item Navigating the xml tree structure
 \item Main functions in package XML
 \item XPath
\ei
\eb

\end{frame}

%------------------------------------------------

\begin{frame}
\frametitle{Parsing}

\begin{quotation}
``A parser is a software component that takes input data (frequently text) and builds a data structure ---often some kind of parse tree, abstract syntax tree or other hierarchical structure--- giving a structural representation of the input, checking for correct syntax in the process''
\end{quotation}

{\footnotesize 
\url{http://en.wikipedia.org/wiki/Parsing\#Parser} \\
}

\end{frame}

%------------------------------------------------

\begin{frame}
\frametitle{Parsing XML and HTML Content}

\bb{Parsing XML and HTML?}
Getting data from the web often involves reading and processing content from xml and html documents. This is known as parsing. 

\bigskip
Luckily for us there's the R package \code{"XML"} (by Duncan Temple Lang) that allows us to parse such types of documents.
\eb

\end{frame}

%------------------------------------------------

\begin{frame}
\begin{center}
\Huge{\hilit{R Package \code{"XML"}}}
\end{center}
\end{frame}

%------------------------------------------------

\begin{frame}
\frametitle{R Package XML}

The package \code{"XML"} is designed for 2 major purposes
\begin{enumerate}
 \item parsing xml / html content
 \item writing xml / html content
\end{enumerate}

\bigskip

We won't cover the functions and utilities that have to do with writing xml / html content

\end{frame}

%------------------------------------------------

\begin{frame}
\frametitle{What can we do with "XML"?}

We'll cover 4 major types of tasks that we can perform with \code{"XML"}
\begin{enumerate}
 \item parsing (i.e. \textit{reading}) xml / html content
 \item obtaining descriptive information about parsed contents
 \item navigating the tree structure (i.e. accessing its components)
 \item querying and extracting data from parsed contents
\end{enumerate}

\end{frame}

%------------------------------------------------

\begin{frame}[fragile]
\frametitle{Using \code{"XML"}}

Remember to install \code{"XML"} first
\begin{knitrout}\footnotesize
\definecolor{shadecolor}{rgb}{0.969, 0.969, 0.969}\color{fgcolor}\begin{kframe}
\begin{alltt}
\hlcom{# installing xml}
\hlkwd{install.packages}\hlstd{(}\hlstr{"xml"}\hlstd{,} \hlkwc{dependencies} \hlstd{=} \hlnum{TRUE}\hlstd{)}

\hlcom{# load XML}
\hlkwd{library}\hlstd{(XML)}
\end{alltt}
\end{kframe}
\end{knitrout}



\end{frame}

%------------------------------------------------

\begin{frame}
\begin{center}
\Huge{\hilit{Parsing Functions}}
\end{center}
\end{frame}

%------------------------------------------------

\begin{frame}
\frametitle{Parsing Functions}

Main parsing functions in \code{"XML"}
\bbi
 \item \code{xmlParse()}
 \item \code{xmlTreeParse()}
 \item \code{htmlParse()}
 \item \code{htmlTreeParse()}
\ei

\end{frame}

%------------------------------------------------

\begin{frame}
\frametitle{Function \code{xmlParse()}}

\bb{\code{xmlParse()}}
\bi
\item \code{"XML"} comes with the \textit{almighty} parser function {\hilit \code{xmlParse()}}
 \item the main input for {\hilit \code{xmlParse()}} is a file: either a local file, a complete URL or a text string
 \bi
 \item[ex1:] \code{xmlParse("Documents/file.xml")}
 \item[ex2:] \code{xmlParse("http://www.xyz.com/some\_file.xml")}
 \item[ex3:] \code{xmlParse(xml\_string, asText=TRUE)}
 \ei
 \item the rest of the 20+ parameters are optional, and provide options to control the parsing procedure
\ei
\eb

\end{frame}

%------------------------------------------------

\begin{frame}[fragile]
\frametitle{\code{xmlParse()}}

Ultra simple example:
\begin{knitrout}\footnotesize
\definecolor{shadecolor}{rgb}{0.969, 0.969, 0.969}\color{fgcolor}\begin{kframe}
\begin{alltt}
\hlstd{doc} \hlkwb{<-} \hlkwd{xmlParse}\hlstd{(}\hlstr{"<foo><bar>Some text</bar></foo>"}\hlstd{,}
                \hlkwc{asText} \hlstd{=} \hlnum{TRUE}\hlstd{)}

\hlstd{doc}
\end{alltt}
\begin{verbatim}
## <?xml version="1.0"?>
## <foo>
##   <bar>Some text</bar>
## </foo>
## 
\end{verbatim}
\end{kframe}
\end{knitrout}

\end{frame}

%------------------------------------------------

\begin{frame}
\frametitle{}
\begin{center}
\ig[width=10cm]{images/xml_parse.pdf}
\end{center}
\end{frame}

%------------------------------------------------

\begin{frame}[fragile]
\frametitle{\code{xmlParse()} default behavior}

\bb{Default behavior of \code{xmlParse()}}
\bi
 \item it is a DOM parser: it reads an XML document into a hierarchical structure representation
 \item it builds an XML tree as a native C-level data structure \\
 (not an R data structure)
 \item it returns an object of class {\hilit \code{"XMLInternalDocument"}}
 \item can read content from compressed files without us needing to explicitly uncompress the file
 \item it does NOT handle \code{HTTPS} (secured HTTP)
\ei
\eb

\end{frame}

%------------------------------------------------

\begin{frame}[fragile]
\frametitle{\code{xmlParse()} default behavior}

Simple usage of \code{xmlParse()} on an XML document:
\begin{knitrout}\scriptsize
\definecolor{shadecolor}{rgb}{0.969, 0.969, 0.969}\color{fgcolor}\begin{kframe}
\begin{alltt}
\hlcom{# parsing an xml document}
\hlstd{doc1} \hlkwb{=} \hlkwd{xmlParse}\hlstd{(}\hlstr{"http://www.xmlfiles.com/examples/plant_catalog.xml"}\hlstd{)}
\end{alltt}
\end{kframe}
\end{knitrout}

by default \code{xmlParse()} returns an object of class {\hilit \code{"XMLInternalDocument"}} which is a C-level internal data structure

\begin{knitrout}\footnotesize
\definecolor{shadecolor}{rgb}{0.969, 0.969, 0.969}\color{fgcolor}\begin{kframe}
\begin{alltt}
\hlcom{# class }
\hlkwd{class}\hlstd{(doc1)}
\end{alltt}
\begin{verbatim}
## [1] "XMLInternalDocument" "XMLAbstractDocument"
\end{verbatim}
\end{kframe}
\end{knitrout}

\end{frame}

%------------------------------------------------

\begin{frame}[fragile]
\frametitle{About \code{xmlParse()} (con't)}

\bb{Argument \code{useInternalNodes = FALSE}}
Instead of parsing content as an internal C-level structure, we can parse it into an R structure by specifying the parameter {\hilit \code{useInternalNodes = FALSE}}
\eb
\begin{knitrout}\scriptsize
\definecolor{shadecolor}{rgb}{0.969, 0.969, 0.969}\color{fgcolor}\begin{kframe}
\begin{alltt}
\hlcom{# parsing an xml document into an R structure}
\hlstd{doc2} \hlkwb{=} \hlkwd{xmlParse}\hlstd{(}\hlstr{"http://www.xmlfiles.com/examples/plant_catalog.xml"}\hlstd{,}
                \hlkwc{useInternalNodes} \hlstd{=} \hlnum{FALSE}\hlstd{)}
\end{alltt}
\end{kframe}
\end{knitrout}

the output is of class {\hilit \code{"XMLDocument"}} and is implemented as a hierarchy of lists

\end{frame}

%------------------------------------------------

\begin{frame}[fragile]
\frametitle{About \code{xmlParse()} (con't)}
\begin{knitrout}\scriptsize
\definecolor{shadecolor}{rgb}{0.969, 0.969, 0.969}\color{fgcolor}\begin{kframe}
\begin{alltt}
\hlcom{# parsing an xml document into an R structure}
\hlstd{doc2} \hlkwb{=} \hlkwd{xmlParse}\hlstd{(}\hlstr{"http://www.xmlfiles.com/examples/plant_catalog.xml"}\hlstd{,}
                \hlkwc{useInternalNodes} \hlstd{=} \hlnum{FALSE}\hlstd{)}
\end{alltt}
\end{kframe}
\end{knitrout}

\begin{knitrout}\footnotesize
\definecolor{shadecolor}{rgb}{0.969, 0.969, 0.969}\color{fgcolor}\begin{kframe}
\begin{alltt}
\hlcom{# class }
\hlkwd{class}\hlstd{(doc2)}
\end{alltt}
\begin{verbatim}
## [1] "XMLDocument"         "XMLAbstractDocument"
\end{verbatim}
\begin{alltt}
\hlkwd{is.list}\hlstd{(doc2)}
\end{alltt}
\begin{verbatim}
## [1] TRUE
\end{verbatim}
\end{kframe}
\end{knitrout}

\end{frame}

%------------------------------------------------

\begin{frame}[fragile]
\frametitle{About \code{xmlTreeParse()}}

\bb{Argument \code{useInternalNodes = FALSE}}
\code{"XML"} provides the function {\hilit \code{xmlTreeParse()}} as a convenient synonym for \code{xmlParse(file, useInternalNodes = FALSE)}

\begin{knitrout}\scriptsize
\definecolor{shadecolor}{rgb}{0.969, 0.969, 0.969}\color{fgcolor}\begin{kframe}
\begin{alltt}
\hlcom{# parse an xml document into an R structure}
\hlstd{doc3} \hlkwb{=} \hlkwd{xmlTreeParse}\hlstd{(}\hlstr{"http://www.xmlfiles.com/examples/plant_catalog.xml"}\hlstd{)}
\end{alltt}
\end{kframe}
\end{knitrout}

As expected, the output is of class {\hilit \code{"XMLDocument"}}

\begin{knitrout}\footnotesize
\definecolor{shadecolor}{rgb}{0.969, 0.969, 0.969}\color{fgcolor}\begin{kframe}
\begin{alltt}
\hlcom{# class }
\hlkwd{class}\hlstd{(doc3)}
\end{alltt}
\begin{verbatim}
## [1] "XMLDocument"         "XMLAbstractDocument"
\end{verbatim}
\end{kframe}
\end{knitrout}
\eb

\end{frame}

%------------------------------------------------

\begin{frame}[fragile]
\frametitle{HTML Content}

\bb{Parsing HTML content}
In theory, we could use {\hilit \code{xmlParse()}} with its default settings to parse HTML documents. 

\bigskip
However \code{xmlParse()} ---with its default behavior--- will not work properly when HTML documents are not well-formed:
\bi
 \item no xml declaration
 \item no DOCTYPE
 \item no closure of tags
\ei
\eb

\end{frame}

%------------------------------------------------

\begin{frame}[fragile]
\frametitle{\code{xmlParse()} and HTML Content}

\bb{Argument \code{isHTML = TRUE}}
One option to parse HTML documents is by using \code{xmlParse()} with the argument {\hilit \code{isHTML = TRUE}}

\begin{knitrout}\footnotesize
\definecolor{shadecolor}{rgb}{0.969, 0.969, 0.969}\color{fgcolor}\begin{kframe}
\begin{alltt}
\hlcom{# parsing an html document with 'xmlParse()'}
\hlstd{doc4} \hlkwb{=} \hlkwd{xmlParse}\hlstd{(}\hlstr{"http://www.r-project.org/mail.html"}\hlstd{,}
                \hlkwc{isHTML} \hlstd{=} \hlnum{TRUE}\hlstd{)}
\end{alltt}
\end{kframe}
\end{knitrout}

the output is of class {\hilit \code{"HTMLInternalDocument"}}



\begin{knitrout}\footnotesize
\definecolor{shadecolor}{rgb}{0.969, 0.969, 0.969}\color{fgcolor}\begin{kframe}
\begin{alltt}
\hlcom{# class }
\hlkwd{class}\hlstd{(doc4)}
\end{alltt}
\begin{verbatim}
## [1] "HTMLInternalDocument" "HTMLInternalDocument" "XMLInternalDocument" 
## [4] "XMLAbstractDocument"
\end{verbatim}
\end{kframe}
\end{knitrout}
\eb

\end{frame}

%------------------------------------------------

\begin{frame}[fragile]
\frametitle{\code{htmlParse()} and HTML Content}

\bb{Function \code{htmlParse()}}
Another option is to use the function {\hilit \code{htmlParse()}} which is equivalent to \code{xmlParse(file, isHTML = TRUE)}

\begin{knitrout}\footnotesize
\definecolor{shadecolor}{rgb}{0.969, 0.969, 0.969}\color{fgcolor}\begin{kframe}
\begin{alltt}
\hlcom{# parsing an html document with 'htmlParse()'}
\hlstd{doc5} \hlkwb{=} \hlkwd{htmlParse}\hlstd{(}\hlstr{"http://www.r-project.org/mail.html"}\hlstd{)}
\end{alltt}
\end{kframe}
\end{knitrout}

again, the output is of class {\hilit \code{"HTMLInternalDocument"}}

\begin{knitrout}\footnotesize
\definecolor{shadecolor}{rgb}{0.969, 0.969, 0.969}\color{fgcolor}\begin{kframe}
\begin{alltt}
\hlcom{# class }
\hlkwd{class}\hlstd{(doc5)}
\end{alltt}
\begin{verbatim}
## [1] "HTMLInternalDocument" "HTMLInternalDocument" "XMLInternalDocument" 
## [4] "XMLAbstractDocument"
\end{verbatim}
\end{kframe}
\end{knitrout}
\eb

\end{frame}

%------------------------------------------------

\begin{frame}[fragile]
\frametitle{Function \code{htmlTreeParse()}}

\bb{Function \code{htmlTreeParse()}}
To parse content into an R structure we have to use {\hilit \code{htmlTreeParse()}} which is equivalent to \code{htmlParse(file, useInternalNodes = FALSE)}

\begin{knitrout}\footnotesize
\definecolor{shadecolor}{rgb}{0.969, 0.969, 0.969}\color{fgcolor}\begin{kframe}
\begin{alltt}
\hlcom{# parsing an html document into an  R structure}
\hlstd{doc6} \hlkwb{=} \hlkwd{htmlTreeParse}\hlstd{(}\hlstr{"http://www.r-project.org/mail.html"}\hlstd{)}
\end{alltt}
\end{kframe}
\end{knitrout}

in this case the output is of class {\hilit \code{"XMLDocumentContent"}}

\begin{knitrout}\footnotesize
\definecolor{shadecolor}{rgb}{0.969, 0.969, 0.969}\color{fgcolor}\begin{kframe}
\begin{alltt}
\hlcom{# class }
\hlkwd{class}\hlstd{(doc6)}
\end{alltt}
\begin{verbatim}
## [1] "XMLDocumentContent"
\end{verbatim}
\end{kframe}
\end{knitrout}
\eb

\end{frame}

%------------------------------------------------

\begin{frame}[fragile]
\frametitle{HTML Content}

\bb{About parsing HTML documents}
\bi
 \item \code{xmlParse()} can do the job but only on well-formed HTML
 \item it is better to be conservative and use the argument {\hilit \code{isHTML = TRUE}}, which is equivalent to using \code{htmlParse()}
 \item we can use {\hilit \code{htmlParse()}} or {\hilit \code{htmlTreeParse()}} which try to correct not well-formed docs by using heuristics that will take care of the missing elements
 \item in a worst-case scenario we can use {\hilit \code{tidyHTML()}} from the R package \code{"RTidyHTML"}, and then pass the result to \code{htmlParse()}
\ei
\eb

\end{frame}

%------------------------------------------------

\begin{frame}[fragile]
\frametitle{Parsing Functions Summary}

\bb{\code{xmlParse(file)}}
\bi
 \item main parsing function
 \item returns class \code{"XMLInternalDocument"} (C-level structure)
\ei
\eb

\bb{\code{xmlTreeParse(file)}}
\bi
 \item returns class \code{"XMLDocument"} (R data structure)
 \item equivalent to \code{xmlParse(file, useInternalNodes = FALSE)}
\ei
\eb

\end{frame}

%------------------------------------------------

\begin{frame}[fragile]
\frametitle{Parsing Functions Summary}

\bb{\code{htmlParse(file)}}
\bi
 \item especially suited for parsing HTML content
 \item returns class \code{"HTMLInternalDocument"} (C-level structure)
 \item equivalent to \code{xmlParse(file, isHTML = TRUE)}
\ei
\eb

\bb{\code{htmlTreeParse(file)}}
\bi
 \item especially suited for parsing HTML content
 \item returns class \code{"XMLDocumentContent"} (R data structure)
 \item equivalent to
 \begin{itemize}
  \item \code{xmlParse(file, isHTML = TRUE, useInternalNodes = FALSE)}
  \item \code{htmlParse(file, useInternalNodes = FALSE)}
 \end{itemize}
\ei
\eb

\end{frame}

%------------------------------------------------

\begin{frame}
\frametitle{Parsing Functions}

\begin{center}
 \begin{tabular}{l l}
  \hline
  Function & relation with \code{xmlParse()} \\
  \hline
  \code{xmlParse()} & \textit{default} \\
  & \\
  \code{xmlTreeParse()} & \code{useInternalNodes = FALSE} \\
  & \\
  \code{htmlParse()} & \code{isHTML = TRUE} \\
  & \\
  \code{htmlTreeParse()} & \code{isHTML = TRUE} \\
  & \code{useInternalNodes = FALSE} \\
  \hline
 \end{tabular}
\end{center}

\end{frame}

%------------------------------------------------

\begin{frame}
\begin{center}
\Huge{\hilit{Working with \\ Parsed Documents}}
\end{center}
\end{frame}

%------------------------------------------------

\begin{frame}
\frametitle{Parsed Documents}

\bb{\code{xmlRoot()} and \code{xmlChildren()}}
Having parsed an XML / HTML document, we can use 2 main functions to start working on the tree structure:
\bi
 \item {\hilit \code{xmlRoot()}} gets access to the root node and its elements 
 \item {\hilit \code{xmlChildren()}} gets access to the child elements of a given node
\ei
\eb

\end{frame}

%------------------------------------------------

\begin{frame}[fragile]
\frametitle{Conceptual Diagram}
\begin{center}
\ig[width=10cm]{images/xml_tree_navigate.pdf}
\end{center}
\end{frame}

%------------------------------------------------

\begin{frame}
\frametitle{Some Additional Functions}

\begin{center}
{\mdlit Functions for a given node}
\end{center}

\begin{center}
 \begin{tabular}{l l}
  \hline
  Function & Description \\
  \hline
  \code{xmlName()} & name of the node \\
  \code{xmlSize()} & number of subnodes \\
  \code{xmlAttrs()} & named character vector of all attributes \\
  \code{xmlGetAttr()} & value of a single attribute \\
  \code{xmlValue()} & contents of a leaf node \\
  \code{xmlParent()} & name of parent node \\
  \code{xmlAncestors()} & name of ancestor nodes \\
  \code{getSibling()} & siblings to the right or to the left \\
  \code{xmlNamespace()} & the namespace (if there's one) \\  
  \hline
 \end{tabular}
\end{center}

{\scriptsize The applicability of the functions depends on the class of objects we are working on}

\end{frame}

%------------------------------------------------

\begin{frame}[fragile]
\frametitle{Toy Example: Movies XML}

\begin{knitrout}\tiny
\definecolor{shadecolor}{rgb}{0.969, 0.969, 0.969}\color{fgcolor}\begin{kframe}
\begin{alltt}
\hlcom{# define some xml content}
\hlstd{xml_string} \hlkwb{=} \hlkwd{c}\hlstd{(}
  \hlstr{'<?xml version="1.0" encoding="UTF-8"?>'}\hlstd{,}
  \hlstr{'<movies>'}\hlstd{,}
  \hlstr{'<movie mins="126" lang="eng">'}\hlstd{,}
  \hlstr{'<title>Good Will Hunting</title>'}\hlstd{,}
  \hlstr{'<director>'}\hlstd{,}
  \hlstr{'<first_name>Gus</first_name>'}\hlstd{,}
  \hlstr{'<last_name>Van Sant</last_name>'}\hlstd{,}
  \hlstr{'</director>'}\hlstd{,}
  \hlstr{'<year>1998</year>'}\hlstd{,}
  \hlstr{'<genre>drama</genre>'}\hlstd{,}
  \hlstr{'</movie>'}\hlstd{,}
  \hlstr{'<movie mins="106" lang="spa">'}\hlstd{,}
  \hlstr{'<title>Y tu mama tambien</title>'}\hlstd{,}
  \hlstr{'<director>'}\hlstd{,}
  \hlstr{'<first_name>Alfonso</first_name>'}\hlstd{,}
  \hlstr{'<last_name>Cuaron</last_name>'}\hlstd{,}
  \hlstr{'</director>'}\hlstd{,}
  \hlstr{'<year>2001</year>'}\hlstd{,}
  \hlstr{'<genre>drama</genre>'}\hlstd{,}
  \hlstr{'</movie>'}\hlstd{,}
  \hlstr{'</movies>'}\hlstd{)}

\hlcom{# parse xml content}
\hlstd{movies_xml} \hlkwb{<-} \hlkwd{xmlParse}\hlstd{(xml_string,} \hlkwc{asText} \hlstd{=} \hlnum{TRUE}\hlstd{)}
\end{alltt}
\end{kframe}
\end{knitrout}

\end{frame}

%------------------------------------------------

\begin{frame}[fragile]
\frametitle{Toy Example: Movies XML}
\begin{knitrout}\tiny
\definecolor{shadecolor}{rgb}{0.969, 0.969, 0.969}\color{fgcolor}\begin{kframe}
\begin{alltt}
\hlcom{# check movies_xml}
\hlstd{movies_xml}
\end{alltt}
\begin{verbatim}
## <?xml version="1.0" encoding="UTF-8"?>
## <movies>
##   <movie mins="126" lang="eng">
##     <title>Good Will Hunting</title>
##     <director>
##       <first_name>Gus</first_name>
##       <last_name>Van Sant</last_name>
##     </director>
##     <year>1998</year>
##     <genre>drama</genre>
##   </movie>
##   <movie mins="106" lang="spa">
##     <title>Y tu mama tambien</title>
##     <director>
##       <first_name>Alfonso</first_name>
##       <last_name>Cuaron</last_name>
##     </director>
##     <year>2001</year>
##     <genre>drama</genre>
##   </movie>
## </movies>
## 
\end{verbatim}
\end{kframe}
\end{knitrout}

\end{frame}

%------------------------------------------------

\begin{frame}[fragile]
\frametitle{Movies XML: Root Node}

\begin{columns}[t]
\begin{column}{0.5\textwidth}
\begin{knitrout}\tiny
\definecolor{shadecolor}{rgb}{0.969, 0.969, 0.969}\color{fgcolor}\begin{kframe}
\begin{alltt}
\hlcom{# examine class}
\hlcom{# (movies_xml is a C-level object)}
\hlkwd{class}\hlstd{(movies_xml)}
\end{alltt}
\begin{verbatim}
## [1] "XMLInternalDocument" "XMLAbstractDocument"
\end{verbatim}
\begin{alltt}
\hlcom{# get root node}
\hlstd{root} \hlkwb{<-} \hlkwd{xmlRoot}\hlstd{(movies_xml)}

\hlcom{# examine class}
\hlkwd{class}\hlstd{(root)}
\end{alltt}
\begin{verbatim}
## [1] "XMLInternalElementNode" "XMLInternalNode"        "XMLAbstractNode"
\end{verbatim}
\end{kframe}
\end{knitrout}
\end{column}

\begin{column}{0.5\textwidth}
\begin{knitrout}\tiny
\definecolor{shadecolor}{rgb}{0.969, 0.969, 0.969}\color{fgcolor}\begin{kframe}
\begin{alltt}
\hlcom{# display root node}
\hlstd{root}
\end{alltt}
\begin{verbatim}
## <movies>
##   <movie mins="126" lang="eng">
##     <title>Good Will Hunting</title>
##     <director>
##       <first_name>Gus</first_name>
##       <last_name>Van Sant</last_name>
##     </director>
##     <year>1998</year>
##     <genre>drama</genre>
##   </movie>
##   <movie mins="106" lang="spa">
##     <title>Y tu mama tambien</title>
##     <director>
##       <first_name>Alfonso</first_name>
##       <last_name>Cuaron</last_name>
##     </director>
##     <year>2001</year>
##     <genre>drama</genre>
##   </movie>
## </movies>
\end{verbatim}
\end{kframe}
\end{knitrout}
\end{column}
\end{columns}

\end{frame}

%------------------------------------------------

\begin{frame}[fragile]
\frametitle{Movies XML: movie children}

\begin{knitrout}\tiny
\definecolor{shadecolor}{rgb}{0.969, 0.969, 0.969}\color{fgcolor}\begin{kframe}
\begin{alltt}
\hlcom{# children of root node}
\hlstd{movie_child} \hlkwb{<-} \hlkwd{xmlChildren}\hlstd{(root)}

\hlstd{movie_child}
\end{alltt}
\begin{verbatim}
## $movie
## <movie mins="126" lang="eng">
##   <title>Good Will Hunting</title>
##   <director>
##     <first_name>Gus</first_name>
##     <last_name>Van Sant</last_name>
##   </director>
##   <year>1998</year>
##   <genre>drama</genre>
## </movie> 
## 
## $movie
## <movie mins="106" lang="spa">
##   <title>Y tu mama tambien</title>
##   <director>
##     <first_name>Alfonso</first_name>
##     <last_name>Cuaron</last_name>
##   </director>
##   <year>2001</year>
##   <genre>drama</genre>
## </movie> 
## 
## attr(,"class")
## [1] "XMLInternalNodeList" "XMLNodeList"
\end{verbatim}
\end{kframe}
\end{knitrout}

\end{frame}

%------------------------------------------------

\begin{frame}[fragile]
\frametitle{Movies XML: movie children}
\begin{knitrout}\tiny
\definecolor{shadecolor}{rgb}{0.969, 0.969, 0.969}\color{fgcolor}\begin{kframe}
\begin{alltt}
\hlcom{# first movie}
\hlstd{goodwill} \hlkwb{<-} \hlstd{movie_child[[}\hlnum{1}\hlstd{]]}
\hlstd{goodwill}
\end{alltt}
\begin{verbatim}
## <movie mins="126" lang="eng">
##   <title>Good Will Hunting</title>
##   <director>
##     <first_name>Gus</first_name>
##     <last_name>Van Sant</last_name>
##   </director>
##   <year>1998</year>
##   <genre>drama</genre>
## </movie>
\end{verbatim}
\begin{alltt}
\hlcom{# second movie}
\hlstd{tumama} \hlkwb{<-} \hlstd{movie_child[[}\hlnum{2}\hlstd{]]}
\hlstd{tumama}
\end{alltt}
\begin{verbatim}
## <movie mins="106" lang="spa">
##   <title>Y tu mama tambien</title>
##   <director>
##     <first_name>Alfonso</first_name>
##     <last_name>Cuaron</last_name>
##   </director>
##   <year>2001</year>
##   <genre>drama</genre>
## </movie>
\end{verbatim}
\end{kframe}
\end{knitrout}

\end{frame}

%------------------------------------------------

\begin{frame}[fragile]
\frametitle{Movies XML: movie children}

\begin{columns}[t]
\begin{column}{0.5\textwidth}
\begin{knitrout}\tiny
\definecolor{shadecolor}{rgb}{0.969, 0.969, 0.969}\color{fgcolor}\begin{kframe}
\begin{alltt}
\hlcom{# node name}
\hlkwd{xmlName}\hlstd{(goodwill)}
\end{alltt}
\begin{verbatim}
## [1] "movie"
\end{verbatim}
\begin{alltt}
\hlcom{# number of children}
\hlkwd{xmlSize}\hlstd{(goodwill)}
\end{alltt}
\begin{verbatim}
## [1] 4
\end{verbatim}
\begin{alltt}
\hlcom{# node attributes}
\hlkwd{xmlAttrs}\hlstd{(goodwill)}
\end{alltt}
\begin{verbatim}
##  mins  lang 
## "126" "eng"
\end{verbatim}
\begin{alltt}
\hlcom{# get specific attribute value}
\hlkwd{xmlGetAttr}\hlstd{(goodwill,} \hlkwc{name} \hlstd{=} \hlstr{'lang'}\hlstd{)}
\end{alltt}
\begin{verbatim}
## [1] "eng"
\end{verbatim}
\end{kframe}
\end{knitrout}
\end{column}

\begin{column}{0.5\textwidth}
\begin{knitrout}\tiny
\definecolor{shadecolor}{rgb}{0.969, 0.969, 0.969}\color{fgcolor}\begin{kframe}
\begin{alltt}
\hlcom{# node name}
\hlkwd{xmlName}\hlstd{(tumama)}
\end{alltt}
\begin{verbatim}
## [1] "movie"
\end{verbatim}
\begin{alltt}
\hlcom{# number of children}
\hlkwd{xmlSize}\hlstd{(tumama)}
\end{alltt}
\begin{verbatim}
## [1] 4
\end{verbatim}
\begin{alltt}
\hlcom{# node attributes}
\hlkwd{xmlAttrs}\hlstd{(tumama)}
\end{alltt}
\begin{verbatim}
##  mins  lang 
## "106" "spa"
\end{verbatim}
\begin{alltt}
\hlcom{# get specific attribute value}
\hlkwd{xmlGetAttr}\hlstd{(tumama,} \hlkwc{name} \hlstd{=} \hlstr{'lang'}\hlstd{)}
\end{alltt}
\begin{verbatim}
## [1] "spa"
\end{verbatim}
\end{kframe}
\end{knitrout}
\end{column}
\end{columns}

\end{frame}

%------------------------------------------------

\begin{frame}[fragile]
\frametitle{Movies XML: movie Good Will Hunting}

\begin{columns}[t]
\begin{column}{0.5\textwidth}
\begin{knitrout}\tiny
\definecolor{shadecolor}{rgb}{0.969, 0.969, 0.969}\color{fgcolor}\begin{kframe}
\begin{alltt}
\hlcom{# node content (as character string)}
\hlkwd{xmlValue}\hlstd{(goodwill)}
\end{alltt}
\begin{verbatim}
## [1] "Good Will HuntingGusVan Sant1998drama"
\end{verbatim}
\begin{alltt}
\hlcom{# child nodes of goodwill node}
\hlkwd{xmlChildren}\hlstd{(goodwill)}
\end{alltt}
\begin{verbatim}
## $title
## <title>Good Will Hunting</title> 
## 
## $director
## <director>
##   <first_name>Gus</first_name>
##   <last_name>Van Sant</last_name>
## </director> 
## 
## $year
## <year>1998</year> 
## 
## $genre
## <genre>drama</genre> 
## 
## attr(,"class")
## [1] "XMLInternalNodeList" "XMLNodeList"
\end{verbatim}
\end{kframe}
\end{knitrout}
\end{column}

\begin{column}{0.5\textwidth}
\begin{knitrout}\tiny
\definecolor{shadecolor}{rgb}{0.969, 0.969, 0.969}\color{fgcolor}\begin{kframe}
\begin{alltt}
\hlcom{# director nodes of goodwill node}
\hlstd{gusvan} \hlkwb{<-} \hlkwd{xmlChildren}\hlstd{(goodwill)[[}\hlnum{2}\hlstd{]]}
\hlstd{gusvan}
\end{alltt}
\begin{verbatim}
## <director>
##   <first_name>Gus</first_name>
##   <last_name>Van Sant</last_name>
## </director>
\end{verbatim}
\begin{alltt}
\hlcom{# parent}
\hlkwd{xmlParent}\hlstd{(gusvan)}
\end{alltt}
\begin{verbatim}
## <movie mins="126" lang="eng">
##   <title>Good Will Hunting</title>
##   <director>
##     <first_name>Gus</first_name>
##     <last_name>Van Sant</last_name>
##   </director>
##   <year>1998</year>
##   <genre>drama</genre>
## </movie>
\end{verbatim}
\end{kframe}
\end{knitrout}
\end{column}
\end{columns}

\end{frame}

%------------------------------------------------

\begin{frame}[fragile]
\frametitle{Movies XML: movie Good Will Hunting}

\begin{columns}[t]
\begin{column}{0.5\textwidth}
\begin{knitrout}\tiny
\definecolor{shadecolor}{rgb}{0.969, 0.969, 0.969}\color{fgcolor}\begin{kframe}
\begin{alltt}
\hlcom{# director children}
\hlkwd{xmlChildren}\hlstd{(gusvan)}
\end{alltt}
\begin{verbatim}
## $first_name
## <first_name>Gus</first_name> 
## 
## $last_name
## <last_name>Van Sant</last_name> 
## 
## attr(,"class")
## [1] "XMLInternalNodeList" "XMLNodeList"
\end{verbatim}
\end{kframe}
\end{knitrout}
\end{column}

\begin{column}{0.5\textwidth}
\begin{knitrout}\tiny
\definecolor{shadecolor}{rgb}{0.969, 0.969, 0.969}\color{fgcolor}\begin{kframe}
\begin{alltt}
\hlcom{# sibling of goodwill node}
\hlkwd{getSibling}\hlstd{(goodwill)}
\end{alltt}
\begin{verbatim}
## <movie mins="106" lang="spa">
##   <title>Y tu mama tambien</title>
##   <director>
##     <first_name>Alfonso</first_name>
##     <last_name>Cuaron</last_name>
##   </director>
##   <year>2001</year>
##   <genre>drama</genre>
## </movie>
\end{verbatim}
\end{kframe}
\end{knitrout}
\end{column}
\end{columns}

\end{frame}

%------------------------------------------------

\begin{frame}
\begin{center}
\Huge{\hilit{Looping Over Nodes}}
\end{center}
\end{frame}

%------------------------------------------------

\begin{frame}[fragile]
\frametitle{Looping Over Nodes}

\bb{Looping over nodes}
Extracting data from an XML / HTML document involves applying a given function to a subset of nodes. This means iterating over such subset.
\eb

\end{frame}

%------------------------------------------------

\begin{frame}[fragile]
\frametitle{Looping Over Nodes}

There are various ways to loop over a subset of nodes:
\bbi
 \item the most basic approach is with \code{sapply()} or \code{lapply()}
 \item anoter way is by using the ad-hoc functions {\hilit \code{xmlApply()}} and {\hilit \code{xmlSApply()}}, which are simple wrappers for the \code{lapply()} and \code{sapply()} functions. 
\ei

\end{frame}

%------------------------------------------------

\begin{frame}[fragile]
\frametitle{Looping Over Nodes}

Some iteration examples with {\hilit \code{sapply()}}

\begin{columns}[t]
\begin{column}{0.5\textwidth}
\begin{knitrout}\tiny
\definecolor{shadecolor}{rgb}{0.969, 0.969, 0.969}\color{fgcolor}\begin{kframe}
\begin{alltt}
\hlcom{# length}
\hlkwd{sapply}\hlstd{(movie_child, length)}
\end{alltt}
\begin{verbatim}
## movie movie 
##     1     1
\end{verbatim}
\begin{alltt}
\hlcom{# names in child nodes}
\hlkwd{sapply}\hlstd{(movie_child, names)}
\end{alltt}
\begin{verbatim}
##          movie      movie     
## title    "title"    "title"   
## director "director" "director"
## year     "year"     "year"    
## genre    "genre"    "genre"
\end{verbatim}
\begin{alltt}
\hlkwd{sapply}\hlstd{(movie_child, xmlSize)}
\end{alltt}
\begin{verbatim}
## movie movie 
##     4     4
\end{verbatim}
\end{kframe}
\end{knitrout}
\end{column}

\begin{column}{0.5\textwidth}
\begin{knitrout}\tiny
\definecolor{shadecolor}{rgb}{0.969, 0.969, 0.969}\color{fgcolor}\begin{kframe}
\begin{alltt}
\hlcom{# attributes of root child nodes}
\hlkwd{sapply}\hlstd{(movie_child, xmlAttrs)}
\end{alltt}
\begin{verbatim}
##      movie movie
## mins "126" "106"
## lang "eng" "spa"
\end{verbatim}
\begin{alltt}
\hlcom{# names in child nodes}
\hlkwd{sapply}\hlstd{(movie_child, xmlValue)}
\end{alltt}
\begin{verbatim}
##                                     movie 
##   "Good Will HuntingGusVan Sant1998drama" 
##                                     movie 
## "Y tu mama tambienAlfonsoCuaron2001drama"
\end{verbatim}
\end{kframe}
\end{knitrout}
\end{column}
\end{columns}

\end{frame}

%------------------------------------------------

\begin{frame}[fragile]
\frametitle{Looping Over Nodes}

{\hilit \code{xmlApply()}} and {\hilit \code{xmlSApply()}} operate on the sub-nodes of an \code{XMLNode}:

\begin{columns}[t]
\begin{column}{0.5\textwidth}
\begin{knitrout}\tiny
\definecolor{shadecolor}{rgb}{0.969, 0.969, 0.969}\color{fgcolor}\begin{kframe}
\begin{alltt}
\hlcom{# names in child nodes}
\hlkwd{xmlSApply}\hlstd{(root, names)}
\end{alltt}
\begin{verbatim}
##          movie      movie     
## title    "title"    "title"   
## director "director" "director"
## year     "year"     "year"    
## genre    "genre"    "genre"
\end{verbatim}
\begin{alltt}
\hlcom{# size of movie children}
\hlkwd{xmlSApply}\hlstd{(root, xmlSize)}
\end{alltt}
\begin{verbatim}
## movie movie 
##     4     4
\end{verbatim}
\end{kframe}
\end{knitrout}
\end{column}

\begin{column}{0.5\textwidth}
\begin{knitrout}\tiny
\definecolor{shadecolor}{rgb}{0.969, 0.969, 0.969}\color{fgcolor}\begin{kframe}
\begin{alltt}
\hlcom{# attributes of root child nodes}
\hlkwd{xmlSApply}\hlstd{(root, xmlAttrs)}
\end{alltt}
\begin{verbatim}
##      movie movie
## mins "126" "106"
## lang "eng" "spa"
\end{verbatim}
\begin{alltt}
\hlcom{# names in child nodes}
\hlkwd{xmlSApply}\hlstd{(root, xmlValue)}
\end{alltt}
\begin{verbatim}
##                                     movie 
##   "Good Will HuntingGusVan Sant1998drama" 
##                                     movie 
## "Y tu mama tambienAlfonsoCuaron2001drama"
\end{verbatim}
\end{kframe}
\end{knitrout}
\end{column}
\end{columns}

\end{frame}

%------------------------------------------------

\begin{frame}[fragile]
\frametitle{Looping Over Nodes}

\begin{columns}[t]
\begin{column}{0.5\textwidth}
\begin{knitrout}\tiny
\definecolor{shadecolor}{rgb}{0.969, 0.969, 0.969}\color{fgcolor}\begin{kframe}
\begin{alltt}
\hlcom{# length of nodes in movie 1}
\hlkwd{xmlSApply}\hlstd{(root[[}\hlnum{1}\hlstd{]], length)}
\end{alltt}
\begin{verbatim}
##    title director     year    genre 
##        1        1        1        1
\end{verbatim}
\begin{alltt}
\hlcom{# size in child nodes in movie 1}
\hlkwd{xmlSApply}\hlstd{(root[[}\hlnum{1}\hlstd{]], xmlSize)}
\end{alltt}
\begin{verbatim}
##    title director     year    genre 
##        1        2        1        1
\end{verbatim}
\begin{alltt}
\hlcom{# attribute values of nodes in movie 1}
\hlkwd{xmlSApply}\hlstd{(root[[}\hlnum{1}\hlstd{]], xmlValue)}
\end{alltt}
\begin{verbatim}
##               title            director                year 
## "Good Will Hunting"       "GusVan Sant"              "1998" 
##               genre 
##             "drama"
\end{verbatim}
\end{kframe}
\end{knitrout}
\end{column}

\begin{column}{0.5\textwidth}
\begin{knitrout}\tiny
\definecolor{shadecolor}{rgb}{0.969, 0.969, 0.969}\color{fgcolor}\begin{kframe}
\begin{alltt}
\hlcom{# length of nodes in movie 2}
\hlkwd{xmlSApply}\hlstd{(root[[}\hlnum{2}\hlstd{]], length)}
\end{alltt}
\begin{verbatim}
##    title director     year    genre 
##        1        1        1        1
\end{verbatim}
\begin{alltt}
\hlcom{# size in child nodes in movie 2}
\hlkwd{xmlSApply}\hlstd{(root[[}\hlnum{2}\hlstd{]], xmlSize)}
\end{alltt}
\begin{verbatim}
##    title director     year    genre 
##        1        2        1        1
\end{verbatim}
\begin{alltt}
\hlcom{# attribute values of nodes in movie 2}
\hlkwd{xmlSApply}\hlstd{(root[[}\hlnum{2}\hlstd{]], xmlValue)}
\end{alltt}
\begin{verbatim}
##               title            director                year 
## "Y tu mama tambien"     "AlfonsoCuaron"              "2001" 
##               genre 
##             "drama"
\end{verbatim}
\end{kframe}
\end{knitrout}
\end{column}
\end{columns}

\end{frame}

%------------------------------------------------

\begin{frame}
\begin{center}
\Huge{\hilit{Case Study}}
\end{center}
\end{frame}

%------------------------------------------------

\begin{frame}
\frametitle{XML}

Example from \code{www.xmlfiles.com}

\url{http://www.xmlfiles.com/examples/plant_catalog.xml}

\end{frame}

%------------------------------------------------

\begin{frame}
\begin{center}
\Huge{\hilit{XPath Language}}
\end{center}
\end{frame}

%------------------------------------------------

\begin{frame}[fragile]
\frametitle{XPath}

\bb{Querying Trees}
The real parsing power comes from the ability to \textbf{locate nodes and extract information from them}. For this, we need to be able to perform queries on the parsed content.
\eb

\bb{XPath}
The solution is provided by \textbf{XPath}, which is a language to navigate through elements and attributes in an XML/HTML document
\eb

\end{frame}

%------------------------------------------------

\begin{frame}
\frametitle{XPath}

\bb{XPath}
\bi
 \item is a language for finding information in an XML document
 \item uses path expressions to select nodes or node-sets in an XML document
 \item works by identifying patterns to match data or content
 \item includes over 100 built-in functions
\ei
\eb

\end{frame}

%------------------------------------------------

\begin{frame}
\frametitle{About XPath}

\bb{XPath Syntax}
XPath uses \textbf{path expressions} to select nodes in an XML document. It has a computational model to identify sets of nodes (node-sets)
\eb

\bb{XPath Syntax}
We can specify paths through the tree structure:
\bi
 \item based on node names
 \item based on node content
 \item based on a node's relationship to other nodes
\ei
\eb

\end{frame}

%------------------------------------------------

\begin{frame}[fragile]
\frametitle{About XPath}

\bb{XPath Syntax}
The key concept is knowing how to write XPath expressions. XPath expressions have a syntax similar to the way files are located in a hierarchy of directories/folders in a computer file system. For instance:
\eb

{\hilit \code{/movies/movie[1]}}

\bigskip
is the XPath expression to locate the first {\hilit \code{movie}} element that is the child of the {\hilit \code{movies}} element

\end{frame}

%------------------------------------------------

\begin{frame}
\frametitle{Selecting Nodes}

\bb{XPath Syntax}
The main path expressions (i.e. symbols) are:
\eb

\begin{center}
 \begin{tabular}{l l}
  \hline
  Symbol & Description \\
  \hline
  \code{/} & selects from the root node \\
  \code{//} & selects nodes anywhere \\
  \code{.} & selects the current node \\
  \code{..} & Selects the parent of the current node \\
  \code{@} & Selects attributes \\
  \code{[]} & Square brackets to indicate attributes \\
  \hline
 \end{tabular}
\end{center}

\end{frame}

%------------------------------------------------

\begin{frame}
\frametitle{Selecting Unknown Nodes}

\bb{XPath wildcards for unknown nodes}
XPath wildcards can be used to select unknown XML elements
\eb

\begin{center}
 \begin{tabular}{l l}
  \hline
  Symbol & Description \\
  \hline
  \code{*} & matches any element node \\
  \code{@*} & matches any attribute node \\
  \code{node()} & matches any node of any kind \\
  \hline
 \end{tabular}
\end{center}

\end{frame}

%------------------------------------------------

\begin{frame}
\frametitle{Movies Tree Structure}
\begin{center}
\ig[width=10cm]{images/xpath_tree.pdf}
\end{center}
\end{frame}

%------------------------------------------------

\begin{frame}
\frametitle{XPath: movie nodes}
\begin{center}
\ig[width=10cm]{images/xpath_movie.pdf}
\end{center}
\end{frame}

%------------------------------------------------

\begin{frame}
\frametitle{XPath: movie title nodes}
\begin{center}
\ig[width=10cm]{images/xpath_title.pdf}
\end{center}
\end{frame}

%------------------------------------------------

\begin{frame}
\frametitle{XPath: movie director's first name nodes}
\begin{center}
\ig[width=10cm]{images/xpath_firstname.pdf}
\end{center}
\end{frame}

%------------------------------------------------

\begin{frame}
\frametitle{XPath: movie director nodes}
\begin{center}
\ig[width=10cm]{images/xpath_director.pdf}
\end{center}
\end{frame}

%------------------------------------------------

\begin{frame}
\frametitle{XPath: last name nodes}
\begin{center}
\ig[width=10cm]{images/xpath_lastname.pdf}
\end{center}
\end{frame}

%------------------------------------------------

\begin{frame}
\frametitle{XPath: title node of movie in Spanish}
\begin{center}
\ig[width=10cm]{images/xpath_ytmt.pdf}
\end{center}
\end{frame}

%------------------------------------------------

\begin{frame}
\begin{center}
\Huge{\hilit{Querying parsed documents}}
\end{center}
\end{frame}

%------------------------------------------------

\begin{frame}[fragile]
\frametitle{XPath in "XML"}

\bb{XPath in \code{"XML"}}
To work with XPath expressions using the \code{"XML"} package, we have the auxiliary function {\hilit \code{getNodeSet()}} that accepts XPath expressions in order to 
select node-sets. Its main usage is:
\begin{verbatim}
    getNodeSet(doc, path)
\end{verbatim}
\eb

where {\hilit \code{doc}} is an object of class \code{"XMLInternalDocument"} and {\hilit \code{path}} is a string giving the XPath expression to be evaluated

\end{frame}

%------------------------------------------------

\begin{frame}
\frametitle{Some References}

\bi
 \item An Introduction to the XML Package for R \\
{\scriptsize \url{http://www.omegahat.org/RSXML/Tour.pdf}}
 \item A Short Introduction to the XML package for R \\
{\scriptsize \url{http://www.omegahat.org/RSXML/shortIntro.pdf}}
 \item R and Splus XML Parsers \\
 {\scriptsize \url{http://www.omegahat.org/RSXML/Overview.html}}
 \item XML and Web Technlogies for Data Sciences with R \\
 by Deb Nolan and Duncan Temple Lang
\ei

\end{frame}

%------------------------------------------------


\end{document}
