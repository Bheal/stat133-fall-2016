\documentclass[12pt]{beamer}\usepackage[]{graphicx}\usepackage[]{color}
%% maxwidth is the original width if it is less than linewidth
%% otherwise use linewidth (to make sure the graphics do not exceed the margin)
\makeatletter
\def\maxwidth{ %
  \ifdim\Gin@nat@width>\linewidth
    \linewidth
  \else
    \Gin@nat@width
  \fi
}
\makeatother

\definecolor{fgcolor}{rgb}{0.345, 0.345, 0.345}
\newcommand{\hlnum}[1]{\textcolor[rgb]{0.686,0.059,0.569}{#1}}%
\newcommand{\hlstr}[1]{\textcolor[rgb]{0.192,0.494,0.8}{#1}}%
\newcommand{\hlcom}[1]{\textcolor[rgb]{0.678,0.584,0.686}{\textit{#1}}}%
\newcommand{\hlopt}[1]{\textcolor[rgb]{0,0,0}{#1}}%
\newcommand{\hlstd}[1]{\textcolor[rgb]{0.345,0.345,0.345}{#1}}%
\newcommand{\hlkwa}[1]{\textcolor[rgb]{0.161,0.373,0.58}{\textbf{#1}}}%
\newcommand{\hlkwb}[1]{\textcolor[rgb]{0.69,0.353,0.396}{#1}}%
\newcommand{\hlkwc}[1]{\textcolor[rgb]{0.333,0.667,0.333}{#1}}%
\newcommand{\hlkwd}[1]{\textcolor[rgb]{0.737,0.353,0.396}{\textbf{#1}}}%
\let\hlipl\hlkwb

\usepackage{framed}
\makeatletter
\newenvironment{kframe}{%
 \def\at@end@of@kframe{}%
 \ifinner\ifhmode%
  \def\at@end@of@kframe{\end{minipage}}%
  \begin{minipage}{\columnwidth}%
 \fi\fi%
 \def\FrameCommand##1{\hskip\@totalleftmargin \hskip-\fboxsep
 \colorbox{shadecolor}{##1}\hskip-\fboxsep
     % There is no \\@totalrightmargin, so:
     \hskip-\linewidth \hskip-\@totalleftmargin \hskip\columnwidth}%
 \MakeFramed {\advance\hsize-\width
   \@totalleftmargin\z@ \linewidth\hsize
   \@setminipage}}%
 {\par\unskip\endMakeFramed%
 \at@end@of@kframe}
\makeatother

\definecolor{shadecolor}{rgb}{.97, .97, .97}
\definecolor{messagecolor}{rgb}{0, 0, 0}
\definecolor{warningcolor}{rgb}{1, 0, 1}
\definecolor{errorcolor}{rgb}{1, 0, 0}
\newenvironment{knitrout}{}{} % an empty environment to be redefined in TeX

\usepackage{alltt}
\usepackage{graphicx}
\usepackage{tikz}
\setbeameroption{hide notes}
\setbeamertemplate{note page}[plain]
\usepackage{listings}

% get rid of junk
\usetheme{default}
\usefonttheme[onlymath]{serif}
\beamertemplatenavigationsymbolsempty
\hypersetup{pdfpagemode=UseNone} % don't show bookmarks on initial view

% named colors
\definecolor{offwhite}{RGB}{255,250,240}
\definecolor{gray}{RGB}{155,155,155}

\ifx\notescolors\undefined % slides

  \definecolor{foreground}{RGB}{80,80,80}
  \definecolor{background}{RGB}{255,255,255}
  \definecolor{title}{RGB}{255,199,0}
  \definecolor{subtitle}{RGB}{89,132,212}
  \definecolor{hilit}{RGB}{248,117,79}
  \definecolor{vhilit}{RGB}{255,111,207}
  \definecolor{lolit}{RGB}{200,200,200}
  \definecolor{lit}{RGB}{255,199,0}
  \definecolor{mdlit}{RGB}{89,132,212}
  \definecolor{link}{RGB}{248,117,79}

\else % notes
  \definecolor{background}{RGB}{255,255,255}
  \definecolor{foreground}{RGB}{24,24,24}
  \definecolor{title}{RGB}{27,94,134}
  \definecolor{subtitle}{RGB}{22,175,124}
  \definecolor{hilit}{RGB}{122,0,128}
  \definecolor{vhilit}{RGB}{255,0,128}
  \definecolor{lolit}{RGB}{95,95,95}
\fi
\definecolor{nhilit}{RGB}{128,0,128}  % hilit color in notes
\definecolor{nvhilit}{RGB}{255,0,128} % vhilit for notes

\newcommand{\hilit}{\color{hilit}}
\newcommand{\vhilit}{\color{vhilit}}
\newcommand{\nhilit}{\color{nhilit}}
\newcommand{\nvhilit}{\color{nvhilit}}
\newcommand{\lit}{\color{lit}}
\newcommand{\mdlit}{\color{mdlit}}
\newcommand{\lolit}{\color{lolit}}

% use those colors
\setbeamercolor{titlelike}{fg=title}
\setbeamercolor{subtitle}{fg=subtitle}
\setbeamercolor{frametitle}{fg=gray}
\setbeamercolor{structure}{fg=subtitle}
\setbeamercolor{institute}{fg=lolit}
\setbeamercolor{normal text}{fg=foreground,bg=background}
%\setbeamercolor{item}{fg=foreground} % color of bullets
%\setbeamercolor{subitem}{fg=hilit}
%\setbeamercolor{itemize/enumerate subbody}{fg=lolit}
\setbeamertemplate{itemize subitem}{{\textendash}}
\setbeamerfont{itemize/enumerate subbody}{size=\footnotesize}
\setbeamerfont{itemize/enumerate subitem}{size=\footnotesize}

% center title of slides
\setbeamertemplate{blocks}[rounded]
\setbeamertemplate{frametitle}[default][center]
% margins
\setbeamersize{text margin left=25pt,text margin right=25pt}

% page number
\setbeamertemplate{footline}{%
    \raisebox{5pt}{\makebox[\paperwidth]{\hfill\makebox[20pt]{\lolit
          \scriptsize\insertframenumber}}}\hspace*{5pt}}

% add a bit of space at the top of the notes page
\addtobeamertemplate{note page}{\setlength{\parskip}{12pt}}

% default link color
\hypersetup{colorlinks, urlcolor={link}}

\ifx\notescolors\undefined % slides
  % set up listing environment
  \lstset{language=bash,
          basicstyle=\ttfamily\scriptsize,
          frame=single,
          commentstyle=,
          backgroundcolor=\color{darkgray},
          showspaces=false,
          showstringspaces=false
          }
\else % notes
  \lstset{language=bash,
          basicstyle=\ttfamily\scriptsize,
          frame=single,
          commentstyle=,
          backgroundcolor=\color{offwhite},
          showspaces=false,
          showstringspaces=false
          }
\fi

% a few macros
\newcommand{\code}[1]{\texttt{#1}}
\newcommand{\hicode}[1]{{\hilit \texttt{#1}}}
\newcommand{\bb}[1]{\begin{block}{#1}}
\newcommand{\eb}{\end{block}}
\newcommand{\bi}{\begin{itemize}}
%\newcommand{\bbi}{\vspace{24pt} \begin{itemize} \itemsep8pt}
\newcommand{\bbi}{\vspace{4pt} \begin{itemize} \itemsep8pt}
\newcommand{\ei}{\end{itemize}}
\newcommand{\bv}{\begin{verbatim}}
\newcommand{\ev}{\end{verbatim}}
\newcommand{\ig}{\includegraphics}
\newcommand{\subt}[1]{{\footnotesize \color{subtitle} {#1}}}
\newcommand{\ttsm}{\tt \small}
\newcommand{\ttfn}{\tt \footnotesize}
\newcommand{\figh}[2]{\centerline{\includegraphics[height=#2\textheight]{#1}}}
\newcommand{\figw}[2]{\centerline{\includegraphics[width=#2\textwidth]{#1}}}



%------------------------------------------------
% end of header
%------------------------------------------------

\title{Control Flow Structures}
\subtitle{STAT 133}
\author{\href{http://www.gastonsanchez.com}{Gaston Sanchez}}
\institute{\href{https://github.com/ucb-stat133/stat133-fall-2016}{\tt \scriptsize \color{foreground} github.com/ucb-stat133/stat133-fall-2016}}
\date{}
\IfFileExists{upquote.sty}{\usepackage{upquote}}{}
\begin{document}


{
  \setbeamertemplate{footline}{} % no page number here
  \frame{
    \titlepage
  } 
}


%------------------------------------------------

\begin{frame}
\begin{center}
\Huge{\hilit{Control Flow Structures}}
\end{center}
\end{frame}

%------------------------------------------------

\begin{frame}
\frametitle{Control Flow}

There are many times where you don't just want to execute one statement after another: you need to control the flow of execution.

\end{frame}

%------------------------------------------------

\begin{frame}
\frametitle{Main Idea}

\begin{center}
{\mdlit \Large Execute some code when \\ a condition is fulfilled}
\end{center}

\end{frame}

%------------------------------------------------

\begin{frame}
\frametitle{Control Flow Structures}

\bi
  \item if-then-else
  \item switch cases
  \item repeat loop
  \item while loop
  \item for loop
\ei

\end{frame}

%------------------------------------------------

\begin{frame}
\begin{center}
\Huge{\hilit{If-Then-Else}}
\end{center}
\end{frame}

%------------------------------------------------

\begin{frame}[fragile]
\frametitle{If-Then-Else}

\textbf{If-then-else} statements make it possible to choose between two (possibly compound) expressions depending on the value of a \textbf{logical} condition.

\begin{knitrout}\footnotesize
\definecolor{shadecolor}{rgb}{0.969, 0.969, 0.969}\color{fgcolor}\begin{kframe}
\begin{alltt}
\hlcom{# absolute value}
\hlstd{num} \hlkwb{<-} \hlkwd{rnorm}\hlstd{(}\hlnum{1}\hlstd{)}
\hlkwa{if} \hlstd{(num} \hlopt{>=} \hlnum{0}\hlstd{) \{}
  \hlstd{num}
\hlstd{\}} \hlkwa{else} \hlstd{\{}
  \hlopt{-}\hlstd{num}
\hlstd{\}}
\end{alltt}
\begin{verbatim}
## [1] 0.05950904
\end{verbatim}
\end{kframe}
\end{knitrout}

\end{frame}

%------------------------------------------------

\begin{frame}[fragile]
\frametitle{If-Then-Else}

\textbf{If-then-else} statements make it possible to choose between two (possibly compound) expressions depending on the value of a \textbf{logical} condition.

\begin{knitrout}\footnotesize
\definecolor{shadecolor}{rgb}{0.969, 0.969, 0.969}\color{fgcolor}\begin{kframe}
\begin{alltt}
\hlkwa{if} \hlstd{(condition) expression1} \hlkwa{else} \hlstd{expression2}
\end{alltt}
\end{kframe}
\end{knitrout}

If \code{condition} is true then \code{expression1} is evaluated otherwise \code{expression2} is executed

\end{frame}

%------------------------------------------------

\begin{frame}[fragile]
\frametitle{If-Then-Else}

If-then-else with \textbf{simple} expressions (equivalent forms):

\begin{knitrout}\footnotesize
\definecolor{shadecolor}{rgb}{0.969, 0.969, 0.969}\color{fgcolor}\begin{kframe}
\begin{alltt}
\hlcom{# simple expressions don't require braces}
\hlkwa{if} \hlstd{(condition) expression1} \hlkwa{else} \hlstd{expression2}

\hlcom{# it's good practice to use braces, even with simple expressions}
\hlkwa{if} \hlstd{(condition) \{}
  \hlstd{expression1}
\hlstd{\}} \hlkwa{else} \hlstd{\{}
  \hlstd{expression2}
\hlstd{\}}
\end{alltt}
\end{kframe}
\end{knitrout}

\end{frame}

%------------------------------------------------

\begin{frame}[fragile]
\frametitle{If-Then-Else}

Equivalent forms:
\begin{knitrout}\footnotesize
\definecolor{shadecolor}{rgb}{0.969, 0.969, 0.969}\color{fgcolor}\begin{kframe}
\begin{alltt}
\hlcom{# simple if-then-else}
\hlkwa{if} \hlstd{(}\hlnum{5} \hlopt{>} \hlnum{2}\hlstd{)} \hlnum{5} \hlopt{*} \hlnum{2} \hlkwa{else} \hlnum{5} \hlopt{/} \hlnum{2}

\hlcom{# simple if-then-else}
\hlkwa{if} \hlstd{(}\hlnum{5} \hlopt{>} \hlnum{2}\hlstd{) \{}
  \hlnum{5} \hlopt{*} \hlnum{2}
\hlstd{\}} \hlkwa{else} \hlstd{\{}
  \hlnum{5} \hlopt{/} \hlnum{2}
\hlstd{\}}
\end{alltt}
\end{kframe}
\end{knitrout}

\end{frame}

%------------------------------------------------

\begin{frame}[fragile]
\frametitle{If-Then-Else}

If-then-else with \textbf{compound} expressions:

\begin{knitrout}\footnotesize
\definecolor{shadecolor}{rgb}{0.969, 0.969, 0.969}\color{fgcolor}\begin{kframe}
\begin{alltt}
\hlcom{# compound expressions require braces}
\hlkwa{if} \hlstd{(condition) \{}
  \hlstd{expression1}
  \hlstd{expression2}
  \hlstd{...}
\hlstd{\}} \hlkwa{else} \hlstd{\{}
  \hlstd{expression3}
  \hlstd{expression4}
  \hlstd{...}
\hlstd{\}}
\end{alltt}
\end{kframe}
\end{knitrout}

\end{frame}

%------------------------------------------------

\begin{frame}[fragile]
\frametitle{Example: If-Then-Else}

Equivalent forms:
\begin{knitrout}\footnotesize
\definecolor{shadecolor}{rgb}{0.969, 0.969, 0.969}\color{fgcolor}\begin{kframe}
\begin{alltt}
\hlcom{# simple if-then-else}
\hlkwa{if} \hlstd{(}\hlnum{5} \hlopt{>} \hlnum{2}\hlstd{)} \hlnum{5} \hlopt{*} \hlnum{2} \hlkwa{else} \hlnum{5} \hlopt{/} \hlnum{2}
\end{alltt}
\begin{verbatim}
## [1] 10
\end{verbatim}
\begin{alltt}
\hlcom{# simple if-then-else}
\hlkwa{if} \hlstd{(}\hlnum{5} \hlopt{>} \hlnum{2}\hlstd{) \{}
  \hlnum{5} \hlopt{*} \hlnum{2}
\hlstd{\}} \hlkwa{else} \hlstd{\{}
  \hlnum{5} \hlopt{/} \hlnum{2}
\hlstd{\}}
\end{alltt}
\begin{verbatim}
## [1] 10
\end{verbatim}
\end{kframe}
\end{knitrout}

\end{frame}

%------------------------------------------------

\begin{frame}
\frametitle{If-Then-Else}

\bi
  \item {\hilit \code{if()}} takes a \textbf{logical} condition
  \item the condition must be a logical value \textbf{of length one}
  \item it executes the next statement if the condition is true
  \item if the condition is false, then it executes the false expression
\ei

\end{frame}

%------------------------------------------------

\begin{frame}[fragile]
\frametitle{2 types of If conditions}

\begin{columns}[t]
\begin{column}{0.45\textwidth}
\textbf{if-then-else}
\begin{knitrout}\footnotesize
\definecolor{shadecolor}{rgb}{0.969, 0.969, 0.969}\color{fgcolor}\begin{kframe}
\begin{alltt}
\hlcom{# if and else}
\hlkwa{if} \hlstd{(condition) \{}
  \hlstd{expression_true}
\hlstd{\}} \hlkwa{else} \hlstd{\{}
  \hlstd{expression_false}
\hlstd{\}}
\end{alltt}
\end{kframe}
\end{knitrout}
\end{column}

\begin{column}{0.45\textwidth}
\textbf{Just if}
\begin{knitrout}\footnotesize
\definecolor{shadecolor}{rgb}{0.969, 0.969, 0.969}\color{fgcolor}\begin{kframe}
\begin{alltt}
\hlcom{# simple if}
\hlkwa{if} \hlstd{(condition) \{}
  \hlstd{expression_true}
\hlstd{\}}
\end{alltt}
\end{kframe}
\end{knitrout}
\end{column}
\end{columns}

\bigskip
It is also possible to just have the if clause (without \code{else})

\end{frame}

%------------------------------------------------

\begin{frame}[fragile]
\frametitle{Just If}

Just \code{if}
\begin{knitrout}\footnotesize
\definecolor{shadecolor}{rgb}{0.969, 0.969, 0.969}\color{fgcolor}\begin{kframe}
\begin{alltt}
\hlcom{# just if}
\hlkwa{if} \hlstd{(condition) \{}
  \hlstd{expression1}
  \hlstd{...}
\hlstd{\}}
\end{alltt}
\end{kframe}
\end{knitrout}

Equivalent to:
\begin{knitrout}\footnotesize
\definecolor{shadecolor}{rgb}{0.969, 0.969, 0.969}\color{fgcolor}\begin{kframe}
\begin{alltt}
\hlcom{# just if (else NULL)}
\hlkwa{if} \hlstd{(condition) \{}
  \hlstd{expression1}
  \hlstd{...}
\hlstd{\}} \hlkwa{else NULL}
\end{alltt}
\end{kframe}
\end{knitrout}

\end{frame}

%------------------------------------------------

\begin{frame}
\frametitle{If and Else}

\bi
  \item \code{if()} takes a \textbf{logical} condition
  \item the condition must be a logical value of length one
  \item it executes the next statement if the condition is true
  \item if the condition is false, and there is no \code{else}, then it stops
  \item if the condition is false, and there is an \code{else}, then it executes the false expression
\ei

\end{frame}

%------------------------------------------------

\begin{frame}
\frametitle{Reminder of Comparison Operators}

\begin{center}
 \begin{tabular}{l l}
  \hline
   operation & usage \\
  \hline
  less than & \code{x < x} \\
  greater than & \code{x > y} \\
  less than or equal & \code{x <= y} \\  
  greater than or equal  & \code{x >= y} \\
  equality & \code{x == y} \\
  different & \code{x != y} \\
  \hline
 \end{tabular}
\end{center}

Comparison operators produce logical values

\end{frame}

%------------------------------------------------

\begin{frame}
\frametitle{Reminder of Logical Operators}

\begin{center}
 \begin{tabular}{l l}
  \hline
   operation & usage \\
  \hline
  NOT & \code{!x} \\
  AND (elementwise) & \code{x \& y} \\
  AND (1st element) & \code{x \&\& y} \\  
  OR (elementwise)  & \code{x | y} \\
  OR (1st element) & \code{x || y} \\
  exclusive OR & \code{xor(x, y)} \\
  \hline
 \end{tabular}
\end{center}

Logical operators produce logical values

\end{frame}

%------------------------------------------------

\begin{frame}[fragile]
\frametitle{Just \code{if}'s behavior}

\begin{columns}[t]
\begin{column}{0.45\textwidth}
\begin{knitrout}\footnotesize
\definecolor{shadecolor}{rgb}{0.969, 0.969, 0.969}\color{fgcolor}\begin{kframe}
\begin{alltt}
\hlcom{# this prints}
\hlkwa{if} \hlstd{(}\hlnum{TRUE}\hlstd{) \{}
  \hlkwd{print}\hlstd{(}\hlstr{"It is true"}\hlstd{)}
\hlstd{\}}


\hlcom{# this does not print}
\hlkwa{if} \hlstd{(}\hlnum{FALSE}\hlstd{) \{}
  \hlkwd{print}\hlstd{(}\hlstr{"It is false"}\hlstd{)}
\hlstd{\}}
\end{alltt}
\end{kframe}
\end{knitrout}
\end{column}

\begin{column}{0.45\textwidth}
\begin{knitrout}\footnotesize
\definecolor{shadecolor}{rgb}{0.969, 0.969, 0.969}\color{fgcolor}\begin{kframe}
\begin{alltt}
\hlcom{# this does not print}
\hlkwa{if} \hlstd{(}\hlopt{!}\hlnum{TRUE}\hlstd{) \{}
  \hlkwd{print}\hlstd{(}\hlstr{"It is not true"}\hlstd{)}
\hlstd{\}}


\hlcom{# this prints}
\hlkwa{if} \hlstd{(}\hlopt{!}\hlnum{FALSE}\hlstd{) \{}
  \hlkwd{print}\hlstd{(}\hlstr{"It is not false"}\hlstd{)}
\hlstd{\}}
\end{alltt}
\end{kframe}
\end{knitrout}
\end{column}
\end{columns}

\end{frame}

%------------------------------------------------

\begin{frame}[fragile]
\frametitle{Just \code{if}}

\begin{knitrout}\footnotesize
\definecolor{shadecolor}{rgb}{0.969, 0.969, 0.969}\color{fgcolor}\begin{kframe}
\begin{alltt}
\hlstd{x} \hlkwb{<-} \hlnum{7}

\hlkwa{if} \hlstd{(x} \hlopt{>=} \hlnum{0}\hlstd{) \{}
  \hlkwd{print}\hlstd{(}\hlstr{"it is positive"}\hlstd{)}
\hlstd{\}}
\end{alltt}
\begin{verbatim}
## [1] "it is positive"
\end{verbatim}
\begin{alltt}
\hlkwa{if} \hlstd{(}\hlkwd{is.numeric}\hlstd{(x)) \{}
  \hlkwd{print}\hlstd{(}\hlstr{"it is numeric"}\hlstd{)}
\hlstd{\}}
\end{alltt}
\begin{verbatim}
## [1] "it is numeric"
\end{verbatim}
\end{kframe}
\end{knitrout}

\end{frame}

%------------------------------------------------

\begin{frame}[fragile]
\frametitle{If and Else}

\begin{knitrout}\footnotesize
\definecolor{shadecolor}{rgb}{0.969, 0.969, 0.969}\color{fgcolor}\begin{kframe}
\begin{alltt}
\hlstd{y} \hlkwb{<-} \hlopt{-}\hlnum{5}

\hlkwa{if} \hlstd{(y} \hlopt{>=} \hlnum{0}\hlstd{) \{}
  \hlkwd{print}\hlstd{(}\hlstr{"it is positive"}\hlstd{)}
\hlstd{\}} \hlkwa{else} \hlstd{\{}
  \hlkwd{print}\hlstd{(}\hlstr{"it is negative"}\hlstd{)}
\hlstd{\}}
\end{alltt}
\begin{verbatim}
## [1] "it is negative"
\end{verbatim}
\end{kframe}
\end{knitrout}
The \code{else} statement must occur on the same line as the closing brace from the \code{if} clause!

\end{frame}

%------------------------------------------------

\begin{frame}[fragile]
\frametitle{If and Else}

The logical condition must be of length one!
\begin{knitrout}\footnotesize
\definecolor{shadecolor}{rgb}{0.969, 0.969, 0.969}\color{fgcolor}\begin{kframe}
\begin{alltt}
\hlkwa{if} \hlstd{(}\hlkwd{c}\hlstd{(}\hlnum{TRUE}\hlstd{,} \hlnum{TRUE}\hlstd{)) \{}
  \hlkwd{print}\hlstd{(}\hlstr{"it is positive"}\hlstd{)}
\hlstd{\}} \hlkwa{else} \hlstd{\{}
  \hlkwd{print}\hlstd{(}\hlstr{"it is negative"}\hlstd{)}
\hlstd{\}}
\end{alltt}


{\ttfamily\noindent\color{warningcolor}{\#\# Warning in if (c(TRUE, TRUE)) \{: the condition has length > 1 and only the first element will be used}}\begin{verbatim}
## [1] "it is positive"
\end{verbatim}
\end{kframe}
\end{knitrout}

\end{frame}

%------------------------------------------------

\begin{frame}[fragile]
\frametitle{If and Else}

What's the length of the logical condition?
\begin{knitrout}\footnotesize
\definecolor{shadecolor}{rgb}{0.969, 0.969, 0.969}\color{fgcolor}\begin{kframe}
\begin{alltt}
\hlstd{x} \hlkwb{<-} \hlnum{3}
\hlstd{y} \hlkwb{<-} \hlnum{4}

\hlkwa{if} \hlstd{(x} \hlopt{>} \hlnum{0} \hlopt{&} \hlstd{y} \hlopt{>} \hlnum{0}\hlstd{) \{}
  \hlkwd{print}\hlstd{(}\hlstr{"they are not negative"}\hlstd{)}
\hlstd{\}} \hlkwa{else} \hlstd{\{}
  \hlkwd{print}\hlstd{(}\hlstr{"they may be negative"}\hlstd{)}
\hlstd{\}}
\end{alltt}
\begin{verbatim}
## [1] "they are not negative"
\end{verbatim}
\end{kframe}
\end{knitrout}

\end{frame}

%------------------------------------------------

\begin{frame}[fragile]
\frametitle{If and Else}

If there's is a single statement, you can omit the braces:
\begin{knitrout}\footnotesize
\definecolor{shadecolor}{rgb}{0.969, 0.969, 0.969}\color{fgcolor}\begin{kframe}
\begin{alltt}
\hlkwa{if} \hlstd{(}\hlnum{TRUE}\hlstd{) \{} \hlkwd{print}\hlstd{(}\hlstr{"It is true"}\hlstd{) \}}

\hlkwa{if} \hlstd{(}\hlnum{TRUE}\hlstd{)} \hlkwd{print}\hlstd{(}\hlstr{"It is true"}\hlstd{)}

\hlcom{# valid but not recommended}
\hlkwa{if} \hlstd{(}\hlnum{TRUE}\hlstd{)}
  \hlkwd{print}\hlstd{(}\hlstr{"It is true"}\hlstd{)}
\end{alltt}
\end{kframe}
\end{knitrout}

\end{frame}

%------------------------------------------------

\begin{frame}[fragile]
\frametitle{Equivalent ways}

\begin{columns}[t]
\begin{column}{0.45\textwidth}
No braces:
\begin{knitrout}\footnotesize
\definecolor{shadecolor}{rgb}{0.969, 0.969, 0.969}\color{fgcolor}\begin{kframe}
\begin{alltt}
\hlcom{# ok}
\hlkwa{if} \hlstd{(}\hlnum{TRUE}\hlstd{)} \hlkwd{print}\hlstd{(}\hlstr{"It's true"}\hlstd{)}


\hlcom{# valid but not recommended}
\hlkwa{if} \hlstd{(}\hlnum{TRUE}\hlstd{)}
  \hlkwd{print}\hlstd{(}\hlstr{"It's true"}\hlstd{)}
\end{alltt}
\end{kframe}
\end{knitrout}
\end{column}

\begin{column}{0.45\textwidth}
With braces:
\begin{knitrout}\footnotesize
\definecolor{shadecolor}{rgb}{0.969, 0.969, 0.969}\color{fgcolor}\begin{kframe}
\begin{alltt}
\hlcom{# ok}
\hlkwa{if} \hlstd{(}\hlnum{TRUE}\hlstd{) \{}\hlkwd{print}\hlstd{(}\hlstr{"It's true"}\hlstd{)\}}


\hlcom{# recommended}
\hlkwa{if} \hlstd{(}\hlnum{TRUE}\hlstd{) \{}
  \hlkwd{print}\hlstd{(}\hlstr{"It's true"}\hlstd{)}
\hlstd{\}}
\end{alltt}
\end{kframe}
\end{knitrout}
\end{column}
\end{columns}

\end{frame}

%------------------------------------------------

\begin{frame}[fragile]
\frametitle{If and Else}

If there are multiple statements, you must use braces:
\begin{knitrout}\footnotesize
\definecolor{shadecolor}{rgb}{0.969, 0.969, 0.969}\color{fgcolor}\begin{kframe}
\begin{alltt}
\hlkwa{if} \hlstd{(x} \hlopt{>} \hlnum{0}\hlstd{) \{}
  \hlstd{a} \hlkwb{<-} \hlstd{x}\hlopt{^}\hlstd{(}\hlnum{2}\hlstd{)}
  \hlstd{b} \hlkwb{<-} \hlnum{3} \hlopt{*} \hlstd{a} \hlopt{+} \hlnum{34.8} \hlopt{-} \hlkwd{exp}\hlstd{(}\hlnum{2}\hlstd{)}
\hlstd{\}}
\end{alltt}
\end{kframe}
\end{knitrout}

\end{frame}

%------------------------------------------------

\begin{frame}[fragile]
\frametitle{Multiple If's}

Multiple conditions can be defined by combining \code{if} and \code{else} repeatedly:
\begin{knitrout}\footnotesize
\definecolor{shadecolor}{rgb}{0.969, 0.969, 0.969}\color{fgcolor}\begin{kframe}
\begin{alltt}
\hlkwd{set.seed}\hlstd{(}\hlnum{9}\hlstd{)}
\hlstd{x} \hlkwb{<-} \hlkwd{round}\hlstd{(}\hlkwd{rnorm}\hlstd{(}\hlnum{1}\hlstd{),} \hlnum{1}\hlstd{)}

\hlkwa{if} \hlstd{(x} \hlopt{>} \hlnum{0}\hlstd{) \{}
  \hlkwd{print}\hlstd{(}\hlstr{"x is positive"}\hlstd{)}
\hlstd{\}} \hlkwa{else if} \hlstd{(x} \hlopt{<} \hlnum{0}\hlstd{) \{}
  \hlkwd{print}\hlstd{(}\hlstr{"x is negative"}\hlstd{)}
\hlstd{\}} \hlkwa{else if} \hlstd{(x} \hlopt{==} \hlnum{0}\hlstd{) \{}
  \hlkwd{print}\hlstd{(}\hlstr{"x is zero"}\hlstd{)}
\hlstd{\}}
\end{alltt}
\begin{verbatim}
## [1] "x is negative"
\end{verbatim}
\end{kframe}
\end{knitrout}

\end{frame}

%------------------------------------------------

\begin{frame}[fragile]
\frametitle{Vectorized \code{ifelse()}}

\code{if()} takes a single logical value. If you want to pass a logical vector of conditions, you can use {\hilit \code{ifelse()}}:
\begin{knitrout}\footnotesize
\definecolor{shadecolor}{rgb}{0.969, 0.969, 0.969}\color{fgcolor}\begin{kframe}
\begin{alltt}
\hlstd{true_false} \hlkwb{<-} \hlkwd{c}\hlstd{(}\hlnum{TRUE}\hlstd{,} \hlnum{FALSE}\hlstd{)}

\hlkwd{ifelse}\hlstd{(true_false,} \hlstr{"true"}\hlstd{,} \hlstr{"false"}\hlstd{)}
\end{alltt}
\begin{verbatim}
## [1] "true"  "false"
\end{verbatim}
\end{kframe}
\end{knitrout}

\end{frame}

%------------------------------------------------

\begin{frame}[fragile]
\frametitle{Vectorized If}

\begin{knitrout}\footnotesize
\definecolor{shadecolor}{rgb}{0.969, 0.969, 0.969}\color{fgcolor}\begin{kframe}
\begin{alltt}
\hlcom{# some numbers}
\hlstd{numbers} \hlkwb{<-} \hlkwd{c}\hlstd{(}\hlnum{1}\hlstd{,} \hlnum{0}\hlstd{,} \hlopt{-}\hlnum{4}\hlstd{,} \hlnum{9}\hlstd{,} \hlopt{-}\hlnum{0.9}\hlstd{)}

\hlcom{# are they non-negative or negative?}
\hlkwd{ifelse}\hlstd{(numbers} \hlopt{>=} \hlnum{0}\hlstd{,} \hlstr{"non-neg"}\hlstd{,} \hlstr{"neg"}\hlstd{)}
\end{alltt}
\begin{verbatim}
## [1] "non-neg" "non-neg" "neg"     "non-neg" "neg"
\end{verbatim}
\end{kframe}
\end{knitrout}

\end{frame}

%------------------------------------------------

\begin{frame}[fragile]
\frametitle{Test yourself}

Which option will cause an error:
\begin{columns}[t]
\begin{column}{0.45\textwidth}
\begin{knitrout}\footnotesize
\definecolor{shadecolor}{rgb}{0.969, 0.969, 0.969}\color{fgcolor}\begin{kframe}
\begin{alltt}
\hlcom{# A}
\hlkwa{if} \hlstd{(}\hlkwd{is.numeric}\hlstd{(}\hlnum{1}\hlopt{:}\hlnum{5}\hlstd{)) \{}
  \hlkwd{print}\hlstd{(}\hlstr{'ok'}\hlstd{)}
\hlstd{\}}


\hlcom{# B}
\hlkwa{if} \hlstd{(}\hlstr{"TRUE"}\hlstd{) \{}
  \hlkwd{print}\hlstd{(}\hlstr{'ok'}\hlstd{)}
\hlstd{\}}
\end{alltt}
\end{kframe}
\end{knitrout}
\end{column}

\begin{column}{0.45\textwidth}
\begin{knitrout}\footnotesize
\definecolor{shadecolor}{rgb}{0.969, 0.969, 0.969}\color{fgcolor}\begin{kframe}
\begin{alltt}
\hlcom{# C}
\hlkwa{if} \hlstd{(}\hlnum{0}\hlstd{)} \hlkwd{print}\hlstd{(}\hlstr{'ok'}\hlstd{)}


\hlcom{# D}
\hlkwa{if} \hlstd{(}\hlnum{NA}\hlstd{)} \hlkwd{print}\hlstd{(}\hlstr{'ok'}\hlstd{)}


\hlcom{# E}
\hlkwa{if} \hlstd{(}\hlstr{'yes'}\hlstd{) \{}
  \hlkwd{print}\hlstd{(}\hlstr{'ok'}\hlstd{)}
\hlstd{\}}
\end{alltt}
\end{kframe}
\end{knitrout}
\end{column}
\end{columns}

\end{frame}

%------------------------------------------------

\begin{frame}
\begin{center}
\Huge{Function \hilit{Switch}}
\end{center}
\end{frame}

%------------------------------------------------

\begin{frame}[fragile]
\frametitle{Multiple Selection with \code{switch()}}

{\large When a condition has multiple options, combining several \code{if} and \code{else} can become cumbersome}

\end{frame}

%------------------------------------------------

\begin{frame}[fragile]
\frametitle{}

\begin{knitrout}\footnotesize
\definecolor{shadecolor}{rgb}{0.969, 0.969, 0.969}\color{fgcolor}\begin{kframe}
\begin{alltt}
\hlstd{first_name} \hlkwb{<-} \hlstr{"harry"}

\hlkwa{if} \hlstd{(first_name} \hlopt{==} \hlstr{"harry"}\hlstd{) \{}
  \hlstd{last_name} \hlkwb{<-} \hlstr{"potter"}
\hlstd{\}} \hlkwa{else} \hlstd{\{}
  \hlkwa{if} \hlstd{(first_name} \hlopt{==} \hlstr{"ron"}\hlstd{) \{}
    \hlstd{last_name} \hlkwb{<-} \hlstr{"weasley"}
  \hlstd{\}} \hlkwa{else} \hlstd{\{}
    \hlkwa{if} \hlstd{(first_name} \hlopt{==} \hlstr{"hermione"}\hlstd{) \{}
      \hlstd{last_name} \hlkwb{<-} \hlstr{"granger"}
    \hlstd{\}} \hlkwa{else} \hlstd{\{}
      \hlstd{last_name} \hlkwb{<-} \hlstr{"not available"}
    \hlstd{\}}
  \hlstd{\}}
\hlstd{\}}

\hlstd{last_name}
\end{alltt}
\begin{verbatim}
## [1] "potter"
\end{verbatim}
\end{kframe}
\end{knitrout}

\end{frame}

%------------------------------------------------

\begin{frame}[fragile]
\frametitle{Multiple Selection with \code{switch()}}

\begin{knitrout}\footnotesize
\definecolor{shadecolor}{rgb}{0.969, 0.969, 0.969}\color{fgcolor}\begin{kframe}
\begin{alltt}
\hlstd{first_name} \hlkwb{<-} \hlstr{"ron"}

\hlstd{last_name} \hlkwb{<-} \hlkwd{switch}\hlstd{(}
  \hlstd{first_name,}
  \hlkwc{harry} \hlstd{=} \hlstr{"potter"}\hlstd{,}
  \hlkwc{ron} \hlstd{=} \hlstr{"weasley"}\hlstd{,}
  \hlkwc{hermione} \hlstd{=} \hlstr{"granger"}\hlstd{,}
  \hlstr{"not available"}\hlstd{)}

\hlstd{last_name}
\end{alltt}
\begin{verbatim}
## [1] "weasley"
\end{verbatim}
\end{kframe}
\end{knitrout}

\end{frame}

%------------------------------------------------

\begin{frame}[fragile]
\frametitle{Multiple Selection with \code{switch()}}

\bi
  \item the \code{switch()} function makes it possible to choose between various alternatives
  \item \code{switch()} takes a character string
  \item followed by several named arguments
  \item \code{switch()} will match the input string with the provided arguments
  \item a default value can be given when there's no match
  \item multiple expression can be enclosed by braces
\ei

\end{frame}

%------------------------------------------------

\begin{frame}[fragile]
\frametitle{Multiple Selection with \code{switch()}}

\begin{knitrout}\footnotesize
\definecolor{shadecolor}{rgb}{0.969, 0.969, 0.969}\color{fgcolor}\begin{kframe}
\begin{alltt}
\hlkwd{switch}\hlstd{(expr,}
       \hlkwc{tag1} \hlstd{= rcode_block1,}
       \hlkwc{tag2} \hlstd{= rcode_block2,}
       \hlstd{...}
       \hlstd{)}
\end{alltt}
\end{kframe}
\end{knitrout}
\code{switch()} selects one of the code blocks, depending on the value of \code{expr}

\end{frame}

%------------------------------------------------

\begin{frame}[fragile]
\frametitle{Multiple Selection with \code{switch()}}

\begin{knitrout}\footnotesize
\definecolor{shadecolor}{rgb}{0.969, 0.969, 0.969}\color{fgcolor}\begin{kframe}
\begin{alltt}
\hlstd{operation} \hlkwb{<-} \hlstr{"add"}

\hlstd{result} \hlkwb{<-} \hlkwd{switch}\hlstd{(}
  \hlstd{operation,}
  \hlkwc{add} \hlstd{=} \hlnum{2} \hlopt{+} \hlnum{3}\hlstd{,}
  \hlkwc{product} \hlstd{=} \hlnum{2} \hlopt{*} \hlnum{3}\hlstd{,}
  \hlkwc{division} \hlstd{=} \hlnum{2} \hlopt{/} \hlnum{3}\hlstd{,}
  \hlkwc{other} \hlstd{= \{}
    \hlstd{a} \hlkwb{<-} \hlnum{2} \hlopt{+} \hlnum{3}
    \hlkwd{exp}\hlstd{(}\hlnum{1} \hlopt{/} \hlkwd{sqrt}\hlstd{(a))}
  \hlstd{\}}
\hlstd{)}

\hlstd{result}
\end{alltt}
\begin{verbatim}
## [1] 5
\end{verbatim}
\end{kframe}
\end{knitrout}

\end{frame}

%------------------------------------------------

\begin{frame}[fragile]
\frametitle{Multiple Selection with \code{switch()}}

\bi
  \item \code{switch()} can also take an integer as first argument
  \item in this case the remaining arguments do not need names
  \item instead, the they will have associated integers
\ei

\begin{knitrout}\footnotesize
\definecolor{shadecolor}{rgb}{0.969, 0.969, 0.969}\color{fgcolor}\begin{kframe}
\begin{alltt}
\hlkwd{switch}\hlstd{(}
  \hlnum{4}\hlstd{,}
  \hlstr{"one"}\hlstd{,}
  \hlstr{"two"}\hlstd{,}
  \hlstr{"three"}\hlstd{,}
  \hlstr{"four"}\hlstd{)}
\end{alltt}
\begin{verbatim}
## [1] "four"
\end{verbatim}
\end{kframe}
\end{knitrout}

\end{frame}

%------------------------------------------------

\begin{frame}[fragile]
\frametitle{Empty code blocks in \code{switch()}}

Empty code blocks can be used to make several tags match the same code block:

\begin{knitrout}\footnotesize
\definecolor{shadecolor}{rgb}{0.969, 0.969, 0.969}\color{fgcolor}\begin{kframe}
\begin{alltt}
\hlstd{student} \hlkwb{<-} \hlstr{"ron"}

\hlstd{house} \hlkwb{<-} \hlkwd{switch}\hlstd{(}
  \hlstd{student,}
  \hlkwc{harry} \hlstd{= ,}
  \hlkwc{ron} \hlstd{= ,}
  \hlkwc{hermione} \hlstd{=} \hlstr{"gryffindor"}\hlstd{,}
  \hlkwc{draco} \hlstd{=} \hlstr{"slytherin"}\hlstd{)}
\end{alltt}
\end{kframe}
\end{knitrout}
In this case a value of \code{"harry"}, \code{"ron"} or \code{"hermione"} will cause \code{"gryffindor"}

\end{frame}

%------------------------------------------------

\begin{frame}
\begin{center}
\Huge{\hilit{Loops}}
\end{center}
\end{frame}

%------------------------------------------------

\begin{frame}[fragile]
\frametitle{About Loops}

\bbi
  \item Many times we need to perform a procedure several times
  \item The main idea is that of \textbf{iteration}
  \item For this purpose we use loops
  \item We perform operations as long as some condition is fulfilled
  \item R provides three basic paradigms:
  \bi
    \item \code{for}, \code{repeat}, \code{while}
  \ei
\ei

\end{frame}

%------------------------------------------------

\begin{frame}
\begin{center}
\Huge{\hilit{For Loops}}
\end{center}
\end{frame}

%------------------------------------------------

\begin{frame}
\frametitle{For Loops}

Often we want to repeatedly carry out some computation a fixed number of times. For instance, repeat an operation for each element of a vector. In R this is done with a {\hilit \code{for}} loop 

\end{frame}

%------------------------------------------------

\begin{frame}[fragile]
\frametitle{Motivation}

Consider some numeric vector
\begin{knitrout}\footnotesize
\definecolor{shadecolor}{rgb}{0.969, 0.969, 0.969}\color{fgcolor}\begin{kframe}
\begin{alltt}
\hlstd{prices} \hlkwb{<-} \hlkwd{c}\hlstd{(}\hlnum{2.50}\hlstd{,} \hlnum{2.95}\hlstd{,} \hlnum{3.45}\hlstd{,} \hlnum{3.25}\hlstd{)}

\hlstd{prices}
\end{alltt}
\begin{verbatim}
## [1] 2.50 2.95 3.45 3.25
\end{verbatim}
\end{kframe}
\end{knitrout}

\end{frame}

%------------------------------------------------

\begin{frame}[fragile]
\frametitle{Motivation}

Say you want to display the prices on the console:
\begin{knitrout}\footnotesize
\definecolor{shadecolor}{rgb}{0.969, 0.969, 0.969}\color{fgcolor}\begin{kframe}
\begin{alltt}
\hlkwd{cat}\hlstd{(}\hlstr{"Price 1 is"}\hlstd{, prices[}\hlnum{1}\hlstd{])}
\hlkwd{cat}\hlstd{(}\hlstr{"Price 2 is"}\hlstd{, prices[}\hlnum{2}\hlstd{])}
\hlkwd{cat}\hlstd{(}\hlstr{"Price 3 is"}\hlstd{, prices[}\hlnum{3}\hlstd{])}
\hlkwd{cat}\hlstd{(}\hlstr{"Price 4 is"}\hlstd{, prices[}\hlnum{4}\hlstd{])}
\end{alltt}
\end{kframe}
\end{knitrout}

We are repeating the same operation four times
\begin{knitrout}\footnotesize
\definecolor{shadecolor}{rgb}{0.969, 0.969, 0.969}\color{fgcolor}\begin{kframe}
\begin{verbatim}
## Price 1 is 2.5
## Price 2 is 2.95
## Price 3 is 3.45
## Price 4 is 3.25
\end{verbatim}
\end{kframe}
\end{knitrout}

\end{frame}

%------------------------------------------------

\begin{frame}[fragile]
\frametitle{Motivation}

We can use a \code{for} lope for this purpose:
\begin{knitrout}\footnotesize
\definecolor{shadecolor}{rgb}{0.969, 0.969, 0.969}\color{fgcolor}\begin{kframe}
\begin{alltt}
\hlkwa{for} \hlstd{(i} \hlkwa{in} \hlnum{1}\hlopt{:}\hlnum{4}\hlstd{) \{}
  \hlkwd{cat}\hlstd{(}\hlstr{"Price"}\hlstd{, i,} \hlstr{"is"}\hlstd{, prices[i],} \hlstr{"\textbackslash{}n"}\hlstd{)}
\hlstd{\}}
\end{alltt}
\begin{verbatim}
## Price 1 is 2.5 
## Price 2 is 2.95 
## Price 3 is 3.45 
## Price 4 is 3.25
\end{verbatim}
\end{kframe}
\end{knitrout}

\end{frame}

%------------------------------------------------

\begin{frame}[fragile]
\frametitle{For Loops}

\code{for} loops are used when we know exactly how many times we want the code to repeat
\begin{knitrout}\footnotesize
\definecolor{shadecolor}{rgb}{0.969, 0.969, 0.969}\color{fgcolor}\begin{kframe}
\begin{alltt}
\hlkwa{for} \hlstd{(iterator} \hlkwa{in} \hlstd{times) \{}
  \hlstd{do_something}
\hlstd{\}}
\end{alltt}
\end{kframe}
\end{knitrout}

\code{for} takes an \textit{iterator} variable and a vector of \textit{times} to iterate through

\end{frame}

%------------------------------------------------

\begin{frame}[fragile]
\frametitle{For Loops}

\begin{knitrout}\footnotesize
\definecolor{shadecolor}{rgb}{0.969, 0.969, 0.969}\color{fgcolor}\begin{kframe}
\begin{alltt}
\hlstd{value} \hlkwb{<-} \hlnum{2}

\hlkwa{for} \hlstd{(i} \hlkwa{in} \hlnum{1}\hlopt{:}\hlnum{5}\hlstd{) \{}
  \hlstd{value} \hlkwb{<-} \hlstd{value} \hlopt{*} \hlnum{2}
  \hlkwd{print}\hlstd{(value)}
\hlstd{\}}
\end{alltt}
\begin{verbatim}
## [1] 4
## [1] 8
## [1] 16
## [1] 32
## [1] 64
\end{verbatim}
\end{kframe}
\end{knitrout}

\end{frame}

%------------------------------------------------

\begin{frame}[fragile]
\frametitle{For Loops}

The vector of \textit{times} does not have to be a numeric vector; it can be any vector
\begin{knitrout}\footnotesize
\definecolor{shadecolor}{rgb}{0.969, 0.969, 0.969}\color{fgcolor}\begin{kframe}
\begin{alltt}
\hlstd{value} \hlkwb{<-} \hlnum{2}
\hlstd{times} \hlkwb{<-} \hlkwd{c}\hlstd{(}\hlstr{'1'}\hlstd{,} \hlstr{'2'}\hlstd{,} \hlstr{'3'}\hlstd{,} \hlstr{'4'}\hlstd{,} \hlstr{'5'}\hlstd{)}

\hlkwa{for} \hlstd{(i} \hlkwa{in} \hlstd{times) \{}
  \hlstd{value} \hlkwb{<-} \hlstd{value} \hlopt{*} \hlnum{2}
  \hlkwd{print}\hlstd{(value)}
\hlstd{\}}
\end{alltt}
\begin{verbatim}
## [1] 4
## [1] 8
## [1] 16
## [1] 32
## [1] 64
\end{verbatim}
\end{kframe}
\end{knitrout}

\end{frame}

%------------------------------------------------

\begin{frame}[fragile]
\frametitle{For Loops and Next statement}

Sometimes we need to skip a loop iteration if a given condition is met, this can be done with a \code{next} statement
\begin{knitrout}\footnotesize
\definecolor{shadecolor}{rgb}{0.969, 0.969, 0.969}\color{fgcolor}\begin{kframe}
\begin{alltt}
\hlkwa{for} \hlstd{(iterator} \hlkwa{in} \hlstd{times) \{}
  \hlstd{expr1}
  \hlstd{expr2}
  \hlkwa{if} \hlstd{(condition) \{}
    \hlkwa{next}
  \hlstd{\}}
  \hlstd{expr3}
  \hlstd{expr4}
\hlstd{\}}
\end{alltt}
\end{kframe}
\end{knitrout}

\end{frame}

%------------------------------------------------

\begin{frame}[fragile]
\frametitle{For Loops and Next statement}

\begin{knitrout}\footnotesize
\definecolor{shadecolor}{rgb}{0.969, 0.969, 0.969}\color{fgcolor}\begin{kframe}
\begin{alltt}
\hlstd{x} \hlkwb{<-} \hlnum{2}
\hlkwa{for} \hlstd{(i} \hlkwa{in} \hlnum{1}\hlopt{:}\hlnum{5}\hlstd{) \{}
  \hlstd{y} \hlkwb{<-} \hlstd{x} \hlopt{*} \hlstd{i}
  \hlkwa{if} \hlstd{(y} \hlopt{==} \hlnum{8}\hlstd{) \{}
    \hlkwa{next}
  \hlstd{\}}
  \hlkwd{print}\hlstd{(y)}
\hlstd{\}}
\end{alltt}
\begin{verbatim}
## [1] 2
## [1] 4
## [1] 6
## [1] 10
\end{verbatim}
\end{kframe}
\end{knitrout}

\end{frame}

%------------------------------------------------

\begin{frame}[fragile]
\frametitle{For Loops}

\begin{columns}[t]
\begin{column}{0.45\textwidth}
For loop
\begin{knitrout}\footnotesize
\definecolor{shadecolor}{rgb}{0.969, 0.969, 0.969}\color{fgcolor}\begin{kframe}
\begin{alltt}
\hlcom{# squaring values}
\hlstd{x} \hlkwb{<-} \hlnum{1}\hlopt{:}\hlnum{5}
\hlstd{y} \hlkwb{<-} \hlstd{x}

\hlkwa{for} \hlstd{(i} \hlkwa{in} \hlnum{1}\hlopt{:}\hlnum{5}\hlstd{) \{}
  \hlstd{y[i]} \hlkwb{<-} \hlstd{x[i]}\hlopt{^}\hlnum{2}
\hlstd{\}}

\hlstd{y}
\end{alltt}
\begin{verbatim}
## [1]  1  4  9 16 25
\end{verbatim}
\end{kframe}
\end{knitrout}
\end{column}

\begin{column}{0.45\textwidth}
Vectorized computation
\begin{knitrout}\footnotesize
\definecolor{shadecolor}{rgb}{0.969, 0.969, 0.969}\color{fgcolor}\begin{kframe}
\begin{alltt}
\hlcom{# squaring values}
\hlstd{x} \hlkwb{<-} \hlnum{1}\hlopt{:}\hlnum{5}
\hlstd{y} \hlkwb{<-} \hlstd{x}\hlopt{^}\hlnum{2}
\hlstd{y}
\end{alltt}
\begin{verbatim}
## [1]  1  4  9 16 25
\end{verbatim}
\end{kframe}
\end{knitrout}
\end{column}
\end{columns}

\end{frame}

%------------------------------------------------

\begin{frame}[fragile]
\frametitle{A less boring example}

\begin{knitrout}\footnotesize
\definecolor{shadecolor}{rgb}{0.969, 0.969, 0.969}\color{fgcolor}\begin{kframe}
\begin{alltt}
\hlcom{# Generate a sample dataset (missing values as '-99')}
\hlkwd{set.seed}\hlstd{(}\hlnum{6354}\hlstd{)}
\hlstd{dat} \hlkwb{<-} \hlkwd{data.frame}\hlstd{(}
  \hlkwd{replicate}\hlstd{(}\hlnum{6}\hlstd{,} \hlkwd{sample}\hlstd{(}\hlkwd{c}\hlstd{(}\hlnum{1}\hlopt{:}\hlnum{10}\hlstd{,} \hlopt{-}\hlnum{99}\hlstd{),} \hlnum{6}\hlstd{,} \hlkwc{rep} \hlstd{=} \hlnum{TRUE}\hlstd{))}
\hlstd{)}
\hlkwd{names}\hlstd{(dat)} \hlkwb{<-} \hlstd{letters[}\hlnum{1}\hlopt{:}\hlnum{6}\hlstd{]}

\hlstd{dat}
\end{alltt}
\begin{verbatim}
##    a  b   c d  e  f
## 1  5  9   1 7  8 10
## 2  2  6  10 7  3  7
## 3 10  7   2 9 10  9
## 4  3  2 -99 4  5  6
## 5  8 10   3 3  3  5
## 6  4  5   2 3  1 10
\end{verbatim}
\end{kframe}
\end{knitrout}

\end{frame}

%------------------------------------------------

\begin{frame}[fragile]
\frametitle{A less boring example}

\begin{knitrout}\footnotesize
\definecolor{shadecolor}{rgb}{0.969, 0.969, 0.969}\color{fgcolor}\begin{kframe}
\begin{alltt}
\hlcom{# converting '-99' to NA's}
\hlstd{dat}\hlopt{$}\hlstd{a[dat}\hlopt{$}\hlstd{a} \hlopt{== -}\hlnum{99}\hlstd{]} \hlkwb{<-} \hlnum{NA}
\hlstd{dat}\hlopt{$}\hlstd{b[dat}\hlopt{$}\hlstd{b} \hlopt{== -}\hlnum{99}\hlstd{]} \hlkwb{<-} \hlnum{NA}
\hlstd{dat}\hlopt{$}\hlstd{c[dat}\hlopt{$}\hlstd{c} \hlopt{== -}\hlnum{98}\hlstd{]} \hlkwb{<-} \hlnum{NA}
\hlstd{dat}\hlopt{$}\hlstd{d[dat}\hlopt{$}\hlstd{d} \hlopt{== -}\hlnum{99}\hlstd{]} \hlkwb{<-} \hlnum{NA}
\hlstd{dat}\hlopt{$}\hlstd{e[dat}\hlopt{$}\hlstd{e} \hlopt{== -}\hlnum{99}\hlstd{]} \hlkwb{<-} \hlnum{NA}
\hlstd{dat}\hlopt{$}\hlstd{f[dat}\hlopt{$}\hlstd{g} \hlopt{== -}\hlnum{99}\hlstd{]} \hlkwb{<-} \hlnum{NA}
\end{alltt}
\end{kframe}
\end{knitrout}

With copy-and-paste it is easy to make mistakes. Can you spot any mistakes?

\end{frame}

%------------------------------------------------

\begin{frame}[fragile]
\frametitle{A less boring example}

\begin{knitrout}\footnotesize
\definecolor{shadecolor}{rgb}{0.969, 0.969, 0.969}\color{fgcolor}\begin{kframe}
\begin{alltt}
\hlcom{# converting '-99' to NA's}
\hlstd{dat}\hlopt{$}\hlstd{a[dat}\hlopt{$}\hlstd{a} \hlopt{== -}\hlnum{99}\hlstd{]} \hlkwb{<-} \hlnum{NA}
\hlstd{dat}\hlopt{$}\hlstd{b[dat}\hlopt{$}\hlstd{b} \hlopt{== -}\hlnum{99}\hlstd{]} \hlkwb{<-} \hlnum{NA}
\hlstd{dat}\hlopt{$}\hlstd{c[dat}\hlopt{$}\hlstd{c} \hlopt{== -}\hlnum{98}\hlstd{]} \hlkwb{<-} \hlnum{NA}  \hlcom{# oops}
\hlstd{dat}\hlopt{$}\hlstd{d[dat}\hlopt{$}\hlstd{d} \hlopt{== -}\hlnum{99}\hlstd{]} \hlkwb{<-} \hlnum{NA}
\hlstd{dat}\hlopt{$}\hlstd{e[dat}\hlopt{$}\hlstd{e} \hlopt{== -}\hlnum{99}\hlstd{]} \hlkwb{<-} \hlnum{NA}
\hlstd{dat}\hlopt{$}\hlstd{f[dat}\hlopt{$}\hlstd{g} \hlopt{== -}\hlnum{99}\hlstd{]} \hlkwb{<-} \hlnum{NA}  \hlcom{# oops}
\end{alltt}
\end{kframe}
\end{knitrout}

\end{frame}

%------------------------------------------------

\begin{frame}[fragile]
\frametitle{A less boring example}

\begin{knitrout}\footnotesize
\definecolor{shadecolor}{rgb}{0.969, 0.969, 0.969}\color{fgcolor}\begin{kframe}
\begin{alltt}
\hlcom{# for loop to convert '-99' to NA's}
\hlstd{columns} \hlkwb{<-} \hlstd{letters[}\hlnum{1}\hlopt{:}\hlnum{6}\hlstd{]}
\hlkwa{for} \hlstd{(j} \hlkwa{in} \hlstd{columns) \{}
  \hlstd{dat[ ,j][dat[ ,j]} \hlopt{== -}\hlnum{99}\hlstd{]} \hlkwb{<-} \hlnum{NA}
\hlstd{\}}

\hlstd{dat}
\end{alltt}
\begin{verbatim}
##    a  b  c d  e  f
## 1  5  9  1 7  8 10
## 2  2  6 10 7  3  7
## 3 10  7  2 9 10  9
## 4  3  2 NA 4  5  6
## 5  8 10  3 3  3  5
## 6  4  5  2 3  1 10
\end{verbatim}
\end{kframe}
\end{knitrout}

\end{frame}

%------------------------------------------------

\begin{frame}[fragile]
\frametitle{Nested Loops}

It is common to have nested loops
\begin{knitrout}\footnotesize
\definecolor{shadecolor}{rgb}{0.969, 0.969, 0.969}\color{fgcolor}\begin{kframe}
\begin{alltt}
\hlkwa{for} \hlstd{(iterator1} \hlkwa{in} \hlstd{times1) \{}
  \hlkwa{for} \hlstd{(iterator2} \hlkwa{in} \hlstd{times2) \{}
    \hlstd{expr1}
    \hlstd{expr2}
    \hlstd{...}
  \hlstd{\}}
\hlstd{\}}
\end{alltt}
\end{kframe}
\end{knitrout}

\end{frame}

%------------------------------------------------

\begin{frame}[fragile]
\frametitle{Nested Loops}

\begin{knitrout}\footnotesize
\definecolor{shadecolor}{rgb}{0.969, 0.969, 0.969}\color{fgcolor}\begin{kframe}
\begin{alltt}
\hlcom{# some matrix }
\hlstd{A} \hlkwb{<-} \hlkwd{matrix}\hlstd{(}\hlnum{1}\hlopt{:}\hlnum{12}\hlstd{,} \hlkwc{nrow} \hlstd{=} \hlnum{3}\hlstd{,} \hlkwc{ncol} \hlstd{=} \hlnum{4}\hlstd{)}

\hlstd{A}
\end{alltt}
\begin{verbatim}
##      [,1] [,2] [,3] [,4]
## [1,]    1    4    7   10
## [2,]    2    5    8   11
## [3,]    3    6    9   12
\end{verbatim}
\end{kframe}
\end{knitrout}

\end{frame}

%------------------------------------------------

\begin{frame}[fragile]
\frametitle{Nested Loops}
\begin{knitrout}\footnotesize
\definecolor{shadecolor}{rgb}{0.969, 0.969, 0.969}\color{fgcolor}\begin{kframe}
\begin{alltt}
\hlcom{# obtaining reciprocal of those values < 6}
\hlkwa{for} \hlstd{(i} \hlkwa{in} \hlnum{1}\hlopt{:}\hlkwd{nrow}\hlstd{(A)) \{}
  \hlkwa{for} \hlstd{(j} \hlkwa{in} \hlnum{1}\hlopt{:}\hlkwd{ncol}\hlstd{(A)) \{}
    \hlkwa{if} \hlstd{(A[i,j]} \hlopt{<} \hlnum{6}\hlstd{) A[i,j]} \hlkwb{<-} \hlnum{1} \hlopt{/} \hlstd{A[i,j]}
  \hlstd{\}}
\hlstd{\}}

\hlstd{A}
\end{alltt}
\begin{verbatim}
##           [,1] [,2] [,3] [,4]
## [1,] 1.0000000 0.25    7   10
## [2,] 0.5000000 0.20    8   11
## [3,] 0.3333333 6.00    9   12
\end{verbatim}
\end{kframe}
\end{knitrout}

\end{frame}

%------------------------------------------------

\begin{frame}
\begin{center}
\Huge{\hilit{Repeat Loops}}
\end{center}
\end{frame}

%------------------------------------------------

\begin{frame}[fragile]
\frametitle{Repeat Loops}

\code{repeat} executes the same code over and over until a stop condition is met.

\begin{knitrout}\footnotesize
\definecolor{shadecolor}{rgb}{0.969, 0.969, 0.969}\color{fgcolor}\begin{kframe}
\begin{alltt}
\hlkwa{repeat} \hlstd{\{}
  \hlstd{keep_doing_something}
  \hlkwa{if} \hlstd{(stop_condition)} \hlkwa{break}
\hlstd{\}}
\end{alltt}
\end{kframe}
\end{knitrout}

The \code{break} statement stops the loops

\end{frame}

%------------------------------------------------

\begin{frame}[fragile]
\frametitle{Repeat Loops}

\begin{knitrout}\footnotesize
\definecolor{shadecolor}{rgb}{0.969, 0.969, 0.969}\color{fgcolor}\begin{kframe}
\begin{alltt}
\hlstd{value} \hlkwb{<-} \hlnum{2}

\hlkwa{repeat} \hlstd{\{}
  \hlstd{value} \hlkwb{<-} \hlstd{value} \hlopt{*} \hlnum{2}
  \hlkwd{print}\hlstd{(value)}
  \hlkwa{if} \hlstd{(value} \hlopt{>=} \hlnum{40}\hlstd{)} \hlkwa{break}
\hlstd{\}}
\end{alltt}
\begin{verbatim}
## [1] 4
## [1] 8
## [1] 16
## [1] 32
## [1] 64
\end{verbatim}
\end{kframe}
\end{knitrout}

If you enter an infinite loop, break it by pressing \code{ESC} key

\end{frame}

%------------------------------------------------

\begin{frame}[fragile]
\frametitle{Repeat Loops}

To skip a current iteration, use \code{next}
\begin{knitrout}\footnotesize
\definecolor{shadecolor}{rgb}{0.969, 0.969, 0.969}\color{fgcolor}\begin{kframe}
\begin{alltt}
\hlstd{value} \hlkwb{<-} \hlnum{2}

\hlkwa{repeat} \hlstd{\{}
  \hlstd{value} \hlkwb{<-} \hlstd{value} \hlopt{*} \hlnum{2}
  \hlkwd{print}\hlstd{(value)}
  \hlkwa{if} \hlstd{(value} \hlopt{==} \hlnum{16}\hlstd{) \{}
    \hlstd{value} \hlkwb{<-} \hlstd{value} \hlopt{*} \hlnum{2}
    \hlkwa{next}
  \hlstd{\}}
  \hlkwa{if} \hlstd{(value} \hlopt{>} \hlnum{80}\hlstd{)} \hlkwa{break}
\hlstd{\}}
\end{alltt}
\begin{verbatim}
## [1] 4
## [1] 8
## [1] 16
## [1] 64
## [1] 128
\end{verbatim}
\end{kframe}
\end{knitrout}

\end{frame}

%------------------------------------------------

\begin{frame}
\begin{center}
\Huge{\hilit{While Loops}}
\end{center}
\end{frame}

%------------------------------------------------

\begin{frame}[fragile]
\frametitle{While Loops}

It can also be useful to repeat a computation until a condition is false. A \code{while} loop provides this form of control flow
\begin{knitrout}\footnotesize
\definecolor{shadecolor}{rgb}{0.969, 0.969, 0.969}\color{fgcolor}\begin{kframe}
\begin{alltt}
\hlkwa{while} \hlstd{(condition) \{}
  \hlstd{keep_doing_something}
\hlstd{\}}
\end{alltt}
\end{kframe}
\end{knitrout}
\end{frame}

%------------------------------------------------

\begin{frame}[fragile]
\frametitle{While Loops}

\bi
  \item \code{while} loops are like backward \code{repeat} loops; 
  \item \code{while} checks first and then attempts to execute
  \item computations are carried out for as long as the condition is true
  \item the loop stops when the condition is false
  \item If you enter an infinite loop, break it by pressing \code{ESC} key
\ei

\end{frame}

%------------------------------------------------

\begin{frame}[fragile]
\frametitle{While Loops}

\begin{knitrout}\footnotesize
\definecolor{shadecolor}{rgb}{0.969, 0.969, 0.969}\color{fgcolor}\begin{kframe}
\begin{alltt}
\hlstd{value} \hlkwb{<-} \hlnum{2}

\hlkwa{while} \hlstd{(value} \hlopt{<} \hlnum{40}\hlstd{) \{}
  \hlstd{value} \hlkwb{<-} \hlstd{value} \hlopt{*} \hlnum{2}
  \hlkwd{print}\hlstd{(value)}
\hlstd{\}}
\end{alltt}
\begin{verbatim}
## [1] 4
## [1] 8
## [1] 16
## [1] 32
## [1] 64
\end{verbatim}
\end{kframe}
\end{knitrout}

If you enter an infinite loop, break it by pressing \code{ESC} key

\end{frame}

%------------------------------------------------

\begin{frame}
\frametitle{For Loops and Vectorized Computations}

\bi
  \item R \code{for} loops have a bad reputation for being slow
  \item Experienced users will tell you to avoid for loops in R (me included)
  \item R provides a family of functions that tend to be more efficient than loops (i.e. \code{apply()} functions)
\ei

\end{frame}

%------------------------------------------------

\begin{frame}
\frametitle{For Loops and Vectorized Computations}

\bi
  \item For purposes of learning programming (and flow control structures in R), I won't demonize R loops
  \item You can start solving a problem using a for loop
  \item Once you solved it, try to see if you can find a vectorized alternative
  \item It takes practice and experience to find alternative solutions to for loops
  \item There are cases when using \code{for} loops is inevitable
\ei

\end{frame}

%------------------------------------------------

\begin{frame}
\frametitle{Repeat, While, For}

\bi
  \item If you don't know the number of times something will be done you can use either \code{repeat} or \code{while}
  \item \code{while} evaluates the condition at the beginning
  \item \code{repeat} executes operations until a stop condition is met
  \item If you know the number of times that something will be done, use \code{for}
  \item \code{for} needs an \textit{iterator} and a vector of \textit{times}
\ei

\end{frame}

%------------------------------------------------

\begin{frame}[fragile]
\frametitle{For Loop}

This example is just for demo purposes (not recommended in R)
\begin{knitrout}\footnotesize
\definecolor{shadecolor}{rgb}{0.969, 0.969, 0.969}\color{fgcolor}\begin{kframe}
\begin{alltt}
\hlcom{# empty numeric vector}
\hlstd{x} \hlkwb{<-} \hlkwd{numeric}\hlstd{(}\hlnum{0}\hlstd{)}
\hlstd{x}
\end{alltt}
\begin{verbatim}
## numeric(0)
\end{verbatim}
\begin{alltt}
\hlcom{# for loop to fill x}
\hlkwa{for} \hlstd{(i} \hlkwa{in} \hlnum{1}\hlopt{:}\hlnum{5}\hlstd{) \{}
  \hlstd{x[i]} \hlkwb{<-} \hlstd{i}
\hlstd{\}}
\hlstd{x}
\end{alltt}
\begin{verbatim}
## [1] 1 2 3 4 5
\end{verbatim}
\end{kframe}
\end{knitrout}

\end{frame}

%------------------------------------------------

\begin{frame}[fragile]
\frametitle{For Loop}

If you know the number of times
\begin{knitrout}\footnotesize
\definecolor{shadecolor}{rgb}{0.969, 0.969, 0.969}\color{fgcolor}\begin{kframe}
\begin{alltt}
\hlcom{# empty numeric vector}
\hlstd{x} \hlkwb{<-} \hlkwd{numeric}\hlstd{(}\hlnum{5}\hlstd{)}
\hlstd{x}
\end{alltt}
\begin{verbatim}
## [1] 0 0 0 0 0
\end{verbatim}
\begin{alltt}
\hlcom{# for loop to fill x}
\hlkwa{for} \hlstd{(i} \hlkwa{in} \hlnum{1}\hlopt{:}\hlnum{5}\hlstd{) \{}
  \hlstd{x[i]} \hlkwb{<-} \hlstd{i}
\hlstd{\}}
\hlstd{x}
\end{alltt}
\begin{verbatim}
## [1] 1 2 3 4 5
\end{verbatim}
\end{kframe}
\end{knitrout}

\end{frame}

%------------------------------------------------

\begin{frame}
\frametitle{Quiz Questions}

\bbi
  \item What happens if you pass \code{NA} as a condition to \code{if()}?
  \item What happens if you pass \code{NA} as a condition to \code{ifelse()}?
  \item What types of values can be passed as the first argument to the \code{switch()} function?
  \item How do you stop a \code{repeat} loop executing?
  \item How do you jump to the next iteration of a loop?
\ei

\end{frame}

%------------------------------------------------


\end{document}
