\documentclass[12pt]{beamer}\usepackage[]{graphicx}\usepackage[]{color}
%% maxwidth is the original width if it is less than linewidth
%% otherwise use linewidth (to make sure the graphics do not exceed the margin)
\makeatletter
\def\maxwidth{ %
  \ifdim\Gin@nat@width>\linewidth
    \linewidth
  \else
    \Gin@nat@width
  \fi
}
\makeatother

\definecolor{fgcolor}{rgb}{0.345, 0.345, 0.345}
\newcommand{\hlnum}[1]{\textcolor[rgb]{0.686,0.059,0.569}{#1}}%
\newcommand{\hlstr}[1]{\textcolor[rgb]{0.192,0.494,0.8}{#1}}%
\newcommand{\hlcom}[1]{\textcolor[rgb]{0.678,0.584,0.686}{\textit{#1}}}%
\newcommand{\hlopt}[1]{\textcolor[rgb]{0,0,0}{#1}}%
\newcommand{\hlstd}[1]{\textcolor[rgb]{0.345,0.345,0.345}{#1}}%
\newcommand{\hlkwa}[1]{\textcolor[rgb]{0.161,0.373,0.58}{\textbf{#1}}}%
\newcommand{\hlkwb}[1]{\textcolor[rgb]{0.69,0.353,0.396}{#1}}%
\newcommand{\hlkwc}[1]{\textcolor[rgb]{0.333,0.667,0.333}{#1}}%
\newcommand{\hlkwd}[1]{\textcolor[rgb]{0.737,0.353,0.396}{\textbf{#1}}}%
\let\hlipl\hlkwb

\usepackage{framed}
\makeatletter
\newenvironment{kframe}{%
 \def\at@end@of@kframe{}%
 \ifinner\ifhmode%
  \def\at@end@of@kframe{\end{minipage}}%
  \begin{minipage}{\columnwidth}%
 \fi\fi%
 \def\FrameCommand##1{\hskip\@totalleftmargin \hskip-\fboxsep
 \colorbox{shadecolor}{##1}\hskip-\fboxsep
     % There is no \\@totalrightmargin, so:
     \hskip-\linewidth \hskip-\@totalleftmargin \hskip\columnwidth}%
 \MakeFramed {\advance\hsize-\width
   \@totalleftmargin\z@ \linewidth\hsize
   \@setminipage}}%
 {\par\unskip\endMakeFramed%
 \at@end@of@kframe}
\makeatother

\definecolor{shadecolor}{rgb}{.97, .97, .97}
\definecolor{messagecolor}{rgb}{0, 0, 0}
\definecolor{warningcolor}{rgb}{1, 0, 1}
\definecolor{errorcolor}{rgb}{1, 0, 0}
\newenvironment{knitrout}{}{} % an empty environment to be redefined in TeX

\usepackage{alltt}
\usepackage{graphicx}
\usepackage{tikz}
\setbeameroption{hide notes}
\setbeamertemplate{note page}[plain]
\usepackage{listings}

% get rid of junk
\usetheme{default}
\usefonttheme[onlymath]{serif}
\beamertemplatenavigationsymbolsempty
\hypersetup{pdfpagemode=UseNone} % don't show bookmarks on initial view

% named colors
\definecolor{offwhite}{RGB}{255,250,240}
\definecolor{gray}{RGB}{155,155,155}

\ifx\notescolors\undefined % slides

  \definecolor{foreground}{RGB}{80,80,80}
  \definecolor{background}{RGB}{255,255,255}
  \definecolor{title}{RGB}{255,199,0}
  \definecolor{subtitle}{RGB}{89,132,212}
  \definecolor{hilit}{RGB}{248,117,79}
  \definecolor{vhilit}{RGB}{255,111,207}
  \definecolor{lolit}{RGB}{200,200,200}
  \definecolor{lit}{RGB}{255,199,0}
  \definecolor{mdlit}{RGB}{89,132,212}
  \definecolor{link}{RGB}{248,117,79}

\else % notes
  \definecolor{background}{RGB}{255,255,255}
  \definecolor{foreground}{RGB}{24,24,24}
  \definecolor{title}{RGB}{27,94,134}
  \definecolor{subtitle}{RGB}{22,175,124}
  \definecolor{hilit}{RGB}{122,0,128}
  \definecolor{vhilit}{RGB}{255,0,128}
  \definecolor{lolit}{RGB}{95,95,95}
\fi
\definecolor{nhilit}{RGB}{128,0,128}  % hilit color in notes
\definecolor{nvhilit}{RGB}{255,0,128} % vhilit for notes

\newcommand{\hilit}{\color{hilit}}
\newcommand{\vhilit}{\color{vhilit}}
\newcommand{\nhilit}{\color{nhilit}}
\newcommand{\nvhilit}{\color{nvhilit}}
\newcommand{\lit}{\color{lit}}
\newcommand{\mdlit}{\color{mdlit}}
\newcommand{\lolit}{\color{lolit}}

% use those colors
\setbeamercolor{titlelike}{fg=title}
\setbeamercolor{subtitle}{fg=subtitle}
\setbeamercolor{frametitle}{fg=gray}
\setbeamercolor{structure}{fg=subtitle}
\setbeamercolor{institute}{fg=lolit}
\setbeamercolor{normal text}{fg=foreground,bg=background}
%\setbeamercolor{item}{fg=foreground} % color of bullets
%\setbeamercolor{subitem}{fg=hilit}
%\setbeamercolor{itemize/enumerate subbody}{fg=lolit}
\setbeamertemplate{itemize subitem}{{\textendash}}
\setbeamerfont{itemize/enumerate subbody}{size=\footnotesize}
\setbeamerfont{itemize/enumerate subitem}{size=\footnotesize}

% center title of slides
\setbeamertemplate{blocks}[rounded]
\setbeamertemplate{frametitle}[default][center]
% margins
\setbeamersize{text margin left=25pt,text margin right=25pt}

% page number
\setbeamertemplate{footline}{%
    \raisebox{5pt}{\makebox[\paperwidth]{\hfill\makebox[20pt]{\lolit
          \scriptsize\insertframenumber}}}\hspace*{5pt}}

% add a bit of space at the top of the notes page
\addtobeamertemplate{note page}{\setlength{\parskip}{12pt}}

% default link color
\hypersetup{colorlinks, urlcolor={link}}

\ifx\notescolors\undefined % slides
  % set up listing environment
  \lstset{language=bash,
          basicstyle=\ttfamily\scriptsize,
          frame=single,
          commentstyle=,
          backgroundcolor=\color{darkgray},
          showspaces=false,
          showstringspaces=false
          }
\else % notes
  \lstset{language=bash,
          basicstyle=\ttfamily\scriptsize,
          frame=single,
          commentstyle=,
          backgroundcolor=\color{offwhite},
          showspaces=false,
          showstringspaces=false
          }
\fi

% a few macros
\newcommand{\code}[1]{\texttt{#1}}
\newcommand{\hicode}[1]{{\hilit \texttt{#1}}}
\newcommand{\bb}[1]{\begin{block}{#1}}
\newcommand{\eb}{\end{block}}
\newcommand{\bi}{\begin{itemize}}
%\newcommand{\bbi}{\vspace{24pt} \begin{itemize} \itemsep8pt}
\newcommand{\bbi}{\vspace{4pt} \begin{itemize} \itemsep8pt}
\newcommand{\ei}{\end{itemize}}
\newcommand{\bv}{\begin{verbatim}}
\newcommand{\ev}{\end{verbatim}}
\newcommand{\ig}{\includegraphics}
\newcommand{\subt}[1]{{\footnotesize \color{subtitle} {#1}}}
\newcommand{\ttsm}{\tt \small}
\newcommand{\ttfn}{\tt \footnotesize}
\newcommand{\figh}[2]{\centerline{\includegraphics[height=#2\textheight]{#1}}}
\newcommand{\figw}[2]{\centerline{\includegraphics[width=#2\textwidth]{#1}}}



%------------------------------------------------
% end of header
%------------------------------------------------

\title{Functions - part 2}
\subtitle{STAT 133}
\author{\href{http://www.gastonsanchez.com}{Gaston Sanchez}}
\institute{\href{https://github.com/ucb-stat133/stat133-fall-2016}{\tt \scriptsize \color{foreground} github.com/ucb-stat133/stat133-fall-2016}}
\date{}
\IfFileExists{upquote.sty}{\usepackage{upquote}}{}
\begin{document}


{
  \setbeamertemplate{footline}{} % no page number here
  \frame{
    \titlepage
  } 
}


%------------------------------------------------

\begin{frame}
\begin{center}
\Huge{\hilit{Binary Operators}}
\end{center}
\end{frame}

%------------------------------------------------

\begin{frame}
\frametitle{Binary Operators}

\bi
  \item One type of functions very common in R are \textbf{binary operators}, eg:
  \bi
    \item \code{2 + 5} (sum)
    \item \code{3 / 2} (division)
    \item \code{a \%in\% b} (value matching)
    \item \code{X \%*\% Y} (matrix multiplication)
  \ei
  \item Binary operators are actually functions
  \item These functions take two inputs---hence the term \textit{binary}
  \item It is possible to define your own binary operators
\ei

\end{frame}

%------------------------------------------------

\begin{frame}[fragile]
\frametitle{Binary Operators}

Example:
\begin{knitrout}\footnotesize
\definecolor{shadecolor}{rgb}{0.969, 0.969, 0.969}\color{fgcolor}\begin{kframe}
\begin{alltt}
\hlcom{# addition operator}
\hlnum{2} \hlopt{+} \hlnum{3}

\hlcom{# equivalent to}
\hlstr{'+'}\hlstd{(}\hlnum{2}\hlstd{,} \hlnum{3}\hlstd{)}
\end{alltt}
\end{kframe}
\end{knitrout}

\end{frame}

%------------------------------------------------

\begin{frame}[fragile]
\frametitle{Binary Operators}

\begin{knitrout}\footnotesize
\definecolor{shadecolor}{rgb}{0.969, 0.969, 0.969}\color{fgcolor}\begin{kframe}
\begin{alltt}
\hlcom{# binary operator}
\hlstr{"%p%"} \hlkwb{<-} \hlkwa{function}\hlstd{(}\hlkwc{x}\hlstd{,} \hlkwc{y}\hlstd{) \{}
  \hlkwd{paste}\hlstd{(x, y,} \hlkwc{sep} \hlstd{=} \hlstr{" "}\hlstd{)}
\hlstd{\}}

\hlstr{'good'} \hlopt \hlstr{'morning'}
\end{alltt}
\begin{verbatim}
## [1] "good morning"
\end{verbatim}
\end{kframe}
\end{knitrout}

\end{frame}

%------------------------------------------------

\begin{frame}
\frametitle{How to create a binary operator?}


\bbi
  \item A binary operator is defined as one or more characters surrounded by percent symbols {\hilit \code{\%}}
  \item When defining the function, the entire name must be quoted
  \item Include two arguments
  \item As usual, avoid using names of existing operators:
  \bi
    \item \code{"\%\%"}, \code{\%*\%}, \code{\%/\%}, \code{\%o\%}, \code{\%in\%}
  \ei
\ei

\end{frame}

%------------------------------------------------

\begin{frame}[fragile]
\frametitle{Another example}

Here's another example:
\begin{knitrout}\footnotesize
\definecolor{shadecolor}{rgb}{0.969, 0.969, 0.969}\color{fgcolor}\begin{kframe}
\begin{alltt}
\hlcom{# binary operator}
\hlstr{"%u%"} \hlkwb{<-} \hlkwa{function}\hlstd{(}\hlkwc{x}\hlstd{,} \hlkwc{y}\hlstd{) \{}
  \hlkwd{union}\hlstd{(x, y)}
\hlstd{\}}

\hlnum{1}\hlopt{:}\hlnum{5} \hlopt \hlkwd{c}\hlstd{(}\hlnum{1}\hlstd{,} \hlnum{3}\hlstd{,} \hlnum{5}\hlstd{,} \hlnum{7}\hlstd{,} \hlnum{9}\hlstd{)}
\end{alltt}
\begin{verbatim}
## [1] 1 2 3 4 5 7 9
\end{verbatim}
\end{kframe}
\end{knitrout}

\end{frame}

%------------------------------------------------

\begin{frame}
\begin{center}
\Huge{\hilit{Lazy Evaluation}}
\end{center}
\end{frame}

%------------------------------------------------

\begin{frame}[fragile]
\frametitle{Lazy Evaluation}

Arguments to functions are evaluated lazily, that is, they are evaluated only as needed:
\begin{knitrout}\footnotesize
\definecolor{shadecolor}{rgb}{0.969, 0.969, 0.969}\color{fgcolor}\begin{kframe}
\begin{alltt}
\hlstd{g} \hlkwb{<-} \hlkwa{function}\hlstd{(}\hlkwc{a}\hlstd{,} \hlkwc{b}\hlstd{) \{}
  \hlstd{a} \hlopt{*} \hlstd{a} \hlopt{*} \hlstd{a}
\hlstd{\}}

\hlkwd{g}\hlstd{(}\hlnum{2}\hlstd{)}
\end{alltt}
\begin{verbatim}
## [1] 8
\end{verbatim}
\end{kframe}
\end{knitrout}

\code{g()} never uses the argument \code{b}, so calling \code{g(2)} does not produce an error

\end{frame}

%------------------------------------------------

\begin{frame}[fragile]
\frametitle{Lazy Evaluation}

Another example
\begin{knitrout}\footnotesize
\definecolor{shadecolor}{rgb}{0.969, 0.969, 0.969}\color{fgcolor}\begin{kframe}
\begin{alltt}
\hlstd{g} \hlkwb{<-} \hlkwa{function}\hlstd{(}\hlkwc{a}\hlstd{,} \hlkwc{b}\hlstd{) \{}
  \hlkwd{print}\hlstd{(a)}
  \hlkwd{print}\hlstd{(b)}
\hlstd{\}}

\hlkwd{g}\hlstd{(}\hlnum{2}\hlstd{)}
\end{alltt}
\begin{verbatim}
## [1] 2
\end{verbatim}


{\ttfamily\noindent\bfseries\color{errorcolor}{\#\# Error in print(b): argument "{}b"{} is missing, with no default}}\end{kframe}
\end{knitrout}

Notice that \code{2} got printed before the error was triggered. This is because \code{b} did not have to be evaluated until after \code{print(a)}

\end{frame}

%------------------------------------------------

\begin{frame}
\begin{center}
\Huge{\hilit{Messages}}
\end{center}
\end{frame}

%------------------------------------------------

\begin{frame}[fragile]
\frametitle{}

There are two main functions for generating warnings and errors:
\bbi
  \item {\hilit \code{stop()}}
  \item {\hilit \code{warning()}}
\ei

\bigskip
{\footnotesize There's also the {\hilit \code{stopifnot()}} function}

\end{frame}

%------------------------------------------------

\begin{frame}[fragile]
\frametitle{Stop Execution}

Use {\hilit \code{stop()}} to stop the execution of a function (this will raise an error)
\begin{knitrout}\footnotesize
\definecolor{shadecolor}{rgb}{0.969, 0.969, 0.969}\color{fgcolor}\begin{kframe}
\begin{alltt}
\hlstd{meansd} \hlkwb{<-} \hlkwa{function}\hlstd{(}\hlkwc{x}\hlstd{,} \hlkwc{na.rm} \hlstd{=} \hlnum{FALSE}\hlstd{) \{}
  \hlkwa{if} \hlstd{(}\hlopt{!}\hlkwd{is.numeric}\hlstd{(x)) \{}
    \hlkwd{stop}\hlstd{(}\hlstr{"x is not numeric"}\hlstd{)}
  \hlstd{\}}
  \hlcom{# output}
  \hlkwd{c}\hlstd{(}\hlkwc{mean} \hlstd{=} \hlkwd{mean}\hlstd{(x,} \hlkwc{na.rm} \hlstd{= na.rm),}
    \hlkwc{sd} \hlstd{=} \hlkwd{sd}\hlstd{(x,} \hlkwc{na.rm} \hlstd{= na.rm))}
\hlstd{\}}
\end{alltt}
\end{kframe}
\end{knitrout}

\end{frame}

%------------------------------------------------

\begin{frame}[fragile]
\frametitle{Stop Execution}

Use {\hilit \code{stop()}} to stop the execution of a function (this will raise an error)
\begin{knitrout}\footnotesize
\definecolor{shadecolor}{rgb}{0.969, 0.969, 0.969}\color{fgcolor}\begin{kframe}
\begin{alltt}
\hlcom{# ok}
\hlkwd{meansd}\hlstd{(}\hlkwd{c}\hlstd{(}\hlnum{4}\hlstd{,} \hlnum{5}\hlstd{,} \hlnum{3}\hlstd{,} \hlnum{1}\hlstd{,} \hlnum{2}\hlstd{))}
\end{alltt}
\begin{verbatim}
##     mean       sd 
## 3.000000 1.581139
\end{verbatim}
\begin{alltt}
\hlcom{# this causes an error}
\hlkwd{meansd}\hlstd{(}\hlkwd{c}\hlstd{(}\hlstr{'a'}\hlstd{,} \hlstr{'b'}\hlstd{,} \hlstr{'c'}\hlstd{))}
\end{alltt}


{\ttfamily\noindent\bfseries\color{errorcolor}{\#\# Error in meansd(c("{}a"{}, "{}b"{}, "{}c"{})): x is not numeric}}\end{kframe}
\end{knitrout}

\end{frame}

%------------------------------------------------

\begin{frame}[fragile]
\frametitle{Warning Messages}

Use {\hilit \code{warning()}} to show a warning message
\begin{knitrout}\footnotesize
\definecolor{shadecolor}{rgb}{0.969, 0.969, 0.969}\color{fgcolor}\begin{kframe}
\begin{alltt}
\hlstd{meansd} \hlkwb{<-} \hlkwa{function}\hlstd{(}\hlkwc{x}\hlstd{,} \hlkwc{na.rm} \hlstd{=} \hlnum{FALSE}\hlstd{) \{}
  \hlkwa{if} \hlstd{(}\hlopt{!}\hlkwd{is.numeric}\hlstd{(x)) \{}
    \hlkwd{warning}\hlstd{(}\hlstr{"non-numeric input coerced to numeric"}\hlstd{)}
    \hlstd{x} \hlkwb{<-} \hlkwd{as.numeric}\hlstd{(x)}
  \hlstd{\}}
  \hlcom{# output}
  \hlkwd{c}\hlstd{(}\hlkwc{mean} \hlstd{=} \hlkwd{mean}\hlstd{(x,} \hlkwc{na.rm} \hlstd{= na.rm),}
    \hlkwc{sd} \hlstd{=} \hlkwd{sd}\hlstd{(x,} \hlkwc{na.rm} \hlstd{= na.rm))}
\hlstd{\}}
\end{alltt}
\end{kframe}
\end{knitrout}
A warning is useful when we don't want to stop the execution, but we still want to show potential problems

\end{frame}

%------------------------------------------------

\begin{frame}[fragile]
\frametitle{Warning Messages}

Use {\hilit \code{warning()}} to show a warning message
\begin{knitrout}\footnotesize
\definecolor{shadecolor}{rgb}{0.969, 0.969, 0.969}\color{fgcolor}\begin{kframe}
\begin{alltt}
\hlcom{# ok}
\hlkwd{meansd}\hlstd{(}\hlkwd{c}\hlstd{(}\hlnum{4}\hlstd{,} \hlnum{5}\hlstd{,} \hlnum{3}\hlstd{,} \hlnum{1}\hlstd{,} \hlnum{2}\hlstd{))}
\end{alltt}
\begin{verbatim}
##     mean       sd 
## 3.000000 1.581139
\end{verbatim}
\begin{alltt}
\hlcom{# this causes a warning}
\hlkwd{meansd}\hlstd{(}\hlkwd{c}\hlstd{(}\hlnum{TRUE}\hlstd{,} \hlnum{FALSE}\hlstd{,} \hlnum{TRUE}\hlstd{,} \hlnum{FALSE}\hlstd{))}
\end{alltt}


{\ttfamily\noindent\color{warningcolor}{\#\# Warning in meansd(c(TRUE, FALSE, TRUE, FALSE)): non-numeric input coerced to numeric}}\begin{verbatim}
##      mean        sd 
## 0.5000000 0.5773503
\end{verbatim}
\end{kframe}
\end{knitrout}

\end{frame}

%------------------------------------------------

\begin{frame}[fragile]
\frametitle{Stop Execution}

{\hilit \code{stopifnot()}} ensures the truth of expressions:
\begin{knitrout}\footnotesize
\definecolor{shadecolor}{rgb}{0.969, 0.969, 0.969}\color{fgcolor}\begin{kframe}
\begin{alltt}
\hlstd{meansd} \hlkwb{<-} \hlkwa{function}\hlstd{(}\hlkwc{x}\hlstd{,} \hlkwc{na.rm} \hlstd{=} \hlnum{FALSE}\hlstd{) \{}
  \hlkwd{stopifnot}\hlstd{(}\hlkwd{is.numeric}\hlstd{(x))}
  \hlcom{# output}
  \hlkwd{c}\hlstd{(}\hlkwc{mean} \hlstd{=} \hlkwd{mean}\hlstd{(x,} \hlkwc{na.rm} \hlstd{= na.rm),}
    \hlkwc{sd} \hlstd{=} \hlkwd{sd}\hlstd{(x,} \hlkwc{na.rm} \hlstd{= na.rm))}
\hlstd{\}}

\hlkwd{meansd}\hlstd{(}\hlstr{'hello'}\hlstd{)}
\end{alltt}


{\ttfamily\noindent\bfseries\color{errorcolor}{\#\# Error: is.numeric(x) is not TRUE}}\end{kframe}
\end{knitrout}

\end{frame}

%------------------------------------------------

\begin{frame}
\begin{center}
\Huge{\hilit{Environments and Functions}}
\end{center}
\end{frame}

%------------------------------------------------

\begin{frame}[fragile]
\frametitle{Consider this example}

\begin{knitrout}\footnotesize
\definecolor{shadecolor}{rgb}{0.969, 0.969, 0.969}\color{fgcolor}\begin{kframe}
\begin{alltt}
\hlstd{w} \hlkwb{<-} \hlnum{10}

\hlstd{f} \hlkwb{<-} \hlkwa{function}\hlstd{(}\hlkwc{y}\hlstd{) \{}
  \hlstd{d} \hlkwb{<-} \hlnum{5}
  \hlstd{h} \hlkwb{<-} \hlkwa{function}\hlstd{() \{}
    \hlstd{d} \hlopt{*} \hlstd{(w} \hlopt{+} \hlstd{y)}
  \hlstd{\}}
  \hlkwd{return}\hlstd{(}\hlkwd{h}\hlstd{())}
\hlstd{\}}

\hlkwd{f}\hlstd{(}\hlnum{2}\hlstd{)}
\end{alltt}
\begin{verbatim}
## [1] 60
\end{verbatim}
\end{kframe}
\end{knitrout}

How / Why does \code{f()} work?

\end{frame}

%------------------------------------------------

\begin{frame}[fragile]
\frametitle{Consider this other example}

\begin{knitrout}\footnotesize
\definecolor{shadecolor}{rgb}{0.969, 0.969, 0.969}\color{fgcolor}\begin{kframe}
\begin{alltt}
\hlstd{w} \hlkwb{<-} \hlnum{10}

\hlstd{f} \hlkwb{<-} \hlkwa{function}\hlstd{(}\hlkwc{y}\hlstd{) \{}
  \hlstd{d} \hlkwb{<-} \hlnum{5}
  \hlkwd{return}\hlstd{(}\hlkwd{h}\hlstd{())}
\hlstd{\}}

\hlkwd{f}\hlstd{(}\hlnum{2}\hlstd{)}
\end{alltt}


{\ttfamily\noindent\bfseries\color{errorcolor}{\#\# Error in f(2): could not find function "{}h"{}}}\end{kframe}
\end{knitrout}

Why \code{f()} does not work?

\end{frame}

%------------------------------------------------

\begin{frame}
\frametitle{Environments}

\bbi
  \item All the variables that we create need to be stored somewhere
  \item The place where they are stored is called an \textbf{environment}
  \item R works with enviroments, all of which are in (virtual) memory
  \item Usually, we don't need to explicitly deal with environments
  \item Environments are nested
\ei

\end{frame}

%------------------------------------------------

\begin{frame}[fragile]
\frametitle{Global Environment}

\bi
  \item The user workspace is the \textbf{global environment}
  \item The global environment is the \textbf{top level} environment
  \item It is formally referred to as \code{R\_GlobalEnv}
  \item Variables defined in the global environment can be seen from anywhere
  \item The contents of the global environment are listed with \code{ls()}
\ei

\begin{knitrout}\footnotesize
\definecolor{shadecolor}{rgb}{0.969, 0.969, 0.969}\color{fgcolor}\begin{kframe}
\begin{alltt}
\hlcom{# top level environment}
\hlkwd{environment}\hlstd{()}
\end{alltt}
\begin{verbatim}
## <environment: R_GlobalEnv>
\end{verbatim}
\end{kframe}
\end{knitrout}

\end{frame}

%------------------------------------------------

\begin{frame}
\frametitle{Searching objects}

\bi
  \item When R tries to bind a value to a symbol, it searches through a series of environments to find the appropriate value
  \item To retrieve the value of an object the order is:
  \item Search the current environment
  \item Search the global environment for a symbol name matching the one requested
  \item Search the namespaces of each of the packages on the search list: {\hilit \code{search()}}
\ei

\end{frame}

%------------------------------------------------

\begin{frame}
\frametitle{Environments and Functions}

\bbi
  \item A function consists not only of its arguments and body but also of its \textit{environment}
  \item An environment is made up of the collection of objects present at the time the function comes into existence
  \item When a function is created by evaluating the corresponding expression, the current environment is recorded as a property of the function
\ei

\end{frame}

%------------------------------------------------

\begin{frame}[fragile]
\frametitle{Let's go back to our first example}

\begin{knitrout}\footnotesize
\definecolor{shadecolor}{rgb}{0.969, 0.969, 0.969}\color{fgcolor}\begin{kframe}
\begin{alltt}
\hlstd{w} \hlkwb{<-} \hlnum{10}

\hlstd{f} \hlkwb{<-} \hlkwa{function}\hlstd{(}\hlkwc{y}\hlstd{) \{}
  \hlstd{d} \hlkwb{<-} \hlnum{5}
  \hlstd{h} \hlkwb{<-} \hlkwa{function}\hlstd{() \{}
    \hlstd{d} \hlopt{*} \hlstd{(w} \hlopt{+} \hlstd{y)}
  \hlstd{\}}
  \hlkwd{return}\hlstd{(}\hlkwd{h}\hlstd{())}
\hlstd{\}}

\hlkwd{f}\hlstd{(}\hlnum{2}\hlstd{)}
\end{alltt}
\begin{verbatim}
## [1] 60
\end{verbatim}
\end{kframe}
\end{knitrout}

How does \code{f()} work?

\end{frame}

%------------------------------------------------

\begin{frame}[fragile]
\frametitle{Let's see the environments}

\begin{knitrout}\footnotesize
\definecolor{shadecolor}{rgb}{0.969, 0.969, 0.969}\color{fgcolor}\begin{kframe}
\begin{alltt}
\hlstd{w} \hlkwb{<-} \hlnum{10}   \hlcom{# variable (in global environment)}

\hlcom{# a function (in global environment)}
\hlstd{f} \hlkwb{<-} \hlkwa{function}\hlstd{(}\hlkwc{y}\hlstd{) \{}
  \hlstd{d} \hlkwb{<-} \hlnum{5}     \hlcom{# local variable}
  \hlstd{h} \hlkwb{<-} \hlkwa{function}\hlstd{() \{}    \hlcom{# subfunction}
    \hlstd{d} \hlopt{*} \hlstd{(w} \hlopt{+} \hlstd{y)}     \hlcom{# w is a free variable}
  \hlstd{\}}
  \hlkwd{return}\hlstd{(}\hlkwd{h}\hlstd{())}
\hlstd{\}}

\hlkwd{environment}\hlstd{(f)}
\end{alltt}
\begin{verbatim}
## <environment: R_GlobalEnv>
\end{verbatim}
\end{kframe}
\end{knitrout}

\end{frame}

%------------------------------------------------

\begin{frame}[fragile]
\frametitle{Function Environment}

\bi
  \item \code{w} is a global variable (in global environment)
  \item \code{f()} is a function in the global environment
  \item \code{d} is a local variable---local to \code{f()}
  \item \code{h()} is a subfunction---local to \code{f()}
  \item \code{w} is not an argument but a free variable
\ei

\end{frame}

%------------------------------------------------

\begin{frame}[fragile]
\frametitle{Let's see the environments}

\begin{knitrout}\footnotesize
\definecolor{shadecolor}{rgb}{0.969, 0.969, 0.969}\color{fgcolor}\begin{kframe}
\begin{alltt}
\hlstd{f} \hlkwb{<-} \hlkwa{function}\hlstd{(}\hlkwc{y}\hlstd{) \{}
  \hlstd{d} \hlkwb{<-} \hlnum{5}
  \hlstd{h} \hlkwb{<-} \hlkwa{function}\hlstd{() \{}
    \hlstd{d} \hlopt{*} \hlstd{(w} \hlopt{+} \hlstd{y)}
  \hlstd{\}}
  \hlkwd{print}\hlstd{(}\hlkwd{environment}\hlstd{(h))}   \hlcom{# h()'s environment}
  \hlkwd{return}\hlstd{(}\hlkwd{h}\hlstd{())}
\hlstd{\}}

\hlkwd{environment}\hlstd{(f)}
\end{alltt}
\begin{verbatim}
## <environment: R_GlobalEnv>
\end{verbatim}
\begin{alltt}
\hlkwd{f}\hlstd{(}\hlnum{2}\hlstd{)}
\end{alltt}
\begin{verbatim}
## <environment: 0x7fb252b17d80>
## [1] 60
\end{verbatim}
\end{kframe}
\end{knitrout}

\end{frame}

%------------------------------------------------

\begin{frame}[fragile]
\frametitle{Your turn}

\begin{knitrout}\footnotesize
\definecolor{shadecolor}{rgb}{0.969, 0.969, 0.969}\color{fgcolor}\begin{kframe}
\begin{alltt}
\hlcom{# When executed, what does g() return?}
\hlstd{x} \hlkwb{<-} \hlnum{5}

\hlstd{g} \hlkwb{<-} \hlkwa{function}\hlstd{(}\hlkwc{x} \hlstd{=} \hlnum{FALSE}\hlstd{) \{}
    \hlstd{y} \hlkwb{<-} \hlnum{10}
    \hlkwd{list}\hlstd{(}\hlkwc{x} \hlstd{= x,} \hlkwc{y} \hlstd{= y)}
\hlstd{\}}
\hlkwd{g}\hlstd{()}
\end{alltt}
\end{kframe}
\end{knitrout}
\bi
  \item[A)] \code{x = 5, y = 10}
  \item[B)] \code{x = 0, y = 10}
  \item[C)] \code{x = FALSE, y = 10}
  \item[D)] \code{x = FALSE, y = 5}
\ei

\end{frame}

%------------------------------------------------

\begin{frame}
\frametitle{Variable's Scope}

\bi
  \item A variable's \textbf{scope} is the set of places from which you can see the variable
  \item R will try to find a variable in the current environment
  \item If it doesn't find them it will look in the parent environment
  \item And then that environment's parent
  \item And so on until it reaches the global environment
\ei

\end{frame}

%------------------------------------------------

\begin{frame}
\frametitle{Variable's Scope}

\bi
  \item A variable's \textbf{scope} is the set of places from which you can see the variable
  \item R will try to find a variable in the current environment
  \item If it doesn't find them it will look in the parent environment
  \item And then that environment's parent
  \item And so on until it reaches the global environment
\ei

\end{frame}

%------------------------------------------------

\begin{frame}
\frametitle{Variable Scope}

\bbi
  \item When we define a variable inside a function, the rest of the statements in that function will have access to that variable
\ei

\end{frame}

%------------------------------------------------

\begin{frame}[fragile]
\frametitle{Variable Scope}

\begin{knitrout}\footnotesize
\definecolor{shadecolor}{rgb}{0.969, 0.969, 0.969}\color{fgcolor}\begin{kframe}
\begin{alltt}
\hlstd{f} \hlkwb{<-} \hlkwa{function}\hlstd{(}\hlkwc{x}\hlstd{) \{}
  \hlstd{y} \hlkwb{<-} \hlnum{1}
  \hlstd{g} \hlkwb{<-} \hlkwa{function}\hlstd{(}\hlkwc{x}\hlstd{) \{}
    \hlstd{(x} \hlopt{+} \hlstd{y)} \hlopt{/} \hlnum{2}
  \hlstd{\}}
  \hlkwd{g}\hlstd{(x)}
\hlstd{\}}

\hlkwd{f}\hlstd{(}\hlnum{5}\hlstd{)}
\end{alltt}
\begin{verbatim}
## [1] 3
\end{verbatim}
\end{kframe}
\end{knitrout}

\code{g()} is a subfunction that have access to \code{y} in \code{f}'s environment.

\end{frame}

%------------------------------------------------

\begin{frame}[fragile]
\frametitle{Variable Scope}

\begin{knitrout}\footnotesize
\definecolor{shadecolor}{rgb}{0.969, 0.969, 0.969}\color{fgcolor}\begin{kframe}
\begin{alltt}
\hlstd{f} \hlkwb{<-} \hlkwa{function}\hlstd{(}\hlkwc{x}\hlstd{) \{}
  \hlstd{y} \hlkwb{<-} \hlnum{1}
  \hlkwd{g}\hlstd{(x)}
\hlstd{\}}

\hlstd{g} \hlkwb{<-} \hlkwa{function}\hlstd{(}\hlkwc{x}\hlstd{) \{}
  \hlstd{(x} \hlopt{+} \hlstd{y)} \hlopt{/} \hlnum{2}
\hlstd{\}}

\hlkwd{f}\hlstd{(}\hlnum{5}\hlstd{)}
\end{alltt}


{\ttfamily\noindent\bfseries\color{errorcolor}{\#\# Error in g(x): object 'y' not found}}\end{kframe}
\end{knitrout}

\code{g()} is a function that doesn't have access to \code{y}; \code{g()} can only see things in the global environment

\end{frame}

%------------------------------------------------

\begin{frame}[fragile]
\frametitle{One more thing ...}

Let's look at another exmaple
\begin{knitrout}\footnotesize
\definecolor{shadecolor}{rgb}{0.969, 0.969, 0.969}\color{fgcolor}\begin{kframe}
\begin{alltt}
\hlkwd{mean}\hlstd{(}\hlnum{1}\hlopt{:}\hlnum{5}\hlstd{)}
\end{alltt}
\begin{verbatim}
## [1] 3
\end{verbatim}
\begin{alltt}
\hlstd{mean}
\end{alltt}
\begin{verbatim}
## function (x, ...) 
## UseMethod("mean")
## <bytecode: 0x7fb254ba7150>
## <environment: namespace:base>
\end{verbatim}
\end{kframe}
\end{knitrout}

\end{frame}

%------------------------------------------------

\begin{frame}[fragile]
\frametitle{One more thing ...}

You can do things like this
\begin{knitrout}\footnotesize
\definecolor{shadecolor}{rgb}{0.969, 0.969, 0.969}\color{fgcolor}\begin{kframe}
\begin{alltt}
\hlcom{# confusing but it works}
\hlstd{mean} \hlkwb{<-} \hlnum{1}\hlopt{:}\hlnum{5}
\hlkwd{mean}\hlstd{(mean)}
\end{alltt}
\begin{verbatim}
## [1] 3
\end{verbatim}
\end{kframe}
\end{knitrout}

\end{frame}

%------------------------------------------------

\begin{frame}[fragile]
\frametitle{Some issues}

You can also do things like this
\begin{knitrout}\footnotesize
\definecolor{shadecolor}{rgb}{0.969, 0.969, 0.969}\color{fgcolor}\begin{kframe}
\begin{alltt}
\hlcom{# not a good idea but you can do it}
\hlstd{mean} \hlkwb{<-} \hlkwa{function}\hlstd{(}\hlkwc{x}\hlstd{)} \hlnum{2}\hlopt{*}\hlstd{x} \hlopt{+} \hlnum{5}

\hlkwd{mean}\hlstd{(}\hlnum{1}\hlopt{:}\hlnum{5}\hlstd{)}
\end{alltt}
\begin{verbatim}
## [1]  7  9 11 13 15
\end{verbatim}
\end{kframe}
\end{knitrout}

It seems we've lost the original \code{mean()} function

\end{frame}

%------------------------------------------------

\begin{frame}[fragile]
\frametitle{The \code{::} Operator}

\code{::} operator to the rescue
\begin{knitrout}\footnotesize
\definecolor{shadecolor}{rgb}{0.969, 0.969, 0.969}\color{fgcolor}\begin{kframe}
\begin{alltt}
\hlcom{# my mean}
\hlkwd{mean}\hlstd{(}\hlnum{1}\hlopt{:}\hlnum{5}\hlstd{)}
\end{alltt}
\begin{verbatim}
## [1]  7  9 11 13 15
\end{verbatim}
\begin{alltt}
\hlcom{# base mean}
\hlstd{base}\hlopt{::}\hlkwd{mean}\hlstd{(}\hlnum{1}\hlopt{:}\hlnum{5}\hlstd{)}
\end{alltt}
\begin{verbatim}
## [1] 3
\end{verbatim}
\end{kframe}
\end{knitrout}

Here we use the name espace \code{base} of the R package \code{"base"} to access the original \code{mean()}

\end{frame}

%------------------------------------------------

\begin{frame}[fragile]
\frametitle{Recovering original \code{mean()}}

To recover the original \code{mean()} you have to remove the artificial \code{mean}
\begin{knitrout}\footnotesize
\definecolor{shadecolor}{rgb}{0.969, 0.969, 0.969}\color{fgcolor}\begin{kframe}
\begin{alltt}
\hlcom{# remove my mean}
\hlkwd{rm}\hlstd{(mean)}

\hlcom{# now try it again}
\hlkwd{mean}\hlstd{(}\hlnum{1}\hlopt{:}\hlnum{5}\hlstd{)}
\end{alltt}
\begin{verbatim}
## [1] 3
\end{verbatim}
\end{kframe}
\end{knitrout}

\end{frame}

%------------------------------------------------

\begin{frame}
\begin{center}
\Huge{\hilit{Your Turn}}
\end{center}
\end{frame}

%------------------------------------------------

\begin{frame}[fragile]
\frametitle{Exercise}

R has a function \code{summary()} that when applied on a numeric vector provides something like this:
\begin{knitrout}\footnotesize
\definecolor{shadecolor}{rgb}{0.969, 0.969, 0.969}\color{fgcolor}\begin{kframe}
\begin{alltt}
\hlkwd{summary}\hlstd{(}\hlnum{1}\hlopt{:}\hlnum{10}\hlstd{)}
\end{alltt}
\begin{verbatim}
##    Min. 1st Qu.  Median    Mean 3rd Qu.    Max. 
##    1.00    3.25    5.50    5.50    7.75   10.00
\end{verbatim}
\end{kframe}
\end{knitrout}

Create a \code{describe()} function that takes a numeric vector and returns: minimum, maximum, mean, and standard deviation

\end{frame}

%------------------------------------------------

\begin{frame}[fragile]
\frametitle{Exercise}

First attempt
\begin{knitrout}\footnotesize
\definecolor{shadecolor}{rgb}{0.969, 0.969, 0.969}\color{fgcolor}\begin{kframe}
\begin{alltt}
\hlstd{describe} \hlkwb{<-} \hlkwa{function}\hlstd{(}\hlkwc{x}\hlstd{) \{}
  \hlstd{x_min} \hlkwb{<-} \hlkwd{min}\hlstd{(x)}
  \hlstd{x_max} \hlkwb{<-} \hlkwd{max}\hlstd{(x)}
  \hlstd{x_mean} \hlkwb{<-} \hlkwd{mean}\hlstd{(x)}
  \hlstd{x_sd} \hlkwb{<-} \hlkwd{sd}\hlstd{(x)}
  \hlkwd{return}\hlstd{(}\hlkwd{c}\hlstd{(x_min, x_max, x_mean, x_sd))}
\hlstd{\}}

\hlkwd{describe}\hlstd{(}\hlnum{1}\hlopt{:}\hlnum{10}\hlstd{)}
\end{alltt}
\begin{verbatim}
## [1]  1.00000 10.00000  5.50000  3.02765
\end{verbatim}
\end{kframe}
\end{knitrout}

\end{frame}

%------------------------------------------------

\begin{frame}[fragile]
\frametitle{Exercise}

Second attempt (adding names)
\begin{knitrout}\footnotesize
\definecolor{shadecolor}{rgb}{0.969, 0.969, 0.969}\color{fgcolor}\begin{kframe}
\begin{alltt}
\hlstd{describe} \hlkwb{<-} \hlkwa{function}\hlstd{(}\hlkwc{x}\hlstd{) \{}
  \hlstd{x_min} \hlkwb{<-} \hlkwd{min}\hlstd{(x)}
  \hlstd{x_max} \hlkwb{<-} \hlkwd{max}\hlstd{(x)}
  \hlstd{x_mean} \hlkwb{<-} \hlkwd{mean}\hlstd{(x)}
  \hlstd{x_sd} \hlkwb{<-} \hlkwd{sd}\hlstd{(x)}
  \hlstd{values} \hlkwb{<-} \hlkwd{c}\hlstd{(x_min, x_max, x_mean, x_sd)}
  \hlkwd{names}\hlstd{(values)} \hlkwb{<-} \hlkwd{c}\hlstd{(}\hlstr{"min"}\hlstd{,} \hlstr{"max"}\hlstd{,} \hlstr{"mean"}\hlstd{,} \hlstr{"sd"}\hlstd{)}
  \hlkwd{return}\hlstd{(values)}
\hlstd{\}}

\hlkwd{describe}\hlstd{(}\hlnum{1}\hlopt{:}\hlnum{10}\hlstd{)}
\end{alltt}
\begin{verbatim}
##      min      max     mean       sd 
##  1.00000 10.00000  5.50000  3.02765
\end{verbatim}
\end{kframe}
\end{knitrout}

\end{frame}

%------------------------------------------------

\begin{frame}[fragile]
\frametitle{Exercise}

Third attempt (using a list as output)

\begin{columns}[t]
\begin{column}{0.45\textwidth}
\begin{knitrout}\footnotesize
\definecolor{shadecolor}{rgb}{0.969, 0.969, 0.969}\color{fgcolor}\begin{kframe}
\begin{alltt}
\hlstd{describe} \hlkwb{<-} \hlkwa{function}\hlstd{(}\hlkwc{x}\hlstd{) \{}
  \hlkwd{list}\hlstd{(}
    \hlkwc{min} \hlstd{=} \hlkwd{min}\hlstd{(x),}
    \hlkwc{max} \hlstd{=} \hlkwd{max}\hlstd{(x),}
    \hlkwc{mean} \hlstd{=} \hlkwd{mean}\hlstd{(x),}
    \hlkwc{sd} \hlstd{=} \hlkwd{sd}\hlstd{(x)}
  \hlstd{)}
\hlstd{\}}
\end{alltt}
\end{kframe}
\end{knitrout}
\end{column}

\begin{column}{0.4\textwidth}
\begin{knitrout}\footnotesize
\definecolor{shadecolor}{rgb}{0.969, 0.969, 0.969}\color{fgcolor}\begin{kframe}
\begin{alltt}
\hlkwd{describe}\hlstd{(}\hlnum{1}\hlopt{:}\hlnum{10}\hlstd{)}
\end{alltt}
\begin{verbatim}
## $min
## [1] 1
## 
## $max
## [1] 10
## 
## $mean
## [1] 5.5
## 
## $sd
## [1] 3.02765
\end{verbatim}
\end{kframe}
\end{knitrout}
\end{column}
\end{columns}

\end{frame}

%------------------------------------------------

\begin{frame}[fragile]
\frametitle{Exercise}

Probability Density of the Normal Distribution:
$$
f(x | \mu, \sigma) = \frac{1}{\sigma \sqrt{2 \pi}} e^{- \frac{(x - \mu)^2}{2 \sigma^2}}
$$

Write a function that takes a value $x$ (with parameters $\mu$ and $\sigma$) which computes the probability density distribution of the normal distribution

\end{frame}

%------------------------------------------------

\begin{frame}[fragile]
\frametitle{Exercise}

Normal Distribution:
\begin{knitrout}\footnotesize
\definecolor{shadecolor}{rgb}{0.969, 0.969, 0.969}\color{fgcolor}\begin{kframe}
\begin{alltt}
\hlstd{normal_dist} \hlkwb{<-} \hlkwa{function}\hlstd{(}\hlkwc{x}\hlstd{,} \hlkwc{mu} \hlstd{=} \hlnum{0}\hlstd{,} \hlkwc{sigma} \hlstd{=} \hlnum{1}\hlstd{) \{}
  \hlstd{constant} \hlkwb{<-} \hlnum{1} \hlopt{/} \hlstd{(sigma} \hlopt{*} \hlkwd{sqrt}\hlstd{(}\hlnum{2}\hlopt{*}\hlstd{pi))}
  \hlstd{constant} \hlopt{*} \hlkwd{exp}\hlstd{(}\hlopt{-}\hlstd{((x} \hlopt{-} \hlstd{mu)}\hlopt{^}\hlnum{2}\hlstd{)} \hlopt{/} \hlstd{(}\hlnum{2} \hlopt{*} \hlstd{sigma}\hlopt{^}\hlnum{2}\hlstd{))}
\hlstd{\}}

\hlkwd{normal_dist}\hlstd{(}\hlnum{2}\hlstd{)}
\end{alltt}
\begin{verbatim}
## [1] 0.05399097
\end{verbatim}
\end{kframe}
\end{knitrout}

\end{frame}

%------------------------------------------------

\begin{frame}[fragile]
\frametitle{Argument Matching}

\begin{knitrout}\footnotesize
\definecolor{shadecolor}{rgb}{0.969, 0.969, 0.969}\color{fgcolor}\begin{kframe}
\begin{alltt}
\hlstd{normal_dist} \hlkwb{<-} \hlkwa{function}\hlstd{(}\hlkwc{x}\hlstd{,} \hlkwc{mu} \hlstd{=} \hlnum{0}\hlstd{,} \hlkwc{sigma} \hlstd{=} \hlnum{1}\hlstd{) \{}
  \hlstd{constant} \hlkwb{<-} \hlnum{1} \hlopt{/} \hlstd{(sigma} \hlopt{*} \hlkwd{sqrt}\hlstd{(}\hlnum{2}\hlopt{*}\hlstd{pi))}
  \hlstd{constant} \hlopt{*} \hlkwd{exp}\hlstd{(}\hlopt{-}\hlstd{((x} \hlopt{-} \hlstd{mu)}\hlopt{^}\hlnum{2}\hlstd{)} \hlopt{/} \hlstd{(}\hlnum{2} \hlopt{*} \hlstd{sigma}\hlopt{^}\hlnum{2}\hlstd{))}
\hlstd{\}}

\hlkwd{normal_dist}\hlstd{(}\hlnum{2}\hlstd{)}
\hlkwd{normal_dist}\hlstd{(}\hlnum{2}\hlstd{,} \hlkwc{sigma} \hlstd{=} \hlnum{3}\hlstd{,} \hlkwc{mu} \hlstd{=} \hlnum{1}\hlstd{)}
\hlkwd{normal_dist}\hlstd{(}\hlkwc{mu} \hlstd{=} \hlnum{1}\hlstd{,} \hlkwc{sigma} \hlstd{=} \hlnum{3}\hlstd{,} \hlnum{2}\hlstd{)}
\hlkwd{normal_dist}\hlstd{(}\hlkwc{mu} \hlstd{=} \hlnum{1}\hlstd{,} \hlnum{2}\hlstd{,} \hlkwc{sigma} \hlstd{=} \hlnum{3}\hlstd{)}
\end{alltt}
\end{kframe}
\end{knitrout}

\end{frame}

%------------------------------------------------

\begin{frame}[fragile]
\frametitle{Argument Matching}

R is ``smart'' enough in doing pattern matching with arguments' names (not recommended though)
\begin{knitrout}\footnotesize
\definecolor{shadecolor}{rgb}{0.969, 0.969, 0.969}\color{fgcolor}\begin{kframe}
\begin{alltt}
\hlkwd{normal_dist}\hlstd{(}\hlnum{2}\hlstd{)}
\end{alltt}
\begin{verbatim}
## [1] 0.05399097
\end{verbatim}
\begin{alltt}
\hlkwd{normal_dist}\hlstd{(}\hlnum{2}\hlstd{,} \hlkwc{m} \hlstd{=} \hlnum{0}\hlstd{,} \hlkwc{s} \hlstd{=} \hlnum{1}\hlstd{)}
\end{alltt}
\begin{verbatim}
## [1] 0.05399097
\end{verbatim}
\begin{alltt}
\hlkwd{normal_dist}\hlstd{(}\hlnum{2}\hlstd{,} \hlkwc{sig} \hlstd{=} \hlnum{1}\hlstd{,} \hlkwc{m} \hlstd{=} \hlnum{0}\hlstd{)}
\end{alltt}
\begin{verbatim}
## [1] 0.05399097
\end{verbatim}
\end{kframe}
\end{knitrout}

\end{frame}

%------------------------------------------------

\end{document}
