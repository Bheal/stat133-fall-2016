\documentclass[12pt]{beamer}\usepackage[]{graphicx}\usepackage[]{color}
%% maxwidth is the original width if it is less than linewidth
%% otherwise use linewidth (to make sure the graphics do not exceed the margin)
\makeatletter
\def\maxwidth{ %
  \ifdim\Gin@nat@width>\linewidth
    \linewidth
  \else
    \Gin@nat@width
  \fi
}
\makeatother

\definecolor{fgcolor}{rgb}{0.345, 0.345, 0.345}
\newcommand{\hlnum}[1]{\textcolor[rgb]{0.686,0.059,0.569}{#1}}%
\newcommand{\hlstr}[1]{\textcolor[rgb]{0.192,0.494,0.8}{#1}}%
\newcommand{\hlcom}[1]{\textcolor[rgb]{0.678,0.584,0.686}{\textit{#1}}}%
\newcommand{\hlopt}[1]{\textcolor[rgb]{0,0,0}{#1}}%
\newcommand{\hlstd}[1]{\textcolor[rgb]{0.345,0.345,0.345}{#1}}%
\newcommand{\hlkwa}[1]{\textcolor[rgb]{0.161,0.373,0.58}{\textbf{#1}}}%
\newcommand{\hlkwb}[1]{\textcolor[rgb]{0.69,0.353,0.396}{#1}}%
\newcommand{\hlkwc}[1]{\textcolor[rgb]{0.333,0.667,0.333}{#1}}%
\newcommand{\hlkwd}[1]{\textcolor[rgb]{0.737,0.353,0.396}{\textbf{#1}}}%
\let\hlipl\hlkwb

\usepackage{framed}
\makeatletter
\newenvironment{kframe}{%
 \def\at@end@of@kframe{}%
 \ifinner\ifhmode%
  \def\at@end@of@kframe{\end{minipage}}%
  \begin{minipage}{\columnwidth}%
 \fi\fi%
 \def\FrameCommand##1{\hskip\@totalleftmargin \hskip-\fboxsep
 \colorbox{shadecolor}{##1}\hskip-\fboxsep
     % There is no \\@totalrightmargin, so:
     \hskip-\linewidth \hskip-\@totalleftmargin \hskip\columnwidth}%
 \MakeFramed {\advance\hsize-\width
   \@totalleftmargin\z@ \linewidth\hsize
   \@setminipage}}%
 {\par\unskip\endMakeFramed%
 \at@end@of@kframe}
\makeatother

\definecolor{shadecolor}{rgb}{.97, .97, .97}
\definecolor{messagecolor}{rgb}{0, 0, 0}
\definecolor{warningcolor}{rgb}{1, 0, 1}
\definecolor{errorcolor}{rgb}{1, 0, 0}
\newenvironment{knitrout}{}{} % an empty environment to be redefined in TeX

\usepackage{alltt}
\usepackage{graphicx}
\setbeameroption{hide notes}
\setbeamertemplate{note page}[plain]
\usepackage{listings}

% get rid of junk
\usetheme{default}
\usefonttheme[onlymath]{serif}
\beamertemplatenavigationsymbolsempty
\hypersetup{pdfpagemode=UseNone} % don't show bookmarks on initial view

% named colors
\definecolor{offwhite}{RGB}{255,250,240}
\definecolor{gray}{RGB}{155,155,155}

\ifx\notescolors\undefined % slides

  \definecolor{foreground}{RGB}{80,80,80}
  \definecolor{background}{RGB}{255,255,255}
  \definecolor{title}{RGB}{255,199,0}
  \definecolor{subtitle}{RGB}{89,132,212}
  \definecolor{hilit}{RGB}{248,117,79}
  \definecolor{vhilit}{RGB}{255,111,207}
  \definecolor{lolit}{RGB}{200,200,200}
  \definecolor{lit}{RGB}{255,199,0}
  \definecolor{mdlit}{RGB}{89,132,212}
  \definecolor{link}{RGB}{248,117,79}

\else % notes
  \definecolor{background}{RGB}{255,255,255}
  \definecolor{foreground}{RGB}{24,24,24}
  \definecolor{title}{RGB}{27,94,134}
  \definecolor{subtitle}{RGB}{22,175,124}
  \definecolor{hilit}{RGB}{122,0,128}
  \definecolor{vhilit}{RGB}{255,0,128}
  \definecolor{lolit}{RGB}{95,95,95}
\fi
\definecolor{nhilit}{RGB}{128,0,128}  % hilit color in notes
\definecolor{nvhilit}{RGB}{255,0,128} % vhilit for notes

\newcommand{\hilit}{\color{hilit}}
\newcommand{\vhilit}{\color{vhilit}}
\newcommand{\nhilit}{\color{nhilit}}
\newcommand{\nvhilit}{\color{nvhilit}}
\newcommand{\lit}{\color{lit}}
\newcommand{\mdlit}{\color{mdlit}}
\newcommand{\lolit}{\color{lolit}}

% use those colors
\setbeamercolor{titlelike}{fg=title}
\setbeamercolor{subtitle}{fg=subtitle}
\setbeamercolor{frametitle}{fg=gray}
\setbeamercolor{structure}{fg=subtitle}
\setbeamercolor{institute}{fg=lolit}
\setbeamercolor{normal text}{fg=foreground,bg=background}
%\setbeamercolor{item}{fg=foreground} % color of bullets
%\setbeamercolor{subitem}{fg=hilit}
%\setbeamercolor{itemize/enumerate subbody}{fg=lolit}
\setbeamertemplate{itemize subitem}{{\textendash}}
\setbeamerfont{itemize/enumerate subbody}{size=\footnotesize}
\setbeamerfont{itemize/enumerate subitem}{size=\footnotesize}

% center title of slides
\setbeamertemplate{blocks}[rounded]
\setbeamertemplate{frametitle}[default][center]
% margins
\setbeamersize{text margin left=25pt,text margin right=25pt}

% page number
\setbeamertemplate{footline}{%
    \raisebox{5pt}{\makebox[\paperwidth]{\hfill\makebox[20pt]{\lolit
          \scriptsize\insertframenumber}}}\hspace*{5pt}}

% add a bit of space at the top of the notes page
\addtobeamertemplate{note page}{\setlength{\parskip}{12pt}}

% default link color
\hypersetup{colorlinks, urlcolor={link}}

\ifx\notescolors\undefined % slides
  % set up listing environment
  \lstset{language=bash,
          basicstyle=\ttfamily\scriptsize,
          frame=single,
          commentstyle=,
          backgroundcolor=\color{darkgray},
          showspaces=false,
          showstringspaces=false
          }
\else % notes
  \lstset{language=bash,
          basicstyle=\ttfamily\scriptsize,
          frame=single,
          commentstyle=,
          backgroundcolor=\color{offwhite},
          showspaces=false,
          showstringspaces=false
          }
\fi

% a few macros
\newcommand{\code}[1]{\texttt{#1}}
\newcommand{\hicode}[1]{{\hilit \texttt{#1}}}
\newcommand{\bb}[1]{\begin{block}{#1}}
\newcommand{\eb}{\end{block}}
\newcommand{\bi}{\begin{itemize}}
%\newcommand{\bbi}{\vspace{24pt} \begin{itemize} \itemsep8pt}
\newcommand{\bbi}{\vspace{4pt} \begin{itemize} \itemsep8pt}
\newcommand{\ei}{\end{itemize}}
\newcommand{\bv}{\begin{verbatim}}
\newcommand{\ev}{\end{verbatim}}
\newcommand{\ig}{\includegraphics}
\newcommand{\subt}[1]{{\footnotesize \color{subtitle} {#1}}}
\newcommand{\ttsm}{\tt \small}
\newcommand{\ttfn}{\tt \footnotesize}
\newcommand{\figh}[2]{\centerline{\includegraphics[height=#2\textheight]{#1}}}
\newcommand{\figw}[2]{\centerline{\includegraphics[width=#2\textwidth]{#1}}}



%------------------------------------------------
% end of header
%------------------------------------------------

\title{Good Habits in R Programming}
\subtitle{STAT 133}
\author{\href{http://www.gastonsanchez.com}{Gaston Sanchez}}
\institute{\href{https://github.com/ucb-stat133/stat133-fall-2016}{\tt \scriptsize \color{foreground} github.com/ucb-stat133/stat133-fall-2016}}
\date{}
\IfFileExists{upquote.sty}{\usepackage{upquote}}{}
\begin{document}


{
  \setbeamertemplate{footline}{} % no page number here
  \frame{
    \titlepage
  } 
}

%------------------------------------------------

\begin{frame}
\begin{center}
\Huge{\hilit{Good Coding Habits}}
\end{center}
\end{frame}

%------------------------------------------------

\begin{frame}
\frametitle{Code Habits}

Now that you've worked with various R scripts, written some functions, and done some data manipulation, it's time to look at some good coding practices.

\end{frame}

%------------------------------------------------

\begin{frame}
\frametitle{Code Habits}

Popular style guides among useR's
\bbi
  \item \url{https://google-styleguide.googlecode.com/svn/trunk/Rguide.xml}
  \item \url{http://adv-r.had.co.nz/Style.html}
\ei

\end{frame}

%------------------------------------------------

\begin{frame}[fragile]
\frametitle{}
\begin{center}
\ig[width=10cm]{images/google-style.png}
\end{center}
\end{frame}

%------------------------------------------------

\begin{frame}[fragile]
\frametitle{}
\begin{center}
\ig[width=10cm]{images/hadley-style.png}
\end{center}
\end{frame}

%------------------------------------------------

\begin{frame}
\frametitle{Editor}

\bb{Text Editor}
\bbi
 \item Text editor $\neq$ word processor
 \item Use a good text editor
 \item e.g. vim, sublime text, text wrangler, notepad, etc
 \item With syntax highlighting
 \item Or use an Integrated Development Environment (IDE) like RStudio
\ei
\eb

\end{frame}

%------------------------------------------------

\begin{frame}[fragile]
\frametitle{Without Syntax Highlighting}

\begin{verbatim}
a <- 2
x <- 3
y <- log(sqrt(x))
3*x^7 - pi * x / (y - a)
"some strings"
dat <- read.table(file = 'data.csv', header = TRUE)
\end{verbatim}

\end{frame}

%------------------------------------------------

\begin{frame}[fragile]
\frametitle{Syntax Highlighting}

\begin{knitrout}
\definecolor{shadecolor}{rgb}{0.969, 0.969, 0.969}\color{fgcolor}\begin{kframe}
\begin{alltt}
\hlstd{a} \hlkwb{<-} \hlnum{2}
\hlstd{x} \hlkwb{<-} \hlnum{3}
\hlstd{y} \hlkwb{<-} \hlkwd{log}\hlstd{(}\hlkwd{sqrt}\hlstd{(x))}
\hlnum{3}\hlopt{*}\hlstd{x}\hlopt{^}\hlnum{7} \hlopt{-} \hlstd{pi} \hlopt{*} \hlstd{x} \hlopt{/} \hlstd{(y} \hlopt{-} \hlstd{a)}
\hlstr{"some strings"}
\hlstd{dat} \hlkwb{<-} \hlkwd{read.table}\hlstd{(}\hlkwc{file} \hlstd{=} \hlstr{'data.csv'}\hlstd{,} \hlkwc{header} \hlstd{=} \hlnum{TRUE}\hlstd{)}
\end{alltt}
\end{kframe}
\end{knitrout}

\end{frame}

%------------------------------------------------

\begin{frame}[fragile]
\frametitle{Syntax Highlight}

{\mdlit Without highlighting it's harder to detect syntax errors:}
\begin{verbatim}
numbers <- c("one", "two, "three")

if (x > 0) {
  3 * x + 19
} esle {
  2 * x - 20
}
\end{verbatim}

\end{frame}

%------------------------------------------------

\begin{frame}[fragile]
\frametitle{Syntax Highlight}

{\hilit With highlighting it's easier to detect syntax errors:}
\begin{knitrout}\footnotesize
\definecolor{shadecolor}{rgb}{0.969, 0.969, 0.969}\color{fgcolor}\begin{kframe}
\begin{alltt}
numbers <- \hlkwd{c}(\hlstr{"one"}, \hlstr{"two, "}three")
\end{alltt}
\end{kframe}
\end{knitrout}

\begin{knitrout}\footnotesize
\definecolor{shadecolor}{rgb}{0.969, 0.969, 0.969}\color{fgcolor}\begin{kframe}
\begin{alltt}
\hlkwd{if} (x > 0) \{
  3 * x + 19
\} esle \{
  2 * x - 20
\}
\end{alltt}
\end{kframe}
\end{knitrout}



\end{frame}

%------------------------------------------------

\begin{frame}[fragile]
\frametitle{Your Turn}

{\mdlit Which instruction is free of errors}
\begin{verbatim}
A) mean(numbers, na.mr = TRUE)

B) read.table(~/Documents/rawdata.txt, sep = '\t')

C) barplot(x, horiz = TURE)

D) matrix(1:12, nrow = 3, ncol = 4)
\end{verbatim}

\end{frame}

%------------------------------------------------

\begin{frame}[fragile]
\frametitle{Use an IDE}

\bi
  \item Syntax highlighting
  \item Syntax aware 
  \item Able to evaluate R code
  \bi
    \item by line
    \item by selection
    \item entire file
  \ei
  \item Command completion
\ei

\end{frame}

%------------------------------------------------

\begin{frame}[fragile]
\frametitle{Use an IDE}

Use an IDE with autocompletion
\begin{center}
\ig[width=10cm]{images/rstudio_ide1.png}
\end{center}
\end{frame}

%------------------------------------------------

\begin{frame}[fragile]
\frametitle{Use an IDE}

Use an IDE that provides helpful documentation 
\begin{center}
\ig[width=10cm]{images/rstudio_ide3.png}
\end{center}
\end{frame}

%------------------------------------------------

\begin{frame}
\begin{center}
\Huge{\hilit{Good Source Code}}
\end{center}
\end{frame}

%------------------------------------------------

\begin{frame}[fragile]
\frametitle{Literate Programming}

Think about programs/scripts/code as \textbf{works of literature}

\end{frame}

%------------------------------------------------

\begin{frame}[fragile]
\frametitle{Important Aspects}

\bbi
  \item Indentation of lines
  \item Use of spaces
  \item Use of comments
  \item Naming style
  \item Use of white space
  \item Consistency
\ei

\end{frame}

%------------------------------------------------

\begin{frame}[fragile]
\frametitle{Literate Programming}

\bb{Good source code}
\bbi
  \item Well readable by humans
  \item As much self-explaining as possible
\ei
\eb

\end{frame}

%------------------------------------------------

\begin{frame}
\frametitle{Literate Programming}

{\large ``Let us change our traditional attitude to the construction of programs: Instead of imagining that our main task is to instruct a computer what to do, let us concentrate rather on explaining to human beings what we want a computer to do''}

\bigskip
Donald Knuth. ``Literate Programming (1984)''

\end{frame}

%------------------------------------------------

\begin{frame}[fragile]
\frametitle{Literate Programming}

\bbi
  \item Choose the names of variables carefully
  \item Explain what each variable means
  \item Strive for a program that is comprehensible
  \item Introduce concepts in an order that is best for human understanding
\ei

\bigskip
{\footnotesize (From Donald Knuth's: Literate Programming, 1984)}

\end{frame}

%------------------------------------------------

\begin{frame}[fragile]
\frametitle{Literate Programming}

Instructing a computer what to do
\begin{knitrout}\footnotesize
\definecolor{shadecolor}{rgb}{0.969, 0.969, 0.969}\color{fgcolor}\begin{kframe}
\begin{alltt}
\hlcom{# good for computers (not much for humans)}
\hlkwa{if} \hlstd{(}\hlkwd{is.numeric}\hlstd{(x)} \hlopt{&} \hlstd{x} \hlopt{>} \hlnum{0} \hlopt{&} \hlstd{x} \hlopt \hlnum{1} \hlopt{==} \hlnum{0}\hlstd{)} \hlnum{TRUE} \hlkwa{else} \hlnum{FALSE}
\end{alltt}
\end{kframe}
\end{knitrout}

\pause

\bigskip
Explaining a human being what we want a computer to do
\begin{knitrout}\footnotesize
\definecolor{shadecolor}{rgb}{0.969, 0.969, 0.969}\color{fgcolor}\begin{kframe}
\begin{alltt}
\hlcom{# good for humans}
\hlkwd{is_positive_integer}\hlstd{(x)}
\end{alltt}
\end{kframe}
\end{knitrout}

\end{frame}

%------------------------------------------------

\begin{frame}[fragile]
\frametitle{Literate Programming}

\begin{knitrout}\footnotesize
\definecolor{shadecolor}{rgb}{0.969, 0.969, 0.969}\color{fgcolor}\begin{kframe}
\begin{alltt}
\hlcom{# example}
\hlstd{is_positive_integer} \hlkwb{<-} \hlkwa{function}\hlstd{(}\hlkwc{x}\hlstd{) \{}
  \hlstd{(}\hlkwd{is.numeric}\hlstd{(x)} \hlopt{&} \hlstd{x} \hlopt{>} \hlnum{0} \hlopt{&} \hlstd{x} \hlopt \hlnum{1} \hlopt{==} \hlnum{0}\hlstd{)}
\hlstd{\}}

\hlkwd{is_positive_integer}\hlstd{(}\hlnum{2}\hlstd{)}
\end{alltt}
\begin{verbatim}
## [1] TRUE
\end{verbatim}
\begin{alltt}
\hlkwd{is_positive_integer}\hlstd{(}\hlnum{2.1}\hlstd{)}
\end{alltt}
\begin{verbatim}
## [1] FALSE
\end{verbatim}
\end{kframe}
\end{knitrout}

\end{frame}

%------------------------------------------------

\begin{frame}[fragile]
\frametitle{Indentation}

\bi
  \item Keep your indentation style consistent
  \item There is more than one way of indenting code
  \item There is no ``best'' style that everyone should be following
  \item You can indent using spaces or tabs (but don't mix them)
  \item Can help in detecting errors in your code because it can expose lack of symmetry
  \item Do this systematically (RStudio editor helps a lot)
\ei

\end{frame}

%------------------------------------------------

\begin{frame}[fragile]
\frametitle{Indentation}

\begin{knitrout}\footnotesize
\definecolor{shadecolor}{rgb}{0.969, 0.969, 0.969}\color{fgcolor}\begin{kframe}
\begin{alltt}
\hlcom{# Don't do this!}
\hlkwa{if}\hlstd{(}\hlopt{!}\hlkwd{is.vector}\hlstd{(x)) \{}
\hlkwd{stop}\hlstd{(}\hlstr{'x must be a vector'}\hlstd{)}
\hlstd{\}} \hlkwa{else} \hlstd{\{}
\hlkwa{if}\hlstd{(}\hlkwd{any}\hlstd{(}\hlkwd{is.na}\hlstd{(x)))\{}
\hlstd{x} \hlkwb{<-} \hlstd{x[}\hlopt{!}\hlkwd{is.na}\hlstd{(x)]}
\hlstd{\}}
\hlstd{total} \hlkwb{<-} \hlkwd{length}\hlstd{(x)}
\hlstd{x_sum} \hlkwb{<-} \hlnum{0}
\hlkwa{for} \hlstd{(i} \hlkwa{in} \hlkwd{seq_along}\hlstd{(x)) \{}
  \hlstd{x_sum} \hlkwb{<-} \hlstd{x_sum} \hlopt{+} \hlstd{x[i]}
\hlstd{\}}
\hlstd{x_sum} \hlopt{/} \hlstd{total}
\hlstd{\}}
\end{alltt}
\end{kframe}
\end{knitrout}

\end{frame}

%------------------------------------------------

\begin{frame}[fragile]
\frametitle{Indentation}

\begin{knitrout}\footnotesize
\definecolor{shadecolor}{rgb}{0.969, 0.969, 0.969}\color{fgcolor}\begin{kframe}
\begin{alltt}
\hlcom{# better with indentation}
\hlkwa{if}\hlstd{(}\hlopt{!}\hlkwd{is.vector}\hlstd{(x)) \{}
  \hlkwd{stop}\hlstd{(}\hlstr{'x must be a vector'}\hlstd{)}
\hlstd{\}} \hlkwa{else} \hlstd{\{}
  \hlkwa{if}\hlstd{(}\hlkwd{any}\hlstd{(}\hlkwd{is.na}\hlstd{(x))) \{}
    \hlstd{x} \hlkwb{<-} \hlstd{x[}\hlopt{!}\hlkwd{is.na}\hlstd{(x)]}
  \hlstd{\}}
  \hlstd{total} \hlkwb{<-} \hlkwd{length}\hlstd{(x)}
  \hlstd{x_sum} \hlkwb{<-} \hlnum{0}
  \hlkwa{for} \hlstd{(i} \hlkwa{in} \hlkwd{seq_along}\hlstd{(x)) \{}
    \hlstd{x_sum} \hlkwb{<-} \hlstd{x_sum} \hlopt{+} \hlstd{x[i]}
  \hlstd{\}}
  \hlstd{x_sum} \hlopt{/} \hlstd{total}
\hlstd{\}}
\end{alltt}
\end{kframe}
\end{knitrout}

\end{frame}

%------------------------------------------------

\begin{frame}[fragile]
\frametitle{Indenting Styles}

\begin{knitrout}\footnotesize
\definecolor{shadecolor}{rgb}{0.969, 0.969, 0.969}\color{fgcolor}\begin{kframe}
\begin{alltt}
\hlcom{# style 1}
\hlstd{find_roots} \hlkwb{<-} \hlkwa{function}\hlstd{(}\hlkwc{a} \hlstd{=} \hlnum{1}\hlstd{,} \hlkwc{b} \hlstd{=} \hlnum{1}\hlstd{,} \hlkwc{c} \hlstd{=} \hlnum{0}\hlstd{)}
\hlstd{\{}
  \hlkwa{if} \hlstd{(b}\hlopt{^}\hlnum{2} \hlopt{-} \hlnum{4}\hlopt{*}\hlstd{a}\hlopt{*}\hlstd{c} \hlopt{<} \hlnum{0}\hlstd{)}
  \hlstd{\{}
    \hlkwd{return}\hlstd{(}\hlstr{"No real roots"}\hlstd{)}
  \hlstd{\}} \hlkwa{else}
  \hlstd{\{}
    \hlkwd{return}\hlstd{(}\hlkwd{quadratic}\hlstd{(}\hlkwc{a} \hlstd{= a,} \hlkwc{b} \hlstd{= b,} \hlkwc{c} \hlstd{= c))}
  \hlstd{\}}
\hlstd{\}}
\end{alltt}
\end{kframe}
\end{knitrout}

\end{frame}

%------------------------------------------------

\begin{frame}[fragile]
\frametitle{Indenting Styles}

\begin{knitrout}\footnotesize
\definecolor{shadecolor}{rgb}{0.969, 0.969, 0.969}\color{fgcolor}\begin{kframe}
\begin{alltt}
\hlcom{# style 2}
\hlstd{find_roots} \hlkwb{<-} \hlkwa{function}\hlstd{(}\hlkwc{a} \hlstd{=} \hlnum{1}\hlstd{,} \hlkwc{b} \hlstd{=} \hlnum{1}\hlstd{,} \hlkwc{c} \hlstd{=} \hlnum{0}\hlstd{) \{}
  \hlkwa{if} \hlstd{(b}\hlopt{^}\hlnum{2} \hlopt{-} \hlnum{4}\hlopt{*}\hlstd{a}\hlopt{*}\hlstd{c} \hlopt{<} \hlnum{0}\hlstd{) \{}
    \hlkwd{return}\hlstd{(}\hlstr{"No real roots"}\hlstd{)}
  \hlstd{\}} \hlkwa{else} \hlstd{\{}
    \hlkwd{return}\hlstd{(}\hlkwd{quadratic}\hlstd{(}\hlkwc{a} \hlstd{= a,} \hlkwc{b} \hlstd{= b,} \hlkwc{c} \hlstd{= c))}
  \hlstd{\}}
\hlstd{\}}
\end{alltt}
\end{kframe}
\end{knitrout}

\end{frame}

%------------------------------------------------

\begin{frame}[fragile]
\frametitle{Indentation}

\bb{Benefits of code indentation:}
\bi
  \item Easier to read
  \item Easier to understand
  \item Easier to modify
  \item Easier to maintain
  \item Easier to enhance
\ei
\eb

\end{frame}

%------------------------------------------------

\begin{frame}[fragile]
\frametitle{Reformat Code in RStudio}

\bi
  \item RStudio provides \textit{code reformatting} (use it!)
  \item Click \textbf{Code} on the menu bar
  \item Then click \textbf{Reformat Code}
\ei

\end{frame}

%------------------------------------------------

\begin{frame}[fragile]
\frametitle{}
\begin{center}
\ig[width=10cm]{images/reformat_code.pdf}
\end{center}
\end{frame}

%------------------------------------------------

\begin{frame}[fragile]
\frametitle{Reformat Code in RStudio}

\begin{knitrout}\scriptsize
\definecolor{shadecolor}{rgb}{0.969, 0.969, 0.969}\color{fgcolor}\begin{kframe}
\begin{alltt}
\hlcom{# unformatted code}
\hlstd{quadratic}\hlkwb{<-}\hlkwa{function}\hlstd{(}\hlkwc{a}\hlstd{=}\hlnum{1}\hlstd{,}\hlkwc{b}\hlstd{=}\hlnum{1}\hlstd{,}\hlkwc{c}\hlstd{=}\hlnum{0}\hlstd{)\{}
\hlstd{root}\hlkwb{<-}\hlkwd{sqrt}\hlstd{(b}\hlopt{^}\hlnum{2}\hlopt{-}\hlnum{4}\hlopt{*}\hlstd{a}\hlopt{*}\hlstd{c)}
\hlstd{x1}\hlkwb{<-}\hlstd{(}\hlopt{-}\hlstd{b}\hlopt{+}\hlstd{root)}\hlopt{/}\hlnum{2}\hlopt{*}\hlstd{a}
\hlstd{x2}\hlkwb{<-}\hlstd{(}\hlopt{-}\hlstd{b}\hlopt{-}\hlstd{root)}\hlopt{/}\hlnum{2}\hlopt{*}\hlstd{a}
\hlkwd{list}\hlstd{(}\hlkwc{sol1}\hlstd{=x1,}\hlkwc{sol2}\hlstd{=x2)}
\hlstd{\}}
\end{alltt}
\end{kframe}
\end{knitrout}

\pause
\begin{knitrout}\scriptsize
\definecolor{shadecolor}{rgb}{0.969, 0.969, 0.969}\color{fgcolor}\begin{kframe}
\begin{alltt}
\hlcom{# reformatted code}
\hlstd{quadratic} \hlkwb{<-} \hlkwa{function}\hlstd{(}\hlkwc{a} \hlstd{=} \hlnum{1}\hlstd{,}\hlkwc{b} \hlstd{=} \hlnum{1}\hlstd{,}\hlkwc{c} \hlstd{=} \hlnum{0}\hlstd{) \{}
  \hlstd{root} \hlkwb{<-} \hlkwd{sqrt}\hlstd{(b} \hlopt{^} \hlnum{2} \hlopt{-} \hlnum{4} \hlopt{*} \hlstd{a} \hlopt{*} \hlstd{c)}
  \hlstd{x1} \hlkwb{<-} \hlstd{(}\hlopt{-}\hlstd{b} \hlopt{+} \hlstd{root)} \hlopt{/} \hlnum{2} \hlopt{*} \hlstd{a}
  \hlstd{x2} \hlkwb{<-} \hlstd{(}\hlopt{-}\hlstd{b} \hlopt{-} \hlstd{root)} \hlopt{/} \hlnum{2} \hlopt{*} \hlstd{a}
  \hlkwd{list}\hlstd{(}\hlkwc{sol1} \hlstd{= x1,}\hlkwc{sol2} \hlstd{= x2)}
\hlstd{\}}
\end{alltt}
\end{kframe}
\end{knitrout}

\end{frame}

%------------------------------------------------

\begin{frame}
\begin{center}
\Huge{\hilit{Meaningful Names}}
\end{center}
\end{frame}

%------------------------------------------------

\begin{frame}[fragile]
\frametitle{Naming Style}

Choose a consistent naming style for objects and functions
\bbi
  \item \code{someObject} (lowerCamelCase)
  \item \code{SomeObject} (UpperCamelCase)
  \item \code{some\_object} (underscore separation)
  \item \code{some.object} (dot separation)
\ei

\end{frame}

%------------------------------------------------

\begin{frame}[fragile]
\frametitle{Naming Style}

Avoid using names of standard R objects
\bi
  \item \code{vector}
  \item \code{mean}
  \item \code{list}
  \item \code{data}
  \item \code{c}
  \item \code{colors}
  \item \textit{etc}
\ei

\end{frame}

%------------------------------------------------

\begin{frame}[fragile]
\frametitle{Naming Style}

If you're thinking about using names of R objects, prefer something like this
\bi
  \item \code{xvector}
  \item \code{xmean}
  \item \code{xlist}
  \item \code{xdata}
  \item \code{xc}
  \item \code{xcolors}
  \item \textit{etc}
\ei

\end{frame}

%------------------------------------------------

\begin{frame}[fragile]
\frametitle{Naming Style}

Better to add meaning like this
\bi
  \item \code{mean\_salary}
  \item \code{input\_vector}
  \item \code{data\_list}
  \item \code{data\_table}
  \item \code{first\_last}
  \item \code{some\_colors}
  \item \textit{etc}
\ei

\end{frame}

%------------------------------------------------

\begin{frame}[fragile]
\frametitle{Naming Style}

\begin{knitrout}\footnotesize
\definecolor{shadecolor}{rgb}{0.969, 0.969, 0.969}\color{fgcolor}\begin{kframe}
\begin{alltt}
\hlcom{# what does getThem() do?}
\hlstd{getThem} \hlkwb{<-} \hlkwa{function}\hlstd{(}\hlkwc{values}\hlstd{,} \hlkwc{y}\hlstd{) \{}
  \hlstd{list1} \hlkwb{<-} \hlkwd{c}\hlstd{()}

  \hlkwa{for} \hlstd{(i} \hlkwa{in} \hlstd{values) \{}
    \hlkwa{if} \hlstd{(values[i]} \hlopt{==} \hlstd{y)}
      \hlstd{list1} \hlkwb{<-} \hlkwd{c}\hlstd{(list1, x)}
  \hlstd{\}}
  \hlkwd{return}\hlstd{(list1)}
\hlstd{\}}
\end{alltt}
\end{kframe}
\end{knitrout}

\end{frame}

%------------------------------------------------

\begin{frame}[fragile]
\frametitle{Naming Style}

\begin{knitrout}\footnotesize
\definecolor{shadecolor}{rgb}{0.969, 0.969, 0.969}\color{fgcolor}\begin{kframe}
\begin{alltt}
\hlcom{# this is more meaningful}
\hlstd{getFlaggedCells} \hlkwb{<-} \hlkwa{function}\hlstd{(}\hlkwc{gameBoard}\hlstd{,} \hlkwc{flagged}\hlstd{) \{}
  \hlstd{flaggedCells} \hlkwb{<-} \hlkwd{c}\hlstd{()}

  \hlkwa{for} \hlstd{(cell} \hlkwa{in} \hlstd{gameBoard) \{}
    \hlkwa{if} \hlstd{(gameBoard[cell]} \hlopt{==} \hlstd{flagged)}
      \hlstd{flaggedCells} \hlkwb{<-} \hlkwd{c}\hlstd{(flaggedCells, x)}
  \hlstd{\}}
  \hlkwd{return}\hlstd{(flaggedCells)}
\hlstd{\}}
\end{alltt}
\end{kframe}
\end{knitrout}

\end{frame}

%------------------------------------------------

\begin{frame}[fragile]
\frametitle{Meaningful Distinctions}

\begin{knitrout}\footnotesize
\definecolor{shadecolor}{rgb}{0.969, 0.969, 0.969}\color{fgcolor}\begin{kframe}
\begin{alltt}
\hlcom{# argument names 'a1' and 'a2'?}
\hlstd{move_strings} \hlkwb{<-} \hlkwa{function}\hlstd{(}\hlkwc{a1}\hlstd{,} \hlkwc{a2}\hlstd{) \{}
  \hlkwa{for} \hlstd{(i} \hlkwa{in} \hlkwd{seq_along}\hlstd{(a1)) \{}
    \hlstd{a1[i]} \hlkwb{<-} \hlkwd{toupper}\hlstd{(}\hlkwd{substr}\hlstd{(a1,} \hlnum{1}\hlstd{,} \hlnum{3}\hlstd{))}
  \hlstd{\}}
  \hlstd{a2}
\hlstd{\}}
\end{alltt}
\end{kframe}
\end{knitrout}

\pause
\begin{knitrout}\footnotesize
\definecolor{shadecolor}{rgb}{0.969, 0.969, 0.969}\color{fgcolor}\begin{kframe}
\begin{alltt}
\hlcom{# argument names }
\hlstd{move_strings} \hlkwb{<-} \hlkwa{function}\hlstd{(}\hlkwc{origin}\hlstd{,} \hlkwc{destination}\hlstd{) \{}
  \hlkwa{for} \hlstd{(i} \hlkwa{in} \hlkwd{seq_along}\hlstd{(origin)) \{}
    \hlstd{destination[i]} \hlkwb{<-} \hlkwd{toupper}\hlstd{(}\hlkwd{substr}\hlstd{(origin,} \hlnum{1}\hlstd{,} \hlnum{3}\hlstd{))}
  \hlstd{\}}
  \hlstd{destination}
\hlstd{\}}
\end{alltt}
\end{kframe}
\end{knitrout}

\end{frame}

%------------------------------------------------

\begin{frame}[fragile]
\frametitle{Pronounceable Names}

\begin{knitrout}\footnotesize
\definecolor{shadecolor}{rgb}{0.969, 0.969, 0.969}\color{fgcolor}\begin{kframe}
\begin{alltt}
\hlcom{# cryptic abbreviations }
\hlstd{DtaRcrd102} \hlkwb{<-} \hlkwd{list}\hlstd{(}
  \hlkwc{nm} \hlstd{=} \hlstr{'John Doe'}\hlstd{,}
  \hlkwc{bdg} \hlstd{=} \hlstr{'Valley Life Sciences Building'}\hlstd{,}
  \hlkwc{rm} \hlstd{=} \hlnum{2060}
\hlstd{)}
\end{alltt}
\end{kframe}
\end{knitrout}

\pause
\begin{knitrout}\footnotesize
\definecolor{shadecolor}{rgb}{0.969, 0.969, 0.969}\color{fgcolor}\begin{kframe}
\begin{alltt}
\hlcom{# pronounceable names }
\hlstd{Customer} \hlkwb{<-} \hlkwd{list}\hlstd{(}
  \hlkwc{name} \hlstd{=} \hlstr{'John Doe'}\hlstd{,}
  \hlkwc{building} \hlstd{=} \hlstr{'Valley Life Sciences Building'}\hlstd{,}
  \hlkwc{room} \hlstd{=} \hlnum{2060}
\hlstd{)}
\end{alltt}
\end{kframe}
\end{knitrout}

\end{frame}

%------------------------------------------------

\begin{frame}[fragile]
\frametitle{Your Turn}

Which of the following is NOT a valid name:
\bbi
  \item A) \code{x12345}
  \item B) \code{data\_}
  \item C) \code{oBjEcT}
  \item D) \code{5ummary}
  \item E) \code{data.frame}
\ei

\end{frame}

%------------------------------------------------

\begin{frame}
\frametitle{Syntax}

\bb{White Spaces}
\bi
 \item Use a lot of it
 \item around operators (assignment and arithmetic)
 \item between function arguments and list elements
 \item between matrix/array indices, in particular for missing indices
 \item Split long lines at meaningful places
\ei
\eb

\end{frame}

%------------------------------------------------

\begin{frame}[fragile]
\frametitle{White spaces}

Avoid this
\begin{knitrout}\footnotesize
\definecolor{shadecolor}{rgb}{0.969, 0.969, 0.969}\color{fgcolor}\begin{kframe}
\begin{alltt}
\hlstd{a}\hlkwb{<-}\hlnum{2}
\hlstd{x}\hlkwb{<-}\hlnum{3}
\hlstd{y}\hlkwb{<-}\hlkwd{log}\hlstd{(}\hlkwd{sqrt}\hlstd{(x))}
\hlnum{3}\hlopt{*}\hlstd{x}\hlopt{^}\hlnum{7}\hlopt{-}\hlstd{pi}\hlopt{*}\hlstd{x}\hlopt{/}\hlstd{(y}\hlopt{-}\hlstd{a)}
\end{alltt}
\end{kframe}
\end{knitrout}

Much Better
\begin{knitrout}\footnotesize
\definecolor{shadecolor}{rgb}{0.969, 0.969, 0.969}\color{fgcolor}\begin{kframe}
\begin{alltt}
\hlstd{a} \hlkwb{<-} \hlnum{2}
\hlstd{x} \hlkwb{<-} \hlnum{3}
\hlstd{y} \hlkwb{<-} \hlkwd{log}\hlstd{(}\hlkwd{sqrt}\hlstd{(x))}
\hlnum{3}\hlopt{*}\hlstd{x}\hlopt{^}\hlnum{7} \hlopt{-} \hlstd{pi} \hlopt{*} \hlstd{x} \hlopt{/} \hlstd{(y} \hlopt{-} \hlstd{a)}
\end{alltt}
\end{kframe}
\end{knitrout}

\end{frame}

%------------------------------------------------

\begin{frame}[fragile]
\frametitle{White spaces}

\begin{knitrout}\footnotesize
\definecolor{shadecolor}{rgb}{0.969, 0.969, 0.969}\color{fgcolor}\begin{kframe}
\begin{alltt}
\hlcom{# Avoid this}
\hlkwd{plot}\hlstd{(x,y,}\hlkwc{col}\hlstd{=}\hlkwd{rgb}\hlstd{(}\hlnum{0.5}\hlstd{,}\hlnum{0.7}\hlstd{,}\hlnum{0.4}\hlstd{),}\hlkwc{pch}\hlstd{=}\hlstr{'+'}\hlstd{,}\hlkwc{cex}\hlstd{=}\hlnum{5}\hlstd{)}
\end{alltt}
\end{kframe}
\end{knitrout}

\pause
\begin{knitrout}\footnotesize
\definecolor{shadecolor}{rgb}{0.969, 0.969, 0.969}\color{fgcolor}\begin{kframe}
\begin{alltt}
\hlcom{# OK}
\hlkwd{plot}\hlstd{(x, y,} \hlkwc{col} \hlstd{=} \hlkwd{rgb}\hlstd{(}\hlnum{0.5}\hlstd{,} \hlnum{0.7}\hlstd{,} \hlnum{0.4}\hlstd{),} \hlkwc{pch} \hlstd{=} \hlstr{'+'}\hlstd{,} \hlkwc{cex} \hlstd{=} \hlnum{5}\hlstd{)}
\end{alltt}
\end{kframe}
\end{knitrout}

\end{frame}

%------------------------------------------------

\begin{frame}[fragile]
\frametitle{Readability}

Lines should be broken/wrapped around so that they are less than 80 columns wide
\begin{knitrout}\scriptsize
\definecolor{shadecolor}{rgb}{0.969, 0.969, 0.969}\color{fgcolor}\begin{kframe}
\begin{alltt}
\hlcom{# lines too long}
\hlstd{histogram} \hlkwb{<-} \hlkwa{function}\hlstd{(}\hlkwc{data}\hlstd{)\{}
\hlkwd{hist}\hlstd{(data,} \hlkwc{col} \hlstd{=} \hlstr{'gray90'}\hlstd{,} \hlkwc{xlab} \hlstd{=} \hlstr{'x'}\hlstd{,} \hlkwc{ylab} \hlstd{=} \hlstr{'Frequency'}\hlstd{,} \hlkwc{main}\hlstd{=} \hlstr{'Histogram of x'}\hlstd{)}
\hlkwd{abline}\hlstd{(}\hlkwc{v} \hlstd{=} \hlkwd{c}\hlstd{(}\hlkwd{min}\hlstd{(data),} \hlkwd{max}\hlstd{(data),} \hlkwd{median}\hlstd{(data),} \hlkwd{mean}\hlstd{(data)),}
\hlkwc{col} \hlstd{=} \hlkwd{c}\hlstd{(}\hlstr{'gray30'}\hlstd{,} \hlstr{'gray30'}\hlstd{,} \hlstr{'orange'}\hlstd{,} \hlstr{'tomato'}\hlstd{),} \hlkwc{lty} \hlstd{=} \hlkwd{c}\hlstd{(}\hlnum{2}\hlstd{,}\hlnum{2}\hlstd{,}\hlnum{1}\hlstd{,}\hlnum{1}\hlstd{),} \hlkwc{lwd} \hlstd{=} \hlnum{3}\hlstd{)}
\hlstd{\}}
\end{alltt}
\end{kframe}
\end{knitrout}

\end{frame}

%------------------------------------------------

\begin{frame}[fragile]
\frametitle{Readability}

Lines should be broken/wrapped aroung so that they are less than 80 columns wide
\begin{knitrout}\scriptsize
\definecolor{shadecolor}{rgb}{0.969, 0.969, 0.969}\color{fgcolor}\begin{kframe}
\begin{alltt}
\hlcom{# lines too long}
\hlstd{histogram} \hlkwb{<-} \hlkwa{function}\hlstd{(}\hlkwc{data}\hlstd{) \{}
  \hlkwd{hist}\hlstd{(data,} \hlkwc{col} \hlstd{=} \hlstr{'gray90'}\hlstd{,} \hlkwc{xlab} \hlstd{=} \hlstr{'x'}\hlstd{,} \hlkwc{ylab} \hlstd{=} \hlstr{'Frequency'}\hlstd{,}
       \hlkwc{main} \hlstd{=} \hlstr{'Histogram of x'}\hlstd{)}
  \hlkwd{abline}\hlstd{(}\hlkwc{v} \hlstd{=} \hlkwd{c}\hlstd{(}\hlkwd{min}\hlstd{(data),} \hlkwd{max}\hlstd{(data),} \hlkwd{median}\hlstd{(data),} \hlkwd{mean}\hlstd{(data)),}
         \hlkwc{col} \hlstd{=} \hlkwd{c}\hlstd{(}\hlstr{'gray30'}\hlstd{,} \hlstr{'gray30'}\hlstd{,} \hlstr{'orange'}\hlstd{,} \hlstr{'tomato'}\hlstd{),}
         \hlkwc{lty} \hlstd{=} \hlkwd{c}\hlstd{(}\hlnum{2}\hlstd{,}\hlnum{2}\hlstd{,}\hlnum{1}\hlstd{,}\hlnum{1}\hlstd{),} \hlkwc{lwd} \hlstd{=} \hlnum{3}\hlstd{)}
\hlstd{\}}
\end{alltt}
\end{kframe}
\end{knitrout}

\end{frame}

%------------------------------------------------

\begin{frame}[fragile]
\frametitle{White spaces}

\bbi
  \item Spacing forms the second important part in code indentation and formatting. 
  \item Spacing makes the code more readable
  \item Follow proper spacing through out your coding 
  \item Use spacing consistently
\ei

\end{frame}

%------------------------------------------------

\begin{frame}[fragile]
\frametitle{White spaces}

\begin{knitrout}\footnotesize
\definecolor{shadecolor}{rgb}{0.969, 0.969, 0.969}\color{fgcolor}\begin{kframe}
\begin{alltt}
\hlcom{# this can be improved}
\hlstd{stats} \hlkwb{<-} \hlkwd{c}\hlstd{(}\hlkwd{min}\hlstd{(x),} \hlkwd{max}\hlstd{(x),} \hlkwd{max}\hlstd{(x)}\hlopt{-}\hlkwd{min}\hlstd{(x),}
  \hlkwd{quantile}\hlstd{(x,} \hlkwc{probs}\hlstd{=}\hlnum{0.25}\hlstd{),} \hlkwd{quantile}\hlstd{(x,} \hlkwc{probs}\hlstd{=}\hlnum{0.75}\hlstd{),} \hlkwd{IQR}\hlstd{(x),}
  \hlkwd{median}\hlstd{(x),} \hlkwd{mean}\hlstd{(x),} \hlkwd{sd}\hlstd{(x)}
\hlstd{)}
\end{alltt}
\end{kframe}
\end{knitrout}

\end{frame}

%------------------------------------------------

\begin{frame}[fragile]
\frametitle{White spaces}

Don't be afraid of splitting one long line into individual pieces:
\begin{knitrout}\footnotesize
\definecolor{shadecolor}{rgb}{0.969, 0.969, 0.969}\color{fgcolor}\begin{kframe}
\begin{alltt}
\hlcom{# much better}
\hlstd{stats} \hlkwb{<-} \hlkwd{c}\hlstd{(}
  \hlkwd{min}\hlstd{(x),}
  \hlkwd{max}\hlstd{(x),}
  \hlkwd{max}\hlstd{(x)} \hlopt{-} \hlkwd{min}\hlstd{(x),}
  \hlkwd{quantile}\hlstd{(x,} \hlkwc{probs} \hlstd{=} \hlnum{0.25}\hlstd{),}
  \hlkwd{quantile}\hlstd{(x,} \hlkwc{probs} \hlstd{=} \hlnum{0.75}\hlstd{),}
  \hlkwd{IQR}\hlstd{(x),}
  \hlkwd{median}\hlstd{(x),}
  \hlkwd{mean}\hlstd{(x),}
  \hlkwd{sd}\hlstd{(x)}
\hlstd{)}
\end{alltt}
\end{kframe}
\end{knitrout}

\end{frame}

%------------------------------------------------

\begin{frame}[fragile]
\frametitle{White spaces}

You can even do this:
\begin{knitrout}\footnotesize
\definecolor{shadecolor}{rgb}{0.969, 0.969, 0.969}\color{fgcolor}\begin{kframe}
\begin{alltt}
\hlcom{# also OK}
\hlstd{stats} \hlkwb{<-} \hlkwd{c}\hlstd{(}
  \hlkwc{min}    \hlstd{=} \hlkwd{min}\hlstd{(x),}
  \hlkwc{max}    \hlstd{=} \hlkwd{max}\hlstd{(x),}
  \hlkwc{range}  \hlstd{=} \hlkwd{max}\hlstd{(x)} \hlopt{-} \hlkwd{min}\hlstd{(x),}
  \hlkwc{q1}     \hlstd{=} \hlkwd{quantile}\hlstd{(x,} \hlkwc{probs} \hlstd{=} \hlnum{0.25}\hlstd{),}
  \hlkwc{q3}     \hlstd{=} \hlkwd{quantile}\hlstd{(x,} \hlkwc{probs} \hlstd{=} \hlnum{0.75}\hlstd{),}
  \hlkwc{iqr}    \hlstd{=} \hlkwd{IQR}\hlstd{(x),}
  \hlkwc{median} \hlstd{=} \hlkwd{median}\hlstd{(x),}
  \hlkwc{mean}   \hlstd{=} \hlkwd{mean}\hlstd{(x),}
  \hlkwc{stdev}  \hlstd{=} \hlkwd{sd}\hlstd{(x)}
\hlstd{)}
\end{alltt}
\end{kframe}
\end{knitrout}

\end{frame}

%------------------------------------------------

\begin{frame}[fragile]
\frametitle{White spaces}

\bbi
  \item All commas and semicolons must be followed by single whitespace
  \item All binary operators should maintain a space on either side of the operator
  \item Left parenthesis should start immediately after a function name
  \item All keywords like \code{if, while, for, repeat} should be followed by a single space.
\ei

\end{frame}

%------------------------------------------------

\begin{frame}[fragile]
\frametitle{White spaces}

All binary operators should maintain a space on either side of the operator
\begin{knitrout}\footnotesize
\definecolor{shadecolor}{rgb}{0.969, 0.969, 0.969}\color{fgcolor}\begin{kframe}
\begin{alltt}
\hlcom{# NOT Recommended }
\hlstd{a}\hlkwb{=}\hlstd{b}\hlopt{-}\hlstd{c}
\hlstd{a} \hlkwb{=} \hlstd{b}\hlopt{-}\hlstd{c}
\hlstd{a}\hlkwb{=}\hlstd{b} \hlopt{-} \hlstd{c;}

\hlcom{# Recommended }
\hlstd{a} \hlkwb{=} \hlstd{b} \hlopt{-} \hlstd{c}
\end{alltt}
\end{kframe}
\end{knitrout}

\end{frame}

%------------------------------------------------

\begin{frame}[fragile]
\frametitle{White spaces}

All binary operators should maintain a space on either side of the operator
\begin{knitrout}\footnotesize
\definecolor{shadecolor}{rgb}{0.969, 0.969, 0.969}\color{fgcolor}\begin{kframe}
\begin{alltt}
\hlcom{# Not really recommended }
\hlstd{z} \hlkwb{<-} \hlnum{6}\hlopt{*}\hlstd{x} \hlopt{+} \hlnum{9}\hlopt{*}\hlstd{y}

\hlcom{# Recommended (option 1)}
\hlstd{z} \hlkwb{<-} \hlnum{6} \hlopt{*} \hlstd{x} \hlopt{+} \hlnum{9} \hlopt{*} \hlstd{y}

\hlcom{# Recommended (option 2)}
\hlstd{z} \hlkwb{<-} \hlstd{(}\hlnum{7} \hlopt{*} \hlstd{x)} \hlopt{+} \hlstd{(}\hlnum{9} \hlopt{*} \hlstd{y)}
\end{alltt}
\end{kframe}
\end{knitrout}

\end{frame}

%------------------------------------------------

\begin{frame}[fragile]
\frametitle{White spaces}

Left parenthesis should start immediately after a function name
\begin{knitrout}\footnotesize
\definecolor{shadecolor}{rgb}{0.969, 0.969, 0.969}\color{fgcolor}\begin{kframe}
\begin{alltt}
\hlcom{# NOT Recommended }
\hlkwd{read.table} \hlstd{(}\hlstr{'data.csv'}\hlstd{,} \hlkwc{header} \hlstd{=} \hlnum{TRUE}\hlstd{,} \hlkwc{row.names} \hlstd{=} \hlnum{1}\hlstd{)}

\hlcom{# Recommended }
\hlkwd{read.table}\hlstd{(}\hlstr{'data.csv'}\hlstd{,} \hlkwc{header} \hlstd{=} \hlnum{TRUE}\hlstd{,} \hlkwc{row.names} \hlstd{=} \hlnum{1}\hlstd{)}
\end{alltt}
\end{kframe}
\end{knitrout}

\end{frame}

%------------------------------------------------

\begin{frame}[fragile]
\frametitle{White spaces}

All keywords like \code{if, while, for, repeat} should be followed by a single space.
\begin{knitrout}\footnotesize
\definecolor{shadecolor}{rgb}{0.969, 0.969, 0.969}\color{fgcolor}\begin{kframe}
\begin{alltt}
\hlcom{# not bad}
\hlkwa{if}\hlstd{(}\hlkwd{is.numeric}\hlstd{(object)) \{}
  \hlkwd{mean}\hlstd{(object)}
\hlstd{\}}

\hlcom{# much better}
\hlkwa{if} \hlstd{(}\hlkwd{is.numeric}\hlstd{(object)) \{}
  \hlkwd{mean}\hlstd{(object)}
\hlstd{\}}
\end{alltt}
\end{kframe}
\end{knitrout}

\end{frame}

%------------------------------------------------

\begin{frame}[fragile]
\frametitle{Syntax: Parentheses}

Use parentheses for clarity even if not needed for order of operations. 
\begin{knitrout}\footnotesize
\definecolor{shadecolor}{rgb}{0.969, 0.969, 0.969}\color{fgcolor}\begin{kframe}
\begin{alltt}
\hlstd{a} \hlkwb{<-} \hlnum{2}
\hlstd{x} \hlkwb{<-} \hlnum{3}
\hlstd{y} \hlkwb{<-} \hlnum{4}

\hlstd{a}\hlopt{/}\hlstd{y}\hlopt{*}\hlstd{x}
\end{alltt}
\begin{verbatim}
## [1] 1.5
\end{verbatim}
\begin{alltt}
\hlcom{# better}
\hlstd{(a} \hlopt{/} \hlstd{y)} \hlopt{*} \hlstd{x}
\end{alltt}
\begin{verbatim}
## [1] 1.5
\end{verbatim}
\end{kframe}
\end{knitrout}

\end{frame}

%------------------------------------------------

\begin{frame}[fragile]
\frametitle{Use Parentheses}

\begin{knitrout}\footnotesize
\definecolor{shadecolor}{rgb}{0.969, 0.969, 0.969}\color{fgcolor}\begin{kframe}
\begin{alltt}
\hlcom{# confusing}
\hlnum{1}\hlopt{:}\hlnum{3}\hlopt{^}\hlnum{2}
\end{alltt}
\begin{verbatim}
## [1] 1 2 3 4 5 6 7 8 9
\end{verbatim}
\begin{alltt}
\hlcom{# better}
\hlnum{1}\hlopt{:}\hlstd{(}\hlnum{3}\hlopt{^}\hlnum{2}\hlstd{)}
\end{alltt}
\begin{verbatim}
## [1] 1 2 3 4 5 6 7 8 9
\end{verbatim}
\end{kframe}
\end{knitrout}

\end{frame}

%------------------------------------------------

\begin{frame}
\begin{center}
\Huge{\hilit{\code{\#} Comments}}
\end{center}
\end{frame}

%------------------------------------------------

\begin{frame}
\frametitle{Comments}

\bb{Comment your code}
\bi
 \item Add lots of comments
 \item But don't belabor the obvious
 \item Use blank lines to separate blocks of code and comments to say what the block does
 \item Remember that in a few months, you may not follow your own code any better than a stranger
 \item Some key things to document:
 \bi
    \item summarizing a block of code
    \item explaining a very complicated piece of code
    \item explaining arbitrary constant values
  \ei
\ei
\eb

\end{frame}

%------------------------------------------------

\begin{frame}[fragile]
\frametitle{Line spaces and Comments}

\begin{knitrout}\footnotesize
\definecolor{shadecolor}{rgb}{0.969, 0.969, 0.969}\color{fgcolor}\begin{kframe}
\begin{alltt}
\hlstd{MV} \hlkwb{<-} \hlkwd{get_manifests}\hlstd{(Data, blocks)}
\hlstd{check_MV} \hlkwb{<-} \hlkwd{test_manifest_scaling}\hlstd{(MV, specs}\hlopt{$}\hlstd{scaling)}
\hlstd{gens} \hlkwb{<-} \hlkwd{get_generals}\hlstd{(MV, path_matrix)}
\hlkwd{names}\hlstd{(blocks)} \hlkwb{<-} \hlstd{gens}\hlopt{$}\hlstd{lvs_names}
\hlstd{block_sizes} \hlkwb{<-} \hlkwd{lengths}\hlstd{(blocks)}
\hlstd{blockinds} \hlkwb{<-} \hlkwd{indexify}\hlstd{(blocks)}
\end{alltt}
\end{kframe}
\end{knitrout}

\end{frame}

%------------------------------------------------

\begin{frame}[fragile]
\frametitle{Line spaces and Comments}

\begin{knitrout}\footnotesize
\definecolor{shadecolor}{rgb}{0.969, 0.969, 0.969}\color{fgcolor}\begin{kframe}
\begin{alltt}
\hlcom{# ==================================================}
\hlcom{# Preparing data and blocks indexification}
\hlcom{# ==================================================}
\hlcom{# building data matrix 'MV'}
\hlstd{MV} \hlkwb{<-} \hlkwd{get_manifests}\hlstd{(Data, blocks)}
\hlstd{check_MV} \hlkwb{<-} \hlkwd{test_manifest_scaling}\hlstd{(MV, specs}\hlopt{$}\hlstd{scaling)}

\hlcom{# generals about obs, mvs, lvs}
\hlstd{gens} \hlkwb{<-} \hlkwd{get_generals}\hlstd{(MV, path_matrix)}

\hlcom{# indexing blocks}
\hlkwd{names}\hlstd{(blocks)} \hlkwb{<-} \hlstd{gens}\hlopt{$}\hlstd{lvs_names}
\hlstd{block_sizes} \hlkwb{<-} \hlkwd{lengths}\hlstd{(blocks)}
\hlstd{blockinds} \hlkwb{<-} \hlkwd{indexify}\hlstd{(blocks)}
\end{alltt}
\end{kframe}
\end{knitrout}

\end{frame}

%------------------------------------------------

\begin{frame}[fragile]
\frametitle{Line spaces and Comments}

Different line styles:
\begin{knitrout}\footnotesize
\definecolor{shadecolor}{rgb}{0.969, 0.969, 0.969}\color{fgcolor}\begin{kframe}
\begin{alltt}
\hlcom{####################################################}


\hlcom{# ==================================================}


\hlcom{# **************************************************}


\hlcom{# --------------------------------------------------}
\end{alltt}
\end{kframe}
\end{knitrout}

\end{frame}

%------------------------------------------------

\begin{frame}[fragile]
\frametitle{Line spaces and Comments}

\begin{knitrout}\footnotesize
\definecolor{shadecolor}{rgb}{0.969, 0.969, 0.969}\color{fgcolor}\begin{kframe}
\begin{alltt}
\hlcom{# ==================================================}
\hlcom{# Preparing data and blocks indexification}
\hlcom{# ==================================================}
\hlcom{# building data matrix 'MV'}
\hlstd{MV} \hlkwb{<-} \hlkwd{get_manifests}\hlstd{(Data, blocks)}
\hlstd{check_MV} \hlkwb{<-} \hlkwd{test_manifest_scaling}\hlstd{(MV, specs}\hlopt{$}\hlstd{scaling)}
\end{alltt}
\end{kframe}
\end{knitrout}

\pause
\begin{knitrout}\footnotesize
\definecolor{shadecolor}{rgb}{0.969, 0.969, 0.969}\color{fgcolor}\begin{kframe}
\begin{alltt}
\hlcom{# ---- Preparing data and blocks indexification ----}

\hlcom{# building data matrix 'MV'}
\hlstd{MV} \hlkwb{<-} \hlkwd{get_manifests}\hlstd{(Data, blocks)}
\hlstd{check_MV} \hlkwb{<-} \hlkwd{test_manifest_scaling}\hlstd{(MV, specs}\hlopt{$}\hlstd{scaling)}
\end{alltt}
\end{kframe}
\end{knitrout}

\end{frame}

%------------------------------------------------

\begin{frame}[fragile]
\frametitle{Comments}

Include comments to say what a block does, or what a block is intended for
\begin{knitrout}\footnotesize
\definecolor{shadecolor}{rgb}{0.969, 0.969, 0.969}\color{fgcolor}\begin{kframe}
\begin{alltt}
\hlcom{# =====================================================}
\hlcom{# Data: liga2015}
\hlcom{# =====================================================}
\hlcom{# For this session we'll be using the dataset that }
\hlcom{# comes in the file 'liga2015.csv' (see github repo)}
\hlcom{# This dataset contains basic statistics from the}
\hlcom{# Spanish soccer league during the season 2014-2015}
\end{alltt}
\end{kframe}
\end{knitrout}

\end{frame}

%------------------------------------------------

\begin{frame}[fragile]
\frametitle{Comments}

\begin{knitrout}\footnotesize
\definecolor{shadecolor}{rgb}{0.969, 0.969, 0.969}\color{fgcolor}\begin{kframe}
\begin{alltt}
\hlstd{x} \hlkwb{<-} \hlkwd{matrix}\hlstd{(}\hlnum{1}\hlopt{:}\hlnum{10}\hlstd{,} \hlkwc{nrow} \hlstd{=} \hlnum{2}\hlstd{,} \hlkwc{ncol} \hlstd{=} \hlnum{5}\hlstd{)}

\hlcom{# mean vectors by rows and columns}
\hlstd{xmean1} \hlkwb{<-} \hlkwd{apply}\hlstd{(x,} \hlnum{1}\hlstd{, mean)}
\hlstd{xmean2} \hlkwb{<-} \hlkwd{apply}\hlstd{(x,} \hlnum{2}\hlstd{, mean)}

\hlcom{# Subtract off the mean of each row/column}
\hlstd{y} \hlkwb{<-} \hlkwd{sweep}\hlstd{(x,} \hlnum{1}\hlstd{, xmean1)}
\hlstd{z} \hlkwb{<-} \hlkwd{sweep}\hlstd{(x,} \hlnum{2}\hlstd{, xmean2)}

\hlcom{# Multiply by the mean of each column (for some reason)}
\hlstd{w} \hlkwb{<-} \hlkwd{sweep}\hlstd{(x,} \hlnum{2}\hlstd{, xmean1,} \hlkwc{FUN} \hlstd{=} \hlstr{"*"}\hlstd{)}
\end{alltt}
\end{kframe}
\end{knitrout}

\end{frame}

%------------------------------------------------

\begin{frame}[fragile]
\frametitle{About Comments}

Be careful with your comments (you never know who will end up looking at your code, or where you'll be in the future)
\begin{knitrout}\footnotesize
\definecolor{shadecolor}{rgb}{0.969, 0.969, 0.969}\color{fgcolor}\begin{kframe}
\begin{alltt}
\hlcom{# F***ing piece of code that drives me bananas}

\hlcom{# wtf function}

\hlcom{# best for loop ever}
\end{alltt}
\end{kframe}
\end{knitrout}

\end{frame}

%------------------------------------------------

\begin{frame}
\frametitle{Good coding practices: Syntax}

\bb{Code Files}
\bi
 \item Break code into separate files (<2000-3000 lines per file) 
 \item Give files meaningful names 
 \item Group related functions within a file
\ei
\eb

\end{frame}

%------------------------------------------------

\begin{frame}
\frametitle{R Scripts}

\bb{Include Header information such as}
\bi
 \item Who wrote / programmed it
 \item When was it done
 \item What is it all about
 \item How the code might fit within a larger program
\ei
\eb

\end{frame}

%------------------------------------------------

\begin{frame}[fragile]
\frametitle{R Scripts}

Header example:
\begin{knitrout}\footnotesize
\definecolor{shadecolor}{rgb}{0.969, 0.969, 0.969}\color{fgcolor}\begin{kframe}
\begin{alltt}
\hlcom{# ===================================================}
\hlcom{# Some Title}
\hlcom{# Author(s): First Last}
\hlcom{# Date: month-day-year}
\hlcom{# Description: what this code is about}
\hlcom{# Data: perhaps is designed for a specific data set}
\hlcom{# ===================================================}
\end{alltt}
\end{kframe}
\end{knitrout}

\end{frame}

%------------------------------------------------

\begin{frame}[fragile]
\frametitle{R Scripts}

If you need to load R packages, do so at the beginning of your script, after the header:
\begin{knitrout}\footnotesize
\definecolor{shadecolor}{rgb}{0.969, 0.969, 0.969}\color{fgcolor}\begin{kframe}
\begin{alltt}
\hlcom{# ===================================================}
\hlcom{# Some Title}
\hlcom{# Author(s): First Last}
\hlcom{# Date: month-day-year}
\hlcom{# Description: what this code is about}
\hlcom{# Data: perhaps is designed for a specific data set}
\hlcom{# ===================================================}

\hlkwd{library}\hlstd{(stringr)}
\hlkwd{library}\hlstd{(ggplot2)}
\hlkwd{library}\hlstd{(MASS)}
\end{alltt}
\end{kframe}
\end{knitrout}

\end{frame}

%------------------------------------------------

\begin{frame}
\begin{center}
\Huge{\hilit{Functions}}
\end{center}
\end{frame}

%------------------------------------------------

\begin{frame}
\frametitle{Functions}

\bi
 \item Functions are tools and operations
 \item Functions form the building blocks for larger tasks
 \item Functions allow us to reuse blocks of code easily for later use
 \item Use functions whenever possible
 \item Try to write functions rather than carry out your 
 work using blocks of code
\ei

\end{frame}

%------------------------------------------------

\begin{frame}
\frametitle{Length of Functions}

\bb{Some rules of thumb (in order of preference)}
\bbi
 \item Ideal length between 2 and 4 lines of code
 \item No more than 10 lines
 \item No more than 20 lines
 \item Should not exceed the size of the text editor window
\ei
\eb

\end{frame}

%------------------------------------------------

\begin{frame}
\frametitle{Length of Functions}

\bi
 \item Don't write long functions
 \item Rewrite long functions by converting collections 
 of related expression into separate functions
 \item Smaller functions are easier to debug, easier to understand, 
 and can be combined in a modular fashion
 \item Functions shouldn't be longer than one visible screen 
 (with reasonable font)
\ei

\end{frame}

%------------------------------------------------

\begin{frame}
\frametitle{Length of Functions}

\bi
 \item Separate small functions
 \item are easier to reason about and manage
 \item are easier to test and verify they are correct
 \item are more likely to be reusable
\ei

\end{frame}

%------------------------------------------------

\begin{frame}
\frametitle{Some considerations}

\bi
 \item Think about different scenarios and contexts in which a function might be used
 \item Can you generalize it?
 \item Who will use it?
 \item Who is going to maintain the code?
\ei

\end{frame}

%------------------------------------------------

\begin{frame}
\frametitle{Naming Functions}

\bi
  \item Use descriptive names
  \item Readers (including you) should infer the operation by looking at the call of the function
\ei

\end{frame}

%------------------------------------------------

\begin{frame}
\frametitle{Functions}

\bb{Functions should:}
\bi
 \item be modular (having a single task)
 \item have meaningful name
 \item have a comment describing their purpose, inputs and outputs
\ei
\eb

\end{frame}

%------------------------------------------------

\begin{frame}[fragile]
\frametitle{Functions: Example}

\begin{knitrout}\scriptsize
\definecolor{shadecolor}{rgb}{0.969, 0.969, 0.969}\color{fgcolor}\begin{kframe}
\begin{alltt}
\hlcom{# find mean of Y on the data z, Y in last column, and predict at xnew}
\hlstd{meany} \hlkwb{<-} \hlkwa{function}\hlstd{(}\hlkwc{predpt}\hlstd{,}\hlkwc{nearxy}\hlstd{) \{}
   \hlstd{ycol} \hlkwb{<-} \hlkwd{ncol}\hlstd{(nearxy)}
   \hlkwd{mean}\hlstd{(nearxy[,ycol])}
\hlstd{\}}

\hlcom{# find variance of Y in the neighborhood of predpt}
\hlstd{vary} \hlkwb{<-} \hlkwa{function}\hlstd{(}\hlkwc{predpt}\hlstd{,}\hlkwc{nearxy}\hlstd{) \{}
   \hlstd{ycol} \hlkwb{<-} \hlkwd{ncol}\hlstd{(nearxy)}
   \hlkwd{var}\hlstd{(nearxy[,ycol])}
\hlstd{\}}

\hlcom{# fit linear model to the data z, Y in last column, and predict at xnew}
\hlstd{loclin} \hlkwb{<-} \hlkwa{function}\hlstd{(}\hlkwc{predpt}\hlstd{,}\hlkwc{nearxy}\hlstd{) \{}
   \hlstd{ycol} \hlkwb{<-} \hlkwd{ncol}\hlstd{(nearxy)}
   \hlstd{bhat} \hlkwb{<-} \hlkwd{coef}\hlstd{(}\hlkwd{lm}\hlstd{(nearxy[,ycol]} \hlopt{~} \hlstd{nearxy[,}\hlopt{-}\hlstd{ycol]))}
   \hlkwd{c}\hlstd{(}\hlnum{1}\hlstd{,predpt)} \hlopt \hlstd{bhat}
\hlstd{\}}
\end{alltt}
\end{kframe}
\end{knitrout}

{\footnotesize source: Norm Matloff}

\end{frame}

%------------------------------------------------

\begin{frame}
\frametitle{Functions and Global Variables}

\bi
 \item Functions should not modify global variables
 \item except connections or environments
 \item should not change global \code{par()} settings
\ei

\end{frame}

%------------------------------------------------

\begin{frame}
\begin{center}
\Huge{\hilit{DRY}}
\end{center}
\end{frame}

%------------------------------------------------

\begin{frame}
\begin{center}
\Huge{Don't Repeat Yourself}
\end{center}
\end{frame}

%------------------------------------------------

\begin{frame}
\begin{center}
\Large{Every piece of knowledge must have a single, unambiguous, authoritative representation within a system.}
\end{center}
\end{frame}

%------------------------------------------------

\begin{frame}[fragile]
\frametitle{DRY}

\begin{knitrout}\scriptsize
\definecolor{shadecolor}{rgb}{0.969, 0.969, 0.969}\color{fgcolor}\begin{kframe}
\begin{alltt}
\hlcom{# avoid repetition}
\hlkwd{plot}\hlstd{(x, y,} \hlkwc{type} \hlstd{=} \hlstr{'n'}\hlstd{)}
\hlkwd{points}\hlstd{(x[size} \hlopt{==} \hlstr{'xsmall'}\hlstd{], y[size} \hlopt{==} \hlstr{'xsmall'}\hlstd{],} \hlkwc{col} \hlstd{=} \hlstr{'purple'}\hlstd{)}
\hlkwd{points}\hlstd{(x[size} \hlopt{==} \hlstr{'small'}\hlstd{], y[size} \hlopt{==} \hlstr{'small'}\hlstd{],} \hlkwc{col} \hlstd{=} \hlstr{'blue'}\hlstd{)}
\hlkwd{points}\hlstd{(x[size} \hlopt{==} \hlstr{'medium'}\hlstd{], y[size} \hlopt{==} \hlstr{'medium'}\hlstd{],} \hlkwc{col} \hlstd{=} \hlstr{'green'}\hlstd{)}
\hlkwd{points}\hlstd{(x[size} \hlopt{==} \hlstr{'large'}\hlstd{], y[size} \hlopt{==} \hlstr{'large'}\hlstd{],} \hlkwc{col} \hlstd{=} \hlstr{'orange'}\hlstd{)}
\hlkwd{points}\hlstd{(x[size} \hlopt{==} \hlstr{'xlarge'}\hlstd{], y[size} \hlopt{==} \hlstr{'xlarge'}\hlstd{],} \hlkwc{col} \hlstd{=} \hlstr{'red'}\hlstd{)}
\end{alltt}
\end{kframe}
\end{knitrout}

\pause
\begin{knitrout}\scriptsize
\definecolor{shadecolor}{rgb}{0.969, 0.969, 0.969}\color{fgcolor}\begin{kframe}
\begin{alltt}
\hlcom{# avoid repetition}
\hlstd{size_colors} \hlkwb{<-} \hlkwd{c}\hlstd{(}\hlstr{'purple'}\hlstd{,} \hlstr{'blue'}\hlstd{,} \hlstr{'green'}\hlstd{,} \hlstr{'orange'}\hlstd{,} \hlstr{'red'}\hlstd{)}
\hlkwd{plot}\hlstd{(x, y,} \hlkwc{type} \hlstd{=} \hlstr{'n'}\hlstd{)}
\hlkwa{for} \hlstd{(i} \hlkwa{in} \hlkwd{seq_along}\hlstd{(}\hlkwd{levels}\hlstd{(size))) \{}
  \hlkwd{points}\hlstd{(x[size} \hlopt{==} \hlstd{i], y[size} \hlopt{==} \hlstd{i],} \hlkwc{col} \hlstd{= size_colors[i])}
\hlstd{\}}
\end{alltt}
\end{kframe}
\end{knitrout}

\end{frame}

%------------------------------------------------

\begin{frame}[fragile]
\frametitle{RStudio Code Tools}
\begin{center}
\ig[width=10cm]{images/code_tools.png}
\end{center}
\end{frame}

%------------------------------------------------

\begin{frame}[fragile]
\frametitle{RStudio Shortcuts}
\begin{center}
\ig[width=10cm]{images/shortcuts.png}
\end{center}
\end{frame}

%------------------------------------------------

\begin{frame}
\frametitle{Strongly Recommended}

\bb{Look at other people's code}
\bi
 \item \url{https://github.com/hadley}
 \item \url{https://github.com/yihui}
 \item \url{https://github.com/karthik}
 \item \url{https://github.com/kbroman}
 \item \url{https://github.com/cboettig}
 \item \url{https://github.com/garrettgman}
\ei
\eb

\end{frame}

%------------------------------------------------

\begin{frame}
\frametitle{Your Own Style}

\bi
 \item It takes time to develop a personal style
 \item Try different styles and see which one best fits you
 \item Sometimes you have to adapt to a company's style
 \item There is no one single best style
\ei

\end{frame}

%------------------------------------------------

\begin{frame}[fragile]
\frametitle{Test Yourself}

What's wrong with this function?
\begin{knitrout}\footnotesize
\definecolor{shadecolor}{rgb}{0.969, 0.969, 0.969}\color{fgcolor}\begin{kframe}
\begin{alltt}
\hlstd{average} \hlkwb{<-} \hlkwa{function}\hlstd{(}\hlkwc{x}\hlstd{) \{}
  \hlstd{l} \hlkwb{<-} \hlkwd{length}\hlstd{(x)}
  \hlkwa{for}\hlstd{(i} \hlkwa{in} \hlstd{l) \{}
    \hlstd{y[i]} \hlkwb{<-} \hlstd{x[i]}\hlopt{/}\hlstd{l}
    \hlstd{z} \hlkwb{<-} \hlkwd{sum}\hlstd{(y[}\hlnum{1}\hlopt{:}\hlstd{l])}
    \hlkwd{return}\hlstd{(}\hlkwd{as.numeric}\hlstd{(z))}
  \hlstd{\}}
\hlstd{\}}
\end{alltt}
\end{kframe}
\end{knitrout}

\end{frame}

%------------------------------------------------

\begin{frame}[fragile]
\frametitle{Test Yourself}

What's wrong with this function?
\begin{knitrout}\footnotesize
\definecolor{shadecolor}{rgb}{0.969, 0.969, 0.969}\color{fgcolor}\begin{kframe}
\begin{alltt}
\hlstd{freq_table} \hlkwb{<-} \hlkwa{function}\hlstd{(}\hlkwc{x}\hlstd{) \{}
  \hlstd{table} \hlkwb{<-} \hlkwd{table}\hlstd{(x)}
  \hlstr{'category'} \hlkwb{<-} \hlkwd{levels}\hlstd{(x)}
  \hlstr{'count'} \hlkwb{<-} \hlkwd{print}\hlstd{(table)}
  \hlstr{'prop'} \hlkwb{<-} \hlstd{table}\hlopt{/}\hlkwd{length}\hlstd{(x)}
  \hlstr{'cumcount'} \hlkwb{<-} \hlkwd{print}\hlstd{(table)}
  \hlstr{'cumprop'} \hlkwb{<-} \hlstd{table}\hlopt{/}\hlkwd{length}\hlstd{(x)}
  \hlkwa{if}\hlstd{(}\hlkwd{is.factor}\hlstd{(x)) \{}
    \hlkwd{return}\hlstd{(}\hlkwd{data.frame}\hlstd{(}\hlkwc{rownames}\hlstd{=}\hlkwd{c}\hlstd{(}\hlstr{'category'}\hlstd{,} \hlstr{'count'}\hlstd{,}\hlstr{'prop'}\hlstd{,}
                                 \hlstr{'cumcount'}\hlstd{,}\hlstr{'cumprop'}\hlstd{)))}
  \hlstd{\}} \hlkwa{else} \hlstd{\{}
    \hlkwd{stop}\hlstd{(}\hlstr{'Not a factor'}\hlstd{)}
  \hlstd{\}}
\hlstd{\}}
\end{alltt}
\end{kframe}
\end{knitrout}

\end{frame}

%------------------------------------------------

\begin{frame}
\frametitle{Discussion}

\bbi
 \item What other suggestions do you have?
 \item How could we restructure the code, to make it easier to read?
 \item Grab a buddy and practice ``code review''. We do it for methods and papers, why not code? 
 \item Our code is a major scientific product and the result of a lot of hard work!
\ei

\end{frame}

%------------------------------------------------


\end{document}
