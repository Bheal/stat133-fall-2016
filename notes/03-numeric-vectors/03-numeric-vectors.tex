\documentclass[12pt]{beamer}\usepackage[]{graphicx}\usepackage[]{color}
%% maxwidth is the original width if it is less than linewidth
%% otherwise use linewidth (to make sure the graphics do not exceed the margin)
\makeatletter
\def\maxwidth{ %
  \ifdim\Gin@nat@width>\linewidth
    \linewidth
  \else
    \Gin@nat@width
  \fi
}
\makeatother

\definecolor{fgcolor}{rgb}{0.345, 0.345, 0.345}
\newcommand{\hlnum}[1]{\textcolor[rgb]{0.686,0.059,0.569}{#1}}%
\newcommand{\hlstr}[1]{\textcolor[rgb]{0.192,0.494,0.8}{#1}}%
\newcommand{\hlcom}[1]{\textcolor[rgb]{0.678,0.584,0.686}{\textit{#1}}}%
\newcommand{\hlopt}[1]{\textcolor[rgb]{0,0,0}{#1}}%
\newcommand{\hlstd}[1]{\textcolor[rgb]{0.345,0.345,0.345}{#1}}%
\newcommand{\hlkwa}[1]{\textcolor[rgb]{0.161,0.373,0.58}{\textbf{#1}}}%
\newcommand{\hlkwb}[1]{\textcolor[rgb]{0.69,0.353,0.396}{#1}}%
\newcommand{\hlkwc}[1]{\textcolor[rgb]{0.333,0.667,0.333}{#1}}%
\newcommand{\hlkwd}[1]{\textcolor[rgb]{0.737,0.353,0.396}{\textbf{#1}}}%
\let\hlipl\hlkwb

\usepackage{framed}
\makeatletter
\newenvironment{kframe}{%
 \def\at@end@of@kframe{}%
 \ifinner\ifhmode%
  \def\at@end@of@kframe{\end{minipage}}%
  \begin{minipage}{\columnwidth}%
 \fi\fi%
 \def\FrameCommand##1{\hskip\@totalleftmargin \hskip-\fboxsep
 \colorbox{shadecolor}{##1}\hskip-\fboxsep
     % There is no \\@totalrightmargin, so:
     \hskip-\linewidth \hskip-\@totalleftmargin \hskip\columnwidth}%
 \MakeFramed {\advance\hsize-\width
   \@totalleftmargin\z@ \linewidth\hsize
   \@setminipage}}%
 {\par\unskip\endMakeFramed%
 \at@end@of@kframe}
\makeatother

\definecolor{shadecolor}{rgb}{.97, .97, .97}
\definecolor{messagecolor}{rgb}{0, 0, 0}
\definecolor{warningcolor}{rgb}{1, 0, 1}
\definecolor{errorcolor}{rgb}{1, 0, 0}
\newenvironment{knitrout}{}{} % an empty environment to be redefined in TeX

\usepackage{alltt}
\usepackage{graphicx}
\usepackage{tikz}
\setbeameroption{hide notes}
\setbeamertemplate{note page}[plain]
\usepackage{listings}

% do not include "../" so Makefile does not crash
% get rid of junk
\usetheme{default}
\usefonttheme[onlymath]{serif}
\beamertemplatenavigationsymbolsempty
\hypersetup{pdfpagemode=UseNone} % don't show bookmarks on initial view

% named colors
\definecolor{offwhite}{RGB}{255,250,240}
\definecolor{gray}{RGB}{155,155,155}

\ifx\notescolors\undefined % slides

  \definecolor{foreground}{RGB}{80,80,80}
  \definecolor{background}{RGB}{255,255,255}
  \definecolor{title}{RGB}{255,199,0}
  \definecolor{subtitle}{RGB}{89,132,212}
  \definecolor{hilit}{RGB}{248,117,79}
  \definecolor{vhilit}{RGB}{255,111,207}
  \definecolor{lolit}{RGB}{200,200,200}
  \definecolor{lit}{RGB}{255,199,0}
  \definecolor{mdlit}{RGB}{89,132,212}
  \definecolor{link}{RGB}{248,117,79}

\else % notes
  \definecolor{background}{RGB}{255,255,255}
  \definecolor{foreground}{RGB}{24,24,24}
  \definecolor{title}{RGB}{27,94,134}
  \definecolor{subtitle}{RGB}{22,175,124}
  \definecolor{hilit}{RGB}{122,0,128}
  \definecolor{vhilit}{RGB}{255,0,128}
  \definecolor{lolit}{RGB}{95,95,95}
\fi
\definecolor{nhilit}{RGB}{128,0,128}  % hilit color in notes
\definecolor{nvhilit}{RGB}{255,0,128} % vhilit for notes

\newcommand{\hilit}{\color{hilit}}
\newcommand{\vhilit}{\color{vhilit}}
\newcommand{\nhilit}{\color{nhilit}}
\newcommand{\nvhilit}{\color{nvhilit}}
\newcommand{\lit}{\color{lit}}
\newcommand{\mdlit}{\color{mdlit}}
\newcommand{\lolit}{\color{lolit}}

% use those colors
\setbeamercolor{titlelike}{fg=title}
\setbeamercolor{subtitle}{fg=subtitle}
\setbeamercolor{frametitle}{fg=gray}
\setbeamercolor{structure}{fg=subtitle}
\setbeamercolor{institute}{fg=lolit}
\setbeamercolor{normal text}{fg=foreground,bg=background}
%\setbeamercolor{item}{fg=foreground} % color of bullets
%\setbeamercolor{subitem}{fg=hilit}
%\setbeamercolor{itemize/enumerate subbody}{fg=lolit}
\setbeamertemplate{itemize subitem}{{\textendash}}
\setbeamerfont{itemize/enumerate subbody}{size=\footnotesize}
\setbeamerfont{itemize/enumerate subitem}{size=\footnotesize}

% center title of slides
\setbeamertemplate{blocks}[rounded]
\setbeamertemplate{frametitle}[default][center]
% margins
\setbeamersize{text margin left=25pt,text margin right=25pt}

% page number
\setbeamertemplate{footline}{%
    \raisebox{5pt}{\makebox[\paperwidth]{\hfill\makebox[20pt]{\lolit
          \scriptsize\insertframenumber}}}\hspace*{5pt}}

% add a bit of space at the top of the notes page
\addtobeamertemplate{note page}{\setlength{\parskip}{12pt}}

% default link color
\hypersetup{colorlinks, urlcolor={link}}

\ifx\notescolors\undefined % slides
  % set up listing environment
  \lstset{language=bash,
          basicstyle=\ttfamily\scriptsize,
          frame=single,
          commentstyle=,
          backgroundcolor=\color{darkgray},
          showspaces=false,
          showstringspaces=false
          }
\else % notes
  \lstset{language=bash,
          basicstyle=\ttfamily\scriptsize,
          frame=single,
          commentstyle=,
          backgroundcolor=\color{offwhite},
          showspaces=false,
          showstringspaces=false
          }
\fi

% a few macros
\newcommand{\code}[1]{\texttt{#1}}
\newcommand{\hicode}[1]{{\hilit \texttt{#1}}}
\newcommand{\bb}[1]{\begin{block}{#1}}
\newcommand{\eb}{\end{block}}
\newcommand{\bi}{\begin{itemize}}
%\newcommand{\bbi}{\vspace{24pt} \begin{itemize} \itemsep8pt}
\newcommand{\bbi}{\vspace{4pt} \begin{itemize} \itemsep8pt}
\newcommand{\ei}{\end{itemize}}
\newcommand{\bv}{\begin{verbatim}}
\newcommand{\ev}{\end{verbatim}}
\newcommand{\ig}{\includegraphics}
\newcommand{\subt}[1]{{\footnotesize \color{subtitle} {#1}}}
\newcommand{\ttsm}{\tt \small}
\newcommand{\ttfn}{\tt \footnotesize}
\newcommand{\figh}[2]{\centerline{\includegraphics[height=#2\textheight]{#1}}}
\newcommand{\figw}[2]{\centerline{\includegraphics[width=#2\textwidth]{#1}}}



%------------------------------------------------
% end of header
%------------------------------------------------

\title{Numeric Vectors}
\subtitle{STAT 133}
\author{\href{http://www.gastonsanchez.com}{Gaston Sanchez}}
\institute{\href{https://github.com/ucb-stat133/stat133-fall-2016}{\tt \scriptsize \color{foreground} github.com/ucb-stat133/stat133-fall-2016}}
\date{}
\IfFileExists{upquote.sty}{\usepackage{upquote}}{}
\begin{document}


{
  \setbeamertemplate{footline}{} % no page number here
  \frame{
    \titlepage
  } 
}

%------------------------------------------------

\begin{frame}
\frametitle{Data Types and Structures}

To make the best of the R language, you'll need a strong understanding of the basic \textbf{data types} and \textbf{data structures} and how to operate on them.

\end{frame}

%------------------------------------------------

\begin{frame}
\begin{center}
\Huge{\hilit{Vectors}}
\end{center}
\end{frame}

%------------------------------------------------

\begin{frame}
\frametitle{Vectors Reminder}

\bi
  \item A vector is the most basic data structure in R
  \item Vectors are contiguous cells containing data
  \item Can be of any length (including zero)
  \item R has five basic type of vectors: \\
  integer, double, complex, logical, character
  \item vectors are \textbf{atomic} structures
  \item the values in a vector must be ALL of the same type 
\ei

\end{frame}

%------------------------------------------------

\begin{frame}[fragile]
\frametitle{Vectors}

The most simple type of vectors are scalars or single values:
\begin{knitrout}\footnotesize
\definecolor{shadecolor}{rgb}{0.969, 0.969, 0.969}\color{fgcolor}\begin{kframe}
\begin{alltt}
\hlcom{# integer}
\hlstd{x} \hlkwb{<-} \hlnum{1L}
\hlcom{# double (real)}
\hlstd{y} \hlkwb{<-} \hlnum{5}
\hlcom{# complex}
\hlstd{z} \hlkwb{<-} \hlnum{3} \hlopt{+} \hlnum{5i}
\hlcom{# logical}
\hlstd{a} \hlkwb{<-} \hlnum{TRUE}
\hlcom{# character}
\hlstd{b} \hlkwb{<-} \hlstr{"yosemite"}
\end{alltt}
\end{kframe}
\end{knitrout}

\end{frame}

%------------------------------------------------

\begin{frame}[fragile]
\frametitle{Vectors}

The function to create a vector from individual values is {\hilit \code{c()}}, short for \textbf{concatenate}:

\begin{knitrout}\footnotesize
\definecolor{shadecolor}{rgb}{0.969, 0.969, 0.969}\color{fgcolor}\begin{kframe}
\begin{alltt}
\hlcom{# some vectors}
\hlstd{x} \hlkwb{<-} \hlkwd{c}\hlstd{(}\hlnum{1}\hlstd{,} \hlnum{2}\hlstd{,} \hlnum{3}\hlstd{,} \hlnum{4}\hlstd{,} \hlnum{5}\hlstd{)}

\hlstd{y} \hlkwb{<-} \hlkwd{c}\hlstd{(}\hlstr{"one"}\hlstd{,} \hlstr{"two"}\hlstd{,} \hlstr{"three"}\hlstd{)}

\hlstd{z} \hlkwb{<-} \hlkwd{c}\hlstd{(}\hlnum{TRUE}\hlstd{,} \hlnum{FALSE}\hlstd{,} \hlnum{FALSE}\hlstd{)}
\end{alltt}
\end{kframe}
\end{knitrout}

\end{frame}

%------------------------------------------------

\begin{frame}[fragile]
\frametitle{Atomic Vectors}

If you mix different data values, R will coerce them so they are all of the same type
\begin{knitrout}\footnotesize
\definecolor{shadecolor}{rgb}{0.969, 0.969, 0.969}\color{fgcolor}\begin{kframe}
\begin{alltt}
\hlcom{# mixing numbers and characters}
\hlstd{x} \hlkwb{<-} \hlkwd{c}\hlstd{(}\hlnum{1}\hlstd{,} \hlnum{2}\hlstd{,} \hlnum{3}\hlstd{,} \hlstr{"four"}\hlstd{,} \hlstr{"five"}\hlstd{)}

\hlcom{# mixing numbers and logical values}
\hlstd{y} \hlkwb{<-} \hlkwd{c}\hlstd{(}\hlnum{TRUE}\hlstd{,} \hlnum{FALSE}\hlstd{,} \hlnum{3}\hlstd{,} \hlnum{4}\hlstd{)}

\hlcom{# mixing numbers and logical values}
\hlstd{z} \hlkwb{<-} \hlkwd{c}\hlstd{(}\hlnum{TRUE}\hlstd{,} \hlnum{FALSE}\hlstd{,} \hlstr{"TRUE"}\hlstd{,} \hlstr{"FALSE"}\hlstd{)}

\hlcom{# mixing integer, real, and complex numbers}
\hlstd{w} \hlkwb{<-} \hlkwd{c}\hlstd{(}\hlnum{1L}\hlstd{,} \hlopt{-}\hlnum{0.5}\hlstd{,} \hlnum{3} \hlopt{+} \hlnum{5i}\hlstd{)}
\end{alltt}
\end{kframe}
\end{knitrout}

\end{frame}

%------------------------------------------------

\begin{frame}[fragile]
\frametitle{Vectors of a given class}

Sometimes is useful to initialize vectors of a particular class by simply specifying the number of elements:
\begin{knitrout}\footnotesize
\definecolor{shadecolor}{rgb}{0.969, 0.969, 0.969}\color{fgcolor}\begin{kframe}
\begin{alltt}
\hlcom{# five element vectors }
\hlstd{int} \hlkwb{<-} \hlkwd{integer}\hlstd{(}\hlnum{5}\hlstd{)}
\hlstd{num} \hlkwb{<-} \hlkwd{numeric}\hlstd{(}\hlnum{5}\hlstd{)}
\hlstd{comp} \hlkwb{<-} \hlkwd{complex}\hlstd{(}\hlnum{5}\hlstd{)}
\hlstd{logi} \hlkwb{<-} \hlkwd{logical}\hlstd{(}\hlnum{5}\hlstd{)}
\hlstd{char} \hlkwb{<-} \hlkwd{character}\hlstd{(}\hlnum{5}\hlstd{)}
\end{alltt}
\end{kframe}
\end{knitrout}

\end{frame}

%------------------------------------------------

\begin{frame}[fragile]
\frametitle{Vector class functions}

\bi
  \item \code{integer()}, \code{is.integer()}, \code{as.integer()}
  \item \code{numeric()}, \code{is.numeric()}, \code{as.numeric()}
  \item \code{complex()}, \code{is.complex()}, \code{as.complex()}
  \item \code{logical()}, \code{is.logical()}, \code{as.logical()}
  \item \code{character()}, \code{is.character()}, \code{as.character()}
\ei

\end{frame}

%------------------------------------------------

\begin{frame}[fragile]
\frametitle{Numeric Vectors}

Vectors of sequence of \textbf{integers} can be created with the colon operator {\hilit \code{":"}}
\begin{knitrout}\footnotesize
\definecolor{shadecolor}{rgb}{0.969, 0.969, 0.969}\color{fgcolor}\begin{kframe}
\begin{alltt}
\hlcom{# positive: from 1 to 5}
\hlnum{1}\hlopt{:}\hlnum{5}

\hlcom{# negative: from -7 to -2}
\hlopt{-}\hlnum{7}\hlopt{:-}\hlnum{2}

\hlcom{# decreasing: from 3 to -3}
\hlnum{3}\hlopt{:-}\hlnum{3}
\end{alltt}
\end{kframe}
\end{knitrout}

\end{frame}

%------------------------------------------------

\begin{frame}[fragile]
\frametitle{Numeric Vectors}

More vectors of numeric sequences (not only integers) can be created with the function {\hilit \code{seq()}}
\begin{knitrout}\footnotesize
\definecolor{shadecolor}{rgb}{0.969, 0.969, 0.969}\color{fgcolor}\begin{kframe}
\begin{alltt}
\hlcom{# sequences}
\hlkwd{seq}\hlstd{(}\hlnum{1}\hlstd{)}
\hlkwd{seq}\hlstd{(}\hlkwc{from} \hlstd{=} \hlnum{1}\hlstd{,} \hlkwc{to} \hlstd{=} \hlnum{5}\hlstd{)}
\hlkwd{seq}\hlstd{(}\hlkwc{from} \hlstd{=} \hlopt{-}\hlnum{3}\hlstd{,} \hlkwc{to} \hlstd{=} \hlnum{9}\hlstd{)}
\hlkwd{seq}\hlstd{(}\hlkwc{from} \hlstd{=} \hlopt{-}\hlnum{3}\hlstd{,} \hlkwc{to} \hlstd{=} \hlnum{9}\hlstd{,} \hlkwc{by} \hlstd{=} \hlnum{2}\hlstd{)}
\hlkwd{seq}\hlstd{(}\hlkwc{from} \hlstd{=} \hlopt{-}\hlnum{3}\hlstd{,} \hlkwc{to} \hlstd{=} \hlnum{3}\hlstd{,} \hlkwc{by} \hlstd{=} \hlnum{0.5}\hlstd{)}
\hlkwd{seq}\hlstd{(}\hlkwc{from} \hlstd{=} \hlnum{1}\hlstd{,} \hlkwc{to} \hlstd{=} \hlnum{20}\hlstd{,} \hlkwc{length.out} \hlstd{=} \hlnum{5}\hlstd{)}
\end{alltt}
\end{kframe}
\end{knitrout}

\end{frame}

%------------------------------------------------

\begin{frame}[fragile]
\frametitle{Sequence generation}

Two sequencing variants of \code{seq()} are \code{seq\_along()} and \code{seq\_len()}
\bi
  \item \code{seq\_along()} returns a sequence of integers of the same length as its argument
  \item \code{seq\_len()} generates a sequence from 1 to the value provided
\ei

\end{frame}

%------------------------------------------------

\begin{frame}[fragile]
\frametitle{Sequence generation}

\begin{knitrout}\footnotesize
\definecolor{shadecolor}{rgb}{0.969, 0.969, 0.969}\color{fgcolor}\begin{kframe}
\begin{alltt}
\hlcom{# some flavors}
\hlstd{flavors} \hlkwb{<-} \hlkwd{c}\hlstd{(}\hlstr{"chocolate"}\hlstd{,} \hlstr{"vanilla"}\hlstd{,} \hlstr{"lemon"}\hlstd{)}

\hlcom{# sequence of integers from flavors}
\hlkwd{seq_along}\hlstd{(flavors)}
\end{alltt}
\begin{verbatim}
## [1] 1 2 3
\end{verbatim}
\begin{alltt}
\hlcom{# sequence from 1 to 5}
\hlkwd{seq_len}\hlstd{(}\hlnum{5}\hlstd{)}
\end{alltt}
\begin{verbatim}
## [1] 1 2 3 4 5
\end{verbatim}
\end{kframe}
\end{knitrout}

\end{frame}

%------------------------------------------------

\begin{frame}[fragile]
\frametitle{Replicate elements}

Another way to create vectors is with the replicating function {\hilit \code{rep()}}
\begin{knitrout}\footnotesize
\definecolor{shadecolor}{rgb}{0.969, 0.969, 0.969}\color{fgcolor}\begin{kframe}
\begin{alltt}
\hlkwd{rep}\hlstd{(}\hlnum{1}\hlstd{,} \hlkwc{times} \hlstd{=} \hlnum{5}\hlstd{)}
\hlkwd{rep}\hlstd{(}\hlkwd{c}\hlstd{(}\hlnum{2}\hlstd{,} \hlnum{4}\hlstd{,} \hlnum{6}\hlstd{),} \hlkwc{times} \hlstd{=} \hlnum{2}\hlstd{)}
\hlkwd{rep}\hlstd{(}\hlnum{1}\hlopt{:}\hlnum{3}\hlstd{,} \hlkwc{times} \hlstd{=} \hlkwd{c}\hlstd{(}\hlnum{3}\hlstd{,} \hlnum{2}\hlstd{,} \hlnum{1}\hlstd{))}
\hlkwd{rep}\hlstd{(}\hlkwd{c}\hlstd{(}\hlnum{2}\hlstd{,} \hlnum{4}\hlstd{,} \hlnum{6}\hlstd{),} \hlkwc{each} \hlstd{=} \hlnum{2}\hlstd{)}
\hlkwd{rep}\hlstd{(}\hlkwd{c}\hlstd{(}\hlnum{2}\hlstd{,} \hlnum{4}\hlstd{,} \hlnum{6}\hlstd{),} \hlkwc{length.out} \hlstd{=} \hlnum{5}\hlstd{)}
\hlkwd{rep}\hlstd{(}\hlkwd{c}\hlstd{(}\hlnum{2}\hlstd{,} \hlnum{4}\hlstd{,} \hlnum{6}\hlstd{),} \hlkwc{each} \hlstd{=} \hlnum{2}\hlstd{,} \hlkwc{times} \hlstd{=} \hlnum{2}\hlstd{)}
\end{alltt}
\end{kframe}
\end{knitrout}

\end{frame}

%------------------------------------------------

\begin{frame}[fragile]
\frametitle{Random Vectors}

R provides a series of random number generation functions that can also be used to create numeric vectors
\begin{center}
 \begin{tabular}{l l}
  \hline
   generator & distribution \\
  \hline
  \code{runif()} & uniform \\
  \code{rnorm()} & normal \\
  \code{rbinom()} & binomial \\  
  \code{rbeta()} & beta \\
  \code{rgamma()} & gamma \\
  \code{rgeom()} & geometric \\
  \hline
 \end{tabular}
\end{center}

Check \code{help(?Distributions)} to see the list of all the available distributions
\end{frame}

%------------------------------------------------

\begin{frame}[fragile]
\frametitle{Random Vectors}

\begin{knitrout}\footnotesize
\definecolor{shadecolor}{rgb}{0.969, 0.969, 0.969}\color{fgcolor}\begin{kframe}
\begin{alltt}
\hlkwd{runif}\hlstd{(}\hlkwc{n} \hlstd{=} \hlnum{5}\hlstd{,} \hlkwc{min} \hlstd{=} \hlnum{0}\hlstd{,} \hlkwc{max} \hlstd{=} \hlnum{1}\hlstd{)}

\hlkwd{rnorm}\hlstd{(}\hlkwc{n} \hlstd{=} \hlnum{5}\hlstd{,} \hlkwc{mean} \hlstd{=} \hlnum{0}\hlstd{,} \hlkwc{sd} \hlstd{=} \hlnum{1}\hlstd{)}

\hlkwd{rbinom}\hlstd{(}\hlkwc{n} \hlstd{=} \hlnum{5}\hlstd{,} \hlkwc{size} \hlstd{=} \hlnum{1}\hlstd{,} \hlkwc{prob} \hlstd{=} \hlnum{0.5}\hlstd{)}

\hlkwd{rbeta}\hlstd{(}\hlkwc{n} \hlstd{=} \hlnum{5}\hlstd{,} \hlkwc{shape1} \hlstd{=} \hlnum{0.5}\hlstd{,} \hlkwc{shape2} \hlstd{=} \hlnum{0.5}\hlstd{)}
\end{alltt}
\end{kframe}
\end{knitrout}

\end{frame}

%------------------------------------------------

\begin{frame}[fragile]
\frametitle{Sampled Vectors}

There's also the function \code{sample()} that generates random samples (with and without replacement)
\begin{knitrout}\footnotesize
\definecolor{shadecolor}{rgb}{0.969, 0.969, 0.969}\color{fgcolor}\begin{kframe}
\begin{alltt}
\hlcom{# shuffle}
\hlkwd{sample}\hlstd{(}\hlnum{1}\hlopt{:}\hlnum{10}\hlstd{,} \hlkwc{size} \hlstd{=} \hlnum{10}\hlstd{)}

\hlcom{# sample with replacement}
\hlstd{values} \hlkwb{<-} \hlkwd{c}\hlstd{(}\hlnum{2}\hlstd{,} \hlnum{3}\hlstd{,} \hlnum{6}\hlstd{,} \hlnum{7}\hlstd{,} \hlnum{9}\hlstd{)}
\hlkwd{sample}\hlstd{(values,} \hlkwc{size} \hlstd{=} \hlnum{20}\hlstd{,} \hlkwc{replace} \hlstd{=} \hlnum{TRUE}\hlstd{)}
\end{alltt}
\end{kframe}
\end{knitrout}

\end{frame}

%------------------------------------------------

\begin{frame}
\begin{center}
\Huge{\hilit{Vector Functions}}
\end{center}
\end{frame}

%------------------------------------------------

\begin{frame}[fragile]
\frametitle{Basic Vector Functions}

\bi
  \item \code{length()}
  \item \code{sort()}
  \item \code{rev()}
  \item \code{order()}
  \item \code{unique()}
  \item \code{duplicated()}
\ei

\end{frame}

%------------------------------------------------

\begin{frame}[fragile]
\frametitle{Basic Vector Functions}

\begin{knitrout}\footnotesize
\definecolor{shadecolor}{rgb}{0.969, 0.969, 0.969}\color{fgcolor}\begin{kframe}
\begin{alltt}
\hlcom{# numeric vector}
\hlstd{num} \hlkwb{<-} \hlkwd{c}\hlstd{(}\hlnum{9}\hlstd{,} \hlnum{4}\hlstd{,} \hlnum{5}\hlstd{,} \hlnum{1}\hlstd{,} \hlnum{4}\hlstd{,} \hlnum{1}\hlstd{,} \hlnum{4}\hlstd{,} \hlnum{7}\hlstd{)}

\hlcom{# how many elements?}
\hlkwd{length}\hlstd{(num)}
\end{alltt}
\begin{verbatim}
## [1] 8
\end{verbatim}
\begin{alltt}
\hlcom{# sorting elements}
\hlkwd{sort}\hlstd{(num)}
\end{alltt}
\begin{verbatim}
## [1] 1 1 4 4 4 5 7 9
\end{verbatim}
\begin{alltt}
\hlkwd{sort}\hlstd{(num,} \hlkwc{decreasing} \hlstd{=} \hlnum{TRUE}\hlstd{)}
\end{alltt}
\begin{verbatim}
## [1] 9 7 5 4 4 4 1 1
\end{verbatim}
\end{kframe}
\end{knitrout}

\end{frame}

%------------------------------------------------

\begin{frame}[fragile]
\frametitle{Basic Vector Functions}

\begin{knitrout}\footnotesize
\definecolor{shadecolor}{rgb}{0.969, 0.969, 0.969}\color{fgcolor}\begin{kframe}
\begin{alltt}
\hlcom{# reversed elements}
\hlkwd{rev}\hlstd{(num)}
\end{alltt}
\begin{verbatim}
## [1] 7 4 1 4 1 5 4 9
\end{verbatim}
\begin{alltt}
\hlcom{# position of sorted elements}
\hlkwd{order}\hlstd{(num)}
\end{alltt}
\begin{verbatim}
## [1] 4 6 2 5 7 3 8 1
\end{verbatim}
\begin{alltt}
\hlkwd{order}\hlstd{(num,} \hlkwc{decreasing} \hlstd{=} \hlnum{TRUE}\hlstd{)}
\end{alltt}
\begin{verbatim}
## [1] 1 8 3 2 5 7 4 6
\end{verbatim}
\end{kframe}
\end{knitrout}

\end{frame}

%------------------------------------------------

\begin{frame}[fragile]
\frametitle{Basic Vector Functions}

\begin{knitrout}\footnotesize
\definecolor{shadecolor}{rgb}{0.969, 0.969, 0.969}\color{fgcolor}\begin{kframe}
\begin{alltt}
\hlcom{# unique elements}
\hlkwd{unique}\hlstd{(num)}
\end{alltt}
\begin{verbatim}
## [1] 9 4 5 1 7
\end{verbatim}
\begin{alltt}
\hlcom{# duplicated elements}
\hlkwd{duplicated}\hlstd{(num)}
\end{alltt}
\begin{verbatim}
## [1] FALSE FALSE FALSE FALSE  TRUE  TRUE  TRUE FALSE
\end{verbatim}
\begin{alltt}
\hlstd{num[}\hlkwd{duplicated}\hlstd{(num)]}
\end{alltt}
\begin{verbatim}
## [1] 4 1 4
\end{verbatim}
\end{kframe}
\end{knitrout}

\end{frame}

%------------------------------------------------

\begin{frame}
\begin{center}
\Huge{\hilit{Math Operations}}
\end{center}
\end{frame}

%------------------------------------------------

\begin{frame}
\frametitle{Arithmetic Operators}

\begin{center}
 \begin{tabular}{l l}
  \hline
   operation & usage \\
  \hline
  unary + & \code{+ x} \\
  unary - & \code{- x} \\
  sum & \code{x + y} \\  
  subtraction & \code{x - y} \\
  multiplication & \code{x * y} \\
  division & \code{x / y} \\
  power & \code{x \^{} y} \\
  modul0 (remainder) & \code{x \%\% y} \\
  integer division & \code{x \%/\% y} \\
  \hline
 \end{tabular}
\end{center}

\end{frame}

%------------------------------------------------

\begin{frame}[fragile]
\frametitle{Arithmetic Operators}

\begin{knitrout}\footnotesize
\definecolor{shadecolor}{rgb}{0.969, 0.969, 0.969}\color{fgcolor}\begin{kframe}
\begin{alltt}
\hlopt{+}\hlnum{2}
\hlopt{-}\hlnum{2}
\hlnum{2} \hlopt{+} \hlnum{3}
\hlnum{2} \hlopt{-} \hlnum{3}
\hlnum{2} \hlopt{*} \hlnum{3}
\hlnum{2} \hlopt{/} \hlnum{3}
\hlnum{2} \hlopt{^} \hlnum{3}
\hlnum{2} \hlopt \hlnum{3}
\hlnum{2} \hlopt \hlnum{3}
\end{alltt}
\end{kframe}
\end{knitrout}

\end{frame}

%------------------------------------------------

\begin{frame}[fragile]
\frametitle{Math Functions}

\bi
  \item \code{abs(), sign(), sqrt()}
  \item \code{ceiling(), floor(), trunc(), round(), signif()}
  \item \code{cummax(), cummin(), cumprod(), cumsum()}
  \item \code{log(), log10(), log2(), log1p()}
  \item \code{sin(), cos(), tan()}
  \item \code{acos(), acosh(), asin(), asinh(), atan(), atanh()}
  \item \code{exp(), expm1()}
  \item \code{gamma(), lgamma(), digamma(), trigamma()}
\ei

\end{frame}

%------------------------------------------------

\begin{frame}[fragile]
\frametitle{Math Functions}

\begin{knitrout}\footnotesize
\definecolor{shadecolor}{rgb}{0.969, 0.969, 0.969}\color{fgcolor}\begin{kframe}
\begin{alltt}
\hlkwd{abs}\hlstd{(}\hlkwd{c}\hlstd{(}\hlopt{-}\hlnum{1}\hlstd{,} \hlopt{-}\hlnum{0.5}\hlstd{,} \hlnum{3}\hlstd{,} \hlnum{0.5}\hlstd{))}
\end{alltt}
\begin{verbatim}
## [1] 1.0 0.5 3.0 0.5
\end{verbatim}
\begin{alltt}
\hlkwd{sign}\hlstd{(}\hlkwd{c}\hlstd{(}\hlopt{-}\hlnum{1}\hlstd{,} \hlopt{-}\hlnum{0.5}\hlstd{,} \hlnum{3}\hlstd{,} \hlnum{0.5}\hlstd{))}
\end{alltt}
\begin{verbatim}
## [1] -1 -1  1  1
\end{verbatim}
\begin{alltt}
\hlkwd{round}\hlstd{(}\hlnum{3.14159}\hlstd{,} \hlnum{1}\hlstd{)}
\end{alltt}
\begin{verbatim}
## [1] 3.1
\end{verbatim}
\begin{alltt}
\hlkwd{log10}\hlstd{(}\hlnum{10}\hlstd{)}
\end{alltt}
\begin{verbatim}
## [1] 1
\end{verbatim}
\end{kframe}
\end{knitrout}

\end{frame}

%------------------------------------------------

\begin{frame}
\begin{center}
\Huge{\hilit{Vectorization}}
\end{center}
\end{frame}

%------------------------------------------------

\begin{frame}[fragile]
\frametitle{Vectorized Operations}

A vectorized computation is any computation that when applied to a vector operates on all of its elements
\begin{knitrout}\footnotesize
\definecolor{shadecolor}{rgb}{0.969, 0.969, 0.969}\color{fgcolor}\begin{kframe}
\begin{alltt}
\hlkwd{c}\hlstd{(}\hlnum{1}\hlstd{,} \hlnum{2}\hlstd{,} \hlnum{3}\hlstd{)} \hlopt{+} \hlkwd{c}\hlstd{(}\hlnum{3}\hlstd{,} \hlnum{2}\hlstd{,} \hlnum{1}\hlstd{)}
\end{alltt}
\begin{verbatim}
## [1] 4 4 4
\end{verbatim}
\begin{alltt}
\hlkwd{c}\hlstd{(}\hlnum{1}\hlstd{,} \hlnum{2}\hlstd{,} \hlnum{3}\hlstd{)} \hlopt{*} \hlkwd{c}\hlstd{(}\hlnum{3}\hlstd{,} \hlnum{2}\hlstd{,} \hlnum{1}\hlstd{)}
\end{alltt}
\begin{verbatim}
## [1] 3 4 3
\end{verbatim}
\begin{alltt}
\hlkwd{c}\hlstd{(}\hlnum{1}\hlstd{,} \hlnum{2}\hlstd{,} \hlnum{3}\hlstd{)} \hlopt{^} \hlkwd{c}\hlstd{(}\hlnum{3}\hlstd{,} \hlnum{2}\hlstd{,} \hlnum{1}\hlstd{)}
\end{alltt}
\begin{verbatim}
## [1] 1 4 3
\end{verbatim}
\end{kframe}
\end{knitrout}

\end{frame}

%------------------------------------------------

\begin{frame}[fragile]
\frametitle{Vectorization}

All arithmetic, trigonometric, math and other vector functions are vectorized:
\begin{knitrout}\footnotesize
\definecolor{shadecolor}{rgb}{0.969, 0.969, 0.969}\color{fgcolor}\begin{kframe}
\begin{alltt}
\hlkwd{log}\hlstd{(}\hlkwd{c}\hlstd{(}\hlnum{1}\hlstd{,} \hlnum{2}\hlstd{,} \hlnum{3}\hlstd{))}
\end{alltt}
\begin{verbatim}
## [1] 0.0000000 0.6931472 1.0986123
\end{verbatim}
\begin{alltt}
\hlkwd{cos}\hlstd{(}\hlkwd{seq}\hlstd{(}\hlnum{1}\hlstd{,} \hlnum{3}\hlstd{))}
\end{alltt}
\begin{verbatim}
## [1]  0.5403023 -0.4161468 -0.9899925
\end{verbatim}
\begin{alltt}
\hlkwd{sqrt}\hlstd{(}\hlnum{1}\hlopt{:}\hlnum{3}\hlstd{)}
\end{alltt}
\begin{verbatim}
## [1] 1.000000 1.414214 1.732051
\end{verbatim}
\end{kframe}
\end{knitrout}

\end{frame}

%------------------------------------------------

\begin{frame}[fragile]
\frametitle{Recycling}

When vectorized computations are applied, some problems may occur when dealing with two vectors of different length
\begin{knitrout}\footnotesize
\definecolor{shadecolor}{rgb}{0.969, 0.969, 0.969}\color{fgcolor}\begin{kframe}
\begin{alltt}
\hlkwd{c}\hlstd{(}\hlnum{2}\hlstd{,} \hlnum{1}\hlstd{)} \hlopt{+} \hlkwd{c}\hlstd{(}\hlnum{1}\hlstd{,} \hlnum{2}\hlstd{,} \hlnum{3}\hlstd{)}
\end{alltt}


{\ttfamily\noindent\color{warningcolor}{\#\# Warning in c(2, 1) + c(1, 2, 3): longer object length is not a multiple of shorter object length}}\begin{verbatim}
## [1] 3 3 5
\end{verbatim}
\end{kframe}
\end{knitrout}

\end{frame}

%------------------------------------------------

\begin{frame}[fragile]
\frametitle{Recycling Rule}

The recycling rule states that the shorter vector is replicated enough times so that the result has the length of the longer vector
\begin{knitrout}\footnotesize
\definecolor{shadecolor}{rgb}{0.969, 0.969, 0.969}\color{fgcolor}\begin{kframe}
\begin{alltt}
\hlkwd{c}\hlstd{(}\hlnum{1}\hlstd{,} \hlnum{2}\hlstd{,} \hlnum{3}\hlstd{,} \hlnum{4}\hlstd{)} \hlopt{+} \hlkwd{c}\hlstd{(}\hlnum{2}\hlstd{,} \hlnum{1}\hlstd{)}
\end{alltt}
\begin{verbatim}
## [1] 3 3 5 5
\end{verbatim}
\begin{alltt}
\hlnum{1}\hlopt{:}\hlnum{10} \hlopt{*} \hlnum{1}\hlopt{:}\hlnum{5}
\end{alltt}
\begin{verbatim}
##  [1]  1  4  9 16 25  6 14 24 36 50
\end{verbatim}
\end{kframe}
\end{knitrout}

\end{frame}

%------------------------------------------------

\begin{frame}[fragile]
\frametitle{Recycling Rule}

The Recycling Rule can be very useful, like when operating between a vector and a ``scalar"
\begin{knitrout}\footnotesize
\definecolor{shadecolor}{rgb}{0.969, 0.969, 0.969}\color{fgcolor}\begin{kframe}
\begin{alltt}
\hlstd{x} \hlkwb{<-} \hlkwd{c}\hlstd{(}\hlnum{2}\hlstd{,} \hlnum{4}\hlstd{,} \hlnum{6}\hlstd{,} \hlnum{8}\hlstd{)}
\hlstd{x} \hlopt{+} \hlnum{3}  \hlcom{# add 3 to all elements in x}
\end{alltt}
\begin{verbatim}
## [1]  5  7  9 11
\end{verbatim}
\begin{alltt}
\hlstd{x} \hlopt{/} \hlnum{3}  \hlcom{# divide all elemnts by 3}
\end{alltt}
\begin{verbatim}
## [1] 0.6666667 1.3333333 2.0000000 2.6666667
\end{verbatim}
\begin{alltt}
\hlstd{x} \hlopt{^} \hlnum{3}  \hlcom{# all elements to the power of 3}
\end{alltt}
\begin{verbatim}
## [1]   8  64 216 512
\end{verbatim}
\end{kframe}
\end{knitrout}

\end{frame}

%------------------------------------------------

\begin{frame}
\frametitle{Comparison Operators}

\begin{center}
 \begin{tabular}{l l}
  \hline
   operation & usage \\
  \hline
  less than & \code{x < x} \\
  greater than & \code{x > y} \\
  less than or equal & \code{x <= y} \\  
  greater than or equal  & \code{x >= y} \\
  equality & \code{x == y} \\
  different & \code{x != y} \\
  \hline
 \end{tabular}
\end{center}

Comparison operators produce logical values

\end{frame}

%------------------------------------------------

\begin{frame}[fragile]
\frametitle{Comparison Operators}

\begin{columns}[t]
\begin{column}{0.4\textwidth}
\begin{knitrout}\footnotesize
\definecolor{shadecolor}{rgb}{0.969, 0.969, 0.969}\color{fgcolor}\begin{kframe}
\begin{alltt}
\hlnum{5} \hlopt{>} \hlnum{1}
\hlnum{5} \hlopt{<} \hlnum{7}
\hlnum{5} \hlopt{>} \hlnum{10}
\hlnum{5} \hlopt{>=} \hlnum{5}
\hlnum{5} \hlopt{<=} \hlnum{5}
\hlnum{5} \hlopt{==} \hlnum{5}
\hlnum{5} \hlopt{!=} \hlnum{3}
\hlnum{5} \hlopt{!=} \hlnum{5}
\end{alltt}
\end{kframe}
\end{knitrout}
\end{column}

\begin{column}{0.4\textwidth}
\begin{knitrout}\footnotesize
\definecolor{shadecolor}{rgb}{0.969, 0.969, 0.969}\color{fgcolor}\begin{kframe}
\begin{alltt}
\hlnum{TRUE} \hlopt{>} \hlnum{FALSE}
\hlnum{TRUE} \hlopt{<} \hlnum{FALSE}
\hlnum{TRUE} \hlopt{==} \hlnum{TRUE}
\hlnum{TRUE} \hlopt{!=} \hlnum{FALSE}
\hlnum{TRUE} \hlopt{!=} \hlnum{TRUE}
\end{alltt}
\end{kframe}
\end{knitrout}
\end{column}
\end{columns}

\end{frame}

%------------------------------------------------

\begin{frame}[fragile]
\frametitle{Comparison Operators}

Comparison Operators are also vectorized 
\begin{knitrout}\footnotesize
\definecolor{shadecolor}{rgb}{0.969, 0.969, 0.969}\color{fgcolor}\begin{kframe}
\begin{alltt}
\hlstd{values} \hlkwb{<-} \hlopt{-}\hlnum{3}\hlopt{:}\hlnum{3}

\hlstd{values} \hlopt{>} \hlnum{0}
\end{alltt}
\begin{verbatim}
## [1] FALSE FALSE FALSE FALSE  TRUE  TRUE  TRUE
\end{verbatim}
\begin{alltt}
\hlstd{values} \hlopt{<} \hlnum{0}
\end{alltt}
\begin{verbatim}
## [1]  TRUE  TRUE  TRUE FALSE FALSE FALSE FALSE
\end{verbatim}
\begin{alltt}
\hlstd{values} \hlopt{==} \hlnum{0}
\end{alltt}
\begin{verbatim}
## [1] FALSE FALSE FALSE  TRUE FALSE FALSE FALSE
\end{verbatim}
\end{kframe}
\end{knitrout}

\end{frame}

%------------------------------------------------

\begin{frame}[fragile]
\frametitle{Comparison operators and recycling rule}

\begin{knitrout}\footnotesize
\definecolor{shadecolor}{rgb}{0.969, 0.969, 0.969}\color{fgcolor}\begin{kframe}
\begin{alltt}
\hlkwd{c}\hlstd{(}\hlnum{1}\hlstd{,} \hlnum{2}\hlstd{,} \hlnum{3}\hlstd{,} \hlnum{4}\hlstd{,} \hlnum{5}\hlstd{)} \hlopt{>} \hlnum{2}
\end{alltt}
\begin{verbatim}
## [1] FALSE FALSE  TRUE  TRUE  TRUE
\end{verbatim}
\begin{alltt}
\hlkwd{c}\hlstd{(}\hlnum{1}\hlstd{,} \hlnum{2}\hlstd{,} \hlnum{3}\hlstd{,} \hlnum{4}\hlstd{,} \hlnum{5}\hlstd{)} \hlopt{>=} \hlnum{2}
\end{alltt}
\begin{verbatim}
## [1] FALSE  TRUE  TRUE  TRUE  TRUE
\end{verbatim}
\begin{alltt}
\hlkwd{c}\hlstd{(}\hlnum{1}\hlstd{,} \hlnum{2}\hlstd{,} \hlnum{3}\hlstd{,} \hlnum{4}\hlstd{,} \hlnum{5}\hlstd{)} \hlopt{<} \hlnum{2}
\end{alltt}
\begin{verbatim}
## [1]  TRUE FALSE FALSE FALSE FALSE
\end{verbatim}
\end{kframe}
\end{knitrout}

\end{frame}

%------------------------------------------------

\begin{frame}[fragile]
\frametitle{Comparison operators and recycling rule}

\begin{knitrout}\footnotesize
\definecolor{shadecolor}{rgb}{0.969, 0.969, 0.969}\color{fgcolor}\begin{kframe}
\begin{alltt}
\hlkwd{c}\hlstd{(}\hlnum{1}\hlstd{,} \hlnum{2}\hlstd{,} \hlnum{3}\hlstd{,} \hlnum{4}\hlstd{,} \hlnum{5}\hlstd{)} \hlopt{<=} \hlnum{2}
\end{alltt}
\begin{verbatim}
## [1]  TRUE  TRUE FALSE FALSE FALSE
\end{verbatim}
\begin{alltt}
\hlkwd{c}\hlstd{(}\hlnum{1}\hlstd{,} \hlnum{2}\hlstd{,} \hlnum{3}\hlstd{,} \hlnum{4}\hlstd{,} \hlnum{5}\hlstd{)} \hlopt{==} \hlnum{2}
\end{alltt}
\begin{verbatim}
## [1] FALSE  TRUE FALSE FALSE FALSE
\end{verbatim}
\begin{alltt}
\hlkwd{c}\hlstd{(}\hlnum{1}\hlstd{,} \hlnum{2}\hlstd{,} \hlnum{3}\hlstd{,} \hlnum{4}\hlstd{,} \hlnum{5}\hlstd{)} \hlopt{!=} \hlnum{2}
\end{alltt}
\begin{verbatim}
## [1]  TRUE FALSE  TRUE  TRUE  TRUE
\end{verbatim}
\end{kframe}
\end{knitrout}

\end{frame}

%------------------------------------------------

\begin{frame}[fragile]
\frametitle{Comparison operators}

When comparing vectors of different types, one is coerced to the type of the other, the (decreasing) order of precedence being character, complex, numeric, integer, logical

\begin{knitrout}\footnotesize
\definecolor{shadecolor}{rgb}{0.969, 0.969, 0.969}\color{fgcolor}\begin{kframe}
\begin{alltt}
\hlstr{'5'} \hlopt{==} \hlnum{5}
\end{alltt}
\begin{verbatim}
## [1] TRUE
\end{verbatim}
\begin{alltt}
\hlnum{5L} \hlopt{==} \hlnum{5}
\end{alltt}
\begin{verbatim}
## [1] TRUE
\end{verbatim}
\begin{alltt}
\hlnum{5} \hlopt{+} \hlnum{0i} \hlopt{==} \hlnum{5}
\end{alltt}
\begin{verbatim}
## [1] TRUE
\end{verbatim}
\end{kframe}
\end{knitrout}

\end{frame}

%------------------------------------------------

\begin{frame}[fragile]
\frametitle{Comparison Operators}

In addition to comparison operators, we have the functions \code{all()} and \code{any()}
\begin{knitrout}\footnotesize
\definecolor{shadecolor}{rgb}{0.969, 0.969, 0.969}\color{fgcolor}\begin{kframe}
\begin{alltt}
\hlkwd{all}\hlstd{(}\hlkwd{c}\hlstd{(}\hlnum{1}\hlstd{,} \hlnum{2}\hlstd{,} \hlnum{3}\hlstd{,} \hlnum{4}\hlstd{,} \hlnum{5}\hlstd{)} \hlopt{>} \hlnum{0}\hlstd{)}

\hlkwd{all}\hlstd{(}\hlkwd{c}\hlstd{(}\hlnum{1}\hlstd{,} \hlnum{2}\hlstd{,} \hlnum{3}\hlstd{,} \hlnum{4}\hlstd{,} \hlnum{5}\hlstd{)} \hlopt{>} \hlnum{1}\hlstd{)}

\hlkwd{any}\hlstd{(}\hlkwd{c}\hlstd{(}\hlnum{1}\hlstd{,} \hlnum{2}\hlstd{,} \hlnum{3}\hlstd{,} \hlnum{4}\hlstd{,} \hlnum{5}\hlstd{)} \hlopt{<} \hlnum{0}\hlstd{)}

\hlkwd{any}\hlstd{(}\hlkwd{c}\hlstd{(}\hlnum{1}\hlstd{,} \hlnum{2}\hlstd{,} \hlnum{3}\hlstd{,} \hlnum{4}\hlstd{,} \hlnum{5}\hlstd{)} \hlopt{>} \hlnum{4}\hlstd{)}
\end{alltt}
\end{kframe}
\end{knitrout}

\end{frame}

%------------------------------------------------

\begin{frame}[fragile]
\frametitle{Summary Functions}

\bi
  \item \code{max()} maximum
  \item \code{min()} minimum
  \item \code{range()} range
  \item \code{mean()} mean
  \item \code{var()} variance
  \item \code{sd()} standard deviation
  \item \code{prod()} product of all elements
  \item \code{sum()} sum of all elements
\ei

\end{frame}

%------------------------------------------------

\begin{frame}[fragile]
\frametitle{Summary Functions}

\begin{knitrout}\footnotesize
\definecolor{shadecolor}{rgb}{0.969, 0.969, 0.969}\color{fgcolor}\begin{kframe}
\begin{alltt}
\hlstd{x} \hlkwb{<-} \hlnum{1}\hlopt{:}\hlnum{7}
\hlkwd{max}\hlstd{(x)}
\hlkwd{min}\hlstd{(x)}
\hlkwd{range}\hlstd{(x)}
\hlkwd{mean}\hlstd{(x)}
\hlkwd{var}\hlstd{(x)}
\hlkwd{sd}\hlstd{(x)}
\hlkwd{prod}\hlstd{(x)}
\hlkwd{sum}\hlstd{(x)}
\end{alltt}
\end{kframe}
\end{knitrout}

\end{frame}

%------------------------------------------------

\begin{frame}
\frametitle{Logical Operators}

\begin{center}
 \begin{tabular}{l l}
  \hline
   operation & usage \\
  \hline
  NOT & \code{!x} \\
  AND (elementwise) & \code{x \& y} \\
  AND (1st element) & \code{x \&\& y} \\  
  OR (elementwise)  & \code{x | y} \\
  OR (1st element) & \code{x || y} \\
  exclusive OR & \code{xor(x, y)} \\
  \hline
 \end{tabular}
\end{center}

Logical operators act on logical and number-like vectors

\end{frame}

%------------------------------------------------

\begin{frame}[fragile]
\frametitle{Logical Operators}

\begin{knitrout}\footnotesize
\definecolor{shadecolor}{rgb}{0.969, 0.969, 0.969}\color{fgcolor}\begin{kframe}
\begin{alltt}
\hlopt{!}\hlnum{TRUE}
\hlopt{!}\hlnum{FALSE}
\hlnum{TRUE} \hlopt{&} \hlnum{TRUE}
\hlnum{TRUE} \hlopt{&} \hlnum{FALSE}
\hlnum{FALSE} \hlopt{&} \hlnum{FALSE}
\hlnum{TRUE} \hlopt{|} \hlnum{TRUE}
\hlnum{TRUE} \hlopt{|} \hlnum{FALSE}
\hlnum{FALSE} \hlopt{|} \hlnum{FALSE}
\hlkwd{xor}\hlstd{(}\hlnum{TRUE}\hlstd{,} \hlnum{FALSE}\hlstd{)}
\hlkwd{xor}\hlstd{(}\hlnum{TRUE}\hlstd{,} \hlnum{TRUE}\hlstd{)}
\hlkwd{xor}\hlstd{(}\hlnum{FALSE}\hlstd{,} \hlnum{FALSE}\hlstd{)}
\end{alltt}
\end{kframe}
\end{knitrout}

\end{frame}

%------------------------------------------------

\begin{frame}[fragile]
\frametitle{Logical and Comparison Operators}

Many operations involve using logical and comparison operators:
\begin{knitrout}\footnotesize
\definecolor{shadecolor}{rgb}{0.969, 0.969, 0.969}\color{fgcolor}\begin{kframe}
\begin{alltt}
\hlstd{x} \hlkwb{<-} \hlnum{5}

\hlstd{(x} \hlopt{>} \hlnum{0}\hlstd{)} \hlopt{&} \hlstd{(x} \hlopt{<} \hlnum{10}\hlstd{)}
\hlstd{(x} \hlopt{>} \hlnum{0}\hlstd{)} \hlopt{|} \hlstd{(x} \hlopt{<} \hlnum{10}\hlstd{)}
\hlstd{(}\hlopt{-}\hlnum{2} \hlopt{*} \hlstd{x} \hlopt{>} \hlnum{0}\hlstd{)} \hlopt{&} \hlstd{(x}\hlopt{/}\hlnum{2} \hlopt{<} \hlnum{10}\hlstd{)}
\hlstd{(}\hlopt{-}\hlnum{2} \hlopt{*} \hlstd{x} \hlopt{>} \hlnum{0}\hlstd{)} \hlopt{|} \hlstd{(x}\hlopt{/}\hlnum{2} \hlopt{<} \hlnum{10}\hlstd{)}
\end{alltt}
\end{kframe}
\end{knitrout}

\end{frame}

%------------------------------------------------

\begin{frame}[fragile]
\frametitle{\code{which()} functions}

\bi
  \item \code{which()}: which indices are \code{TRUE}
  \item \code{which.min()}: location of first minimum
  \item \code{which.max()}: location of first maximum
\ei

\end{frame}

%------------------------------------------------

\begin{frame}[fragile]
\frametitle{Other Functions}

\begin{knitrout}\footnotesize
\definecolor{shadecolor}{rgb}{0.969, 0.969, 0.969}\color{fgcolor}\begin{kframe}
\begin{alltt}
\hlstd{(values} \hlkwb{<-} \hlopt{-}\hlnum{3}\hlopt{:}\hlnum{3}\hlstd{)}
\end{alltt}
\begin{verbatim}
## [1] -3 -2 -1  0  1  2  3
\end{verbatim}
\begin{alltt}
\hlcom{# logical comparison}
\hlstd{values} \hlopt{>} \hlnum{0}
\end{alltt}
\begin{verbatim}
## [1] FALSE FALSE FALSE FALSE  TRUE  TRUE  TRUE
\end{verbatim}
\begin{alltt}
\hlcom{# positions (i.e. indices) of positive values}
\hlkwd{which}\hlstd{(values} \hlopt{>} \hlnum{0}\hlstd{)}
\end{alltt}
\begin{verbatim}
## [1] 5 6 7
\end{verbatim}
\end{kframe}
\end{knitrout}

\end{frame}

%------------------------------------------------

\begin{frame}[fragile]
\frametitle{Function \code{which()}}

\begin{knitrout}\footnotesize
\definecolor{shadecolor}{rgb}{0.969, 0.969, 0.969}\color{fgcolor}\begin{kframe}
\begin{alltt}
\hlcom{# indices of various comparisons}
\hlkwd{which}\hlstd{(values} \hlopt{>} \hlnum{0}\hlstd{)}
\end{alltt}
\begin{verbatim}
## [1] 5 6 7
\end{verbatim}
\begin{alltt}
\hlkwd{which}\hlstd{(values} \hlopt{<} \hlnum{0}\hlstd{)}
\end{alltt}
\begin{verbatim}
## [1] 1 2 3
\end{verbatim}
\begin{alltt}
\hlkwd{which}\hlstd{(values} \hlopt{==} \hlnum{0}\hlstd{)}
\end{alltt}
\begin{verbatim}
## [1] 4
\end{verbatim}
\end{kframe}
\end{knitrout}

\end{frame}

%------------------------------------------------

\begin{frame}[fragile]
\frametitle{Function \code{which()}}

\begin{knitrout}\scriptsize
\definecolor{shadecolor}{rgb}{0.969, 0.969, 0.969}\color{fgcolor}\begin{kframe}
\begin{alltt}
\hlcom{# logical comparison}
\hlstd{values} \hlopt{>} \hlnum{0}
\end{alltt}
\begin{verbatim}
## [1] FALSE FALSE FALSE FALSE  TRUE  TRUE  TRUE
\end{verbatim}
\begin{alltt}
\hlcom{# logical subsetting}
\hlstd{values[values} \hlopt{>} \hlnum{0}\hlstd{]}
\end{alltt}
\begin{verbatim}
## [1] 1 2 3
\end{verbatim}
\begin{alltt}
\hlcom{# positions of positive values}
\hlkwd{which}\hlstd{(values} \hlopt{>} \hlnum{0}\hlstd{)}
\end{alltt}
\begin{verbatim}
## [1] 5 6 7
\end{verbatim}
\begin{alltt}
\hlcom{# numeric subsetting}
\hlstd{values[}\hlkwd{which}\hlstd{(values} \hlopt{>} \hlnum{0}\hlstd{)]}
\end{alltt}
\begin{verbatim}
## [1] 1 2 3
\end{verbatim}
\end{kframe}
\end{knitrout}

\end{frame}

%------------------------------------------------

\begin{frame}[fragile]
\frametitle{\code{which.max()} and \code{which.min()}}

\begin{knitrout}\footnotesize
\definecolor{shadecolor}{rgb}{0.969, 0.969, 0.969}\color{fgcolor}\begin{kframe}
\begin{alltt}
\hlkwd{which.max}\hlstd{(values)}
\end{alltt}
\begin{verbatim}
## [1] 7
\end{verbatim}
\begin{alltt}
\hlkwd{which}\hlstd{(values} \hlopt{==} \hlkwd{max}\hlstd{(values))}
\end{alltt}
\begin{verbatim}
## [1] 7
\end{verbatim}
\begin{alltt}
\hlkwd{which.min}\hlstd{(values)}
\end{alltt}
\begin{verbatim}
## [1] 1
\end{verbatim}
\begin{alltt}
\hlkwd{which}\hlstd{(values} \hlopt{==} \hlkwd{min}\hlstd{(values))}
\end{alltt}
\begin{verbatim}
## [1] 1
\end{verbatim}
\end{kframe}
\end{knitrout}

\end{frame}

%------------------------------------------------

\begin{frame}[fragile]
\frametitle{Set Operations}

Functions to perform set union, intersection, (asymmetric!) difference, equality and membership on two vectors
\bi
  \item \code{union(x, y)}
  \item \code{intersect(x, y)}
  \item \code{setdiff(x, y)}
  \item \code{setequal(x, y)}
  \item \code{is.element(el, set)}
  \item \code{\%in\%} operator
\ei

\end{frame}

%------------------------------------------------

\begin{frame}[fragile]
\frametitle{Set Operations}

\begin{knitrout}\footnotesize
\definecolor{shadecolor}{rgb}{0.969, 0.969, 0.969}\color{fgcolor}\begin{kframe}
\begin{alltt}
\hlstd{x} \hlkwb{<-} \hlkwd{c}\hlstd{(}\hlnum{1}\hlstd{,} \hlnum{2}\hlstd{,} \hlnum{3}\hlstd{,} \hlnum{4}\hlstd{,} \hlnum{5}\hlstd{)}
\hlstd{y} \hlkwb{<-} \hlkwd{c}\hlstd{(}\hlnum{2}\hlstd{,} \hlnum{4}\hlstd{,} \hlnum{6}\hlstd{)}

\hlkwd{union}\hlstd{(x, y)}
\hlkwd{intersect}\hlstd{(x, y)}
\hlkwd{setdiff}\hlstd{(x, y)}
\hlkwd{setequal}\hlstd{(x, y)}
\hlkwd{setequal}\hlstd{(}\hlkwd{c}\hlstd{(}\hlnum{4}\hlstd{,} \hlnum{6}\hlstd{,} \hlnum{2}\hlstd{), y)}
\hlkwd{is.element}\hlstd{(}\hlnum{1}\hlstd{, x)}
\hlkwd{is.element}\hlstd{(}\hlnum{6}\hlstd{, x)}
\hlnum{3} \hlopt \hlstd{x}
\hlnum{3} \hlopt \hlstd{y}
\end{alltt}
\end{kframe}
\end{knitrout}

\end{frame}

%------------------------------------------------

\begin{frame}[fragile]
\frametitle{General Functions}

Functions for inspecting a vector
\bi
  \item \code{class(x)}
  \item \code{length(x)}
  \item \code{head(x)}
  \item \code{tail(x)}
  \item \code{summary(x)}
\ei

\end{frame}

%------------------------------------------------

\begin{frame}[fragile]
\frametitle{General Functions}

\begin{knitrout}\footnotesize
\definecolor{shadecolor}{rgb}{0.969, 0.969, 0.969}\color{fgcolor}\begin{kframe}
\begin{alltt}
\hlstd{ages} \hlkwb{<-} \hlkwd{c}\hlstd{(}\hlnum{21}\hlstd{,} \hlnum{28}\hlstd{,} \hlnum{23}\hlstd{,} \hlnum{25}\hlstd{,} \hlnum{24}\hlstd{,} \hlnum{26}\hlstd{,} \hlnum{27}\hlstd{,} \hlnum{21}\hlstd{)}

\hlkwd{class}\hlstd{(ages)}
\hlkwd{length}\hlstd{(ages)}
\hlkwd{head}\hlstd{(ages)}
\hlkwd{tail}\hlstd{(ages)}
\hlkwd{summary}\hlstd{(ages)}
\end{alltt}
\end{kframe}
\end{knitrout}

\end{frame}

%------------------------------------------------

\begin{frame}[fragile]
\frametitle{Exercise}

Find out what the following expressions return:
\begin{knitrout}\footnotesize
\definecolor{shadecolor}{rgb}{0.969, 0.969, 0.969}\color{fgcolor}\begin{kframe}
\begin{alltt}
\hlnum{1}\hlopt{:}\hlnum{1}

\hlkwd{seq}\hlstd{(}\hlnum{0}\hlstd{,} \hlnum{1}\hlstd{,} \hlkwc{length.out} \hlstd{=} \hlnum{10}\hlstd{)}

\hlkwd{rep}\hlstd{(}\hlkwd{c}\hlstd{(}\hlnum{1}\hlstd{,} \hlnum{2}\hlstd{,} \hlnum{3}\hlstd{),} \hlkwc{times} \hlstd{=} \hlkwd{c}\hlstd{(}\hlnum{1}\hlstd{,} \hlnum{2}\hlstd{,} \hlnum{3}\hlstd{))}
\end{alltt}
\end{kframe}
\end{knitrout}

\end{frame}

%------------------------------------------------

\begin{frame}[fragile]
\frametitle{Exercise}

Write three different ways in which the vector \code{1, 2, 3, 4, 5} can be created:
\pause
\begin{knitrout}\footnotesize
\definecolor{shadecolor}{rgb}{0.969, 0.969, 0.969}\color{fgcolor}\begin{kframe}
\begin{alltt}
\hlkwd{c}\hlstd{(}\hlnum{1}\hlstd{,} \hlnum{2}\hlstd{,} \hlnum{3}\hlstd{,} \hlnum{4}\hlstd{,} \hlnum{5}\hlstd{)}
\hlkwd{seq}\hlstd{(}\hlkwc{from} \hlstd{=} \hlnum{1}\hlstd{,} \hlkwc{to} \hlstd{=} \hlnum{5}\hlstd{)}
\hlnum{1}\hlopt{:}\hlnum{5}

\hlcom{# another option}
\hlnum{0}\hlopt{:}\hlnum{4} \hlopt{+} \hlnum{1}
\end{alltt}
\end{kframe}
\end{knitrout}

\end{frame}

%------------------------------------------------

\begin{frame}[fragile]
\frametitle{Exercise}

Generate a random vector of \code{n=100} elements:
\begin{knitrout}\footnotesize
\definecolor{shadecolor}{rgb}{0.969, 0.969, 0.969}\color{fgcolor}\begin{kframe}
\begin{alltt}
\hlkwd{set.seed}\hlstd{(}\hlnum{1}\hlstd{)}
\hlstd{a} \hlkwb{<-} \hlkwd{rnorm}\hlstd{(}\hlnum{100}\hlstd{)}
\end{alltt}
\end{kframe}
\end{knitrout}

Find the following:
\bi
  \item what's the vector class
  \item what's the mean and standard deviation
  \item what's the sum of all elements in absolute value
  \item how many elements are positive ($\geq$ 0)
  \item how many elements are negative ($<$ 0)
  \item the three smallest and largest numbers
\ei

\end{frame}

%------------------------------------------------

\begin{frame}[fragile]
\frametitle{Exercise}

\begin{columns}[t]
\begin{column}{0.45\textwidth}
\begin{knitrout}\footnotesize
\definecolor{shadecolor}{rgb}{0.969, 0.969, 0.969}\color{fgcolor}\begin{kframe}
\begin{alltt}
\hlcom{# class}
\hlkwd{class}\hlstd{(a)}
\end{alltt}
\begin{verbatim}
## [1] "numeric"
\end{verbatim}
\begin{alltt}
\hlcom{# mean value}
\hlkwd{mean}\hlstd{(a)}
\end{alltt}
\begin{verbatim}
## [1] 0.1088874
\end{verbatim}
\begin{alltt}
\hlcom{# std dev}
\hlkwd{sd}\hlstd{(a)}
\end{alltt}
\begin{verbatim}
## [1] 0.8981994
\end{verbatim}
\end{kframe}
\end{knitrout}
\end{column}

\begin{column}{0.45\textwidth}
\begin{knitrout}\footnotesize
\definecolor{shadecolor}{rgb}{0.969, 0.969, 0.969}\color{fgcolor}\begin{kframe}
\begin{alltt}
\hlcom{# sum of elems in abs-value}
\hlkwd{sum}\hlstd{(}\hlkwd{abs}\hlstd{(a))}
\end{alltt}
\begin{verbatim}
## [1] 71.67207
\end{verbatim}
\begin{alltt}
\hlcom{# how many positive}
\hlkwd{sum}\hlstd{(a} \hlopt{>=} \hlnum{0}\hlstd{)}
\end{alltt}
\begin{verbatim}
## [1] 54
\end{verbatim}
\begin{alltt}
\hlcom{# how many negative}
\hlkwd{sum}\hlstd{(a} \hlopt{<} \hlnum{0}\hlstd{)}
\end{alltt}
\begin{verbatim}
## [1] 46
\end{verbatim}
\end{kframe}
\end{knitrout}
\end{column}
\end{columns}

\end{frame}

%------------------------------------------------

\begin{frame}[fragile]
\frametitle{Exercise}

\begin{knitrout}\footnotesize
\definecolor{shadecolor}{rgb}{0.969, 0.969, 0.969}\color{fgcolor}\begin{kframe}
\begin{alltt}
\hlcom{# 3 smallest numbers}
\hlkwd{sort}\hlstd{(a)[}\hlnum{1}\hlopt{:}\hlnum{3}\hlstd{]}
\end{alltt}
\begin{verbatim}
## [1] -2.214700 -1.989352 -1.804959
\end{verbatim}
\begin{alltt}
\hlcom{# 3 largest numbers}
\hlkwd{sort}\hlstd{(a,} \hlkwc{decreasing} \hlstd{=} \hlnum{TRUE}\hlstd{)[}\hlnum{1}\hlopt{:}\hlnum{3}\hlstd{]}
\end{alltt}
\begin{verbatim}
## [1] 2.401618 2.172612 1.980400
\end{verbatim}
\end{kframe}
\end{knitrout}

\end{frame}

%------------------------------------------------


\end{document}
