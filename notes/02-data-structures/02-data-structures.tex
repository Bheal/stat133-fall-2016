\documentclass[12pt]{beamer}\usepackage[]{graphicx}\usepackage[]{color}
%% maxwidth is the original width if it is less than linewidth
%% otherwise use linewidth (to make sure the graphics do not exceed the margin)
\makeatletter
\def\maxwidth{ %
  \ifdim\Gin@nat@width>\linewidth
    \linewidth
  \else
    \Gin@nat@width
  \fi
}
\makeatother

\definecolor{fgcolor}{rgb}{0.345, 0.345, 0.345}
\newcommand{\hlnum}[1]{\textcolor[rgb]{0.686,0.059,0.569}{#1}}%
\newcommand{\hlstr}[1]{\textcolor[rgb]{0.192,0.494,0.8}{#1}}%
\newcommand{\hlcom}[1]{\textcolor[rgb]{0.678,0.584,0.686}{\textit{#1}}}%
\newcommand{\hlopt}[1]{\textcolor[rgb]{0,0,0}{#1}}%
\newcommand{\hlstd}[1]{\textcolor[rgb]{0.345,0.345,0.345}{#1}}%
\newcommand{\hlkwa}[1]{\textcolor[rgb]{0.161,0.373,0.58}{\textbf{#1}}}%
\newcommand{\hlkwb}[1]{\textcolor[rgb]{0.69,0.353,0.396}{#1}}%
\newcommand{\hlkwc}[1]{\textcolor[rgb]{0.333,0.667,0.333}{#1}}%
\newcommand{\hlkwd}[1]{\textcolor[rgb]{0.737,0.353,0.396}{\textbf{#1}}}%
\let\hlipl\hlkwb

\usepackage{framed}
\makeatletter
\newenvironment{kframe}{%
 \def\at@end@of@kframe{}%
 \ifinner\ifhmode%
  \def\at@end@of@kframe{\end{minipage}}%
  \begin{minipage}{\columnwidth}%
 \fi\fi%
 \def\FrameCommand##1{\hskip\@totalleftmargin \hskip-\fboxsep
 \colorbox{shadecolor}{##1}\hskip-\fboxsep
     % There is no \\@totalrightmargin, so:
     \hskip-\linewidth \hskip-\@totalleftmargin \hskip\columnwidth}%
 \MakeFramed {\advance\hsize-\width
   \@totalleftmargin\z@ \linewidth\hsize
   \@setminipage}}%
 {\par\unskip\endMakeFramed%
 \at@end@of@kframe}
\makeatother

\definecolor{shadecolor}{rgb}{.97, .97, .97}
\definecolor{messagecolor}{rgb}{0, 0, 0}
\definecolor{warningcolor}{rgb}{1, 0, 1}
\definecolor{errorcolor}{rgb}{1, 0, 0}
\newenvironment{knitrout}{}{} % an empty environment to be redefined in TeX

\usepackage{alltt}
\usepackage{graphicx}
\usepackage{tikz}
\setbeameroption{hide notes}
\setbeamertemplate{note page}[plain]
\usepackage{listings}

% do not include "../" so Makefile does not crash
% get rid of junk
\usetheme{default}
\usefonttheme[onlymath]{serif}
\beamertemplatenavigationsymbolsempty
\hypersetup{pdfpagemode=UseNone} % don't show bookmarks on initial view

% named colors
\definecolor{offwhite}{RGB}{255,250,240}
\definecolor{gray}{RGB}{155,155,155}

\ifx\notescolors\undefined % slides

  \definecolor{foreground}{RGB}{80,80,80}
  \definecolor{background}{RGB}{255,255,255}
  \definecolor{title}{RGB}{255,199,0}
  \definecolor{subtitle}{RGB}{89,132,212}
  \definecolor{hilit}{RGB}{248,117,79}
  \definecolor{vhilit}{RGB}{255,111,207}
  \definecolor{lolit}{RGB}{200,200,200}
  \definecolor{lit}{RGB}{255,199,0}
  \definecolor{mdlit}{RGB}{89,132,212}
  \definecolor{link}{RGB}{248,117,79}

\else % notes
  \definecolor{background}{RGB}{255,255,255}
  \definecolor{foreground}{RGB}{24,24,24}
  \definecolor{title}{RGB}{27,94,134}
  \definecolor{subtitle}{RGB}{22,175,124}
  \definecolor{hilit}{RGB}{122,0,128}
  \definecolor{vhilit}{RGB}{255,0,128}
  \definecolor{lolit}{RGB}{95,95,95}
\fi
\definecolor{nhilit}{RGB}{128,0,128}  % hilit color in notes
\definecolor{nvhilit}{RGB}{255,0,128} % vhilit for notes

\newcommand{\hilit}{\color{hilit}}
\newcommand{\vhilit}{\color{vhilit}}
\newcommand{\nhilit}{\color{nhilit}}
\newcommand{\nvhilit}{\color{nvhilit}}
\newcommand{\lit}{\color{lit}}
\newcommand{\mdlit}{\color{mdlit}}
\newcommand{\lolit}{\color{lolit}}

% use those colors
\setbeamercolor{titlelike}{fg=title}
\setbeamercolor{subtitle}{fg=subtitle}
\setbeamercolor{frametitle}{fg=gray}
\setbeamercolor{structure}{fg=subtitle}
\setbeamercolor{institute}{fg=lolit}
\setbeamercolor{normal text}{fg=foreground,bg=background}
%\setbeamercolor{item}{fg=foreground} % color of bullets
%\setbeamercolor{subitem}{fg=hilit}
%\setbeamercolor{itemize/enumerate subbody}{fg=lolit}
\setbeamertemplate{itemize subitem}{{\textendash}}
\setbeamerfont{itemize/enumerate subbody}{size=\footnotesize}
\setbeamerfont{itemize/enumerate subitem}{size=\footnotesize}

% center title of slides
\setbeamertemplate{blocks}[rounded]
\setbeamertemplate{frametitle}[default][center]
% margins
\setbeamersize{text margin left=25pt,text margin right=25pt}

% page number
\setbeamertemplate{footline}{%
    \raisebox{5pt}{\makebox[\paperwidth]{\hfill\makebox[20pt]{\lolit
          \scriptsize\insertframenumber}}}\hspace*{5pt}}

% add a bit of space at the top of the notes page
\addtobeamertemplate{note page}{\setlength{\parskip}{12pt}}

% default link color
\hypersetup{colorlinks, urlcolor={link}}

\ifx\notescolors\undefined % slides
  % set up listing environment
  \lstset{language=bash,
          basicstyle=\ttfamily\scriptsize,
          frame=single,
          commentstyle=,
          backgroundcolor=\color{darkgray},
          showspaces=false,
          showstringspaces=false
          }
\else % notes
  \lstset{language=bash,
          basicstyle=\ttfamily\scriptsize,
          frame=single,
          commentstyle=,
          backgroundcolor=\color{offwhite},
          showspaces=false,
          showstringspaces=false
          }
\fi

% a few macros
\newcommand{\code}[1]{\texttt{#1}}
\newcommand{\hicode}[1]{{\hilit \texttt{#1}}}
\newcommand{\bb}[1]{\begin{block}{#1}}
\newcommand{\eb}{\end{block}}
\newcommand{\bi}{\begin{itemize}}
%\newcommand{\bbi}{\vspace{24pt} \begin{itemize} \itemsep8pt}
\newcommand{\bbi}{\vspace{4pt} \begin{itemize} \itemsep8pt}
\newcommand{\ei}{\end{itemize}}
\newcommand{\bv}{\begin{verbatim}}
\newcommand{\ev}{\end{verbatim}}
\newcommand{\ig}{\includegraphics}
\newcommand{\subt}[1]{{\footnotesize \color{subtitle} {#1}}}
\newcommand{\ttsm}{\tt \small}
\newcommand{\ttfn}{\tt \footnotesize}
\newcommand{\figh}[2]{\centerline{\includegraphics[height=#2\textheight]{#1}}}
\newcommand{\figw}[2]{\centerline{\includegraphics[width=#2\textwidth]{#1}}}



%------------------------------------------------
% end of header
%------------------------------------------------

\title{Data Structures}
\subtitle{STAT 133}
\author{\href{http://www.gastonsanchez.com}{Gaston Sanchez}}
\institute{\href{https://github.com/ucb-stat133/stat133-fall-2016}{\tt \scriptsize \color{foreground} github.com/ucb-stat133/stat133-fall-2016}}
\date{}
\IfFileExists{upquote.sty}{\usepackage{upquote}}{}
\begin{document}


{
  \setbeamertemplate{footline}{} % no page number here
  \frame{
    \titlepage
  } 
}

%------------------------------------------------

\begin{frame}
\frametitle{Data Types and Structures}

To make the best of the R language, you'll need a strong understanding of the basic \textbf{data types} and \textbf{data structures} and how to operate on them.

\end{frame}

%------------------------------------------------

\begin{frame}
\frametitle{Data Structures}

There are various data structures in R (we'll describe them in detail later):

\bi
  \item vectors
  \item matrices (2d arrays)
  \item arrays (in general)
  \item factors
  \item lists
  \item data frames
\ei

\end{frame}

%------------------------------------------------

\begin{frame}
\begin{center}
\Huge{\hilit{Vectors}}
\end{center}
\end{frame}

%------------------------------------------------

\begin{frame}
\frametitle{Vectors}

\bi
  \item A vector is the most basic data structure in R
  \item Vectors are contiguous cells containing data
  \item Can be of any length (including zero)
  \item R has five basic type of vectors: \\
  integer, double, complex, logical, character
\ei

\end{frame}

%------------------------------------------------

\begin{frame}[fragile]
\frametitle{Vectors}

The most simple type of vectors are scalars or single values:
\begin{knitrout}\footnotesize
\definecolor{shadecolor}{rgb}{0.969, 0.969, 0.969}\color{fgcolor}\begin{kframe}
\begin{alltt}
\hlcom{# integer}
\hlstd{x} \hlkwb{<-} \hlnum{1L}
\hlcom{# double (real)}
\hlstd{y} \hlkwb{<-} \hlnum{5}
\hlcom{# complex}
\hlstd{z} \hlkwb{<-} \hlnum{3} \hlopt{+} \hlnum{5i}
\hlcom{# logical}
\hlstd{a} \hlkwb{<-} \hlnum{TRUE}
\hlcom{# character}
\hlstd{b} \hlkwb{<-} \hlstr{"yosemite"}
\end{alltt}
\end{kframe}
\end{knitrout}

\end{frame}

%------------------------------------------------

\begin{frame}
\frametitle{Data modes}

\bi
  \item A \textbf{double} vector stores regular (i.e. real) numbers
  \item An \textbf{integer} vector stores integers (no decimal component)
  \item A \textbf{character} vector stores text
  \item A \textbf{logical} vector stores \code{TRUE}'s and \code{FALSE}'s values
  \item A \textbf{complex} vector stores complex numbers
\ei

\end{frame}

%------------------------------------------------

\begin{frame}
\frametitle{Data Types (or modes)}

\begin{center}
 \begin{tabular}{l l l l}
  \hline
   value & example & mode & storage \\
  \hline
  integer & \code{1L, 2L} & \code{numeric} & \code{integer} \\  
  real & \code{1, -0.5} & \code{numeric} & \code{double} \\
  complex & \code{3 + 5i} & \code{complex} & \code{complex} \\
  logical & \code{TRUE, FALSE} & \code{logical} & \code{logical} \\
  character & \code{"hello"} & \code{character} & \code{character}  \\
  \hline
 \end{tabular}
\end{center}

\end{frame}

%------------------------------------------------

\begin{frame}[fragile]
\frametitle{Special Values}

There are some special values 
\bi
  \item \code{NULL} is the null object (it has length zero)
  \item Missing values are referred to by the symbol \code{NA}
  \item \code{Inf} indicates positive infinite
  \item \code{-Inf} indicates negative infinite
  \item \code{NaN} indicates Not a Number
\ei

\end{frame}

%------------------------------------------------

\begin{frame}[fragile]
\frametitle{Vectors}

The function to create a vector from individual values is {\hilit \code{c()}}, short for \textbf{concatenate}:

\begin{knitrout}\footnotesize
\definecolor{shadecolor}{rgb}{0.969, 0.969, 0.969}\color{fgcolor}\begin{kframe}
\begin{alltt}
\hlcom{# some vectors}
\hlstd{x} \hlkwb{<-} \hlkwd{c}\hlstd{(}\hlnum{1}\hlstd{,} \hlnum{2}\hlstd{,} \hlnum{3}\hlstd{,} \hlnum{4}\hlstd{,} \hlnum{5}\hlstd{)}

\hlstd{y} \hlkwb{<-} \hlkwd{c}\hlstd{(}\hlstr{"one"}\hlstd{,} \hlstr{"two"}\hlstd{,} \hlstr{"three"}\hlstd{)}

\hlstd{z} \hlkwb{<-} \hlkwd{c}\hlstd{(}\hlnum{TRUE}\hlstd{,} \hlnum{FALSE}\hlstd{,} \hlnum{FALSE}\hlstd{)}
\end{alltt}
\end{kframe}
\end{knitrout}
Separate each element by a comma

\end{frame}

%------------------------------------------------

\begin{frame}
\frametitle{Atomic Vectors}

\bi
  \item vectors are \textbf{atomic} structures
  \item the values in a vector must be ALL of the same type 
  \item either all integers, or reals, or complex, or characters, or logicals
  \item you cannot have a vector of different data types
\ei

\end{frame}

%------------------------------------------------

\begin{frame}[fragile]
\frametitle{Atomic Vectors}

If you mix different data values, R will \textbf{implicitly} coerce them so they are all of the same type
\begin{knitrout}\footnotesize
\definecolor{shadecolor}{rgb}{0.969, 0.969, 0.969}\color{fgcolor}\begin{kframe}
\begin{alltt}
\hlcom{# mixing numbers and characters}
\hlstd{x} \hlkwb{<-} \hlkwd{c}\hlstd{(}\hlnum{1}\hlstd{,} \hlnum{2}\hlstd{,} \hlnum{3}\hlstd{,} \hlstr{"four"}\hlstd{,} \hlstr{"five"}\hlstd{)}
\hlstd{x}
\end{alltt}
\begin{verbatim}
## [1] "1"    "2"    "3"    "four" "five"
\end{verbatim}
\begin{alltt}
\hlcom{# mixing numbers and logical values}
\hlstd{y} \hlkwb{<-} \hlkwd{c}\hlstd{(}\hlnum{TRUE}\hlstd{,} \hlnum{FALSE}\hlstd{,} \hlnum{3}\hlstd{,} \hlnum{4}\hlstd{)}
\hlstd{y}
\end{alltt}
\begin{verbatim}
## [1] 1 0 3 4
\end{verbatim}
\end{kframe}
\end{knitrout}

\end{frame}

%------------------------------------------------

\begin{frame}[fragile]
\frametitle{Atomic Vectors}

\begin{knitrout}\footnotesize
\definecolor{shadecolor}{rgb}{0.969, 0.969, 0.969}\color{fgcolor}\begin{kframe}
\begin{alltt}
\hlcom{# mixing numbers and logical values}
\hlstd{z} \hlkwb{<-} \hlkwd{c}\hlstd{(}\hlnum{TRUE}\hlstd{,} \hlnum{FALSE}\hlstd{,} \hlstr{"TRUE"}\hlstd{,} \hlstr{"FALSE"}\hlstd{)}
\hlstd{z}
\end{alltt}
\begin{verbatim}
## [1] "TRUE"  "FALSE" "TRUE"  "FALSE"
\end{verbatim}
\begin{alltt}
\hlcom{# mixing integer, real, and complex numbers}
\hlstd{w} \hlkwb{<-} \hlkwd{c}\hlstd{(}\hlnum{1L}\hlstd{,} \hlopt{-}\hlnum{0.5}\hlstd{,} \hlnum{3} \hlopt{+} \hlnum{5i}\hlstd{)}
\hlstd{w}
\end{alltt}
\begin{verbatim}
## [1]  1.0+0i -0.5+0i  3.0+5i
\end{verbatim}
\end{kframe}
\end{knitrout}

\end{frame}

%------------------------------------------------

\begin{frame}[fragile]
\frametitle{How does R coerce data types?}

\bb{R follows two basic rules of implicit coercion}
If a character is present, R will coerce everything else to characters

\bigskip

If a vector contains logicals and numbers, R will convert the logicals to numbers (\code{TRUE} to 1, \code{FALSE} to 0)
\eb

\end{frame}

%------------------------------------------------

\begin{frame}[fragile]
\frametitle{Coercion functions}

\bb{Coercion Functions}
R provides a set of \textbf{explicit} coercion functions that allow us to ``convert'' one type of data into another
\eb

\bi
  \item \code{as.character()}
  \item \code{as.numeric()}
  \item \code{as.logical()}
\ei

\end{frame}

%------------------------------------------------

\begin{frame}
\frametitle{Conversion between types}

\begin{center}
 \begin{tabular}{l l l l}
  \hline
   from & to & function & conversions \\
  \hline
  logical & numeric & \code{as.numeric} & \code{FALSE} $\rightarrow 0$ \\
   & & & \code{TRUE} $\rightarrow 1$ \\
  logical & character & \code{as.character} & \code{FALSE} $\rightarrow$ \code{"FALSE"} \\
   & & & \code{TRUE} $\rightarrow$ \code{"TRUE"} \\
  character & numeric & \code{as.numeric} & \code{"1", "2"} $\rightarrow$ \code{1, 2} \\
   & & & \code{"A"} $\rightarrow$ \code{NA} \\
  character & logical & \code{as.logical} & \code{"FALSE"} $\rightarrow$ \code{FALSE} \\
   & & & \code{"TRUE"} $\rightarrow$ \code{TRUE} \\
   & & & \textit{other} $\rightarrow$ \code{NA} \\
   numeric & logical & \code{as.logical} & \code{0} $\rightarrow$ \code{FALSE} \\
   & & & \textit{other} $\rightarrow$ \code{1} \\
  numeric & character & \code{as.character} & \code{1, 2} $\rightarrow$ \code{"1", "2"} \\

  \hline
 \end{tabular}
\end{center}

\end{frame}

%------------------------------------------------

\begin{frame}[fragile]
\frametitle{Properties of Vectors}

\bi
  \item all vectors have a length
  \item vector elements can have associated names
  \item vectors are objects of class \code{"vector"}
  \item vectors have a mode (storage mode)
\ei

\end{frame}

%------------------------------------------------

\begin{frame}[fragile]
\frametitle{Properties of Vectors}

\begin{knitrout}\footnotesize
\definecolor{shadecolor}{rgb}{0.969, 0.969, 0.969}\color{fgcolor}\begin{kframe}
\begin{alltt}
\hlcom{# vector with named elements}
\hlstd{x} \hlkwb{<-} \hlkwd{c}\hlstd{(}\hlkwc{a} \hlstd{=} \hlnum{1}\hlstd{,} \hlkwc{b} \hlstd{=} \hlnum{2.5}\hlstd{,} \hlkwc{c} \hlstd{=} \hlnum{3.7}\hlstd{,} \hlkwc{d} \hlstd{=} \hlnum{10}\hlstd{)}
\hlstd{x}
\end{alltt}
\begin{verbatim}
##    a    b    c    d 
##  1.0  2.5  3.7 10.0
\end{verbatim}
\begin{alltt}
\hlkwd{length}\hlstd{(x)}
\end{alltt}
\begin{verbatim}
## [1] 4
\end{verbatim}
\begin{alltt}
\hlkwd{mode}\hlstd{(x)}
\end{alltt}
\begin{verbatim}
## [1] "numeric"
\end{verbatim}
\end{kframe}
\end{knitrout}

\end{frame}

%------------------------------------------------

\begin{frame}
\begin{center}
\Huge{\hilit{Matrices and Arrays}}
\end{center}
\end{frame}

%------------------------------------------------

\begin{frame}[fragile]
\frametitle{From Vectors to Arrays}

We can transform a vector in an \textbf{n-dimensional} array by giving it a dimensions attribute \code{dim}
\begin{knitrout}\footnotesize
\definecolor{shadecolor}{rgb}{0.969, 0.969, 0.969}\color{fgcolor}\begin{kframe}
\begin{alltt}
\hlcom{# positive: from 1 to 8}
\hlstd{x} \hlkwb{<-} \hlnum{1}\hlopt{:}\hlnum{8}

\hlcom{# adding 'dim' attribute}
\hlkwd{dim}\hlstd{(x)} \hlkwb{<-} \hlkwd{c}\hlstd{(}\hlnum{2}\hlstd{,} \hlnum{4}\hlstd{)}
\hlstd{x}
\end{alltt}
\begin{verbatim}
##      [,1] [,2] [,3] [,4]
## [1,]    1    3    5    7
## [2,]    2    4    6    8
\end{verbatim}
\end{kframe}
\end{knitrout}

\end{frame}

%------------------------------------------------

\begin{frame}[fragile]
\frametitle{From Vectors to Arrays}

\bi
  \item a vector can be given a \code{dim} attribute
  \item a \code{dim} attribute is a numeric vector of length \code{n}
  \item R will reorganize the elements of the vector into \code{n} dimensions
  \item each dimension will have as many rows (or columns, etc.) as the \code{n}-th value of the \code{dim} vector
\ei

\end{frame}

%------------------------------------------------

\begin{frame}[fragile]
\frametitle{From Vectors to Arrays}

\begin{knitrout}\footnotesize
\definecolor{shadecolor}{rgb}{0.969, 0.969, 0.969}\color{fgcolor}\begin{kframe}
\begin{alltt}
\hlcom{# dim attribute with 3 dimensions}
\hlkwd{dim}\hlstd{(x)} \hlkwb{<-} \hlkwd{c}\hlstd{(}\hlnum{2}\hlstd{,} \hlnum{2}\hlstd{,} \hlnum{2}\hlstd{)}
\hlstd{x}
\end{alltt}
\begin{verbatim}
## , , 1
## 
##      [,1] [,2]
## [1,]    1    3
## [2,]    2    4
## 
## , , 2
## 
##      [,1] [,2]
## [1,]    5    7
## [2,]    6    8
\end{verbatim}
\end{kframe}
\end{knitrout}

\end{frame}

%------------------------------------------------

\begin{frame}[fragile]
\frametitle{From Vector to Matrix}

A \code{dim} attribute of length 2 will convert a vector into a matrix
\begin{knitrout}\footnotesize
\definecolor{shadecolor}{rgb}{0.969, 0.969, 0.969}\color{fgcolor}\begin{kframe}
\begin{alltt}
\hlcom{# vector to matrix}
\hlstd{A} \hlkwb{<-} \hlnum{1}\hlopt{:}\hlnum{8}
\hlkwd{class}\hlstd{(A)}
\end{alltt}
\begin{verbatim}
## [1] "integer"
\end{verbatim}
\begin{alltt}
\hlkwd{dim}\hlstd{(A)} \hlkwb{<-} \hlkwd{c}\hlstd{(}\hlnum{2}\hlstd{,} \hlnum{4}\hlstd{)}
\hlkwd{class}\hlstd{(A)}
\end{alltt}
\begin{verbatim}
## [1] "matrix"
\end{verbatim}
\end{kframe}
\end{knitrout}

When using \code{dim()}, R always fills up each matrix by rows.

\end{frame}

%------------------------------------------------

\begin{frame}[fragile]
\frametitle{From Vector to Matrix}

To have more control about how a matrix is filled, (by rows or columns), we use the \code{matrix()} function:
\begin{knitrout}\footnotesize
\definecolor{shadecolor}{rgb}{0.969, 0.969, 0.969}\color{fgcolor}\begin{kframe}
\begin{alltt}
\hlcom{# vector to matrix (by rows)}
\hlstd{A} \hlkwb{<-} \hlnum{1}\hlopt{:}\hlnum{8}

\hlkwd{matrix}\hlstd{(A,} \hlkwc{nrow} \hlstd{=} \hlnum{2}\hlstd{,} \hlkwc{ncol} \hlstd{=} \hlnum{4}\hlstd{,} \hlkwc{byrow} \hlstd{=} \hlnum{TRUE}\hlstd{)}
\end{alltt}
\begin{verbatim}
##      [,1] [,2] [,3] [,4]
## [1,]    1    2    3    4
## [2,]    5    6    7    8
\end{verbatim}
\end{kframe}
\end{knitrout}

\end{frame}

%------------------------------------------------

\begin{frame}[fragile]
\frametitle{Matrices}

\bi
  \item Matrices store values in a two-dimensional array
  \item To create a matrix, give a vector to \code{matrix()} and specify number of rows and columns
  \item you can also assign row and column names to a matrix
\ei

\end{frame}

%------------------------------------------------

\begin{frame}[fragile]
\frametitle{Creating a Matrix}

Exercise: create the following matrix
\begin{knitrout}\footnotesize
\definecolor{shadecolor}{rgb}{0.969, 0.969, 0.969}\color{fgcolor}\begin{kframe}
\begin{verbatim}
##      [,1]       [,2]     
## [1,] "Harry"    "Potter" 
## [2,] "Ron"      "Weasley"
## [3,] "Hermione" "Granger"
\end{verbatim}
\end{kframe}
\end{knitrout}

\end{frame}

%------------------------------------------------

\begin{frame}[fragile]
\frametitle{Creating a Matrix}

Solution:
\begin{knitrout}\footnotesize
\definecolor{shadecolor}{rgb}{0.969, 0.969, 0.969}\color{fgcolor}\begin{kframe}
\begin{alltt}
\hlcom{# vector of names}
\hlstd{hp} \hlkwb{<-} \hlkwd{c}\hlstd{(}\hlstr{"Harry"}\hlstd{,} \hlstr{"Potter"}\hlstd{,} \hlstr{"Ron"}\hlstd{,} \hlstr{"Weasley"}\hlstd{,}
        \hlstr{"Hermione"}\hlstd{,} \hlstr{"Granger"}\hlstd{)}

\hlcom{# matrix filled up by rows}
\hlkwd{matrix}\hlstd{(hp,} \hlkwc{nrow} \hlstd{=} \hlnum{3}\hlstd{,} \hlkwc{byrow} \hlstd{=} \hlnum{TRUE}\hlstd{)}
\end{alltt}
\begin{verbatim}
##      [,1]       [,2]     
## [1,] "Harry"    "Potter" 
## [2,] "Ron"      "Weasley"
## [3,] "Hermione" "Granger"
\end{verbatim}
\end{kframe}
\end{knitrout}

\end{frame}

%------------------------------------------------

\begin{frame}[fragile]
\frametitle{Arrays}

\bi
  \item Arrays store values in an n-dimensional array
  \item To create an array, give a vector to \code{array()} and specify number of dimensions
  \item you can also assign dim-names to an array
\ei

\end{frame}

%------------------------------------------------

\begin{frame}[fragile]
\frametitle{Creating an Array}

\begin{knitrout}\scriptsize
\definecolor{shadecolor}{rgb}{0.969, 0.969, 0.969}\color{fgcolor}\begin{kframe}
\begin{alltt}
\hlstd{ar} \hlkwb{<-} \hlkwd{array}\hlstd{(}\hlkwd{c}\hlstd{(}\hlnum{1}\hlopt{:}\hlnum{4}\hlstd{,} \hlnum{5}\hlopt{:}\hlnum{8}\hlstd{,} \hlnum{9}\hlopt{:}\hlnum{12}\hlstd{),} \hlkwc{dim} \hlstd{=} \hlkwd{c}\hlstd{(}\hlnum{2}\hlstd{,} \hlnum{2}\hlstd{,} \hlnum{3}\hlstd{))}
\hlstd{ar}
\end{alltt}
\begin{verbatim}
## , , 1
## 
##      [,1] [,2]
## [1,]    1    3
## [2,]    2    4
## 
## , , 2
## 
##      [,1] [,2]
## [1,]    5    7
## [2,]    6    8
## 
## , , 3
## 
##      [,1] [,2]
## [1,]    9   11
## [2,]   10   12
\end{verbatim}
\end{kframe}
\end{knitrout}

\end{frame}

%------------------------------------------------

\begin{frame}
\begin{center}
\Huge{\hilit{Factors}}
\end{center}
\end{frame}

%------------------------------------------------

\begin{frame}
\frametitle{Factors}

\bi 
  \item A similar structure to vectors are \textbf{factors}
  \item factors are used for handling categorical (i.e. qualitative) data
  \item they are represented as objects of class \code{"factor"}
  \item internally, factors are stored as integers
  \item factors behave much like vectors (but they are not vectors)
\ei
\end{frame}

%------------------------------------------------

\begin{frame}[fragile]
\frametitle{Factors}

To create a factor we use the function {\hilit \code{factor()}}
\begin{knitrout}\footnotesize
\definecolor{shadecolor}{rgb}{0.969, 0.969, 0.969}\color{fgcolor}\begin{kframe}
\begin{alltt}
\hlcom{# factor}
\hlstd{cols} \hlkwb{<-} \hlkwd{c}\hlstd{(}\hlstr{"blue"}\hlstd{,} \hlstr{"red"}\hlstd{,} \hlstr{"blue"}\hlstd{,} \hlstr{"gray"}\hlstd{,} \hlstr{"red"}\hlstd{)}
\hlstd{cols} \hlkwb{<-} \hlkwd{factor}\hlstd{(cols)}
\hlstd{cols}
\end{alltt}
\begin{verbatim}
## [1] blue red  blue gray red 
## Levels: blue gray red
\end{verbatim}
\end{kframe}
\end{knitrout}

The different values in a factor are called \textbf{levels}

\end{frame}

%------------------------------------------------

\begin{frame}
\frametitle{So far ...}

\bi 
  \item Vectors, matrices, and arrays are atomic structures (they can only store one type of data)
  \item Many operations in R need atomic structures to make sure all values are of the same mode
  \item In real life, however, many datasets contain multiple types of information
  \item R provides other data structures to store different types of data
\ei
\end{frame}

%------------------------------------------------

\begin{frame}
\begin{center}
\Huge{\hilit{Lists}}
\end{center}
\end{frame}

%------------------------------------------------

\begin{frame}
\frametitle{Lists}

\bi 
  \item Lists are the most general class of data container
  \item Like vectors, lists group data into a one-dimensional set
  \item Unlike vectors, lists can store all kinds of objects
  \item Lists can be of any length
  \item Elements of a list can be named, or not
\ei
\end{frame}

%------------------------------------------------

\begin{frame}[fragile]
\frametitle{Lists}

To create a list we use the function {\hilit \code{list()}}. The \code{list()} function creates a list the same way \code{c()} creates a vector:
\begin{knitrout}\footnotesize
\definecolor{shadecolor}{rgb}{0.969, 0.969, 0.969}\color{fgcolor}\begin{kframe}
\begin{alltt}
\hlstd{lfriends} \hlkwb{<-} \hlkwd{list}\hlstd{(}\hlstr{"Harry"}\hlstd{,} \hlstr{"Ron"}\hlstd{,} \hlstr{"Hermione"}\hlstd{)}
\hlstd{lfriends}
\end{alltt}
\begin{verbatim}
## [[1]]
## [1] "Harry"
## 
## [[2]]
## [1] "Ron"
## 
## [[3]]
## [1] "Hermione"
\end{verbatim}
\end{kframe}
\end{knitrout}

\end{frame}

%------------------------------------------------

\begin{frame}[fragile]
\frametitle{Lists}

Lists can contain any type of data object (even other lists):
\begin{knitrout}\footnotesize
\definecolor{shadecolor}{rgb}{0.969, 0.969, 0.969}\color{fgcolor}\begin{kframe}
\begin{alltt}
\hlstd{lst} \hlkwb{<-} \hlkwd{list}\hlstd{(}
  \hlkwd{c}\hlstd{(}\hlstr{"Harry"}\hlstd{,} \hlstr{"Ron"}\hlstd{,} \hlstr{"Hermione"}\hlstd{),}
  \hlkwd{matrix}\hlstd{(}\hlnum{1}\hlopt{:}\hlnum{6}\hlstd{,} \hlkwc{nrow} \hlstd{=} \hlnum{2}\hlstd{,} \hlkwc{ncol} \hlstd{=} \hlnum{3}\hlstd{),}
  \hlkwd{factor}\hlstd{(}\hlkwd{c}\hlstd{(}\hlstr{"yes"}\hlstd{,} \hlstr{"no"}\hlstd{,} \hlstr{"no"}\hlstd{,} \hlstr{"no"}\hlstd{,} \hlstr{"yes"}\hlstd{)),}
  \hlstd{lfriends}
\hlstd{)}
\end{alltt}
\end{kframe}
\end{knitrout}

\end{frame}

%------------------------------------------------

\begin{frame}[fragile]
\frametitle{Lists}

Elements in a list can be named:
\begin{knitrout}\footnotesize
\definecolor{shadecolor}{rgb}{0.969, 0.969, 0.969}\color{fgcolor}\begin{kframe}
\begin{alltt}
\hlkwd{list}\hlstd{(}
  \hlkwc{first} \hlstd{=} \hlkwd{c}\hlstd{(}\hlstr{"Harry"}\hlstd{,} \hlstr{"Ron"}\hlstd{,} \hlstr{"Hermione"}\hlstd{),}
  \hlkwc{second} \hlstd{=} \hlkwd{matrix}\hlstd{(}\hlnum{1}\hlopt{:}\hlnum{6}\hlstd{,} \hlkwc{nrow} \hlstd{=} \hlnum{2}\hlstd{,} \hlkwc{ncol} \hlstd{=} \hlnum{3}\hlstd{),}
  \hlkwc{third} \hlstd{=} \hlkwd{factor}\hlstd{(}\hlkwd{c}\hlstd{(}\hlstr{"yes"}\hlstd{,} \hlstr{"no"}\hlstd{,} \hlstr{"no"}\hlstd{,} \hlstr{"no"}\hlstd{,} \hlstr{"yes"}\hlstd{)),}
  \hlkwc{fourth} \hlstd{= lfriends}
\hlstd{)}
\end{alltt}
\end{kframe}
\end{knitrout}

\end{frame}

%------------------------------------------------

\begin{frame}[fragile]
\frametitle{Creating a List}

Exercise: create a list with your first name, middle name, and last name
\begin{knitrout}\footnotesize
\definecolor{shadecolor}{rgb}{0.969, 0.969, 0.969}\color{fgcolor}\begin{kframe}
\begin{verbatim}
## $first
## [1] "Gaston"
## 
## $middle
## NULL
## 
## $last
## [1] "Sanchez"
\end{verbatim}
\end{kframe}
\end{knitrout}

\end{frame}

%------------------------------------------------

\begin{frame}
\begin{center}
\Huge{\hilit{Data Frames}}
\end{center}
\end{frame}

%------------------------------------------------

\begin{frame}
\frametitle{Data Frames}

\bb{Data Frame}
A {\hilit \code{data.frame}} is the primary data structure that R provides for handling tabular data sets (e.g. spreadsheet like).
\eb

\bb{Function \code{data.frame()}}
The {\hilit \code{data.frame()}} function allows us to create data frames 
\eb

\end{frame}

%------------------------------------------------

\begin{frame}
\frametitle{Lists}

\bi 
  \item Data frames are the two-dimensional version of a list
  \item They are the conventional data storage structure for data analysis
  \item A data frame is displayed like a table (or matrix)
  \item A data frame is stored as a list
\ei
\end{frame}

%------------------------------------------------

\begin{frame}[fragile]
\frametitle{Creating a Data Frame}

\begin{knitrout}\footnotesize
\definecolor{shadecolor}{rgb}{0.969, 0.969, 0.969}\color{fgcolor}\begin{kframe}
\begin{alltt}
\hlcom{# creating a data frame (manually)}
\hlstd{elements} \hlkwb{<-} \hlkwd{data.frame}\hlstd{(}
  \hlkwc{name} \hlstd{=} \hlkwd{c}\hlstd{(}\hlstr{'hydrogen'}\hlstd{,} \hlstr{'nitrogen'}\hlstd{,} \hlstr{'oxygen'}\hlstd{),}
  \hlkwc{symbol} \hlstd{=} \hlkwd{c}\hlstd{(}\hlstr{'H'}\hlstd{,} \hlstr{'N'}\hlstd{,} \hlstr{'O'}\hlstd{),}
  \hlkwc{number} \hlstd{=} \hlkwd{c}\hlstd{(}\hlnum{1}\hlstd{,} \hlnum{7}\hlstd{,} \hlnum{8}\hlstd{)}
\hlstd{)}

\hlstd{elements}
\end{alltt}
\begin{verbatim}
##       name symbol number
## 1 hydrogen      H      1
## 2 nitrogen      N      7
## 3   oxygen      O      8
\end{verbatim}
\end{kframe}
\end{knitrout}

(vectors defined inside the data frame function)

\end{frame}

%------------------------------------------------

\begin{frame}
\frametitle{Creating a Data Frame}

\bi 
  \item Give \code{data.frame()} any number of vectors, each separated with a comma
  \item Each vector should be set equal to a name that describes the vector
  \item \code{data.frame()} will turn each vector into a column of the new data frame
  \item All vectors should be of the same length
  \item By default, \code{data.frame()} converts strings into factors
\ei
\end{frame}

%------------------------------------------------

\begin{frame}[fragile]
\frametitle{Creating a Data Frame}

Create the following data frame:
\begin{knitrout}\footnotesize
\definecolor{shadecolor}{rgb}{0.969, 0.969, 0.969}\color{fgcolor}\begin{kframe}
\begin{verbatim}
##        state abbreviation    capital   area
## 1 California           CA Sacramento 163707
## 2   New York           NY     Albany  54475
## 3      Texas           TX     Austin 268601
\end{verbatim}
\end{kframe}
\end{knitrout}

\end{frame}

%------------------------------------------------

\begin{frame}[fragile]
\frametitle{Creating a Data Frame}

Solution:
\begin{knitrout}\footnotesize
\definecolor{shadecolor}{rgb}{0.969, 0.969, 0.969}\color{fgcolor}\begin{kframe}
\begin{alltt}
\hlstd{states} \hlkwb{<-} \hlkwd{data.frame}\hlstd{(}
  \hlkwc{state} \hlstd{=} \hlkwd{c}\hlstd{(}\hlstr{'California'}\hlstd{,} \hlstr{'New York'}\hlstd{,} \hlstr{'Texas'}\hlstd{),}
  \hlkwc{abbreviation} \hlstd{=} \hlkwd{c}\hlstd{(}\hlstr{'CA'}\hlstd{,} \hlstr{'NY'}\hlstd{,} \hlstr{'TX'}\hlstd{),}
  \hlkwc{capital} \hlstd{=} \hlkwd{c}\hlstd{(}\hlstr{'Sacramento'}\hlstd{,} \hlstr{'Albany'}\hlstd{,} \hlstr{'Austin'}\hlstd{),}
  \hlkwc{area} \hlstd{=} \hlkwd{c}\hlstd{(}\hlnum{163707}\hlstd{,} \hlnum{54475}\hlstd{,} \hlnum{268601}\hlstd{),}
  \hlkwc{stringsAsFactors} \hlstd{=} \hlnum{FALSE}
\hlstd{)}

\hlstd{states}
\end{alltt}
\begin{verbatim}
##        state abbreviation    capital   area
## 1 California           CA Sacramento 163707
## 2   New York           NY     Albany  54475
## 3      Texas           TX     Austin 268601
\end{verbatim}
\end{kframe}
\end{knitrout}

\end{frame}

%------------------------------------------------

\begin{frame}[fragile]
\frametitle{R common data structures}
\begin{center}
\ig[width=8cm]{images/data_structures.pdf}
\end{center}
\end{frame}

%------------------------------------------------

\begin{frame}
\begin{center}
\Huge{\hilit{Selecting Values}}
\end{center}
\end{frame}

%------------------------------------------------

\begin{frame}
\frametitle{Selecting Values}
Data manipulation requires you to learn how to select and retrieve values from each of the data structures
\end{frame}

%------------------------------------------------

\begin{frame}
\frametitle{Notation System}

\bb{Notation system to extract values from R objects}
\bbi
  \item to extract values use brackets: \code{[ ]}
  \item inside the brackets specify indices
  \item use as many indices as dimensions in the object
  \item each index is separated by comma
  \item indices can be numbers, logicals, or names
\ei
\eb
\end{frame}

%------------------------------------------------

\begin{frame}[fragile]
\frametitle{Values of Vectors}

\begin{knitrout}\footnotesize
\definecolor{shadecolor}{rgb}{0.969, 0.969, 0.969}\color{fgcolor}\begin{kframe}
\begin{alltt}
\hlcom{# some vector}
\hlstd{vec} \hlkwb{<-} \hlnum{1}\hlopt{:}\hlnum{5}

\hlcom{# adding names}
\hlkwd{names}\hlstd{(vec)} \hlkwb{<-} \hlstd{letters[}\hlnum{1}\hlopt{:}\hlnum{5}\hlstd{]}

\hlstd{vec}
\end{alltt}
\begin{verbatim}
## a b c d e 
## 1 2 3 4 5
\end{verbatim}
\end{kframe}
\end{knitrout}

\end{frame}

%------------------------------------------------

\begin{frame}[fragile]
\frametitle{Extracting values with numbers}

\begin{knitrout}\footnotesize
\definecolor{shadecolor}{rgb}{0.969, 0.969, 0.969}\color{fgcolor}\begin{kframe}
\begin{alltt}
\hlcom{# first element }
\hlstd{vec[}\hlnum{1}\hlstd{]}
\end{alltt}
\begin{verbatim}
## a 
## 1
\end{verbatim}
\begin{alltt}
\hlcom{# third element}
\hlstd{vec[}\hlnum{3}\hlstd{]}
\end{alltt}
\begin{verbatim}
## c 
## 3
\end{verbatim}
\begin{alltt}
\hlcom{# fifth element}
\hlstd{vec[}\hlnum{5}\hlstd{]}
\end{alltt}
\begin{verbatim}
## e 
## 5
\end{verbatim}
\end{kframe}
\end{knitrout}

\end{frame}

%------------------------------------------------

\begin{frame}[fragile]
\frametitle{Extracting values with positive vectors}

\begin{knitrout}\footnotesize
\definecolor{shadecolor}{rgb}{0.969, 0.969, 0.969}\color{fgcolor}\begin{kframe}
\begin{alltt}
\hlcom{# range 1 to 3}
\hlstd{vec[}\hlnum{1}\hlopt{:}\hlnum{3}\hlstd{]}
\end{alltt}
\begin{verbatim}
## a b c 
## 1 2 3
\end{verbatim}
\begin{alltt}
\hlcom{# vector of positive numbers}
\hlstd{vec[}\hlkwd{c}\hlstd{(}\hlnum{1}\hlstd{,} \hlnum{3}\hlstd{,} \hlnum{4}\hlstd{)]}
\end{alltt}
\begin{verbatim}
## a c d 
## 1 3 4
\end{verbatim}
\end{kframe}
\end{knitrout}

\end{frame}

%------------------------------------------------

\begin{frame}[fragile]
\frametitle{Extracting values with negative numbers}

\begin{knitrout}\footnotesize
\definecolor{shadecolor}{rgb}{0.969, 0.969, 0.969}\color{fgcolor}\begin{kframe}
\begin{alltt}
\hlcom{# all values except the first one}
\hlstd{vec[}\hlopt{-}\hlnum{1}\hlstd{]}
\end{alltt}
\begin{verbatim}
## b c d e 
## 2 3 4 5
\end{verbatim}
\begin{alltt}
\hlcom{# all values except 2nd and 4th}
\hlstd{vec[}\hlopt{-}\hlkwd{c}\hlstd{(}\hlnum{2}\hlstd{,} \hlnum{4}\hlstd{)]}
\end{alltt}
\begin{verbatim}
## a c e 
## 1 3 5
\end{verbatim}
\end{kframe}
\end{knitrout}

\end{frame}

%------------------------------------------------

\begin{frame}[fragile]
\frametitle{Extracting values with logicals}

\begin{knitrout}\footnotesize
\definecolor{shadecolor}{rgb}{0.969, 0.969, 0.969}\color{fgcolor}\begin{kframe}
\begin{alltt}
\hlcom{# first element}
\hlstd{vec[}\hlkwd{c}\hlstd{(}\hlnum{TRUE}\hlstd{,} \hlnum{FALSE}\hlstd{,} \hlnum{FALSE}\hlstd{,} \hlnum{FALSE}\hlstd{,} \hlnum{FALSE}\hlstd{)]}
\end{alltt}
\begin{verbatim}
## a 
## 1
\end{verbatim}
\begin{alltt}
\hlcom{# 4th and 5th elements}
\hlstd{vec[}\hlkwd{c}\hlstd{(}\hlnum{FALSE}\hlstd{,} \hlnum{FALSE}\hlstd{,} \hlnum{FALSE}\hlstd{,} \hlnum{TRUE}\hlstd{,} \hlnum{TRUE}\hlstd{)]}
\end{alltt}
\begin{verbatim}
## d e 
## 4 5
\end{verbatim}
\begin{alltt}
\hlcom{# logical negation (2nd and 4th)}
\hlstd{vec[}\hlopt{!}\hlkwd{c}\hlstd{(}\hlnum{TRUE}\hlstd{,} \hlnum{FALSE}\hlstd{,} \hlnum{TRUE}\hlstd{,} \hlnum{FALSE}\hlstd{,} \hlnum{TRUE}\hlstd{)]}
\end{alltt}
\begin{verbatim}
## b d 
## 2 4
\end{verbatim}
\end{kframe}
\end{knitrout}

\end{frame}

%------------------------------------------------

\begin{frame}[fragile]
\frametitle{Extracting values with names}

Since \code{vec} has names, we can use characters to extract named values
\begin{knitrout}\footnotesize
\definecolor{shadecolor}{rgb}{0.969, 0.969, 0.969}\color{fgcolor}\begin{kframe}
\begin{alltt}
\hlcom{# element 'a'}
\hlstd{vec[}\hlstr{'a'}\hlstd{]}
\end{alltt}
\begin{verbatim}
## a 
## 1
\end{verbatim}
\begin{alltt}
\hlcom{# elements 'b' and 'e'}
\hlstd{vec[}\hlkwd{c}\hlstd{(}\hlstr{'b'}\hlstd{,} \hlstr{'e'}\hlstd{)]}
\end{alltt}
\begin{verbatim}
## b e 
## 2 5
\end{verbatim}
\end{kframe}
\end{knitrout}

\end{frame}

%------------------------------------------------

\begin{frame}[fragile]
\frametitle{Extracting values with names}

You can use repeated indices:
\begin{knitrout}\footnotesize
\definecolor{shadecolor}{rgb}{0.969, 0.969, 0.969}\color{fgcolor}\begin{kframe}
\begin{alltt}
\hlcom{# element 2 (3-times)}
\hlstd{vec[}\hlkwd{c}\hlstd{(}\hlnum{2}\hlstd{,} \hlnum{2}\hlstd{,} \hlnum{2}\hlstd{)]}
\end{alltt}
\begin{verbatim}
## b b b 
## 2 2 2
\end{verbatim}
\begin{alltt}
\hlcom{# element 'a' (four times)}
\hlstd{vec[}\hlkwd{c}\hlstd{(}\hlstr{'a'}\hlstd{,} \hlstr{'a'}\hlstd{,} \hlstr{'a'}\hlstd{,} \hlstr{'a'}\hlstd{)]}
\end{alltt}
\begin{verbatim}
## a a a a 
## 1 1 1 1
\end{verbatim}
\end{kframe}
\end{knitrout}

\end{frame}

%------------------------------------------------

\begin{frame}[fragile]
\frametitle{Extracting values of two-dimensional objects}

\begin{knitrout}\footnotesize
\definecolor{shadecolor}{rgb}{0.969, 0.969, 0.969}\color{fgcolor}\begin{kframe}
\begin{alltt}
\hlstd{states}
\end{alltt}
\begin{verbatim}
##        state abbreviation    capital   area
## 1 California           CA Sacramento 163707
## 2   New York           NY     Albany  54475
## 3      Texas           TX     Austin 268601
\end{verbatim}
\end{kframe}
\end{knitrout}

\end{frame}

%------------------------------------------------

\begin{frame}[fragile]
\frametitle{Extracting values of two-dimensional objects}

Extracting a single cell
\begin{knitrout}\footnotesize
\definecolor{shadecolor}{rgb}{0.969, 0.969, 0.969}\color{fgcolor}\begin{kframe}
\begin{alltt}
\hlcom{# value in row 1, and column 1}
\hlstd{states[}\hlnum{1}\hlstd{,}\hlnum{1}\hlstd{]}
\end{alltt}
\begin{verbatim}
## [1] "California"
\end{verbatim}
\begin{alltt}
\hlcom{# value in row 3, and column 3}
\hlstd{states[}\hlnum{3}\hlstd{,}\hlnum{3}\hlstd{]}
\end{alltt}
\begin{verbatim}
## [1] "Austin"
\end{verbatim}
\end{kframe}
\end{knitrout}

\end{frame}

%------------------------------------------------

\begin{frame}[fragile]
\frametitle{Extracting values of two-dimensional objects}

To extract just rows, leave the column index empty
\begin{knitrout}\footnotesize
\definecolor{shadecolor}{rgb}{0.969, 0.969, 0.969}\color{fgcolor}\begin{kframe}
\begin{alltt}
\hlcom{# values in row 1}
\hlstd{states[}\hlnum{1}\hlstd{, ]}
\end{alltt}
\begin{verbatim}
##        state abbreviation    capital   area
## 1 California           CA Sacramento 163707
\end{verbatim}
\begin{alltt}
\hlcom{# values in rows 1 to 2}
\hlstd{states[}\hlnum{1}\hlopt{:}\hlnum{2}\hlstd{, ]}
\end{alltt}
\begin{verbatim}
##        state abbreviation    capital   area
## 1 California           CA Sacramento 163707
## 2   New York           NY     Albany  54475
\end{verbatim}
\end{kframe}
\end{knitrout}

\end{frame}

%------------------------------------------------

\begin{frame}[fragile]
\frametitle{Extracting values of two-dimensional objects}

To extract just columns, leave the row index empty
\begin{knitrout}\footnotesize
\definecolor{shadecolor}{rgb}{0.969, 0.969, 0.969}\color{fgcolor}\begin{kframe}
\begin{alltt}
\hlcom{# values in column 1}
\hlstd{states[ ,} \hlnum{1}\hlstd{]}
\end{alltt}
\begin{verbatim}
## [1] "California" "New York"   "Texas"
\end{verbatim}
\begin{alltt}
\hlcom{# values in columns 2 to 4}
\hlstd{states[ ,} \hlnum{2}\hlopt{:}\hlnum{4}\hlstd{]}
\end{alltt}
\begin{verbatim}
##   abbreviation    capital   area
## 1           CA Sacramento 163707
## 2           NY     Albany  54475
## 3           TX     Austin 268601
\end{verbatim}
\end{kframe}
\end{knitrout}

\end{frame}

%------------------------------------------------

\begin{frame}[fragile]
\frametitle{Extracting values of two-dimensional objects}

Extracting rows with negative indices
\begin{knitrout}\footnotesize
\definecolor{shadecolor}{rgb}{0.969, 0.969, 0.969}\color{fgcolor}\begin{kframe}
\begin{alltt}
\hlcom{# excluding 3rd row}
\hlstd{states[}\hlopt{-}\hlnum{3}\hlstd{, ]}
\end{alltt}
\begin{verbatim}
##        state abbreviation    capital   area
## 1 California           CA Sacramento 163707
## 2   New York           NY     Albany  54475
\end{verbatim}
\begin{alltt}
\hlcom{# excluding 1st and 3rd row}
\hlstd{states[}\hlopt{-}\hlkwd{c}\hlstd{(}\hlnum{1}\hlstd{,} \hlnum{3}\hlstd{), ]}
\end{alltt}
\begin{verbatim}
##      state abbreviation capital  area
## 2 New York           NY  Albany 54475
\end{verbatim}
\end{kframe}
\end{knitrout}

\end{frame}

%------------------------------------------------

\begin{frame}[fragile]
\frametitle{Extracting values of two-dimensional objects}

Extracting columns with negative indices
\begin{knitrout}\footnotesize
\definecolor{shadecolor}{rgb}{0.969, 0.969, 0.969}\color{fgcolor}\begin{kframe}
\begin{alltt}
\hlcom{# excluding 2nd column}
\hlstd{states[ ,} \hlopt{-}\hlnum{2}\hlstd{]}
\end{alltt}
\begin{verbatim}
##        state    capital   area
## 1 California Sacramento 163707
## 2   New York     Albany  54475
## 3      Texas     Austin 268601
\end{verbatim}
\begin{alltt}
\hlcom{# excluding 1st and 2dn columns}
\hlstd{states[ ,} \hlopt{-}\hlkwd{c}\hlstd{(}\hlnum{1}\hlstd{,} \hlnum{2}\hlstd{)]}
\end{alltt}
\begin{verbatim}
##      capital   area
## 1 Sacramento 163707
## 2     Albany  54475
## 3     Austin 268601
\end{verbatim}
\end{kframe}
\end{knitrout}

\end{frame}

%------------------------------------------------

\begin{frame}[fragile]
\frametitle{Extracting values of two-dimensional objects}

You can use repetition of indices
\begin{knitrout}\footnotesize
\definecolor{shadecolor}{rgb}{0.969, 0.969, 0.969}\color{fgcolor}\begin{kframe}
\begin{alltt}
\hlcom{# values in column 1}
\hlstd{states[ ,} \hlkwd{c}\hlstd{(}\hlnum{1}\hlstd{,} \hlnum{1}\hlstd{)]}
\end{alltt}
\begin{verbatim}
##        state    state.1
## 1 California California
## 2   New York   New York
## 3      Texas      Texas
\end{verbatim}
\begin{alltt}
\hlcom{# values in columns 2 to 4}
\hlstd{states[}\hlkwd{c}\hlstd{(}\hlnum{2}\hlstd{,} \hlnum{2}\hlstd{,} \hlnum{3}\hlstd{), ]}
\end{alltt}
\begin{verbatim}
##        state abbreviation capital   area
## 2   New York           NY  Albany  54475
## 2.1 New York           NY  Albany  54475
## 3      Texas           TX  Austin 268601
\end{verbatim}
\end{kframe}
\end{knitrout}

\end{frame}

%------------------------------------------------

\begin{frame}[fragile]
\frametitle{Extracting values of two-dimensional objects}

Extracting values with logicals
\begin{knitrout}\footnotesize
\definecolor{shadecolor}{rgb}{0.969, 0.969, 0.969}\color{fgcolor}\begin{kframe}
\begin{alltt}
\hlcom{# exclude 3rd row}
\hlstd{states[}\hlkwd{c}\hlstd{(}\hlnum{TRUE}\hlstd{,} \hlnum{TRUE}\hlstd{,} \hlnum{FALSE}\hlstd{), ]}
\end{alltt}
\begin{verbatim}
##        state abbreviation    capital   area
## 1 California           CA Sacramento 163707
## 2   New York           NY     Albany  54475
\end{verbatim}
\begin{alltt}
\hlcom{# exclude 3rd column}
\hlstd{states[  ,} \hlkwd{c}\hlstd{(}\hlnum{TRUE}\hlstd{,} \hlnum{TRUE}\hlstd{,} \hlnum{FALSE}\hlstd{,} \hlnum{TRUE}\hlstd{)]}
\end{alltt}
\begin{verbatim}
##        state abbreviation   area
## 1 California           CA 163707
## 2   New York           NY  54475
## 3      Texas           TX 268601
\end{verbatim}
\end{kframe}
\end{knitrout}

\end{frame}

%------------------------------------------------

\begin{frame}[fragile]
\frametitle{Extracting values of two-dimensional objects}

Extracting columns by name
\begin{knitrout}\footnotesize
\definecolor{shadecolor}{rgb}{0.969, 0.969, 0.969}\color{fgcolor}\begin{kframe}
\begin{alltt}
\hlcom{# column 'state'}
\hlstd{states[ ,} \hlstr{'state'}\hlstd{]}
\end{alltt}
\begin{verbatim}
## [1] "California" "New York"   "Texas"
\end{verbatim}
\begin{alltt}
\hlcom{# columns 'state' and 'abbreviation'}
\hlstd{states[ ,} \hlkwd{c}\hlstd{(}\hlstr{'state'}\hlstd{,} \hlstr{'abbreviation'}\hlstd{)]}
\end{alltt}
\begin{verbatim}
##        state abbreviation
## 1 California           CA
## 2   New York           NY
## 3      Texas           TX
\end{verbatim}
\end{kframe}
\end{knitrout}

\end{frame}

%------------------------------------------------

\begin{frame}[fragile]
\frametitle{Extracting values of two-dimensional objects}

When you select one column from a two-dimensional array, R will return a vector. To get a column output, use the argument \code{drop = FALSE}
\begin{knitrout}\footnotesize
\definecolor{shadecolor}{rgb}{0.969, 0.969, 0.969}\color{fgcolor}\begin{kframe}
\begin{alltt}
\hlcom{# columns 'state' and 'abbreviation'}
\hlstd{states[ ,} \hlnum{1}\hlstd{,} \hlkwc{drop} \hlstd{=} \hlnum{FALSE}\hlstd{]}
\end{alltt}
\begin{verbatim}
##        state
## 1 California
## 2   New York
## 3      Texas
\end{verbatim}
\end{kframe}
\end{knitrout}

\end{frame}

%------------------------------------------------

\begin{frame}
\begin{center}
\Huge{\hilit{Dollar signs and Double Brackets}}
\end{center}
\end{frame}

%------------------------------------------------

\begin{frame}
\frametitle{Dollar signs}

R lists and data frames obey an optional second system of notation for extracting values: using the dollar sign {\hilit \code{\$}}

\end{frame}

%------------------------------------------------

\begin{frame}[fragile]
\frametitle{Dollar signs with data frames}

The dollar sign \code{\$} notation works for selecting a column of a data frame using its name
\begin{knitrout}\footnotesize
\definecolor{shadecolor}{rgb}{0.969, 0.969, 0.969}\color{fgcolor}\begin{kframe}
\begin{alltt}
\hlstd{states}\hlopt{$}\hlstd{state}
\end{alltt}
\begin{verbatim}
## [1] "California" "New York"   "Texas"
\end{verbatim}
\begin{alltt}
\hlstd{states}\hlopt{$}\hlstd{abbreviation}
\end{alltt}
\begin{verbatim}
## [1] "CA" "NY" "TX"
\end{verbatim}
\begin{alltt}
\hlstd{states}\hlopt{$}\hlstd{capital}
\end{alltt}
\begin{verbatim}
## [1] "Sacramento" "Albany"     "Austin"
\end{verbatim}
\end{kframe}
\end{knitrout}

\end{frame}

%------------------------------------------------

\begin{frame}[fragile]
\frametitle{Dollar signs with data frames}

You don't need to use quote marks, but you can if you want
\begin{knitrout}\footnotesize
\definecolor{shadecolor}{rgb}{0.969, 0.969, 0.969}\color{fgcolor}\begin{kframe}
\begin{alltt}
\hlstd{states}\hlopt{$}\hlstr{'state'}
\end{alltt}
\begin{verbatim}
## [1] "California" "New York"   "Texas"
\end{verbatim}
\begin{alltt}
\hlstd{states}\hlopt{$}\hlstr{"abbreviation"}
\end{alltt}
\begin{verbatim}
## [1] "CA" "NY" "TX"
\end{verbatim}
\begin{alltt}
\hlstd{states}\hlopt{$}\hlstd{`capital`}
\end{alltt}
\begin{verbatim}
## [1] "Sacramento" "Albany"     "Austin"
\end{verbatim}
\end{kframe}
\end{knitrout}

\end{frame}

%------------------------------------------------

\begin{frame}[fragile]
\frametitle{Dollar signs with lists}

Elements in a named list can also be extracted with the dollar sign:
\begin{knitrout}\footnotesize
\definecolor{shadecolor}{rgb}{0.969, 0.969, 0.969}\color{fgcolor}\begin{kframe}
\begin{alltt}
\hlstd{lst} \hlkwb{<-} \hlkwd{list}\hlstd{(}\hlkwc{numbers} \hlstd{=} \hlnum{1}\hlopt{:}\hlnum{3}\hlstd{,} \hlkwc{letter} \hlstd{=} \hlkwd{c}\hlstd{(}\hlstr{'A'}\hlstd{,} \hlstr{'B'}\hlstd{,} \hlstr{'C'}\hlstd{))}

\hlstd{lst}\hlopt{$}\hlstd{numbers}
\end{alltt}
\begin{verbatim}
## [1] 1 2 3
\end{verbatim}
\begin{alltt}
\hlstd{lst}\hlopt{$}\hlstd{letters}
\end{alltt}
\begin{verbatim}
## NULL
\end{verbatim}
\end{kframe}
\end{knitrout}

\end{frame}

%------------------------------------------------

\begin{frame}[fragile]
\frametitle{Double brackets}

In addition, lists (and data frames) accept a third type of notation that uses double brackets: \code{[[ ]]}
\begin{knitrout}\footnotesize
\definecolor{shadecolor}{rgb}{0.969, 0.969, 0.969}\color{fgcolor}\begin{kframe}
\begin{alltt}
\hlstd{lst[[}\hlnum{1}\hlstd{]]}
\end{alltt}
\begin{verbatim}
## [1] 1 2 3
\end{verbatim}
\begin{alltt}
\hlstd{lst[[}\hlnum{2}\hlstd{]]}
\end{alltt}
\begin{verbatim}
## [1] "A" "B" "C"
\end{verbatim}
\end{kframe}
\end{knitrout}

\end{frame}

%------------------------------------------------

\begin{frame}[fragile]
\frametitle{Double brackets}

Double brackets are used when we want to get access to the individual elements; use double brackets followed by single brackets
\begin{knitrout}\footnotesize
\definecolor{shadecolor}{rgb}{0.969, 0.969, 0.969}\color{fgcolor}\begin{kframe}
\begin{alltt}
\hlstd{lst[[}\hlnum{2}\hlstd{]]}
\end{alltt}
\begin{verbatim}
## [1] "A" "B" "C"
\end{verbatim}
\begin{alltt}
\hlstd{lst[[}\hlnum{2}\hlstd{]][}\hlnum{3}\hlstd{]}
\end{alltt}
\begin{verbatim}
## [1] "C"
\end{verbatim}
\end{kframe}
\end{knitrout}

\end{frame}

%------------------------------------------------

\begin{frame}
\frametitle{Elements-Extraction Notation System}

\begin{center}
 \begin{tabular}{l l l}
  \hline
   object & notation & example \\
  \hline
  vector & \code{[ ]} & \code{v[1:5]} \\  
  factor & \code{[ ]} & \code{g[1:5]} \\  
  matrix & \code{[ , ]} & \code{m[1:5, 1:3]} \\  
  array & \code{[ , , ]} & \code{arr[1, 2, 3]} \\  
        & \code{[ , , , ]} & \code{arr[1, 2, 3, 4]} \\  
  list & \code{[ ]} & \code{lst[3]} \\
       & \code{[[ ]]} & \code{lst[[3]]} \\
       & \code{\$} & \code{lst\$name} \\
  data frame & \code{[ , ]} & \code{df[1, 2]} \\
             & \code{\$} & \code{df\$name} \\
  \hline
 \end{tabular}
\end{center}

\end{frame}

%------------------------------------------------

\end{document}
